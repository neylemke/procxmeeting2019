
  \documentclass[twoside]{article}
  \usepackage[affil-it]{authblk}
  \usepackage{lipsum} % Package to generate dummy text throughout this template
  \usepackage{eurosym}
  \usepackage[sc]{mathpazo} % Use the Palatino font
  \usepackage[T1]{fontenc} % Use 8-bit encoding that has 256 glyphs
  \usepackage[utf8]{inputenc}
  \linespread{1.05} % Line spacing-Palatino needs more space between lines
  \usepackage{microtype} % Slightly tweak font spacing for aesthetics\[IndentingNewLine]
  \usepackage[hmarginratio=1:1,top=32mm,columnsep=20pt]{geometry} % Document margins
  \usepackage{multicol} % Used for the two-column layout of the document
  \usepackage[hang,small,labelfont=bf,up,textfont=it,up]{caption} % Custom captions under//above floats in tables or figures
  \usepackage{booktabs} % Horizontal rules in tables
  \usepackage{float} % Required for tables and figures in the multi-column environment-they need to be placed in specific locations with the[H] (e.g. \begin{table}[H])
  \usepackage{hyperref} % For hyperlinks in the PDF
  \usepackage{lettrine} % The lettrine is the first enlarged letter at the beginning of the text
  \usepackage{paralist} % Used for the compactitem environment which makes bullet points with less space between them
  \usepackage{abstract} % Allows abstract customization
  \renewcommand{\abstractnamefont}{\normalfont\bfseries} 
  %\renewcommand{\abstracttextfont}{\normalfont\small\itshape} % Set the abstract itself to small italic text\[IndentingNewLine]
  \usepackage{titlesec} % Allows customization of titles
  \renewcommand\thesection{\Roman{section}} % Roman numerals for the sections
  \renewcommand\thesubsection{\Roman{subsection}} % Roman numerals for subsections
  \titleformat{\section}[block]{\large\scshape\centering}{\thesection.}{1em}{} % Change the look of the section titles
  \titleformat{\subsection}[block]{\large}{\thesubsection.}{1em}{} % Change the look of the section titles
  \usepackage{fancyhdr} % Headers and footers
  \pagestyle{fancy} % All pages have headers and footers
  \fancyhead{} % Blank out the default header
  \fancyfoot{} % Blank out the default footer
  \fancyhead[C]{X-meeting $\bullet$ October 2019 $\bullet$ Campos do  Jord\~ao} % Custom header text
  \fancyfoot[RO,LE]{} % Custom footer text
  %----------------------------------------------------------------------------------------
  % TITLE SECTION
  %---------------------------------------------------------------------------------------- 
 
 \title{\vspace{-15mm}\fontsize{24pt}{10pt}\selectfont\textbf{ TOLL receptor gene family evolution in insects }} % Article title
  
  
  \author{ Let\'{\i}cia Ferreira Lima, Rodrigo Jardim, Renata Schama }
  
  \affil{ Laborat\'orio de Biologia Computacional e Sistemas,  Oswaldo Cruz Institute- Fiocruz }
  \vspace{-5mm}
  \date{}
  
  %---------------------------------------------------------------------------------------- 
  
  \begin{document}
  
  
  \maketitle % Insert title
  
  
  \thispagestyle{fancy} % All pages have headers and footers
  %----------------------------------------------------------------------------------------  
  % ABSTRACT
  
  %----------------------------------------------------------------------------------------  
  
  \begin{abstract}
  Arthropoda can be found in almost every habitat on earth and many species are intimately related to human life,  for instance,  arthropod-born diseases as malaria and yellow fever. Insects presence in most habitats and their wide variety of diet and behavior also means that they encounter various microorganisms many of which may be pathogenic,  as result insects have evolved mechanisms for recognition and elimination pathogens,  which include the signaling Toll pathway. Toll receptor is transmembrane protein essential for embryonic development and immunity,  the induction of the Toll pathway by Gram-positive bacteria or fungi leads to the activation of cellular immunity as well as the systemic production of certain antimicrobial peptides. The Toll receptor is activated when the proteolytically cleaved ligand Spatzle binds to the receptor,  eventually leading to the binding with other protein of pathway and activation of NF-kB factors. To date,  nine genes have been found in Drosophila melanogaster genome and similar numbers were found in other insects. Insects are ideal models for the study of the diversity of gene families and their evolutionary mechanisms because they have a well studied and well-known phylogeny and there are many examples of evolutionary specialization that have arisen in different bloodlines,  such as hematophagy.  Phylogenetic analyses using the Toll domain of each sequence retrieved from 40 insects genome was able to divide the Toll family into three well supports clades. The results revealed that insect Toll domain formed three major clusters. Here,  we have shown that there is a variety of Toll copies in insect groups,  with duplications and gene losses,  our findings also show that the Toll9 does indeed appear to be closer to vertebrates than the other groups,  and may indicate that it may be the most closely related group. In this study,  Toll9 genes in the Hymenoptera order were not found,  which may suggest that the gene was lost in the order. Insects immune evolution like the Toll pathway is important for an understanding of vector biology and behavior,  helping in aspects of vector control and disease transmission.
  
  Funding:  \\ 
  \end{abstract}
  \end{document} 