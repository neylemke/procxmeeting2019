
  \documentclass[twoside]{article}
  \usepackage[affil-it]{authblk}
  \usepackage{lipsum} % Package to generate dummy text throughout this template
  \usepackage{eurosym}
  \usepackage[sc]{mathpazo} % Use the Palatino font
  \usepackage[T1]{fontenc} % Use 8-bit encoding that has 256 glyphs
  \usepackage[utf8]{inputenc}
  \linespread{1.05} % Line spacing-Palatino needs more space between lines
  \usepackage{microtype} % Slightly tweak font spacing for aesthetics\[IndentingNewLine]
  \usepackage[hmarginratio=1:1,top=32mm,columnsep=20pt]{geometry} % Document margins
  \usepackage{multicol} % Used for the two-column layout of the document
  \usepackage[hang,small,labelfont=bf,up,textfont=it,up]{caption} % Custom captions under//above floats in tables or figures
  \usepackage{booktabs} % Horizontal rules in tables
  \usepackage{float} % Required for tables and figures in the multi-column environment-they need to be placed in specific locations with the[H] (e.g. \begin{table}[H])
  \usepackage{hyperref} % For hyperlinks in the PDF
  \usepackage{lettrine} % The lettrine is the first enlarged letter at the beginning of the text
  \usepackage{paralist} % Used for the compactitem environment which makes bullet points with less space between them
  \usepackage{abstract} % Allows abstract customization
  \renewcommand{\abstractnamefont}{\normalfont\bfseries} 
  %\renewcommand{\abstracttextfont}{\normalfont\small\itshape} % Set the abstract itself to small italic text\[IndentingNewLine]
  \usepackage{titlesec} % Allows customization of titles
  \renewcommand\thesection{\Roman{section}} % Roman numerals for the sections
  \renewcommand\thesubsection{\Roman{subsection}} % Roman numerals for subsections
  \titleformat{\section}[block]{\large\scshape\centering}{\thesection.}{1em}{} % Change the look of the section titles
  \titleformat{\subsection}[block]{\large}{\thesubsection.}{1em}{} % Change the look of the section titles
  \usepackage{fancyhdr} % Headers and footers
  \pagestyle{fancy} % All pages have headers and footers
  \fancyhead{} % Blank out the default header
  \fancyfoot{} % Blank out the default footer
  \fancyhead[C]{X-meeting $\bullet$ October 2019 $\bullet$ Campos do  Jord\~ao} % Custom header text
  \fancyfoot[RO,LE]{} % Custom footer text
  %----------------------------------------------------------------------------------------
  % TITLE SECTION
  %---------------------------------------------------------------------------------------- 
 
 \title{\vspace{-15mm}\fontsize{24pt}{10pt}\selectfont\textbf{ A container-based pipeline for bacterial genome assembly and annotation }} % Article title
  
  
  \author{ Felipe Marques de Almeida, Georgios Joannis Pappas Junior }
  
  \affil{ Universidade de Bras\'{\i}lia }
  \vspace{-5mm}
  \date{}
  
  %---------------------------------------------------------------------------------------- 
  
  \begin{document}
  
  
  \maketitle % Insert title
  
  
  \thispagestyle{fancy} % All pages have headers and footers
  %----------------------------------------------------------------------------------------  
  % ABSTRACT
  
  %----------------------------------------------------------------------------------------  
  
  \begin{abstract}
  Advances in DNA sequencing technologies are reshaping bacterial genomics studies enabling chromosome level assemblies,  at a fraction of cost and time,  paving the way to population level genomic surveys. At the present,  the computational analysis of sequencing data is the main hindrance to the field and withholds its move into mainstream clinical settings. To overcome this barrier we developed a complete container-based pipeline for bacterial genomics analysis,  meaning that given raw sequencing data from multiple platforms (Illumina,  Pacbio and Oxford Nanopore),  it performs genome assembly and annotation,  enabling identification and visualization of antibiotic resistance genes,  virulence factors,  prophages and integrative elements. In general terms the annotation phase of the pipeline can be executed in a few hours in a laptop. The assembly module,  despite requiring a large amount of memory (>64 Gb RAM),  can be executed in a day. The pipeline is designed to be modular,  taking into account different analytical scenarios readily configured by the user. We also leverage the use of operating system virtualization meaning that there is no need for user installation of required pipeline components. When ready,  all modules will be made available through GitHub. In conclusion,  this pipeline offers a seamless exposition of computational tools to bridge the gap toward routine bacterial genomics.
  
  Funding: Conselho Nacional de Desenvolvimento Cient\'{\i}fico e Tecnol\'ogico (CNPq) e Universidade de Bras\'{\i}lia (UnB) \\ 
  \end{abstract}
  \end{document} 