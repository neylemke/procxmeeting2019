
  \documentclass[twoside]{article}
  \usepackage[affil-it]{authblk}
  \usepackage{lipsum} % Package to generate dummy text throughout this template
  \usepackage{eurosym}
  \usepackage[sc]{mathpazo} % Use the Palatino font
  \usepackage[T1]{fontenc} % Use 8-bit encoding that has 256 glyphs
  \usepackage[utf8]{inputenc}
  \linespread{1.05} % Line spacing-Palatino needs more space between lines
  \usepackage{microtype} % Slightly tweak font spacing for aesthetics\[IndentingNewLine]
  \usepackage[hmarginratio=1:1,top=32mm,columnsep=20pt]{geometry} % Document margins
  \usepackage{multicol} % Used for the two-column layout of the document
  \usepackage[hang,small,labelfont=bf,up,textfont=it,up]{caption} % Custom captions under//above floats in tables or figures
  \usepackage{booktabs} % Horizontal rules in tables
  \usepackage{float} % Required for tables and figures in the multi-column environment-they need to be placed in specific locations with the[H] (e.g. \begin{table}[H])
  \usepackage{hyperref} % For hyperlinks in the PDF
  \usepackage{lettrine} % The lettrine is the first enlarged letter at the beginning of the text
  \usepackage{paralist} % Used for the compactitem environment which makes bullet points with less space between them
  \usepackage{abstract} % Allows abstract customization
  \renewcommand{\abstractnamefont}{\normalfont\bfseries} 
  %\renewcommand{\abstracttextfont}{\normalfont\small\itshape} % Set the abstract itself to small italic text\[IndentingNewLine]
  \usepackage{titlesec} % Allows customization of titles
  \renewcommand\thesection{\Roman{section}} % Roman numerals for the sections
  \renewcommand\thesubsection{\Roman{subsection}} % Roman numerals for subsections
  \titleformat{\section}[block]{\large\scshape\centering}{\thesection.}{1em}{} % Change the look of the section titles
  \titleformat{\subsection}[block]{\large}{\thesubsection.}{1em}{} % Change the look of the section titles
  \usepackage{fancyhdr} % Headers and footers
  \pagestyle{fancy} % All pages have headers and footers
  \fancyhead{} % Blank out the default header
  \fancyfoot{} % Blank out the default footer
  \fancyhead[C]{X-meeting $\bullet$ October 2019 $\bullet$ Campos do  Jord\~ao} % Custom header text
  \fancyfoot[RO,LE]{} % Custom footer text
  %----------------------------------------------------------------------------------------
  % TITLE SECTION
  %---------------------------------------------------------------------------------------- 
 
 \title{\vspace{-15mm}\fontsize{24pt}{10pt}\selectfont\textbf{ Reconstruction of metabolic pathways of Klebsiella spp. bacteria for improve the biologic control of Mediterranean fly (Ceratitis capitata). }} % Article title
  
  
  \author{ Luis Augusto Franco L\'opez, C\'esar Alberto Bravo Pariente }
  
  \affil{ Universidade Estadual de Santa Cruz,  Ilh\'eus,  Bahia,  Brazil }
  \vspace{-5mm}
  \date{}
  
  %---------------------------------------------------------------------------------------- 
  
  \begin{document}
  
  
  \maketitle % Insert title
  
  
  \thispagestyle{fancy} % All pages have headers and footers
  %----------------------------------------------------------------------------------------  
  % ABSTRACT
  
  %----------------------------------------------------------------------------------------  
  
  \begin{abstract}
  Pests are known to cause significant damage to crops and affect agriculture productivity. Mediterranian fruit fly (Ceratitis capitata) is a serious horticultural pest,  it attacks a range of cultivated fruits and vegetables. Several findings show that bacteria are present in the Mediterranean fruit fly microbiota and one of the most important bacteria is Klebsiella spp. Nowadays,  scientists develop multiple methods to control pests as the Sterile Insect Technique (SIT) to release infertile males produced by irradiation. There are some metabolic pathways in Klebsiella spp. that can improve the behavior of the irradiated Mediterranean fruit fly males at the moment of liberation. The possible relationship between these organisms produces multiple quantities of biological data,  to explain it,  is necessary the use of mathematical models and computational scientific techniques. Processing these data in an efficient way facilitates its interpretation and scientific application. The objective of the present work is to reconstruct metabolic pathways of three strains of Klebsiella spp.,  from genome scale and choose one strain that can improve the biological control of C. capitata. Two approaches were used; 1) Stoichiometric reconstructions of pathways and 2) Directed graphs to analyze and evaluate the pathway analysis of the metabolic pathway generated with Elementary Modes (EM) analysis. From genome scale the sequences of the three bacteria were used to find possible metabolic pathways metabolites present in the nitrogen metabolism at these sequences using BLAST tool,  those genes are related with symbiotic interaction between bacteria Klebsiella spp and C. capitata. The metabolites in the sequence indicated a possible metabolic pathway in the organism. Using KEGG database with the possible metabolic pathways involved we generate a directed graph and with it,  then we produce a stoichiometric matrix with rows representing the metabolites that participate in reactions and columns with the number of molecules of metabolites involved (stoichiometric coefficients) in one reaction. Based on this information we analyzed and evaluate the pathway analysis with EM. The above-discussed considerations suggest that we can identify possible metabolic pathways involved in the process of symbiosis between Klebsiella sp. and C. capitata and we can generate a strategie to improve the production of infertile males of C. Capitata for the liberation and population control.
  
  Funding:  \\ 
  \end{abstract}
  \end{document} 