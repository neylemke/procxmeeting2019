
  \documentclass[twoside]{article}
  \usepackage[affil-it]{authblk}
  \usepackage{lipsum} % Package to generate dummy text throughout this template
  \usepackage{eurosym}
  \usepackage[sc]{mathpazo} % Use the Palatino font
  \usepackage[T1]{fontenc} % Use 8-bit encoding that has 256 glyphs
  \usepackage[utf8]{inputenc}
  \linespread{1.05} % Line spacing-Palatino needs more space between lines
  \usepackage{microtype} % Slightly tweak font spacing for aesthetics\[IndentingNewLine]
  \usepackage[hmarginratio=1:1,top=32mm,columnsep=20pt]{geometry} % Document margins
  \usepackage{multicol} % Used for the two-column layout of the document
  \usepackage[hang,small,labelfont=bf,up,textfont=it,up]{caption} % Custom captions under//above floats in tables or figures
  \usepackage{booktabs} % Horizontal rules in tables
  \usepackage{float} % Required for tables and figures in the multi-column environment-they need to be placed in specific locations with the[H] (e.g. \begin{table}[H])
  \usepackage{hyperref} % For hyperlinks in the PDF
  \usepackage{lettrine} % The lettrine is the first enlarged letter at the beginning of the text
  \usepackage{paralist} % Used for the compactitem environment which makes bullet points with less space between them
  \usepackage{abstract} % Allows abstract customization
  \renewcommand{\abstractnamefont}{\normalfont\bfseries} 
  %\renewcommand{\abstracttextfont}{\normalfont\small\itshape} % Set the abstract itself to small italic text\[IndentingNewLine]
  \usepackage{titlesec} % Allows customization of titles
  \renewcommand\thesection{\Roman{section}} % Roman numerals for the sections
  \renewcommand\thesubsection{\Roman{subsection}} % Roman numerals for subsections
  \titleformat{\section}[block]{\large\scshape\centering}{\thesection.}{1em}{} % Change the look of the section titles
  \titleformat{\subsection}[block]{\large}{\thesubsection.}{1em}{} % Change the look of the section titles
  \usepackage{fancyhdr} % Headers and footers
  \pagestyle{fancy} % All pages have headers and footers
  \fancyhead{} % Blank out the default header
  \fancyfoot{} % Blank out the default footer
  \fancyhead[C]{X-meeting $\bullet$ October 2019 $\bullet$ Campos do  Jord\~ao} % Custom header text
  \fancyfoot[RO,LE]{} % Custom footer text
  %----------------------------------------------------------------------------------------
  % TITLE SECTION
  %---------------------------------------------------------------------------------------- 
 
 \title{\vspace{-15mm}\fontsize{24pt}{10pt}\selectfont\textbf{ HT Atlas v1.0 database: redefining human and mouse housekeeping genes by mining massive RNA-seq datasets }} % Article title
  
  
  \author{ Bidossessi Wilfried Hounkpe, Francine Chenou, Franciele de Lima, Erich Vinicius de Paula }
  
  \affil{ Faculty of Medical Sciences,  Unicamp }
  \vspace{-5mm}
  \date{}
  
  %---------------------------------------------------------------------------------------- 
  
  \begin{document}
  
  
  \maketitle % Insert title
  
  
  \thispagestyle{fancy} % All pages have headers and footers
  %----------------------------------------------------------------------------------------  
  % ABSTRACT
  
  %----------------------------------------------------------------------------------------  
  
  \begin{abstract}
  Housekeeping (HK) genes are constitutively expressed genes that are required for the maintenance of basic cellular functions. Despite their importance in the calibration of gene expression,  as well as the understanding of many genomic and evolutionary features,  important discrepancies have been observed in studies that previously identified these genes. Here,  we present Housekeeping Transcript Atlas (HT Atlas v1.0,  www.housekeeping.unicamp.br) a web-based database which addresses some of the previously observed limitations in the identification of these genes,  and offers a more accurate database of human and mouse HK genes and transcripts. The database was generated by mining massive human and mouse RNA-seq data sets from GTEx portal and ARCHS4 database. In total,  12, 482 and 507 high-quality RNA-seq data sets from 82 human non-disease tissues/cells and 15 healthy tissues/cells of C57BL/6 wild type mouse,  were respectively included in our workflow. 2, 158 human transcripts from 2, 176 genes fulfilled our criteria and were refered as  HK transcripts and HK genes. In the mouse database,  3, 024 HK transcripts from 3, 277 HK genes were identified. From the web interface,  user can visualize the expression of those transcripts across tissues and download full lists of HK genes and transcripts. HT Atlas v1.0 also offers the most stable and suitable tissue selective reference transcripts for normalization of qPCR experiments. Some reference transcript-specific primers and predicted modifiers of gene expression for some of these HK transcripts are also proposed. All of these resources can be accessed and downloaded from any computer or small device web browsers. The database is a dockerized ShinyApp that can also be pulled from docker hub (bidossessi/ht\_atlas\_v1.0) in order to be locally deployed by the user.
  
  Funding: FAPESP grants \# 2016/14172-6,   2015/24666-3; CNPq Brazil. grant \# 309317/2016 \\ 
  \end{abstract}
  \end{document} 