
  \documentclass[twoside]{article}
  \usepackage[affil-it]{authblk}
  \usepackage{lipsum} % Package to generate dummy text throughout this template
  \usepackage{eurosym}
  \usepackage[sc]{mathpazo} % Use the Palatino font
  \usepackage[T1]{fontenc} % Use 8-bit encoding that has 256 glyphs
  \usepackage[utf8]{inputenc}
  \linespread{1.05} % Line spacing-Palatino needs more space between lines
  \usepackage{microtype} % Slightly tweak font spacing for aesthetics\[IndentingNewLine]
  \usepackage[hmarginratio=1:1,top=32mm,columnsep=20pt]{geometry} % Document margins
  \usepackage{multicol} % Used for the two-column layout of the document
  \usepackage[hang,small,labelfont=bf,up,textfont=it,up]{caption} % Custom captions under//above floats in tables or figures
  \usepackage{booktabs} % Horizontal rules in tables
  \usepackage{float} % Required for tables and figures in the multi-column environment-they need to be placed in specific locations with the[H] (e.g. \begin{table}[H])
  \usepackage{hyperref} % For hyperlinks in the PDF
  \usepackage{lettrine} % The lettrine is the first enlarged letter at the beginning of the text
  \usepackage{paralist} % Used for the compactitem environment which makes bullet points with less space between them
  \usepackage{abstract} % Allows abstract customization
  \renewcommand{\abstractnamefont}{\normalfont\bfseries} 
  %\renewcommand{\abstracttextfont}{\normalfont\small\itshape} % Set the abstract itself to small italic text\[IndentingNewLine]
  \usepackage{titlesec} % Allows customization of titles
  \renewcommand\thesection{\Roman{section}} % Roman numerals for the sections
  \renewcommand\thesubsection{\Roman{subsection}} % Roman numerals for subsections
  \titleformat{\section}[block]{\large\scshape\centering}{\thesection.}{1em}{} % Change the look of the section titles
  \titleformat{\subsection}[block]{\large}{\thesubsection.}{1em}{} % Change the look of the section titles
  \usepackage{fancyhdr} % Headers and footers
  \pagestyle{fancy} % All pages have headers and footers
  \fancyhead{} % Blank out the default header
  \fancyfoot{} % Blank out the default footer
  \fancyhead[C]{X-meeting $\bullet$ October 2019 $\bullet$ Campos do  Jord\~ao} % Custom header text
  \fancyfoot[RO,LE]{} % Custom footer text
  %----------------------------------------------------------------------------------------
  % TITLE SECTION
  %---------------------------------------------------------------------------------------- 
 
 \title{\vspace{-15mm}\fontsize{24pt}{10pt}\selectfont\textbf{ INSIGHTS OF THE IRF1 DNA BINDING DOMAIN: MODELING AND DYNAMICS OF ITS 3D STRUCTURE }} % Article title
  
  
  \author{ Cinthia Caroline Alves, Eduardo Ant\^onio Donadi, Silvana Giuliatti }
  
  \affil{ Ribeirao Preto Medical School }
  \vspace{-5mm}
  \date{}
  
  %---------------------------------------------------------------------------------------- 
  
  \begin{document}
  
  
  \maketitle % Insert title
  
  
  \thispagestyle{fancy} % All pages have headers and footers
  %----------------------------------------------------------------------------------------  
  % ABSTRACT
  
  %----------------------------------------------------------------------------------------  
  
  \begin{abstract}
  Interferon regulatory factor 1 (IRF1) is a member of a closely related proteins referred as the IRF family,  which encodes a transcriptional factor (TF) responsible for the transcriptional regulation of several interferon-inducible genes involved in innate and adaptive immunity. This protein contains a DNA-binding domain (DBD) in its N-terminal region which forms a helix-turn-helix motif and recognizes the interferon-stimulated response element (ISRE) in the promoters of target genes. Mutations at its DBD was associated with cancer development,  such as gastric cancer. Then,  in the present study,  we investigate the structure and dynamics of the IRF1 DBD to better understand its role as a transcription factor. In the Protein Data Bank (PDB - https://www.rcsb.org/),  DBD crystal of the IRF1 TF (PDB code: 1IF1,  with 3$\AA$ of resolution) presented missing atoms in side chains,  which does not allow protein structure dynamics study. Then,  this crystal was used as template to perform homology modelling of human IRF1 DBD  which encompasses the residues 5-113 (UniProt code: P10914 - http://uniprot.org/) using MODELLER 9.14  to generate 5 models. These models were evaluated according their stereochemical properties quality (PROCHECK) and visual analysis (PyMOL2.0) comparing template tertiary structure with homology modeling models. The best quality model was used for molecular dynamic simulations of 100 ns using GROMACS 5.1 to analyze the protein behavior in solution. Homology modeling allowed satisfactory IRF1 DBD prediction models in comparison to the template: quality torsion angle assessment of the best predicted model presented 85.4\% of the residues in the core region of phi-psi torsion angles,  while template presents 57.6\% of the residues in allowed regions,  both of them presented a unique residue in disallowed regions of phi-psi torsion angles. The best quality model showed the lowest root-mean-square deviation (RMSD) of 0.325$\AA$ after model-template alignment. After energy minimization and equilibrium steps of the chosen model,  the molecular dynamic simulation was done. In general,  the predicted model presented low conformational deviation over time with the highest fluctuations observed in loops and terminal regions. In conclusion,  homology modeling generated a satisfactory 3D structure of the IRF1 DBD,  which can be very useful for in silico analysis as template to model DBD and protein-DNA interaction studies.
  
  Funding: CAPES,  CNPq \\ 
  \end{abstract}
  \end{document} 