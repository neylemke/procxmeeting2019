
  \documentclass[twoside]{article}
  \usepackage[affil-it]{authblk}
  \usepackage{lipsum} % Package to generate dummy text throughout this template
  \usepackage{eurosym}
  \usepackage[sc]{mathpazo} % Use the Palatino font
  \usepackage[T1]{fontenc} % Use 8-bit encoding that has 256 glyphs
  \usepackage[utf8]{inputenc}
  \linespread{1.05} % Line spacing-Palatino needs more space between lines
  \usepackage{microtype} % Slightly tweak font spacing for aesthetics\[IndentingNewLine]
  \usepackage[hmarginratio=1:1,top=32mm,columnsep=20pt]{geometry} % Document margins
  \usepackage{multicol} % Used for the two-column layout of the document
  \usepackage[hang,small,labelfont=bf,up,textfont=it,up]{caption} % Custom captions under//above floats in tables or figures
  \usepackage{booktabs} % Horizontal rules in tables
  \usepackage{float} % Required for tables and figures in the multi-column environment-they need to be placed in specific locations with the[H] (e.g. \begin{table}[H])
  \usepackage{hyperref} % For hyperlinks in the PDF
  \usepackage{lettrine} % The lettrine is the first enlarged letter at the beginning of the text
  \usepackage{paralist} % Used for the compactitem environment which makes bullet points with less space between them
  \usepackage{abstract} % Allows abstract customization
  \renewcommand{\abstractnamefont}{\normalfont\bfseries} 
  %\renewcommand{\abstracttextfont}{\normalfont\small\itshape} % Set the abstract itself to small italic text\[IndentingNewLine]
  \usepackage{titlesec} % Allows customization of titles
  \renewcommand\thesection{\Roman{section}} % Roman numerals for the sections
  \renewcommand\thesubsection{\Roman{subsection}} % Roman numerals for subsections
  \titleformat{\section}[block]{\large\scshape\centering}{\thesection.}{1em}{} % Change the look of the section titles
  \titleformat{\subsection}[block]{\large}{\thesubsection.}{1em}{} % Change the look of the section titles
  \usepackage{fancyhdr} % Headers and footers
  \pagestyle{fancy} % All pages have headers and footers
  \fancyhead{} % Blank out the default header
  \fancyfoot{} % Blank out the default footer
  \fancyhead[C]{X-meeting $\bullet$ October 2019 $\bullet$ Campos do  Jord\~ao} % Custom header text
  \fancyfoot[RO,LE]{} % Custom footer text
  %----------------------------------------------------------------------------------------
  % TITLE SECTION
  %---------------------------------------------------------------------------------------- 
 
 \title{\vspace{-15mm}\fontsize{24pt}{10pt}\selectfont\textbf{ Text Mining for Biological Data: An Update from the Last Decade }} % Article title
  
  
  \author{ Camilla Reginatto De Pierri, Diogo de Jesus Soares Machado, Bruno Thiago de Lima Nichio, Antonio Camilo da Silva Filho, Fabio de Oliveira Pedrosa, Roberto Tadeu Raittz }
  
  \affil{ Federal University of Paran\'a }
  \vspace{-5mm}
  \date{}
  
  %---------------------------------------------------------------------------------------- 
  
  \begin{document}
  
  
  \maketitle % Insert title
  
  
  \thispagestyle{fancy} % All pages have headers and footers
  %----------------------------------------------------------------------------------------  
  % ABSTRACT
  
  %----------------------------------------------------------------------------------------  
  
  \begin{abstract}
  Scientific literature is the basis of research in any field of study. Analysis of information from scientific texts is an important strategy to define the starting point and evaluate the state of the art in a given field of research,  as well as assisting in the construction of hypotheses and interpretation of results. In the biological area,  with the growing number of scientific texts deposited in public databases,  the task of identifying relevant studies becomes complex and time consuming. To deal with this large amount of information,  Text Mining (TM) approaches efficiently handle knowledge seeking. TM is a process that refers to the extraction of information found in texts. The advantage of TM techniques is the ability to improve bibliographic search,  facilitate analysis and data storage,  making the search process refined and accurate. Currently,  there are series of TM tools that contemplate different methodologies with potential for study in the most varied biological scenarios. However,  we noticed that most studies using TM as a research tool do not relate findings to a set of strategies,  which in our view,  may limit the discovery of knowledge. To assist researchers in choosing the best TM strategy,  we conducted a literature review on the topic TM and the main tools available. The criteria for study selection were: 1) TM tools developed and / or implemented from 2009 to 2019; 2) Only tools available through scientific publication; 3) Only tools that the research was published in PubMed database. We have identified 41 TM tools with the most varied applications. These include MedlineRanker,  DataShield,  FACTA +,  SAPIENTA,  BioC and DISEASES with the highest number of citations,  according to Google Scholar,  Scopus and Web of Science. The methodology of the tools selected in this research involves processes of information retrieval,  machine learning,  natural language processing and computational and statistical language,  focusing mainly on the study and identification of events related to genes and proteins. We found that text mining is not a simple keyword search in databases. Several automated processes and methods are required for the extraction of knowledge from texts. The use of one or more TM approaches is valuable for identifying relevant concepts and uncovering hidden knowledge in light of unexplored subjects.
  
  Funding: CAPES,  Funda\c{c}\~ao Arauc\'aria \\ 
  \end{abstract}
  \end{document} 