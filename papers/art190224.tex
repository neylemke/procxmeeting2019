
  \documentclass[twoside]{article}
  \usepackage[affil-it]{authblk}
  \usepackage{lipsum} % Package to generate dummy text throughout this template
  \usepackage{eurosym}
  \usepackage[sc]{mathpazo} % Use the Palatino font
  \usepackage[T1]{fontenc} % Use 8-bit encoding that has 256 glyphs
  \usepackage[utf8]{inputenc}
  \linespread{1.05} % Line spacing-Palatino needs more space between lines
  \usepackage{microtype} % Slightly tweak font spacing for aesthetics\[IndentingNewLine]
  \usepackage[hmarginratio=1:1,top=32mm,columnsep=20pt]{geometry} % Document margins
  \usepackage{multicol} % Used for the two-column layout of the document
  \usepackage[hang,small,labelfont=bf,up,textfont=it,up]{caption} % Custom captions under//above floats in tables or figures
  \usepackage{booktabs} % Horizontal rules in tables
  \usepackage{float} % Required for tables and figures in the multi-column environment-they need to be placed in specific locations with the[H] (e.g. \begin{table}[H])
  \usepackage{hyperref} % For hyperlinks in the PDF
  \usepackage{lettrine} % The lettrine is the first enlarged letter at the beginning of the text
  \usepackage{paralist} % Used for the compactitem environment which makes bullet points with less space between them
  \usepackage{abstract} % Allows abstract customization
  \renewcommand{\abstractnamefont}{\normalfont\bfseries} 
  %\renewcommand{\abstracttextfont}{\normalfont\small\itshape} % Set the abstract itself to small italic text\[IndentingNewLine]
  \usepackage{titlesec} % Allows customization of titles
  \renewcommand\thesection{\Roman{section}} % Roman numerals for the sections
  \renewcommand\thesubsection{\Roman{subsection}} % Roman numerals for subsections
  \titleformat{\section}[block]{\large\scshape\centering}{\thesection.}{1em}{} % Change the look of the section titles
  \titleformat{\subsection}[block]{\large}{\thesubsection.}{1em}{} % Change the look of the section titles
  \usepackage{fancyhdr} % Headers and footers
  \pagestyle{fancy} % All pages have headers and footers
  \fancyhead{} % Blank out the default header
  \fancyfoot{} % Blank out the default footer
  \fancyhead[C]{X-meeting $\bullet$ October 2019 $\bullet$ Campos do  Jord\~ao} % Custom header text
  \fancyfoot[RO,LE]{} % Custom footer text
  %----------------------------------------------------------------------------------------
  % TITLE SECTION
  %---------------------------------------------------------------------------------------- 
 
 \title{\vspace{-15mm}\fontsize{24pt}{10pt}\selectfont\textbf{ Exploring lncRNAS in Alzheimer's Disease }} % Article title
  
  
  \author{ Beatriz Miranda, Willian Orlando-Castillo, Silvana Giuliatti }
  
  \affil{ University of Cauca }
  \vspace{-5mm}
  \date{}
  
  %---------------------------------------------------------------------------------------- 
  
  \begin{document}
  
  
  \maketitle % Insert title
  
  
  \thispagestyle{fancy} % All pages have headers and footers
  %----------------------------------------------------------------------------------------  
  % ABSTRACT
  
  %----------------------------------------------------------------------------------------  
  
  \begin{abstract}
  Alzheimer's disease (AD) is a progressive and chronic neurodegenerative disorder,  recognized as a multifactorial disease,  which limits pharmacological options by the multiple pathways involved in the pathogenesis. To date there are no effective therapies for AD,  only the symptoms are treated. Recent studies have shown that non-coding RNAs (ncRNAs),  including long-non coding RNAs (lncRNAs),  are involved in the pathogenesis of AD. It is believed that there is a strong relationship between the structure of lncRNAs and the functions they carry out. However,  due to lack of conservation in its primary nucleotide sequence,  functional studies of lNcRNAs with the objective of developing new therapeutic methods imply a challenging process. Since the experimental methods for obtaining tertiary structures are not easy to perform,  in silico approaches are valuable tools that contribute to this process of understanding the disease and structural bioinformatics can contribute to the prediction of these structures. Therefore,  the goal of this project is to predict the tertiary structures of lncRNAs involved in AD. In this first step,  we begin by modeling the structure of lncRNA BACE1-AS,  which is highly expressed in AD patients. The unit described by Faghihi et al. (2008) (NR\_037803.2) was used. Secondary structure prediction was performed using the Mfold software (Zuke,  2000) and the tertiary structure was modeled using the 3dRNA v2.04 software (Wang and Xiao,  2017). The analysis of tertiary structures was performed using MolProbity (Chen et al.,  2010) and Pymol (Delano,  2002) softwares. 29 secondary structures were obtained and the selected model presented a value of -240.61 kcal / mol. BACE1-AS primary and secondary structures were used for the tertiary structure modelling. Five models were successfully generated. After analysis by MolProbity and visual analysis,  a model was selected for further studies such as normal mode analysis.
  
  Funding: CNPq \\ 
  \end{abstract}
  \end{document} 