
  \documentclass[twoside]{article}
  \usepackage[affil-it]{authblk}
  \usepackage{lipsum} % Package to generate dummy text throughout this template
  \usepackage{eurosym}
  \usepackage[sc]{mathpazo} % Use the Palatino font
  \usepackage[T1]{fontenc} % Use 8-bit encoding that has 256 glyphs
  \usepackage[utf8]{inputenc}
  \linespread{1.05} % Line spacing-Palatino needs more space between lines
  \usepackage{microtype} % Slightly tweak font spacing for aesthetics\[IndentingNewLine]
  \usepackage[hmarginratio=1:1,top=32mm,columnsep=20pt]{geometry} % Document margins
  \usepackage{multicol} % Used for the two-column layout of the document
  \usepackage[hang,small,labelfont=bf,up,textfont=it,up]{caption} % Custom captions under//above floats in tables or figures
  \usepackage{booktabs} % Horizontal rules in tables
  \usepackage{float} % Required for tables and figures in the multi-column environment-they need to be placed in specific locations with the[H] (e.g. \begin{table}[H])
  \usepackage{hyperref} % For hyperlinks in the PDF
  \usepackage{lettrine} % The lettrine is the first enlarged letter at the beginning of the text
  \usepackage{paralist} % Used for the compactitem environment which makes bullet points with less space between them
  \usepackage{abstract} % Allows abstract customization
  \renewcommand{\abstractnamefont}{\normalfont\bfseries} 
  %\renewcommand{\abstracttextfont}{\normalfont\small\itshape} % Set the abstract itself to small italic text\[IndentingNewLine]
  \usepackage{titlesec} % Allows customization of titles
  \renewcommand\thesection{\Roman{section}} % Roman numerals for the sections
  \renewcommand\thesubsection{\Roman{subsection}} % Roman numerals for subsections
  \titleformat{\section}[block]{\large\scshape\centering}{\thesection.}{1em}{} % Change the look of the section titles
  \titleformat{\subsection}[block]{\large}{\thesubsection.}{1em}{} % Change the look of the section titles
  \usepackage{fancyhdr} % Headers and footers
  \pagestyle{fancy} % All pages have headers and footers
  \fancyhead{} % Blank out the default header
  \fancyfoot{} % Blank out the default footer
  \fancyhead[C]{X-meeting $\bullet$ October 2019 $\bullet$ Campos do  Jord\~ao} % Custom header text
  \fancyfoot[RO,LE]{} % Custom footer text
  %----------------------------------------------------------------------------------------
  % TITLE SECTION
  %---------------------------------------------------------------------------------------- 
 
 \title{\vspace{-15mm}\fontsize{24pt}{10pt}\selectfont\textbf{ Gender-based differences in gene expression and alternative splicing profiles of glioma patients }} % Article title
  
  
  \author{ Gabriela Der Agopian Guardia, Felipe R. C. dos Santos, Luiz O. Penalva, PEDRO A F GALANTE }
  
  \affil{ Children’s Cancer Research Institute,  UT Health San Antonio,  San Antonio,  Texas,  USA }
  \vspace{-5mm}
  \date{}
  
  %---------------------------------------------------------------------------------------- 
  
  \begin{document}
  
  
  \maketitle % Insert title
  
  
  \thispagestyle{fancy} % All pages have headers and footers
  %----------------------------------------------------------------------------------------  
  % ABSTRACT
  
  %----------------------------------------------------------------------------------------  
  
  \begin{abstract}
  Gliomas represent the most common malignant tumors of the central nervous system,  predominantly arising from astrocytes and glial progenitors. The World Health Organization classifies malignant gliomas into 3 histological grades (II-IV) based on the level of malignancy. Glioma standard treatment comprises maximal safe surgical resection followed by radiotherapy and/or chemotherapy depending on the tumor stage. Despite the severe treatment regimen,  glioma patients present extremely poor survival rates,  e.g.,  5-year survival rate of 19.6\% for glioblastoma (grade IV) patients. Therefore,  the identification of therapeutic targets for the development of more effective treatment approaches is still an active research field. In this context,  previous studies have also shown that not only the prevalence of gliomas is higher in men than women,  but also men usually have poorer responses to standard treatment. However,  the molecular bases of these differences have not yet been fully elucidated from a genomic perspective,  thus hampering the development of gender-specific therapeutic options that could substantially improve patients overall survival. In this work,  we investigate gender-based differences in gene expression and alternative splicing profiles of primary glioma patients (grades II-IV) to reveal potential modulators of cancer risk and outcome. For each tumor grade,  we identify sets of genes which are up-regulated in men or women patients,  and whose expression levels do not substantially differ between genders on healthy (cortex) samples. Similarly,  we also investigate genes with alternative splicing variants prevalent in men or women patients which do not show significant splicing deregulation on healthy cortex. Next,  we show that expression/splicing imbalances are not restricted to protein-coding genes,  but also observed in the class of long non-coding RNAs,  which corroborates their role on several brain disorders. Finally,  we explore the association between the expression levels of all identified genes and glioma patient survival to unveil genes exclusively associated with prognosis of male or female patients. In summary,  our study established gender-based transcriptomic differences of distinct malignant glioma grades at both the expression and alternative splicing levels. Altogether,  these findings contribute to a better understanding of gliomas aggressiveness and may explain differences in therapeutic responses between men and women patients.
  
  Funding: FAPESP \\ 
  \end{abstract}
  \end{document} 