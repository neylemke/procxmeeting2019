
  \documentclass[twoside]{article}
  \usepackage[affil-it]{authblk}
  \usepackage{lipsum} % Package to generate dummy text throughout this template
  \usepackage{eurosym}
  \usepackage[sc]{mathpazo} % Use the Palatino font
  \usepackage[T1]{fontenc} % Use 8-bit encoding that has 256 glyphs
  \usepackage[utf8]{inputenc}
  \linespread{1.05} % Line spacing-Palatino needs more space between lines
  \usepackage{microtype} % Slightly tweak font spacing for aesthetics\[IndentingNewLine]
  \usepackage[hmarginratio=1:1,top=32mm,columnsep=20pt]{geometry} % Document margins
  \usepackage{multicol} % Used for the two-column layout of the document
  \usepackage[hang,small,labelfont=bf,up,textfont=it,up]{caption} % Custom captions under//above floats in tables or figures
  \usepackage{booktabs} % Horizontal rules in tables
  \usepackage{float} % Required for tables and figures in the multi-column environment-they need to be placed in specific locations with the[H] (e.g. \begin{table}[H])
  \usepackage{hyperref} % For hyperlinks in the PDF
  \usepackage{lettrine} % The lettrine is the first enlarged letter at the beginning of the text
  \usepackage{paralist} % Used for the compactitem environment which makes bullet points with less space between them
  \usepackage{abstract} % Allows abstract customization
  \renewcommand{\abstractnamefont}{\normalfont\bfseries} 
  %\renewcommand{\abstracttextfont}{\normalfont\small\itshape} % Set the abstract itself to small italic text\[IndentingNewLine]
  \usepackage{titlesec} % Allows customization of titles
  \renewcommand\thesection{\Roman{section}} % Roman numerals for the sections
  \renewcommand\thesubsection{\Roman{subsection}} % Roman numerals for subsections
  \titleformat{\section}[block]{\large\scshape\centering}{\thesection.}{1em}{} % Change the look of the section titles
  \titleformat{\subsection}[block]{\large}{\thesubsection.}{1em}{} % Change the look of the section titles
  \usepackage{fancyhdr} % Headers and footers
  \pagestyle{fancy} % All pages have headers and footers
  \fancyhead{} % Blank out the default header
  \fancyfoot{} % Blank out the default footer
  \fancyhead[C]{X-meeting $\bullet$ October 2019 $\bullet$ Campos do  Jord\~ao} % Custom header text
  \fancyfoot[RO,LE]{} % Custom footer text
  %----------------------------------------------------------------------------------------
  % TITLE SECTION
  %---------------------------------------------------------------------------------------- 
 
 \title{\vspace{-15mm}\fontsize{24pt}{10pt}\selectfont\textbf{ Global co-expression network analysis unveils important aspects of evolution and transcriptional regulation in soybean (Glycine max) }} % Article title
  
  
  \author{ Fabr\'{\i}cio de Almeida Silva, Fabricio Brum Machado, Kanhu Charan Moharana, Rajesh Kumar Gazara, Thiago Venancio }
  
  \affil{ Universidade Estadual do Norte Fluminense Darcy Ribeiro }
  \vspace{-5mm}
  \date{}
  
  %---------------------------------------------------------------------------------------- 
  
  \begin{document}
  
  
  \maketitle % Insert title
  
  
  \thispagestyle{fancy} % All pages have headers and footers
  %----------------------------------------------------------------------------------------  
  % ABSTRACT
  
  %----------------------------------------------------------------------------------------  
  
  \begin{abstract}
  Soybean (Glycine max (L.) Merr.) is one of the most important crops worldwide,  representing a significant fraction of Brazilian GNP. Gene co-expression networks (GCN) have been largely used to elucidate regulatory complexity and evolution of genes and their functions. Here,  we have reconstructed a GCN using 1298 publicly available samples from 12 distinct tissues. Sequencing reads were mapped against the soybean genome (Wm82.a2.v1) and relative transcript abundance estimated in Transcripts per million mapped reads (TPM). The network was reconstructed and visualized with the R packages WGCNA and igraph,  respectively. The top 10\% most highly connected genes with the highest module membership were considered intramodular hubs. We explored the network properties and found critically important modules that are up-regulated in specific tissues. Enrichment analyses of these modules revealed biological processes and pathways that are essential to some particular tissues and may have elementary contributions to the plant development. We also identified transcription factors (TFs) among intramodular hubs,  which may be important regulators,  shaping the transcriptional landscape in particular tissues. The top hubs for each module were identified and we found that they tend to encode proteins with critical roles,  such as succinate dehydrogenase,  peroxidases,  and RNA polymerase subunits. Further,  we analyzed the distribution of soybean paralogous genes across the network modules to better comprehend the fate of duplicate genes in polyploid organisms. Most of the duplicate gene pairs were present in different modules,  supporting their subfunctionalization in different tissues and providing insights on the evolutionary importance of polyploidization in soybean genome complexity and evolution.
  
  Funding: CAPES,  CNPq,  FAPERJ \\ 
  \end{abstract}
  \end{document} 