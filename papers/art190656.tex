
  \documentclass[twoside]{article}
  \usepackage[affil-it]{authblk}
  \usepackage{lipsum} % Package to generate dummy text throughout this template
  \usepackage{eurosym}
  \usepackage[sc]{mathpazo} % Use the Palatino font
  \usepackage[T1]{fontenc} % Use 8-bit encoding that has 256 glyphs
  \usepackage[utf8]{inputenc}
  \linespread{1.05} % Line spacing-Palatino needs more space between lines
  \usepackage{microtype} % Slightly tweak font spacing for aesthetics\[IndentingNewLine]
  \usepackage[hmarginratio=1:1,top=32mm,columnsep=20pt]{geometry} % Document margins
  \usepackage{multicol} % Used for the two-column layout of the document
  \usepackage[hang,small,labelfont=bf,up,textfont=it,up]{caption} % Custom captions under//above floats in tables or figures
  \usepackage{booktabs} % Horizontal rules in tables
  \usepackage{float} % Required for tables and figures in the multi-column environment-they need to be placed in specific locations with the[H] (e.g. \begin{table}[H])
  \usepackage{hyperref} % For hyperlinks in the PDF
  \usepackage{lettrine} % The lettrine is the first enlarged letter at the beginning of the text
  \usepackage{paralist} % Used for the compactitem environment which makes bullet points with less space between them
  \usepackage{abstract} % Allows abstract customization
  \renewcommand{\abstractnamefont}{\normalfont\bfseries} 
  %\renewcommand{\abstracttextfont}{\normalfont\small\itshape} % Set the abstract itself to small italic text\[IndentingNewLine]
  \usepackage{titlesec} % Allows customization of titles
  \renewcommand\thesection{\Roman{section}} % Roman numerals for the sections
  \renewcommand\thesubsection{\Roman{subsection}} % Roman numerals for subsections
  \titleformat{\section}[block]{\large\scshape\centering}{\thesection.}{1em}{} % Change the look of the section titles
  \titleformat{\subsection}[block]{\large}{\thesubsection.}{1em}{} % Change the look of the section titles
  \usepackage{fancyhdr} % Headers and footers
  \pagestyle{fancy} % All pages have headers and footers
  \fancyhead{} % Blank out the default header
  \fancyfoot{} % Blank out the default footer
  \fancyhead[C]{X-meeting $\bullet$ October 2019 $\bullet$ Campos do  Jord\~ao} % Custom header text
  \fancyfoot[RO,LE]{} % Custom footer text
  %----------------------------------------------------------------------------------------
  % TITLE SECTION
  %---------------------------------------------------------------------------------------- 
 
 \title{\vspace{-15mm}\fontsize{24pt}{10pt}\selectfont\textbf{ Using macrophage genes expression to build and validate a molecular model of host-parasite interaction }} % Article title
  
  
  \author{ Felipe Caixeta Moreira, Ana Maria Caetano Faria, Tatiani Uceli Maioli, Leandro Martins de Freitas, Paolo Tieri, Filippo Castiglione }
  
  \affil{ Universidade Federal de Minas Gerais }
  \vspace{-5mm}
  \date{}
  
  %---------------------------------------------------------------------------------------- 
  
  \begin{document}
  
  
  \maketitle % Insert title
  
  
  \thispagestyle{fancy} % All pages have headers and footers
  %----------------------------------------------------------------------------------------  
  % ABSTRACT
  
  %----------------------------------------------------------------------------------------  
  
  \begin{abstract}
  Introduction: Macrophages are mononuclear phagocytes that constitute the first line of defense against pathogens. They are also the immune effector cells that,  upon activation,  are able to kill intracellular organisms and are the primary host cells of Leishmania spp. parasites,  the obligate intracellular pathogens that cause leishmaniasis. This group of disease has a spectrum of clinical manifestations ranging from self-healing cutaneous ulcers to severe visceral alterations. In mammals,  macrophages are the main host for Leishmania amastigote. In regard of this,  the aim of this work was to evaluate the genes that are most expressed in macrophages infected with Leishmania.
Methods and Results: In the developing domain of big data,  the role of a data miner is pivotal in the prominent increase in the number of data published The techniques are used in combination with functional transcriptomic,  measurement of expression profiles and functional interactions from cells and molecules of many different organisms. The responses of host cells to pathogenic microorganisms are among the most-well studied examples of cellular responses to external stimuli. Pathogen-induced phenotypic changes in host cells are often accompanied by marked changes in transcriptomic expression. In the present study,  we collected mining results from results from different databases of cells infected with Leishmania ssp. and we compared those results to differences in THP1 macrophages-gene expression at 12 and 24 hours after Leishmania major infection. We employed a data mining approach,  such as the R and Cytoscape software,  to filter and select the most prominent genes,  we analyses the gene expression profile of 30 THP-1 macrophage diamond genes,  12 and 24 hours after L. major infection. The data were validated with qPCR and we have shown no different expressed genes in THP1 macrophages as we found in the mining data. 
Conclusion: These analyses provide insights into the interplay between human macrophages and Leishmania parasites,  and constitute an important general resource for the study of which genes are the most regulated during the host-parasite interaction. Therefore,  there is no difference in gene expression 12hpi and 24hpi,  which can be an interesting result that shows the influence of Leishmania major during infection.

Financial Support: FAPEMIG,  CNPq,  CAPES.
  
  Funding: Fapemig,  Capes,  CNPq,  CNR \\ 
  \end{abstract}
  \end{document} 