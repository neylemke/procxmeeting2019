
  \documentclass[twoside]{article}
  \usepackage[affil-it]{authblk}
  \usepackage{lipsum} % Package to generate dummy text throughout this template
  \usepackage{eurosym}
  \usepackage[sc]{mathpazo} % Use the Palatino font
  \usepackage[T1]{fontenc} % Use 8-bit encoding that has 256 glyphs
  \usepackage[utf8]{inputenc}
  \linespread{1.05} % Line spacing-Palatino needs more space between lines
  \usepackage{microtype} % Slightly tweak font spacing for aesthetics\[IndentingNewLine]
  \usepackage[hmarginratio=1:1,top=32mm,columnsep=20pt]{geometry} % Document margins
  \usepackage{multicol} % Used for the two-column layout of the document
  \usepackage[hang,small,labelfont=bf,up,textfont=it,up]{caption} % Custom captions under//above floats in tables or figures
  \usepackage{booktabs} % Horizontal rules in tables
  \usepackage{float} % Required for tables and figures in the multi-column environment-they need to be placed in specific locations with the[H] (e.g. \begin{table}[H])
  \usepackage{hyperref} % For hyperlinks in the PDF
  \usepackage{lettrine} % The lettrine is the first enlarged letter at the beginning of the text
  \usepackage{paralist} % Used for the compactitem environment which makes bullet points with less space between them
  \usepackage{abstract} % Allows abstract customization
  \renewcommand{\abstractnamefont}{\normalfont\bfseries} 
  %\renewcommand{\abstracttextfont}{\normalfont\small\itshape} % Set the abstract itself to small italic text\[IndentingNewLine]
  \usepackage{titlesec} % Allows customization of titles
  \renewcommand\thesection{\Roman{section}} % Roman numerals for the sections
  \renewcommand\thesubsection{\Roman{subsection}} % Roman numerals for subsections
  \titleformat{\section}[block]{\large\scshape\centering}{\thesection.}{1em}{} % Change the look of the section titles
  \titleformat{\subsection}[block]{\large}{\thesubsection.}{1em}{} % Change the look of the section titles
  \usepackage{fancyhdr} % Headers and footers
  \pagestyle{fancy} % All pages have headers and footers
  \fancyhead{} % Blank out the default header
  \fancyfoot{} % Blank out the default footer
  \fancyhead[C]{X-meeting $\bullet$ October 2019 $\bullet$ Campos do  Jord\~ao} % Custom header text
  \fancyfoot[RO,LE]{} % Custom footer text
  %----------------------------------------------------------------------------------------
  % TITLE SECTION
  %---------------------------------------------------------------------------------------- 
 
 \title{\vspace{-15mm}\fontsize{24pt}{10pt}\selectfont\textbf{ Targeting audience and tailoring courses using the ISCB Competency Framework: An application survey from RSG-Brazil Educational Committee }} % Article title
  
  
  \author{ Maira Rodrigues de Camargo Neves, Raquel Riyuzo, Nilson Coimbra }
  
  \affil{ Universidade Federal de Minas Gerais }
  \vspace{-5mm}
  \date{}
  
  %---------------------------------------------------------------------------------------- 
  
  \begin{document}
  
  
  \maketitle % Insert title
  
  
  \thispagestyle{fancy} % All pages have headers and footers
  %----------------------------------------------------------------------------------------  
  % ABSTRACT
  
  %----------------------------------------------------------------------------------------  
  
  \begin{abstract}
  Reactivated in 2015,  the ISCB Regional Student Group Brazil (RSG-Brazil) is a vibrant student
network with a mission to aid the training of the next-generation of bioinformaticians and
computational biologists among Brazil. This student group has the support of the Student
Council from the International Society of Computational Biology (ISCB-SC),  which is
dedicated to advancing the scientific understanding of living systems through computation. In
view of its mission,  the Educational Committee of RSG-Brazil released an ongoing survey, 
released in April of 2018,  to identify complementary needs on the ongoing formation in the
field of bioinformatics and computational biology. In this work,  we re-devised the ISCB
competency framework,  an international effort that identified 16 core competencies required
by professional working in the fields related to computational biology,  to only 5 categories in
order to increase the community engagement on the form,  as well as reduce the time demands
for respondents. Respondents were asked to declare interest in learning more about one or more
of the 9 pre-defined topics of Bioinformatics shown in the form,  or to make new suggestions.
Also,  target audience was asked to identify themselves within one of the three profiles of
Bioinformatics audience and their roles or subcategories,  as defined by the ISCB Competency
Framework (Bioinformatics user,  Bioinformatics scientist and Bioinformatics developer,  and
their subcategories). The survey was released in May of 2018 and has so far collected 216
responses,  from 21 (out of 26) Brazil states,  with over 85\% engagement from undergrad or
graduate students. We have found that most of the respondents declared as Bioinformatics
users,  working in academia. We also have identified particularities in the interest in topics by
country region and by Bioinformatics profile. With this effort of the Educational Committee, 
we expect to establish partnership between the Regional Student Group Brazil and
universities/companies to delivery tailored courses to the student community of computational
biology in Brazil.
  
  Funding:  \\ 
  \end{abstract}
  \end{document} 