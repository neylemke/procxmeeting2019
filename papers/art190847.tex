
  \documentclass[twoside]{article}
  \usepackage[affil-it]{authblk}
  \usepackage{lipsum} % Package to generate dummy text throughout this template
  \usepackage{eurosym}
  \usepackage[sc]{mathpazo} % Use the Palatino font
  \usepackage[T1]{fontenc} % Use 8-bit encoding that has 256 glyphs
  \usepackage[utf8]{inputenc}
  \linespread{1.05} % Line spacing-Palatino needs more space between lines
  \usepackage{microtype} % Slightly tweak font spacing for aesthetics\[IndentingNewLine]
  \usepackage[hmarginratio=1:1,top=32mm,columnsep=20pt]{geometry} % Document margins
  \usepackage{multicol} % Used for the two-column layout of the document
  \usepackage[hang,small,labelfont=bf,up,textfont=it,up]{caption} % Custom captions under//above floats in tables or figures
  \usepackage{booktabs} % Horizontal rules in tables
  \usepackage{float} % Required for tables and figures in the multi-column environment-they need to be placed in specific locations with the[H] (e.g. \begin{table}[H])
  \usepackage{hyperref} % For hyperlinks in the PDF
  \usepackage{lettrine} % The lettrine is the first enlarged letter at the beginning of the text
  \usepackage{paralist} % Used for the compactitem environment which makes bullet points with less space between them
  \usepackage{abstract} % Allows abstract customization
  \renewcommand{\abstractnamefont}{\normalfont\bfseries} 
  %\renewcommand{\abstracttextfont}{\normalfont\small\itshape} % Set the abstract itself to small italic text\[IndentingNewLine]
  \usepackage{titlesec} % Allows customization of titles
  \renewcommand\thesection{\Roman{section}} % Roman numerals for the sections
  \renewcommand\thesubsection{\Roman{subsection}} % Roman numerals for subsections
  \titleformat{\section}[block]{\large\scshape\centering}{\thesection.}{1em}{} % Change the look of the section titles
  \titleformat{\subsection}[block]{\large}{\thesubsection.}{1em}{} % Change the look of the section titles
  \usepackage{fancyhdr} % Headers and footers
  \pagestyle{fancy} % All pages have headers and footers
  \fancyhead{} % Blank out the default header
  \fancyfoot{} % Blank out the default footer
  \fancyhead[C]{X-meeting $\bullet$ October 2019 $\bullet$ Campos do  Jord\~ao} % Custom header text
  \fancyfoot[RO,LE]{} % Custom footer text
  %----------------------------------------------------------------------------------------
  % TITLE SECTION
  %---------------------------------------------------------------------------------------- 
 
 \title{\vspace{-15mm}\fontsize{24pt}{10pt}\selectfont\textbf{ Variants encompassing the Agouti signaling protein gene are associated with dilution of grey shades in Nellore (Bos indicus) cattle }} % Article title
  
  
  \author{ Beatriz Batista Trigo, Marco Milanesi, Adam Taiti Harth Utsunomiya, Jos\'e Fernando Garcia, Yuri T. Utsunomiya }
  
  \affil{ 1 Department of Support,  Production and Animal Health,  School of Veterinary Medicine,  S\~ao Paulo State University (Unesp),  Ara\c{c}atuba/SP,  Brazil; 2 International Atomic Energy Agency (IAEA) Collaborating Centre on Animal Genomics and Bioinformatics,  Ara\c{c}atuba/SP,  Brazil; 3 Department of Preventive Veterinary Medicine and Animal Reproduction,  School of Agricultural and Veterinarian Sciences,  S\~ao Paulo State University (Unesp),  Jaboticabal/SP,  Brazil }
  \vspace{-5mm}
  \date{}
  
  %---------------------------------------------------------------------------------------- 
  
  \begin{document}
  
  
  \maketitle % Insert title
  
  
  \thispagestyle{fancy} % All pages have headers and footers
  %----------------------------------------------------------------------------------------  
  % ABSTRACT
  
  %----------------------------------------------------------------------------------------  
  
  \begin{abstract}
  Nellore cattle (Bos indicus) are well known for their resilience to infectious diseases,  survival in low input systems and tolerance to heat. Traits that are often speculated to contribute to heat tolerance are the skin pigmentation and the coat color. Nellore cattle posses dark-black skin,  and although their coat color pattern can range widely from solid red to black-spotted,  the Brazilian herds were massively selected for white coat. For this type of coat,  whereas females are completely white,  males can exhibit shades of dark gray in the head,  neck,  hump and knees. Curiously,  some males are also completely white,  and they can transmit this phenotype to their progeny in a dominant mode of inheritance. The present study focused on mapping the functional candidate gene responsible for the dominant white phenotype in Nellore cattle. Bulls were evaluated based on visual scores for gray shading in a subjective scale from 1 to 5 by six evaluators,  where the smallest score (1) was attributed to all white animals and the largest (5) attributed to animals with black shades. The final score for each animal was decided based on majority voting. A total of 379 animals were evaluated,  from which 131 were chosen and included in this study based on extreme scores. These animals were genotyped with the Illumina BovineHD Bead Chip assay,  which included ~777, 000 single nucleotide polymorphism (SNP) markers. After a standard genotype quality control,  a genome-wide association analysis (GWAS) was conducted in order to map loci linked to the white phenotype. Considering a Bonferroni-corrected significance level of alpha = 1.15e-07,  we detected highly significant associations (p = 7.636928e-127) mapping to chromosome 13,  in the vicinity of the Agouti signaling protein gene (ASIP). Given that ASIP is a well-known color dilution gene,  and that the expression pattern of ASIP in the coats of model organisms follows closely the distribution of grey shades found in Nellore cattle,  the gene is a promising functional candidate. We are now sequencing the whole genomes of 8 Nellore bulls (4 black and 4 white) for scrutiny of the identified locus in the search for the causal mutation.
  
  Funding: Beatriz Batista Trigo  is supported by CAPES (Coordination for the Improvement of Higher Education Personnel) scholarship and Marco Milanesi was supported by grant 2016/05787-7,  S\~ao Paulo Research Foundation (FAPESP) \\ 
  \end{abstract}
  \end{document} 