
  \documentclass[twoside]{article}
  \usepackage[affil-it]{authblk}
  \usepackage{lipsum} % Package to generate dummy text throughout this template
  \usepackage{eurosym}
  \usepackage[sc]{mathpazo} % Use the Palatino font
  \usepackage[T1]{fontenc} % Use 8-bit encoding that has 256 glyphs
  \usepackage[utf8]{inputenc}
  \linespread{1.05} % Line spacing-Palatino needs more space between lines
  \usepackage{microtype} % Slightly tweak font spacing for aesthetics\[IndentingNewLine]
  \usepackage[hmarginratio=1:1,top=32mm,columnsep=20pt]{geometry} % Document margins
  \usepackage{multicol} % Used for the two-column layout of the document
  \usepackage[hang,small,labelfont=bf,up,textfont=it,up]{caption} % Custom captions under//above floats in tables or figures
  \usepackage{booktabs} % Horizontal rules in tables
  \usepackage{float} % Required for tables and figures in the multi-column environment-they need to be placed in specific locations with the[H] (e.g. \begin{table}[H])
  \usepackage{hyperref} % For hyperlinks in the PDF
  \usepackage{lettrine} % The lettrine is the first enlarged letter at the beginning of the text
  \usepackage{paralist} % Used for the compactitem environment which makes bullet points with less space between them
  \usepackage{abstract} % Allows abstract customization
  \renewcommand{\abstractnamefont}{\normalfont\bfseries} 
  %\renewcommand{\abstracttextfont}{\normalfont\small\itshape} % Set the abstract itself to small italic text\[IndentingNewLine]
  \usepackage{titlesec} % Allows customization of titles
  \renewcommand\thesection{\Roman{section}} % Roman numerals for the sections
  \renewcommand\thesubsection{\Roman{subsection}} % Roman numerals for subsections
  \titleformat{\section}[block]{\large\scshape\centering}{\thesection.}{1em}{} % Change the look of the section titles
  \titleformat{\subsection}[block]{\large}{\thesubsection.}{1em}{} % Change the look of the section titles
  \usepackage{fancyhdr} % Headers and footers
  \pagestyle{fancy} % All pages have headers and footers
  \fancyhead{} % Blank out the default header
  \fancyfoot{} % Blank out the default footer
  \fancyhead[C]{X-meeting $\bullet$ October 2019 $\bullet$ Campos do  Jord\~ao} % Custom header text
  \fancyfoot[RO,LE]{} % Custom footer text
  %----------------------------------------------------------------------------------------
  % TITLE SECTION
  %---------------------------------------------------------------------------------------- 
 
 \title{\vspace{-15mm}\fontsize{24pt}{10pt}\selectfont\textbf{ Molecular modeling and pharmacophore based virtual screening of Sterol 24-C-  methyltransferase from Leishmania  brasiliensis }} % Article title
  
  
  \author{ Fabr\'{\i}cio Santos Barbosa, Tarcisio Silva Melo, Bruno Silva Andrade }
  
  \affil{ Universidade Estadual do Sudoeste da Bahia,  Brazil }
  \vspace{-5mm}
  \date{}
  
  %---------------------------------------------------------------------------------------- 
  
  \begin{document}
  
  
  \maketitle % Insert title
  
  
  \thispagestyle{fancy} % All pages have headers and footers
  %----------------------------------------------------------------------------------------  
  % ABSTRACT
  
  %----------------------------------------------------------------------------------------  
  
  \begin{abstract}
  Acording Brazilian Ministry of Health,  Leishmaniasis is described as one of the most important neglected diseases of Brazil,  as well as in other 12 Latin American countries. Leishmania braziliensis Vianna is responsible for causing the tegumentary form of leishmaniasis which generates cutaneous injuries by immune cells destruction during its binnary division. The enzyme 24-sterol C-methyltransferase (EC: 2.1.1.41) belongs to the transferase family and is responsible for catalyzing the transfer of methyl group in reactions for ergosterol synthesis in order to maintain membrane fluidity and permeability. The aim of this work was searching for potential inhibitors for 24-sterol C-methyltransferase addressing to block ergosterol production,  as well as with minimum toxicological effects for the hosts. In a first step we used a molecular homology modeling approach using MODELLER software,  for obtaining an 3D model of this enzyme. Using the AMBER14 package,  we subjected the 3D model to 20.000 cycles of energy minimization followed by 10 nanosseconds of molecular dynamics. Additionally,  we performed a pharmacophore based virtual screening using as start points known drugs with leishmanicidal activity – in this step we used PharmaGist (http://bioinfo3d.cs.tau.ac.il/PharmaGist/php.php) for generating the sdf output molecular aligment. Furthermore,  we subjected the molecular alignment to ZincPharmer (http://zincpharmer.csb.pitt.edu/pharmer.html) for searching pharmacophore-drug-like ligands,  which returned 3025 molecules. All molecules selected in pharmacophore studies were used for molecular docking calculations by AutoDock Vina software. Considering punctuation criterion as well as stereochemical and binding characteristics we selected the 30 best ligands with affinity energies below -12.0 Kcal/Mol. Molecular interactions and 2D interaction maps were generated with PyMOL 2.1.1 and Discovery Studio,  respectively. In a further step we will perform molecular dynamics of 50 nanosseconds for  the 10 best complexes.
  
  Funding:  \\ 
  \end{abstract}
  \end{document} 