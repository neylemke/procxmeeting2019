
  \documentclass[twoside]{article}
  \usepackage[affil-it]{authblk}
  \usepackage{lipsum} % Package to generate dummy text throughout this template
  \usepackage{eurosym}
  \usepackage[sc]{mathpazo} % Use the Palatino font
  \usepackage[T1]{fontenc} % Use 8-bit encoding that has 256 glyphs
  \usepackage[utf8]{inputenc}
  \linespread{1.05} % Line spacing-Palatino needs more space between lines
  \usepackage{microtype} % Slightly tweak font spacing for aesthetics\[IndentingNewLine]
  \usepackage[hmarginratio=1:1,top=32mm,columnsep=20pt]{geometry} % Document margins
  \usepackage{multicol} % Used for the two-column layout of the document
  \usepackage[hang,small,labelfont=bf,up,textfont=it,up]{caption} % Custom captions under//above floats in tables or figures
  \usepackage{booktabs} % Horizontal rules in tables
  \usepackage{float} % Required for tables and figures in the multi-column environment-they need to be placed in specific locations with the[H] (e.g. \begin{table}[H])
  \usepackage{hyperref} % For hyperlinks in the PDF
  \usepackage{lettrine} % The lettrine is the first enlarged letter at the beginning of the text
  \usepackage{paralist} % Used for the compactitem environment which makes bullet points with less space between them
  \usepackage{abstract} % Allows abstract customization
  \renewcommand{\abstractnamefont}{\normalfont\bfseries} 
  %\renewcommand{\abstracttextfont}{\normalfont\small\itshape} % Set the abstract itself to small italic text\[IndentingNewLine]
  \usepackage{titlesec} % Allows customization of titles
  \renewcommand\thesection{\Roman{section}} % Roman numerals for the sections
  \renewcommand\thesubsection{\Roman{subsection}} % Roman numerals for subsections
  \titleformat{\section}[block]{\large\scshape\centering}{\thesection.}{1em}{} % Change the look of the section titles
  \titleformat{\subsection}[block]{\large}{\thesubsection.}{1em}{} % Change the look of the section titles
  \usepackage{fancyhdr} % Headers and footers
  \pagestyle{fancy} % All pages have headers and footers
  \fancyhead{} % Blank out the default header
  \fancyfoot{} % Blank out the default footer
  \fancyhead[C]{X-meeting $\bullet$ October 2019 $\bullet$ Campos do  Jord\~ao} % Custom header text
  \fancyfoot[RO,LE]{} % Custom footer text
  %----------------------------------------------------------------------------------------
  % TITLE SECTION
  %---------------------------------------------------------------------------------------- 
 
 \title{\vspace{-15mm}\fontsize{24pt}{10pt}\selectfont\textbf{ The relationships between variability,  architecture and mutation co-occurrence in the HIV-1 integrase: implications of Raltegravir treatment. }} % Article title
  
  
  \author{ Lucas de Almeida Machado, Ana Carolina Ramos Guimar\~aes }
  
  \affil{ Fiocruz }
  \vspace{-5mm}
  \date{}
  
  %---------------------------------------------------------------------------------------- 
  
  \begin{document}
  
  
  \maketitle % Insert title
  
  
  \thispagestyle{fancy} % All pages have headers and footers
  %----------------------------------------------------------------------------------------  
  % ABSTRACT
  
  %----------------------------------------------------------------------------------------  
  
  \begin{abstract}
  One of the primary drug targets in the therapy against immunodeficiency virus type 1 (HIV-1) is the integrase - the enzyme responsible for the integration of the viral DNA into the host genome. The integrase inhibitor Raltegravir has been widely used in antiretroviral therapy; however,  Raltegravir-resistant HIV-1 strains have become a worldwide problem. Here,  we compared the variability of each position of the HIV-1 integrase sequence in clinical isolates of Raltegravir-treated and drug-na\"{\i}ve patients by calculating their Shannon entropies. We also built tridimensional models of the HIV-1 integrase and a mutation co-occurrence network. The relationship between variability,  architecture,  and co-occurrence was investigated. To investigate the  

It was observed that positions bearing major resistance-related mutations are highly conserved among non-treated patients and variable among the treated ones. The integrase structure showed that the highest-entropy residues are in the vicinity of the host DNA,  and their variations may impact the protein-DNA interface. The co-occurrence network and structural analysis support the hypothesis that the resistance-related E138K mutation compensates for mutated DNA-anchoring lysine residues.  

The study results reveal patterns by which the integrase adapts during the Raltegravir therapy; this information can be useful to rethink the drugs currently used or to guide the development of new ones.
  
  Funding: CAPES \\ 
  \end{abstract}
  \end{document} 