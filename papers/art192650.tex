
  \documentclass[twoside]{article}
  \usepackage[affil-it]{authblk}
  \usepackage{lipsum} % Package to generate dummy text throughout this template
  \usepackage{eurosym}
  \usepackage[sc]{mathpazo} % Use the Palatino font
  \usepackage[T1]{fontenc} % Use 8-bit encoding that has 256 glyphs
  \usepackage[utf8]{inputenc}
  \linespread{1.05} % Line spacing-Palatino needs more space between lines
  \usepackage{microtype} % Slightly tweak font spacing for aesthetics\[IndentingNewLine]
  \usepackage[hmarginratio=1:1,top=32mm,columnsep=20pt]{geometry} % Document margins
  \usepackage{multicol} % Used for the two-column layout of the document
  \usepackage[hang,small,labelfont=bf,up,textfont=it,up]{caption} % Custom captions under//above floats in tables or figures
  \usepackage{booktabs} % Horizontal rules in tables
  \usepackage{float} % Required for tables and figures in the multi-column environment-they need to be placed in specific locations with the[H] (e.g. \begin{table}[H])
  \usepackage{hyperref} % For hyperlinks in the PDF
  \usepackage{lettrine} % The lettrine is the first enlarged letter at the beginning of the text
  \usepackage{paralist} % Used for the compactitem environment which makes bullet points with less space between them
  \usepackage{abstract} % Allows abstract customization
  \renewcommand{\abstractnamefont}{\normalfont\bfseries} 
  %\renewcommand{\abstracttextfont}{\normalfont\small\itshape} % Set the abstract itself to small italic text\[IndentingNewLine]
  \usepackage{titlesec} % Allows customization of titles
  \renewcommand\thesection{\Roman{section}} % Roman numerals for the sections
  \renewcommand\thesubsection{\Roman{subsection}} % Roman numerals for subsections
  \titleformat{\section}[block]{\large\scshape\centering}{\thesection.}{1em}{} % Change the look of the section titles
  \titleformat{\subsection}[block]{\large}{\thesubsection.}{1em}{} % Change the look of the section titles
  \usepackage{fancyhdr} % Headers and footers
  \pagestyle{fancy} % All pages have headers and footers
  \fancyhead{} % Blank out the default header
  \fancyfoot{} % Blank out the default footer
  \fancyhead[C]{X-meeting $\bullet$ October 2019 $\bullet$ Campos do  Jord\~ao} % Custom header text
  \fancyfoot[RO,LE]{} % Custom footer text
  %----------------------------------------------------------------------------------------
  % TITLE SECTION
  %---------------------------------------------------------------------------------------- 
 
 \title{\vspace{-15mm}\fontsize{24pt}{10pt}\selectfont\textbf{ RNA-Seq of endogenous human stem cells and tumors to identify cancer-specific therapeutic targets }} % Article title
  
  
  \author{ Isabela Pimentel de Almeida, Main\'a Bitar, Elizabeth O'Brien, Grace Borchert, Charlotte Woods, Guy BArry }
  
  \affil{ Universidade de S\~ao Paulo }
  \vspace{-5mm}
  \date{}
  
  %---------------------------------------------------------------------------------------- 
  
  \begin{document}
  
  
  \maketitle % Insert title
  
  
  \thispagestyle{fancy} % All pages have headers and footers
  %----------------------------------------------------------------------------------------  
  % ABSTRACT
  
  %----------------------------------------------------------------------------------------  
  
  \begin{abstract}
  Stem cells are characterized by their capacity for self-renewal,  long-term viability and ability to differentiate into multiple types of specialized cells. Similarly,  cancer cells are also capable of self-renewal,  which allows aggressive and unlimited tumor growth. Interestingly,  pathways that are normally associated with stem cell development overlap significantly with cancer progression. Therefore,  endogenous stem cell populations residing outside the tumor are significantly affected by cancer treatments as they target common proliferative signaling pathways. Here we investigate for the first time the similarities and differences between various types of endogenous adult human stem cells and publicly available patient-derived glioblastoma and medulloblastoma primary tumors based on transcript expression via RNA-Seq experiments. Additionally,  we profiled the known gene targets of all currently FDA approved drugs for cancer treatment in our stem cells data. The study included Kallisto and STAR/Rsem methods for transcript quantification and also different methods for comparison of expression profiles. Comparing the transcriptomes of normal human stem cells and cancer cells represents an alternative approach to identify better drug targets,  with potentially less severe side effects. As proof of principle,  we used our data to uncover clinically relevant antisense oligonucleotides (ASOs) targeted to candidate transcripts that were highly expressed in glioblastoma but negligibly expressed in stem cells. We observed a marked decrease in proliferation of primary glioblastoma cell-lines treated with these ASOs. This strategy may be further applied to virtually all cancer types and improve cancer treatment by both assessing existing FDA approved drugs and proposing new targets.  Therefore,  our findings may support the development of alternative therapies that specifically target the malignant cells within a tumor.
  
  Funding:  \\ 
  \end{abstract}
  \end{document} 