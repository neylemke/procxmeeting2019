
  \documentclass[twoside]{article}
  \usepackage[affil-it]{authblk}
  \usepackage{lipsum} % Package to generate dummy text throughout this template
  \usepackage{eurosym}
  \usepackage[sc]{mathpazo} % Use the Palatino font
  \usepackage[T1]{fontenc} % Use 8-bit encoding that has 256 glyphs
  \usepackage[utf8]{inputenc}
  \linespread{1.05} % Line spacing-Palatino needs more space between lines
  \usepackage{microtype} % Slightly tweak font spacing for aesthetics\[IndentingNewLine]
  \usepackage[hmarginratio=1:1,top=32mm,columnsep=20pt]{geometry} % Document margins
  \usepackage{multicol} % Used for the two-column layout of the document
  \usepackage[hang,small,labelfont=bf,up,textfont=it,up]{caption} % Custom captions under//above floats in tables or figures
  \usepackage{booktabs} % Horizontal rules in tables
  \usepackage{float} % Required for tables and figures in the multi-column environment-they need to be placed in specific locations with the[H] (e.g. \begin{table}[H])
  \usepackage{hyperref} % For hyperlinks in the PDF
  \usepackage{lettrine} % The lettrine is the first enlarged letter at the beginning of the text
  \usepackage{paralist} % Used for the compactitem environment which makes bullet points with less space between them
  \usepackage{abstract} % Allows abstract customization
  \renewcommand{\abstractnamefont}{\normalfont\bfseries} 
  %\renewcommand{\abstracttextfont}{\normalfont\small\itshape} % Set the abstract itself to small italic text\[IndentingNewLine]
  \usepackage{titlesec} % Allows customization of titles
  \renewcommand\thesection{\Roman{section}} % Roman numerals for the sections
  \renewcommand\thesubsection{\Roman{subsection}} % Roman numerals for subsections
  \titleformat{\section}[block]{\large\scshape\centering}{\thesection.}{1em}{} % Change the look of the section titles
  \titleformat{\subsection}[block]{\large}{\thesubsection.}{1em}{} % Change the look of the section titles
  \usepackage{fancyhdr} % Headers and footers
  \pagestyle{fancy} % All pages have headers and footers
  \fancyhead{} % Blank out the default header
  \fancyfoot{} % Blank out the default footer
  \fancyhead[C]{X-meeting $\bullet$ October 2019 $\bullet$ Campos do  Jord\~ao} % Custom header text
  \fancyfoot[RO,LE]{} % Custom footer text
  %----------------------------------------------------------------------------------------
  % TITLE SECTION
  %---------------------------------------------------------------------------------------- 
 
 \title{\vspace{-15mm}\fontsize{24pt}{10pt}\selectfont\textbf{ Assessment of intratumoral genetic heterogeneity scores (ITGH) and its association with clinical parameters across several cancer types }} % Article title
  
  
  \author{ Filipe Ferreira dos Santos, Cibele Masotti, Isac de Castro, Anamaria A. Camargo, PEDRO A F GALANTE }
  
  \affil{ Molecular Oncology Center,  Hospital S\'{\i}rio-Liban\^es,  S\~ao Paulo,  Brazil }
  \vspace{-5mm}
  \date{}
  
  %---------------------------------------------------------------------------------------- 
  
  \begin{document}
  
  
  \maketitle % Insert title
  
  
  \thispagestyle{fancy} % All pages have headers and footers
  %----------------------------------------------------------------------------------------  
  % ABSTRACT
  
  %----------------------------------------------------------------------------------------  
  
  \begin{abstract}
  In the age of precision medicine,  the use of molecular data has become increasingly common in clinical oncology routine,  especially with the advent of Cancer Gene Panels (CGPs). Given the relevant influence of intratumoral genetic heterogeneity (ITGH) on the prognosis and treatment of patients,  its better understanding is vital. Recent advances in sequencing technologies and computational algorithms allowed the development of tools that estimate ITGH based on mutant allele frequencies (MAFs). However,  several practical and methodological limitations make it difficult to get into the routine of clinical oncology. MATH score is a method capable of estimating ITGH from single biopsies taking into account parameters that are known to disrupt MAF estimates such as sample purity. This study aimed to measure the ITGH of all 33 cancer types from exome (WXS) data of the TCGA project and the MSK-IMPACT 410 cancer gene panel with MATH score in order to evaluate its association with several clinical parameters. Univariate survival analysis was done with stratified 5-Fold cross-validation strategy and ROC curves. For multivariate analysis,  a modified cox regression model was used,  joining the Monte-Carlo cross-validation strategy and the Bootstrap resampling method. ITGH varied significantly among patients and cancer types,  where generally more aggressive tumors have higher levels of ITGH such as OV. In addition,  MATH is a good prognostic marker of OS for UCEC,  UCS and LGG and PFI for UCEC and LGG with WXS data. Similarly,  significant results were obtained for OS in LUSC and for PFI in LUAD from CGP data. Additionally,  higher MATH was associated with high TNM (stages III and IV) staging for UCEC (clinical),  COAD,  SKCM,  KIRP,  and ESCA (pathological) with WXS data and for colorectal cancer with CGP data. In addition,  higher MATH was also related to the presence of metastasis in UCEC,  ESCA,  COAD,  KIRP,  and KIRC with WXS data. With MSK-IMPACT 410,  higher ITGH was associated not only with presence,  but also with a higher risk of metastasis in patients with colorectal cancer. On the other hand,  MATH was not significantly different between responders and non-responders to immunotherapy in both the WXS and CGP data. Therefore,  MATH presents itself as a promising tool for oncologists due to its simple use and easily interpreted results,  in addition to the aforementioned associations with survival,  staging and metastasis. In conclusion,  MATH may have several applications in the near future regarding patient prognosis and therapeutic decision making.
  
  Funding: Coordena\c{c}\~ao de Aperfei\c{c}oamento de Pessoal de N\'{\i}vel Superior - Brasil (CAPES) - C\'odigo de Financiamento 001,  Funda\c{c}\~ao de Amparo \`a Pesquisa do Estado de S\~ao Paulo (FAPESP) - processo n$^o$ 2017/17974-9 \\ 
  \end{abstract}
  \end{document} 