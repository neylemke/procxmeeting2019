
  \documentclass[twoside]{article}
  \usepackage[affil-it]{authblk}
  \usepackage{lipsum} % Package to generate dummy text throughout this template
  \usepackage{eurosym}
  \usepackage[sc]{mathpazo} % Use the Palatino font
  \usepackage[T1]{fontenc} % Use 8-bit encoding that has 256 glyphs
  \usepackage[utf8]{inputenc}
  \linespread{1.05} % Line spacing-Palatino needs more space between lines
  \usepackage{microtype} % Slightly tweak font spacing for aesthetics\[IndentingNewLine]
  \usepackage[hmarginratio=1:1,top=32mm,columnsep=20pt]{geometry} % Document margins
  \usepackage{multicol} % Used for the two-column layout of the document
  \usepackage[hang,small,labelfont=bf,up,textfont=it,up]{caption} % Custom captions under//above floats in tables or figures
  \usepackage{booktabs} % Horizontal rules in tables
  \usepackage{float} % Required for tables and figures in the multi-column environment-they need to be placed in specific locations with the[H] (e.g. \begin{table}[H])
  \usepackage{hyperref} % For hyperlinks in the PDF
  \usepackage{lettrine} % The lettrine is the first enlarged letter at the beginning of the text
  \usepackage{paralist} % Used for the compactitem environment which makes bullet points with less space between them
  \usepackage{abstract} % Allows abstract customization
  \renewcommand{\abstractnamefont}{\normalfont\bfseries} 
  %\renewcommand{\abstracttextfont}{\normalfont\small\itshape} % Set the abstract itself to small italic text\[IndentingNewLine]
  \usepackage{titlesec} % Allows customization of titles
  \renewcommand\thesection{\Roman{section}} % Roman numerals for the sections
  \renewcommand\thesubsection{\Roman{subsection}} % Roman numerals for subsections
  \titleformat{\section}[block]{\large\scshape\centering}{\thesection.}{1em}{} % Change the look of the section titles
  \titleformat{\subsection}[block]{\large}{\thesubsection.}{1em}{} % Change the look of the section titles
  \usepackage{fancyhdr} % Headers and footers
  \pagestyle{fancy} % All pages have headers and footers
  \fancyhead{} % Blank out the default header
  \fancyfoot{} % Blank out the default footer
  \fancyhead[C]{X-meeting $\bullet$ October 2019 $\bullet$ Campos do  Jord\~ao} % Custom header text
  \fancyfoot[RO,LE]{} % Custom footer text
  %----------------------------------------------------------------------------------------
  % TITLE SECTION
  %---------------------------------------------------------------------------------------- 
 
 \title{\vspace{-15mm}\fontsize{24pt}{10pt}\selectfont\textbf{ Using  Drosophila melanogaster Y chromosome heterochromatic sequences as a model to construct complete oligopaints }} % Article title
  
  
  \author{ Isabela Pimentel de Almeida, Maria Dulcetti Vibranovski, Antonio Bernardo de Carvalho }
  
  \affil{ Universidade Federal do Rio de Janeiro }
  \vspace{-5mm}
  \date{}
  
  %---------------------------------------------------------------------------------------- 
  
  \begin{document}
  
  
  \maketitle % Insert title
  
  
  \thispagestyle{fancy} % All pages have headers and footers
  %----------------------------------------------------------------------------------------  
  % ABSTRACT
  
  %----------------------------------------------------------------------------------------  
  
  \begin{abstract}
  Researches that focuses on understanding the sequence and organization of heterochromatin allows essential functions for the organism to be better understood. One of the main obstacles in studies with the Y chromosome is related to the heterochromatic state of this structure,  which in Drosophila melanogaster is formed by approximately 41 Mb of highly repetitive sequences. With applications in cytogenetic studies,  the protocol for constructing oligopaints probes does not usually include such sequences. Considering that there are only 1381 probes for the D. melanogaster Y chromosome,  whereas for its other chromosomes this value is at least ten times greater,  oligopaints for this structure do not allow analyzes as deep as for others. Using D. melanogaster Y chromosome as a model,  we aim to identify unique repetitive sequences of it through the YGS (Y chromosome Genome Scan) method,  increasing the number of known probes for this structure and allowing to construct its complete oligopaint. Thus,  it is possible to compile the methodologies involved in the development of a new technique that will allow to construct oligopaints of the Y chromosome of any species of interest. The analyses of the results and the own efficiency and quality of the obtained oligopaints will be given through the direct comparison between these and those generated from the 1381 probes currently known for the Y chromosome.
  
  Funding: CAPES e FAPESP (Jovem Pesquisador 2015/20844-4) \\ 
  \end{abstract}
  \end{document} 