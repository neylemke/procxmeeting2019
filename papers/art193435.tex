
  \documentclass[twoside]{article}
  \usepackage[affil-it]{authblk}
  \usepackage{lipsum} % Package to generate dummy text throughout this template
  \usepackage{eurosym}
  \usepackage[sc]{mathpazo} % Use the Palatino font
  \usepackage[T1]{fontenc} % Use 8-bit encoding that has 256 glyphs
  \usepackage[utf8]{inputenc}
  \linespread{1.05} % Line spacing-Palatino needs more space between lines
  \usepackage{microtype} % Slightly tweak font spacing for aesthetics\[IndentingNewLine]
  \usepackage[hmarginratio=1:1,top=32mm,columnsep=20pt]{geometry} % Document margins
  \usepackage{multicol} % Used for the two-column layout of the document
  \usepackage[hang,small,labelfont=bf,up,textfont=it,up]{caption} % Custom captions under//above floats in tables or figures
  \usepackage{booktabs} % Horizontal rules in tables
  \usepackage{float} % Required for tables and figures in the multi-column environment-they need to be placed in specific locations with the[H] (e.g. \begin{table}[H])
  \usepackage{hyperref} % For hyperlinks in the PDF
  \usepackage{lettrine} % The lettrine is the first enlarged letter at the beginning of the text
  \usepackage{paralist} % Used for the compactitem environment which makes bullet points with less space between them
  \usepackage{abstract} % Allows abstract customization
  \renewcommand{\abstractnamefont}{\normalfont\bfseries} 
  %\renewcommand{\abstracttextfont}{\normalfont\small\itshape} % Set the abstract itself to small italic text\[IndentingNewLine]
  \usepackage{titlesec} % Allows customization of titles
  \renewcommand\thesection{\Roman{section}} % Roman numerals for the sections
  \renewcommand\thesubsection{\Roman{subsection}} % Roman numerals for subsections
  \titleformat{\section}[block]{\large\scshape\centering}{\thesection.}{1em}{} % Change the look of the section titles
  \titleformat{\subsection}[block]{\large}{\thesubsection.}{1em}{} % Change the look of the section titles
  \usepackage{fancyhdr} % Headers and footers
  \pagestyle{fancy} % All pages have headers and footers
  \fancyhead{} % Blank out the default header
  \fancyfoot{} % Blank out the default footer
  \fancyhead[C]{X-meeting $\bullet$ October 2019 $\bullet$ Campos do  Jord\~ao} % Custom header text
  \fancyfoot[RO,LE]{} % Custom footer text
  %----------------------------------------------------------------------------------------
  % TITLE SECTION
  %---------------------------------------------------------------------------------------- 
 
 \title{\vspace{-15mm}\fontsize{24pt}{10pt}\selectfont\textbf{ Analysis of Affinity and Selectivity of Novel Inhibitors of Polyketide Synthase 13 of Mycobacterium tuberculosis by Molecular Dynamics Simulation and Binding Free Energy Calculations }} % Article title
  
  
  \author{ Jorddy Neves Cruz, Jo\~ao Marcos Pereira Gal\'ucio, Paulo Henrique Taube, Kau\^e Santana da Costa, Eloisa Helena de Aguiar Andrade }
  
  \affil{ Museu Paraense Emilio Goeldi,  Botany Coordination. Adolpho Ducke Laboratory,  Bel\'em,  Par\'a Brazil. }
  \vspace{-5mm}
  \date{}
  
  %---------------------------------------------------------------------------------------- 
  
  \begin{document}
  
  
  \maketitle % Insert title
  
  
  \thispagestyle{fancy} % All pages have headers and footers
  %----------------------------------------------------------------------------------------  
  % ABSTRACT
  
  %----------------------------------------------------------------------------------------  
  
  \begin{abstract}
  Polyketide Synthase 13 (Pks13) is an essential enzyme that forms mycolic acids,  which are critical for viability and virulence of Mycobacterium tuberculosis. Pks13 performs the final assembly step of the mycolic acid synthesis,  i.e.,  the Claisen-type condensation of a C26 a-alkyl branch and C40–60 meromycolate precursors. In the present study,  we investigated the binding mode,  the affinity,  and selectivity of novel inhibitors,  Tam5 and Tam6,  against the Pks13 binding pocket by molecular dynamics simulation (MD) and binding free energy calculations. Our analyses showed that all Pks13-inhibitors systems reach the stabilization after 30 ns of MD,  exhibiting for protein backbone an average RMSD value of 1.61 and 1.59 $\AA$ and the inhibitors also showed a high affinity to the residues of binding pocket exhibiting the following energies (?Gbind) -46.26 $\pm$ 0.07 kcal.mol-1 and -36.52 $\pm$ 0.05 kcal.mol-1,  respectively. Ligand pairwise per-residue energy decomposition analysis showed that Ser1636,  Tyr1637,  Asn1640,  Ala1667,  Phe1670,  and Tyr1674,  exhibited the most energetic contribution for ligands stabilization in Pks13 binding pocket. These preliminary results will be useful to further in silico studies that aim to develop novel analog inhibitors with improved selectivity and affinity against Pks13 binding pocket.
  
  Funding:  \\ 
  \end{abstract}
  \end{document} 