
  \documentclass[twoside]{article}
  \usepackage[affil-it]{authblk}
  \usepackage{lipsum} % Package to generate dummy text throughout this template
  \usepackage{eurosym}
  \usepackage[sc]{mathpazo} % Use the Palatino font
  \usepackage[T1]{fontenc} % Use 8-bit encoding that has 256 glyphs
  \usepackage[utf8]{inputenc}
  \linespread{1.05} % Line spacing-Palatino needs more space between lines
  \usepackage{microtype} % Slightly tweak font spacing for aesthetics\[IndentingNewLine]
  \usepackage[hmarginratio=1:1,top=32mm,columnsep=20pt]{geometry} % Document margins
  \usepackage{multicol} % Used for the two-column layout of the document
  \usepackage[hang,small,labelfont=bf,up,textfont=it,up]{caption} % Custom captions under//above floats in tables or figures
  \usepackage{booktabs} % Horizontal rules in tables
  \usepackage{float} % Required for tables and figures in the multi-column environment-they need to be placed in specific locations with the[H] (e.g. \begin{table}[H])
  \usepackage{hyperref} % For hyperlinks in the PDF
  \usepackage{lettrine} % The lettrine is the first enlarged letter at the beginning of the text
  \usepackage{paralist} % Used for the compactitem environment which makes bullet points with less space between them
  \usepackage{abstract} % Allows abstract customization
  \renewcommand{\abstractnamefont}{\normalfont\bfseries} 
  %\renewcommand{\abstracttextfont}{\normalfont\small\itshape} % Set the abstract itself to small italic text\[IndentingNewLine]
  \usepackage{titlesec} % Allows customization of titles
  \renewcommand\thesection{\Roman{section}} % Roman numerals for the sections
  \renewcommand\thesubsection{\Roman{subsection}} % Roman numerals for subsections
  \titleformat{\section}[block]{\large\scshape\centering}{\thesection.}{1em}{} % Change the look of the section titles
  \titleformat{\subsection}[block]{\large}{\thesubsection.}{1em}{} % Change the look of the section titles
  \usepackage{fancyhdr} % Headers and footers
  \pagestyle{fancy} % All pages have headers and footers
  \fancyhead{} % Blank out the default header
  \fancyfoot{} % Blank out the default footer
  \fancyhead[C]{X-meeting $\bullet$ October 2019 $\bullet$ Campos do  Jord\~ao} % Custom header text
  \fancyfoot[RO,LE]{} % Custom footer text
  %----------------------------------------------------------------------------------------
  % TITLE SECTION
  %---------------------------------------------------------------------------------------- 
 
 \title{\vspace{-15mm}\fontsize{24pt}{10pt}\selectfont\textbf{ Characterization of the mitochondrial genome of Phellinotus piptadeniae (Basidiomycota,  Hymenochaetales) and insights on the phylogeny of Agaricomycetes through comparative mitogenomics }} % Article title
  
  
  \author{ Daniel Silva Ara\'ujo, paula  luize camargos fonseca, Gabriel Quintanilha, Ruth Barros, Bertram Brenig, Vasco Ariston de Carvalho Azevedo, Elisandro Ricardo Drechsler dos Santos, Arist\'oteles G\'oes Neto }
  
  \affil{ University G\"ottingen }
  \vspace{-5mm}
  \date{}
  
  %---------------------------------------------------------------------------------------- 
  
  \begin{document}
  
  
  \maketitle % Insert title
  
  
  \thispagestyle{fancy} % All pages have headers and footers
  %----------------------------------------------------------------------------------------  
  % ABSTRACT
  
  %----------------------------------------------------------------------------------------  
  
  \begin{abstract}
  The Fungi kingdom includes extremely diverse organisms and its taxonomy is constantly going through changes because of its diversity and discovery of new species. The elucidation of the phylogenetic relationships between fungi by analysis of mitogenomes is a common method,  mainly because the mitochondrial genome is well conserved between different groups of fungi. In this work,  the mitogenome of the wood-decaying fungus Phellinotus piptadeniae,  a new species and genus of Agaricomycetes described by our group,  was sequenced and assembled using the programs FastQC and SPAdes. The mitochondrial contig was identified with a BLASTn search against other mitogenomes,  and the annotation was done using MITOS2,  MFannot and RNAweasel. Next,  we used the available Agaricomycetes mitogenomes in NCBI and JGI to build a phylogenetic tree including P. piptadeniae. Available fungal mitogenomes were selected and annotated using aforementioned programs and an alignment was built using MAFFT program based on the 14 core mitochondrial genes,  rps3 protein and two ribosomal RNA. The mitogenome of P. piptadeniae has a size of 137, 790 bp with 24.1\% GC content,  14 mitochondrial core-genes,  2 rRNAs,  30 tRNAs,  47 introns and 62 unidentified open reading frames. We also found coding domains for homing-endonucleases (HEs) (GIY or LAGLIDAGD) and dpo. For our comparative mitogenomics analysis,  a total of 55 mitogenomes were retrieved from public databases and different genes,  whose locations in the genome were conserved in the species of a same family or genus,  were identified,  such as: (i) atp6,  cob,  cox1,  cox2,  cox3,  nad2,  nad4,  nad4l,  nad5,  rrnL and rrnS for Ganoderma genus; (ii) atp6,  atp9,  cob,  cox1,  cox2,  cox3,  nad1,  nad2,  nad3,  nad4l,  nad5 and nad6 for Hymenochaetaceae family; and (iii) also all the 17 genes for Russulaceae family. Deviations of these patterns may be explained by the presence of HEs in these mitogenomes: from all the mitogenomes used in our study,  there were coding domains for HE in 50 of them. The phylogenetic tree was built by distance-based method on the genetic set comprised of the 14 core-genes,  rrnL,  rrnS and rps3. The use of the mitogenome as a tool for a better understanding of fungal phylogenetic relationships proved to be useful but the deficiency of enough mitogenomes of all different families in the Agaricomycetes class in public databases still is a problem to overcome since unbalanced sampling influences on the quality of the results.
  
  Funding: CNPq \\ 
  \end{abstract}
  \end{document} 