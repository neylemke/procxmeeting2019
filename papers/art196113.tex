
  \documentclass[twoside]{article}
  \usepackage[affil-it]{authblk}
  \usepackage{lipsum} % Package to generate dummy text throughout this template
  \usepackage{eurosym}
  \usepackage[sc]{mathpazo} % Use the Palatino font
  \usepackage[T1]{fontenc} % Use 8-bit encoding that has 256 glyphs
  \usepackage[utf8]{inputenc}
  \linespread{1.05} % Line spacing-Palatino needs more space between lines
  \usepackage{microtype} % Slightly tweak font spacing for aesthetics\[IndentingNewLine]
  \usepackage[hmarginratio=1:1,top=32mm,columnsep=20pt]{geometry} % Document margins
  \usepackage{multicol} % Used for the two-column layout of the document
  \usepackage[hang,small,labelfont=bf,up,textfont=it,up]{caption} % Custom captions under//above floats in tables or figures
  \usepackage{booktabs} % Horizontal rules in tables
  \usepackage{float} % Required for tables and figures in the multi-column environment-they need to be placed in specific locations with the[H] (e.g. \begin{table}[H])
  \usepackage{hyperref} % For hyperlinks in the PDF
  \usepackage{lettrine} % The lettrine is the first enlarged letter at the beginning of the text
  \usepackage{paralist} % Used for the compactitem environment which makes bullet points with less space between them
  \usepackage{abstract} % Allows abstract customization
  \renewcommand{\abstractnamefont}{\normalfont\bfseries} 
  %\renewcommand{\abstracttextfont}{\normalfont\small\itshape} % Set the abstract itself to small italic text\[IndentingNewLine]
  \usepackage{titlesec} % Allows customization of titles
  \renewcommand\thesection{\Roman{section}} % Roman numerals for the sections
  \renewcommand\thesubsection{\Roman{subsection}} % Roman numerals for subsections
  \titleformat{\section}[block]{\large\scshape\centering}{\thesection.}{1em}{} % Change the look of the section titles
  \titleformat{\subsection}[block]{\large}{\thesubsection.}{1em}{} % Change the look of the section titles
  \usepackage{fancyhdr} % Headers and footers
  \pagestyle{fancy} % All pages have headers and footers
  \fancyhead{} % Blank out the default header
  \fancyfoot{} % Blank out the default footer
  \fancyhead[C]{X-meeting $\bullet$ October 2019 $\bullet$ Campos do  Jord\~ao} % Custom header text
  \fancyfoot[RO,LE]{} % Custom footer text
  %----------------------------------------------------------------------------------------
  % TITLE SECTION
  %---------------------------------------------------------------------------------------- 
 
 \title{\vspace{-15mm}\fontsize{24pt}{10pt}\selectfont\textbf{ Molecular modeling  and pharmacophore based virtual screening of The Nicotinic acetylcholine receptor of Halyomorpha halys }} % Article title
  
  
  \author{ Beatriz Pereira do Nascimento, Fabr\'{\i}cio Santos Barbosa, T. S. Melo, Bruno Silva Andrade }
  
  \affil{ Universidade Estadual do Sudoeste da Bahia,  Brazil }
  \vspace{-5mm}
  \date{}
  
  %---------------------------------------------------------------------------------------- 
  
  \begin{document}
  
  
  \maketitle % Insert title
  
  
  \thispagestyle{fancy} % All pages have headers and footers
  %----------------------------------------------------------------------------------------  
  % ABSTRACT
  
  %----------------------------------------------------------------------------------------  
  
  \begin{abstract}
  The irrational use of fertilizers as a pest control treatment has become increasingly a potential problem for the industrial agricultural sector. In addition,  nicotinoid resistant pests has been increasing over the years,  and for this reason searching alternative compounds for controling and erradication o these pests is crucial for crop production next years. Halyomorpha halys is popularly known as brown marmorated stink bug,  and it spreads in soybean crop,  damaging most of the grains in formation,  as well as is responsible for the reduction in seed yield and quality. The aim of this work was to construct the nAChR 3D structure of H. Halys as well as perform a virtual screening study in order to find new compounds which can complex and inhibit this receptor. The nAChR 3D structure was modeled using homology modeling approach by SWISS MODEL software. Known nAChR inhibitors were used to perform a pharcophore alignment with Pharmagist (http://bioinfo3d.cs.tau.ac.il/PharmaGist/) and after it was submited to ZincPharmer (http://zincpharmer.csb.pitt.edu/) for seaching for pharmacophore-like ligands in ZINC database. In a second step we docked 1.000 selected molecules into the nAChR active site AutoDock Vina software. The five complexes with best enegy affinity values were submited to AMBER 14 package for Molecular Dynamics simulations.
  
  Funding: 1 – Programa de P\'os-gradua\c{c}\~ao em Qu\'{\i}mica. Universidade Estadual do Sudoeste da Bahia,  Brazil. 2- Departamento de Ci\^encias e Tecnologia (DCT). Universidade Estadual do Sudoeste da Bahia,  Brazil 3 Laborat\'orio de Bioinform\'atica e Qu\'{\i}mica Computacional – LBQC. Universidade Estadual do Sudoeste da Bahia,  Brazil \\ 
  \end{abstract}
  \end{document} 