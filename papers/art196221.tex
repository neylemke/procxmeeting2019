
  \documentclass[twoside]{article}
  \usepackage[affil-it]{authblk}
  \usepackage{lipsum} % Package to generate dummy text throughout this template
  \usepackage{eurosym}
  \usepackage[sc]{mathpazo} % Use the Palatino font
  \usepackage[T1]{fontenc} % Use 8-bit encoding that has 256 glyphs
  \usepackage[utf8]{inputenc}
  \linespread{1.05} % Line spacing-Palatino needs more space between lines
  \usepackage{microtype} % Slightly tweak font spacing for aesthetics\[IndentingNewLine]
  \usepackage[hmarginratio=1:1,top=32mm,columnsep=20pt]{geometry} % Document margins
  \usepackage{multicol} % Used for the two-column layout of the document
  \usepackage[hang,small,labelfont=bf,up,textfont=it,up]{caption} % Custom captions under//above floats in tables or figures
  \usepackage{booktabs} % Horizontal rules in tables
  \usepackage{float} % Required for tables and figures in the multi-column environment-they need to be placed in specific locations with the[H] (e.g. \begin{table}[H])
  \usepackage{hyperref} % For hyperlinks in the PDF
  \usepackage{lettrine} % The lettrine is the first enlarged letter at the beginning of the text
  \usepackage{paralist} % Used for the compactitem environment which makes bullet points with less space between them
  \usepackage{abstract} % Allows abstract customization
  \renewcommand{\abstractnamefont}{\normalfont\bfseries} 
  %\renewcommand{\abstracttextfont}{\normalfont\small\itshape} % Set the abstract itself to small italic text\[IndentingNewLine]
  \usepackage{titlesec} % Allows customization of titles
  \renewcommand\thesection{\Roman{section}} % Roman numerals for the sections
  \renewcommand\thesubsection{\Roman{subsection}} % Roman numerals for subsections
  \titleformat{\section}[block]{\large\scshape\centering}{\thesection.}{1em}{} % Change the look of the section titles
  \titleformat{\subsection}[block]{\large}{\thesubsection.}{1em}{} % Change the look of the section titles
  \usepackage{fancyhdr} % Headers and footers
  \pagestyle{fancy} % All pages have headers and footers
  \fancyhead{} % Blank out the default header
  \fancyfoot{} % Blank out the default footer
  \fancyhead[C]{X-meeting $\bullet$ October 2019 $\bullet$ Campos do  Jord\~ao} % Custom header text
  \fancyfoot[RO,LE]{} % Custom footer text
  %----------------------------------------------------------------------------------------
  % TITLE SECTION
  %---------------------------------------------------------------------------------------- 
 
 \title{\vspace{-15mm}\fontsize{24pt}{10pt}\selectfont\textbf{ Assessment of fecal microbiome differences in captive and non-captive howler monkeys: implications for conservation planning and management }} % Article title
  
  
  \author{ Raquel Riyuzo de Almeida Franco, Gustavo Ribeiro Fernandes, Jo\~ao Carlos Setubal, Aline Maria da Silva }
  
  \affil{ Departamento de Bioqu\'{\i}mica,  Instituto de Qu\'{\i}mica,  Universidade de Sao Paulo }
  \vspace{-5mm}
  \date{}
  
  %---------------------------------------------------------------------------------------- 
  
  \begin{document}
  
  
  \maketitle % Insert title
  
  
  \thispagestyle{fancy} % All pages have headers and footers
  %----------------------------------------------------------------------------------------  
  % ABSTRACT
  
  %----------------------------------------------------------------------------------------  
  
  \begin{abstract}
  Howler monkeys (Alouatta spp) are endemic of South American tropical forests and are highly susceptible to the yellow fever virus,  thus playing an important role as sentinels of outbreaks. Brazil is experiencing its worst yellow fever outbreak in decades,  which is wiping out some wild howler monkey populations. One strategy to regrow endangered populations is the introduction in the wild of captive - bred animals. Due to the significance of the gut microbiome in animal health,  understanding its composition and function may help conservation planning and management of these species. The present study aims to compare the gut microbiomes of captive and non - captive howler monkeys. Fecal samples from Sao Paulo (Brazil) Zoo Park captive animals and non - captive animals that live in the park’s Atlantic rain forest patch were obtained in two different seasons in two consecutive years,  followed by 16S amplicon and shotgun sequencing. Data analysis revealed differences in the microbial community structure,  diversity,  and function between the two populations,  with non - captive individuals showing higher phylogenetic diversity indices and enrichment of specific metabolic functions. The microbiome of non - captive monkeys appears to be more susceptible to seasonal changes than the microbiomes of captive individuals,  perhaps due to seasonal changes in food availability. In the microbiota of captive animals we did not identify members of the genus Faecalibacterium,  which was identified in the non - captive samples,  and which is an abundant genus in the healthy human gut microbiome. Several novel bacterial and viral genomes were recovered from the shotgun sequences.
  
  Funding: Fapesp,  Capes,  Cnpq \\ 
  \end{abstract}
  \end{document} 