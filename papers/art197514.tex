
  \documentclass[twoside]{article}
  \usepackage[affil-it]{authblk}
  \usepackage{lipsum} % Package to generate dummy text throughout this template
  \usepackage{eurosym}
  \usepackage[sc]{mathpazo} % Use the Palatino font
  \usepackage[T1]{fontenc} % Use 8-bit encoding that has 256 glyphs
  \usepackage[utf8]{inputenc}
  \linespread{1.05} % Line spacing-Palatino needs more space between lines
  \usepackage{microtype} % Slightly tweak font spacing for aesthetics\[IndentingNewLine]
  \usepackage[hmarginratio=1:1,top=32mm,columnsep=20pt]{geometry} % Document margins
  \usepackage{multicol} % Used for the two-column layout of the document
  \usepackage[hang,small,labelfont=bf,up,textfont=it,up]{caption} % Custom captions under//above floats in tables or figures
  \usepackage{booktabs} % Horizontal rules in tables
  \usepackage{float} % Required for tables and figures in the multi-column environment-they need to be placed in specific locations with the[H] (e.g. \begin{table}[H])
  \usepackage{hyperref} % For hyperlinks in the PDF
  \usepackage{lettrine} % The lettrine is the first enlarged letter at the beginning of the text
  \usepackage{paralist} % Used for the compactitem environment which makes bullet points with less space between them
  \usepackage{abstract} % Allows abstract customization
  \renewcommand{\abstractnamefont}{\normalfont\bfseries} 
  %\renewcommand{\abstracttextfont}{\normalfont\small\itshape} % Set the abstract itself to small italic text\[IndentingNewLine]
  \usepackage{titlesec} % Allows customization of titles
  \renewcommand\thesection{\Roman{section}} % Roman numerals for the sections
  \renewcommand\thesubsection{\Roman{subsection}} % Roman numerals for subsections
  \titleformat{\section}[block]{\large\scshape\centering}{\thesection.}{1em}{} % Change the look of the section titles
  \titleformat{\subsection}[block]{\large}{\thesubsection.}{1em}{} % Change the look of the section titles
  \usepackage{fancyhdr} % Headers and footers
  \pagestyle{fancy} % All pages have headers and footers
  \fancyhead{} % Blank out the default header
  \fancyfoot{} % Blank out the default footer
  \fancyhead[C]{X-meeting $\bullet$ October 2019 $\bullet$ Campos do  Jord\~ao} % Custom header text
  \fancyfoot[RO,LE]{} % Custom footer text
  %----------------------------------------------------------------------------------------
  % TITLE SECTION
  %---------------------------------------------------------------------------------------- 
 
 \title{\vspace{-15mm}\fontsize{24pt}{10pt}\selectfont\textbf{ The rare lncRNA GOLLD is widespread and structurally conserved among Mycobacterium tRNA arrays }} % Article title
  
  
  \author{ Sergio Mascarenhas Morgado, Deborah Antunes, Ernesto Raul Caffarena, Ana Carolina Paulo Vicente }
  
  \affil{ Oswaldo Cruz Institute }
  \vspace{-5mm}
  \date{}
  
  %---------------------------------------------------------------------------------------- 
  
  \begin{document}
  
  
  \maketitle % Insert title
  
  
  \thispagestyle{fancy} % All pages have headers and footers
  %----------------------------------------------------------------------------------------  
  % ABSTRACT
  
  %----------------------------------------------------------------------------------------  
  
  \begin{abstract}
  Noncoding RNA (ncRNA) genes produce transcripts involved in catalytic or regulatory functions,  some of them presenting highly complex structures. GOLLD RNA is the third-largest bacterial ncRNAs known (~800 bp); however,  its function is still unknown. The GOLLD RNA gene is generally found associated with tRNA genes and supposed to be chromosome- and phage-encoded in bacteria from Lactobacillales and Actinomycetales orders. Besides,  the only inferred GOLLD RNA structure was mainly based on metagenomic sequences. To explore GOLLD in bacterial genomes,  we mined GOLLD gene in thousands of Mycobacterium and virus genomes using Infernal software,  identifying it in 350 mycobacteria (including two in megaplasmids) and 39 virus genomes,  mainly associated with tRNA arrays. Mycobacterium GOLLD genes are highly diverse and distributed in three clades: Mycobacterium exclusive; Mycobacterium and mycobacteriophages; and mycobacteriophage exclusive. We also determined the secondary structure of each clade using R2R software based on GOLLD alignments generated by Infernal software. All clades displayed a 3' half conserved structure including utter E-loops pseudoknots substructures,  also shared by non-Mycobacterium GOLLD while the 5' half motif was different among the clades. In some cases,  an ORF,  coding a tRNA or transposase gene,  was predicted in the  5' half motif. Moreover,  in vitro assays determined the expression of GOLLD RNA gene present in a plasmid harbored by a Mycobacterium isolate from Atlantic Forest soil. Our study showed that the long ncRNA GOLLD is widespread within Mycobacterium in association with tRNA arrays,  besides strengthening its structure,  previously predicted based in metagenomic sequences.
  
  Funding: Conselho Nacional de Desenvolvimento Cient\'{\i}fico e Tecnol\'ogico (CNPq); Coordena\c{c}\~ao de Aperfei\c{c}oamento de Pessoal de N\'{\i}vel Superior -Brasil (CAPES); Oswaldo Cruz Institute \\ 
  \end{abstract}
  \end{document} 