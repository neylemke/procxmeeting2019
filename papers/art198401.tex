
  \documentclass[twoside]{article}
  \usepackage[affil-it]{authblk}
  \usepackage{lipsum} % Package to generate dummy text throughout this template
  \usepackage{eurosym}
  \usepackage[sc]{mathpazo} % Use the Palatino font
  \usepackage[T1]{fontenc} % Use 8-bit encoding that has 256 glyphs
  \usepackage[utf8]{inputenc}
  \linespread{1.05} % Line spacing-Palatino needs more space between lines
  \usepackage{microtype} % Slightly tweak font spacing for aesthetics\[IndentingNewLine]
  \usepackage[hmarginratio=1:1,top=32mm,columnsep=20pt]{geometry} % Document margins
  \usepackage{multicol} % Used for the two-column layout of the document
  \usepackage[hang,small,labelfont=bf,up,textfont=it,up]{caption} % Custom captions under//above floats in tables or figures
  \usepackage{booktabs} % Horizontal rules in tables
  \usepackage{float} % Required for tables and figures in the multi-column environment-they need to be placed in specific locations with the[H] (e.g. \begin{table}[H])
  \usepackage{hyperref} % For hyperlinks in the PDF
  \usepackage{lettrine} % The lettrine is the first enlarged letter at the beginning of the text
  \usepackage{paralist} % Used for the compactitem environment which makes bullet points with less space between them
  \usepackage{abstract} % Allows abstract customization
  \renewcommand{\abstractnamefont}{\normalfont\bfseries} 
  %\renewcommand{\abstracttextfont}{\normalfont\small\itshape} % Set the abstract itself to small italic text\[IndentingNewLine]
  \usepackage{titlesec} % Allows customization of titles
  \renewcommand\thesection{\Roman{section}} % Roman numerals for the sections
  \renewcommand\thesubsection{\Roman{subsection}} % Roman numerals for subsections
  \titleformat{\section}[block]{\large\scshape\centering}{\thesection.}{1em}{} % Change the look of the section titles
  \titleformat{\subsection}[block]{\large}{\thesubsection.}{1em}{} % Change the look of the section titles
  \usepackage{fancyhdr} % Headers and footers
  \pagestyle{fancy} % All pages have headers and footers
  \fancyhead{} % Blank out the default header
  \fancyfoot{} % Blank out the default footer
  \fancyhead[C]{X-meeting $\bullet$ October 2019 $\bullet$ Campos do  Jord\~ao} % Custom header text
  \fancyfoot[RO,LE]{} % Custom footer text
  %----------------------------------------------------------------------------------------
  % TITLE SECTION
  %---------------------------------------------------------------------------------------- 
 
 \title{\vspace{-15mm}\fontsize{24pt}{10pt}\selectfont\textbf{ A molecular docking and ADMET study of a promising compound of the Brazilian semi arid with inhibitory potencial of IKK-B }} % Article title
  
  
  \author{ Wagner Rodrigues de Assis Soares, Tarcisio Silva Melo, Bruno Silva Andrade }
  
  \affil{ Universidade Estadual dom Sudoeste da Bahia }
  \vspace{-5mm}
  \date{}
  
  %---------------------------------------------------------------------------------------- 
  
  \begin{document}
  
  
  \maketitle % Insert title
  
  
  \thispagestyle{fancy} % All pages have headers and footers
  %----------------------------------------------------------------------------------------  
  % ABSTRACT
  
  %----------------------------------------------------------------------------------------  
  
  \begin{abstract}
  The enzyme IKK-$\beta$ modulates nuclear transcription factor (NF-?B) action directly affecting the transcription response of genes encoding proteins important for immune and inflammatory response. The Brazilian semi-arid compounds can be configured as scafolds for new anti-inflammatory agents. Several binders isolated from Brazilian semi arid plants were evaluated,  selecting the best complexes with the IKK-$\beta$ structure and their ADMET characteristics. Binder structures were designed and deposited in the Semi arid Molecule Database (SAM Database),  hosted in the Laboratory of Bioinformatics and Computational Chemistry (LBQC-UESB). The enzyme structure was downloaded from PDB database (4KIK) and computational chemistry tools (Marvin Sketch,  Autodock Vina,  Pymol 1.7,  Discovery Studio 4.0 and Osiris Property Explorer) were used to prepare the structures of the compounds for the molecular docking assay and evaluation of their ADMET characteristics. The SAM0850 binder (-6.5kcal/mol) had lower affinity energy with IKK-$\beta$,  when compared to ATP (-7.3 kcal/mol),  estaurosporine (-7, 7kcal/mol),  GSK-7 (-9, 5kcal/mol) and higher affinity energy than acetylsalicylic acid (-5.8kcal/mol) and mesalasin (-5.5kcal/mol)/mol). The compound SAM0850 did not demonstrate toxicity in the silicon prediction,  the molecular dynamics of the complexes and in vitro tests assays wiil be the next steps.
  
  Funding:  \\ 
  \end{abstract}
  \end{document} 