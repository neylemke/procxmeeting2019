
  \documentclass[twoside]{article}
  \usepackage[affil-it]{authblk}
  \usepackage{lipsum} % Package to generate dummy text throughout this template
  \usepackage{eurosym}
  \usepackage[sc]{mathpazo} % Use the Palatino font
  \usepackage[T1]{fontenc} % Use 8-bit encoding that has 256 glyphs
  \usepackage[utf8]{inputenc}
  \linespread{1.05} % Line spacing-Palatino needs more space between lines
  \usepackage{microtype} % Slightly tweak font spacing for aesthetics\[IndentingNewLine]
  \usepackage[hmarginratio=1:1,top=32mm,columnsep=20pt]{geometry} % Document margins
  \usepackage{multicol} % Used for the two-column layout of the document
  \usepackage[hang,small,labelfont=bf,up,textfont=it,up]{caption} % Custom captions under//above floats in tables or figures
  \usepackage{booktabs} % Horizontal rules in tables
  \usepackage{float} % Required for tables and figures in the multi-column environment-they need to be placed in specific locations with the[H] (e.g. \begin{table}[H])
  \usepackage{hyperref} % For hyperlinks in the PDF
  \usepackage{lettrine} % The lettrine is the first enlarged letter at the beginning of the text
  \usepackage{paralist} % Used for the compactitem environment which makes bullet points with less space between them
  \usepackage{abstract} % Allows abstract customization
  \renewcommand{\abstractnamefont}{\normalfont\bfseries} 
  %\renewcommand{\abstracttextfont}{\normalfont\small\itshape} % Set the abstract itself to small italic text\[IndentingNewLine]
  \usepackage{titlesec} % Allows customization of titles
  \renewcommand\thesection{\Roman{section}} % Roman numerals for the sections
  \renewcommand\thesubsection{\Roman{subsection}} % Roman numerals for subsections
  \titleformat{\section}[block]{\large\scshape\centering}{\thesection.}{1em}{} % Change the look of the section titles
  \titleformat{\subsection}[block]{\large}{\thesubsection.}{1em}{} % Change the look of the section titles
  \usepackage{fancyhdr} % Headers and footers
  \pagestyle{fancy} % All pages have headers and footers
  \fancyhead{} % Blank out the default header
  \fancyfoot{} % Blank out the default footer
  \fancyhead[C]{X-meeting $\bullet$ October 2019 $\bullet$ Campos do  Jord\~ao} % Custom header text
  \fancyfoot[RO,LE]{} % Custom footer text
  %----------------------------------------------------------------------------------------
  % TITLE SECTION
  %---------------------------------------------------------------------------------------- 
 
 \title{\vspace{-15mm}\fontsize{24pt}{10pt}\selectfont\textbf{ Warfarin dosing prediction in Brazilian patients using Algorithms based on Regression and Neural Network Models }} % Article title
  
  
  \author{ Jennifer Eliana Montoya Neyra, Paulo Caleb J. L. Santos, J\'ulia Maria Pavan Soler }
  
  \affil{ Laboratory of Genetics and Molecular Cardiology,  Faculdade de Medicina FMUSP,  Heart Institute (InCor),  University of S\~ao Paulo. }
  \vspace{-5mm}
  \date{}
  
  %---------------------------------------------------------------------------------------- 
  
  \begin{document}
  
  
  \maketitle % Insert title
  
  
  \thispagestyle{fancy} % All pages have headers and footers
  %----------------------------------------------------------------------------------------  
  % ABSTRACT
  
  %----------------------------------------------------------------------------------------  
  
  \begin{abstract}
  To analyze the performance of warfarin dosing prediction through Multiple Linear Regression (MLR) and MultiLayer Perceptron (MLP) algorithms in Brazilian patients from the Heart Institute (InCor-USP),  were used demographic,  genetic and clinical informations of 749 individuals with a maintenance doses in a stable state of warfarin. In addition,  International Normalized Ratio (INR) values,  between 2 and 3,  were used to monitor how well the blood-thinning medication is working to prevent blood clots. The dataset was partitioned in 599 individuals to the training group and 150 individuals to the test group. From the available characteristics,  16 variables corresponding to the IWPC algorithm were evaluated and the result compared with the set of all 74 variables analyzed under the MLR and MLP algorithms. The mean absolute error (MAE) was used to assess the accuracy of the models,  which is a metric widely used in warfarin prediction studies. 
The results show for 74 variables that the MLP had a better performance (MAE = 7.87 mg/week,  SD = 10.06) compared to the MLR using the same variables (MAE = 8.49 mg/week,  SD = 11.13),  and also showed better results that using the variables proposed by the IWPC for both the MLR (MAE = 7.99 mg/week,  SD = 10.86) and MLP (MAE = 8.44 mg/week,  SD = 11.06) models. The performance of the MLP and MLR models tested in this study showed the well-known tendency of the MLP model to obtain better results when are analyzed a greater number of characteristics. This allows us to consider this type of neural network as a good candidate for the prediction of warfarin maintenance doses,  taking into account that at present about of 600 variables have been related to anticoagulant therapy.
  
  Funding: CAPES \\ 
  \end{abstract}
  \end{document} 