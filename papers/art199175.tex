
  \documentclass[twoside]{article}
  \usepackage[affil-it]{authblk}
  \usepackage{lipsum} % Package to generate dummy text throughout this template
  \usepackage{eurosym}
  \usepackage[sc]{mathpazo} % Use the Palatino font
  \usepackage[T1]{fontenc} % Use 8-bit encoding that has 256 glyphs
  \usepackage[utf8]{inputenc}
  \linespread{1.05} % Line spacing-Palatino needs more space between lines
  \usepackage{microtype} % Slightly tweak font spacing for aesthetics\[IndentingNewLine]
  \usepackage[hmarginratio=1:1,top=32mm,columnsep=20pt]{geometry} % Document margins
  \usepackage{multicol} % Used for the two-column layout of the document
  \usepackage[hang,small,labelfont=bf,up,textfont=it,up]{caption} % Custom captions under//above floats in tables or figures
  \usepackage{booktabs} % Horizontal rules in tables
  \usepackage{float} % Required for tables and figures in the multi-column environment-they need to be placed in specific locations with the[H] (e.g. \begin{table}[H])
  \usepackage{hyperref} % For hyperlinks in the PDF
  \usepackage{lettrine} % The lettrine is the first enlarged letter at the beginning of the text
  \usepackage{paralist} % Used for the compactitem environment which makes bullet points with less space between them
  \usepackage{abstract} % Allows abstract customization
  \renewcommand{\abstractnamefont}{\normalfont\bfseries} 
  %\renewcommand{\abstracttextfont}{\normalfont\small\itshape} % Set the abstract itself to small italic text\[IndentingNewLine]
  \usepackage{titlesec} % Allows customization of titles
  \renewcommand\thesection{\Roman{section}} % Roman numerals for the sections
  \renewcommand\thesubsection{\Roman{subsection}} % Roman numerals for subsections
  \titleformat{\section}[block]{\large\scshape\centering}{\thesection.}{1em}{} % Change the look of the section titles
  \titleformat{\subsection}[block]{\large}{\thesubsection.}{1em}{} % Change the look of the section titles
  \usepackage{fancyhdr} % Headers and footers
  \pagestyle{fancy} % All pages have headers and footers
  \fancyhead{} % Blank out the default header
  \fancyfoot{} % Blank out the default footer
  \fancyhead[C]{X-meeting $\bullet$ October 2019 $\bullet$ Campos do  Jord\~ao} % Custom header text
  \fancyfoot[RO,LE]{} % Custom footer text
  %----------------------------------------------------------------------------------------
  % TITLE SECTION
  %---------------------------------------------------------------------------------------- 
 
 \title{\vspace{-15mm}\fontsize{24pt}{10pt}\selectfont\textbf{ Generative Adversarial Neural Networks for a Multiomics Approach in the Mycobacterium Tuberculosis Complex Analysis }} % Article title
  
  
  \author{ Salvador S\'anchez Vinces, Ana Marcia de S\'a Guimar\~aes, Ronaldo Fumio Hashimoto }
  
  \affil{ University of S\~ao Paulo }
  \vspace{-5mm}
  \date{}
  
  %---------------------------------------------------------------------------------------- 
  
  \begin{document}
  
  
  \maketitle % Insert title
  
  
  \thispagestyle{fancy} % All pages have headers and footers
  %----------------------------------------------------------------------------------------  
  % ABSTRACT
  
  %----------------------------------------------------------------------------------------  
  
  \begin{abstract}
  This work presents the application of generative Deep Neural Network methods to Mycobacterium tuberculosis (MTB) gene expression data (with preliminary results). These methods allow generating data with the same distribution as the original samples,  as well as facilitating selective data generation of subgroups of original samples,  and allowing some degree of manipulation to generate states of gene expression profiles. With such a variety of data,  it is possible to establish further processing that facilitates analysis of genetic information (genome and transcriptome). The aim is to deepen the development of this new area of application of generative deep learning methods,  studying characteristics and required preprocessing of the input biological data and optimizing the structures of neural networks for searching biologically plausible and integrated results at different levels of genetic information,  and thus obtain data of interest in the amount required to make robust inferences using different models (e.g.,  identification and comparison of phenotypes by co-expression). For implementation and testing of the proposed model,  we find convenient to work with MTB as reference microorganism within the MTB complex,  for the relatively greater amount of information available,  for example they have small but complex genome (approx. 4000 genes) and because their expression mechanisms are relatively well understood (approx. 40\% of their genome has been characterized).  The generated data (gene expression) were evaluated using qualitative distribution metrics such as histograms and t-SNE,  and the effect on gene co-expression models. For both histograms and t-SNE,  the generative model achieves a very similar distribution of values compared to original samples. Co-expression analysis shows a positive increase in the number of genes and modules inferred from the generated data when compared to the ones obtained from original data,  such as modules neglected by the latter.
  
  Funding: CAPES \\ 
  \end{abstract}
  \end{document} 