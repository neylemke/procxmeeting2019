
  \documentclass[twoside]{article}
  \usepackage[affil-it]{authblk}
  \usepackage{lipsum} % Package to generate dummy text throughout this template
  \usepackage{eurosym}
  \usepackage[sc]{mathpazo} % Use the Palatino font
  \usepackage[T1]{fontenc} % Use 8-bit encoding that has 256 glyphs
  \usepackage[utf8]{inputenc}
  \linespread{1.05} % Line spacing-Palatino needs more space between lines
  \usepackage{microtype} % Slightly tweak font spacing for aesthetics\[IndentingNewLine]
  \usepackage[hmarginratio=1:1,top=32mm,columnsep=20pt]{geometry} % Document margins
  \usepackage{multicol} % Used for the two-column layout of the document
  \usepackage[hang,small,labelfont=bf,up,textfont=it,up]{caption} % Custom captions under//above floats in tables or figures
  \usepackage{booktabs} % Horizontal rules in tables
  \usepackage{float} % Required for tables and figures in the multi-column environment-they need to be placed in specific locations with the[H] (e.g. \begin{table}[H])
  \usepackage{hyperref} % For hyperlinks in the PDF
  \usepackage{lettrine} % The lettrine is the first enlarged letter at the beginning of the text
  \usepackage{paralist} % Used for the compactitem environment which makes bullet points with less space between them
  \usepackage{abstract} % Allows abstract customization
  \renewcommand{\abstractnamefont}{\normalfont\bfseries} 
  %\renewcommand{\abstracttextfont}{\normalfont\small\itshape} % Set the abstract itself to small italic text\[IndentingNewLine]
  \usepackage{titlesec} % Allows customization of titles
  \renewcommand\thesection{\Roman{section}} % Roman numerals for the sections
  \renewcommand\thesubsection{\Roman{subsection}} % Roman numerals for subsections
  \titleformat{\section}[block]{\large\scshape\centering}{\thesection.}{1em}{} % Change the look of the section titles
  \titleformat{\subsection}[block]{\large}{\thesubsection.}{1em}{} % Change the look of the section titles
  \usepackage{fancyhdr} % Headers and footers
  \pagestyle{fancy} % All pages have headers and footers
  \fancyhead{} % Blank out the default header
  \fancyfoot{} % Blank out the default footer
  \fancyhead[C]{X-meeting $\bullet$ October 2019 $\bullet$ Campos do  Jord\~ao} % Custom header text
  \fancyfoot[RO,LE]{} % Custom footer text
  %----------------------------------------------------------------------------------------
  % TITLE SECTION
  %---------------------------------------------------------------------------------------- 
 
 \title{\vspace{-15mm}\fontsize{24pt}{10pt}\selectfont\textbf{ An integrative approach to understand species delimitation in Petunia }} % Article title
  
  
  \author{ ANA L\'UCIA ANVERSA SEGATTO, MAIKEL RECK-KORTMANN, CAROLINE TURCHETTO, LORETA BRAND\~AO DE FREITAS }
  
  \affil{ UNIVERSIDADE FEDERAL DO RIO GRANDE DO SUL }
  \vspace{-5mm}
  \date{}
  
  %---------------------------------------------------------------------------------------- 
  
  \begin{document}
  
  
  \maketitle % Insert title
  
  
  \thispagestyle{fancy} % All pages have headers and footers
  %----------------------------------------------------------------------------------------  
  % ABSTRACT
  
  %----------------------------------------------------------------------------------------  
  
  \begin{abstract}
  Plants are considered to be plastic organisms,  not only due to the phenotypic
plasticity they usually present but also because they can hybridize frequently.
Thereby,  it is difficult to determine if the morphological variation seen,  in what is
considered one plant species,  is morphological plasticity,  hybrid organisms,  or
local adaptation. Taking this into account,  simulated datasets have shown that
the most used methods of species delimitation are negatively influenced by
gene flow and incomplete lineage sorting. The long corolla tube clade of
Petunia,  formed by Petunia exserta,  Petunia secreta and three subspecies of
Petunia axillaris,  presents taxa with morphological variation,  genetic sharing, 
and disjoint distribution that make it a good system to study the effects of
population phenomena in species delimitation. In this work,  our objective was to
clarify the evolutionary relationships among long corolla tube taxa,  including
canonical,  non-canonical and geographic disjointed individuals within each
taxon. To do this,  we sequenced eight nuclear regions,  five plastid DNA
markers,  and genotyped seven microsatellite loci. The phylogenetic
relationships among the lineages were estimated using the Bayesian Inference
as implemented in BEAST 1.7,  and species delimitation was conducted using
the Program BPP 3.4. The software Structure 2.3 was used to perform
clustering analysis based on microsatellite data. The supergene tree showed
better resolution than the species tree. Petunia species previously described as
belonging to the long tube clade formed a monophyletic group in the supergene
tree and were distributed in two main subgroups. Multiple runs of BPP with
different MCMC parameters and guide trees gave similar results; the posterior
probabilities values were higher considering eight species. In contrast, 
microsatellite markers analyses indicated the occurrence of two genetic
components. The long corolla tube clade of Petunia encompasses markedly
different taxa,  regarding morphology and life habits. The multigene and multimarker approaches used here to disentangle the evolutionary relations among
these taxa confirmed their genetic identity; however,  it did not agree with the
current taxonomic classification. The observed scenario possibly involves a
complex interaction of different environmental,  phenotypic,  and genetic
phenomena,  which is similar to what is proposed to a great number of species.
Understanding the relationship between genetic diversity and other variability
sources is extremely important to the preservation of evolutionary lineages, 
mainly in front of the global environmental changes,  and can contribute to
understanding how the taxa identity is maintained. Incongruence between
different data sources may be the key to understand the evolutionary history
and new Bioinformatics approaches are necessary to deal with that.
  
  Funding: CNPq,  CAPES,  PPGBM-UFRGS. \\ 
  \end{abstract}
  \end{document} 