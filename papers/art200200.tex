
  \documentclass[twoside]{article}
  \usepackage[affil-it]{authblk}
  \usepackage{lipsum} % Package to generate dummy text throughout this template
  \usepackage{eurosym}
  \usepackage[sc]{mathpazo} % Use the Palatino font
  \usepackage[T1]{fontenc} % Use 8-bit encoding that has 256 glyphs
  \usepackage[utf8]{inputenc}
  \linespread{1.05} % Line spacing-Palatino needs more space between lines
  \usepackage{microtype} % Slightly tweak font spacing for aesthetics\[IndentingNewLine]
  \usepackage[hmarginratio=1:1,top=32mm,columnsep=20pt]{geometry} % Document margins
  \usepackage{multicol} % Used for the two-column layout of the document
  \usepackage[hang,small,labelfont=bf,up,textfont=it,up]{caption} % Custom captions under//above floats in tables or figures
  \usepackage{booktabs} % Horizontal rules in tables
  \usepackage{float} % Required for tables and figures in the multi-column environment-they need to be placed in specific locations with the[H] (e.g. \begin{table}[H])
  \usepackage{hyperref} % For hyperlinks in the PDF
  \usepackage{lettrine} % The lettrine is the first enlarged letter at the beginning of the text
  \usepackage{paralist} % Used for the compactitem environment which makes bullet points with less space between them
  \usepackage{abstract} % Allows abstract customization
  \renewcommand{\abstractnamefont}{\normalfont\bfseries} 
  %\renewcommand{\abstracttextfont}{\normalfont\small\itshape} % Set the abstract itself to small italic text\[IndentingNewLine]
  \usepackage{titlesec} % Allows customization of titles
  \renewcommand\thesection{\Roman{section}} % Roman numerals for the sections
  \renewcommand\thesubsection{\Roman{subsection}} % Roman numerals for subsections
  \titleformat{\section}[block]{\large\scshape\centering}{\thesection.}{1em}{} % Change the look of the section titles
  \titleformat{\subsection}[block]{\large}{\thesubsection.}{1em}{} % Change the look of the section titles
  \usepackage{fancyhdr} % Headers and footers
  \pagestyle{fancy} % All pages have headers and footers
  \fancyhead{} % Blank out the default header
  \fancyfoot{} % Blank out the default footer
  \fancyhead[C]{X-meeting $\bullet$ October 2019 $\bullet$ Campos do  Jord\~ao} % Custom header text
  \fancyfoot[RO,LE]{} % Custom footer text
  %----------------------------------------------------------------------------------------
  % TITLE SECTION
  %---------------------------------------------------------------------------------------- 
 
 \title{\vspace{-15mm}\fontsize{24pt}{10pt}\selectfont\textbf{ Taxonomy and comparative genomics of a Corynebacterium ulcerans strain isolated from Pig,  previously identified as C. pseudotuberculosis }} % Article title
  
  
  \author{ Jana\'{\i}na Can\'ario Cerqueira, Rodrigo Profeta Silveira Santos, Alessandra Lima da Silva, Raquel Enma Hurtado Castillo, Marcelle Oliveira Almeida, Thiago de Jesus Sousa, Diego Lucas Neres Rodrigues, Juan Luis Valdez Baez, Francielly Rodrigues da Costa, anne cybelle pinto gomide, Henrique Figueiredo, Alice Rebecca Wattam, Artur Silva, Vasco Ariston de Carvalho Azevedo, Marcus Vinicius Can\'ario Viana }
  
  \affil{ University of Virginia }
  \vspace{-5mm}
  \date{}
  
  %---------------------------------------------------------------------------------------- 
  
  \begin{document}
  
  
  \maketitle % Insert title
  
  
  \thispagestyle{fancy} % All pages have headers and footers
  %----------------------------------------------------------------------------------------  
  % ABSTRACT
  
  %----------------------------------------------------------------------------------------  
  
  \begin{abstract}
  The bacterial strain PO100/5 was isolated from a skin abscess of a pig (Sus scrofa domesticus) in the Alentejo region of southern Portugal. It was identified as Corynebacterium pseudotuberculosis using biochemical tests (Api Coryne\textsuperscript{\textcopyright} kit),  multiplex PCR and Pulsed Field Gel Electrophoresis. After genome sequencing and in silico analyses,  the strain was re-identified as C. ulcerans and deposited in Genbank. This species can harbor the Phospholipase D (PLD) toxin and diphtheria toxin (DT),  has variety of mammalian hosts as reservoirs,  and emerged in the last 30 years as the main cause of diphtheria in humans. To better understand the taxonomy of C. ulcerans and improve the identification methods,  we compared strain PO100/5 to other Corynebacterium genomes publicly available. The taxonomic identification of this strain was refined using 16S gene sequence identity,  phylogenetic analysis of the genes 16S and rpoB,  phylogenomic analysis using the nucleotide sequence of 1, 000 shared genes,  and the Average Nucleotide Identity (ANI). The identity of the 16S gene was calculated in relation to the type strains C. ulcerans NCTC7910 (99.673\%),  C. pseudotuberculosis ATC19410 (99.476\%) and C. dipthteriae NCTC11397 (97.709\%). Those values were above the 97\% cutoff used to identify the same species. Two strains had 100\% identity with PO100/5: C. ulcerans W25 (wild boar,  Germany) and KL1196 (deer,  Germany). All phylogenies (16S,  rpoB and phylogenomics) showed a clade composed of C. ulcerans strains PO100/5,  W25 and KL1196 separated from the other C. ulcerans strains and from the closest species C. pseudotuberculosis and C. diphtheriae. Additionally,  C. ulcerans strains were separated in three clades. The ANI analysis showed that within strains PO100/5,  W25 and KL1199 the pairwise values varied from 99.59 to 100\%,  and from 90.13 to 90.34\% when those were compared to other C. ulcerans strains. Within the other C. ulcerans strains,  those values varied from 95.32 to 99.94\%. According to ANI method,  genomes with an identity below 95\% are considered different species. The genome plasticity was accessed by prophage and genomic island prediction. Four prophages were predicted in the genome of strain PO100/5,  one of them harboring the DT gene (tox). Eight and sixteen genomic islands were predicted by comparing PO100/5 to the type strains C. ulcerans NCTC7910 and C. pseudotuberculosis ACTT19410,  respectively. The genomic circular map showed three regions shared by strains PO100/5,  W25 and KL1199 that are absent in other C. ulcerans and the closest species C. pseudotuberculosis and C. diphtheriae. These regions could be used as molecular markers for identification by multiplex PCR. The results suggest that strains PO100/5,  W25 and KL1199 could represent a new species that diverged from a common ancestor with C. ulcerans. Further genomic analysis will be performed to test the hypothesis.
  
  Funding: UFMG,  UFPA,  CNPq \\ 
  \end{abstract}
  \end{document} 