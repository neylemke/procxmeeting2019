
  \documentclass[twoside]{article}
  \usepackage[affil-it]{authblk}
  \usepackage{lipsum} % Package to generate dummy text throughout this template
  \usepackage{eurosym}
  \usepackage[sc]{mathpazo} % Use the Palatino font
  \usepackage[T1]{fontenc} % Use 8-bit encoding that has 256 glyphs
  \usepackage[utf8]{inputenc}
  \linespread{1.05} % Line spacing-Palatino needs more space between lines
  \usepackage{microtype} % Slightly tweak font spacing for aesthetics\[IndentingNewLine]
  \usepackage[hmarginratio=1:1,top=32mm,columnsep=20pt]{geometry} % Document margins
  \usepackage{multicol} % Used for the two-column layout of the document
  \usepackage[hang,small,labelfont=bf,up,textfont=it,up]{caption} % Custom captions under//above floats in tables or figures
  \usepackage{booktabs} % Horizontal rules in tables
  \usepackage{float} % Required for tables and figures in the multi-column environment-they need to be placed in specific locations with the[H] (e.g. \begin{table}[H])
  \usepackage{hyperref} % For hyperlinks in the PDF
  \usepackage{lettrine} % The lettrine is the first enlarged letter at the beginning of the text
  \usepackage{paralist} % Used for the compactitem environment which makes bullet points with less space between them
  \usepackage{abstract} % Allows abstract customization
  \renewcommand{\abstractnamefont}{\normalfont\bfseries} 
  %\renewcommand{\abstracttextfont}{\normalfont\small\itshape} % Set the abstract itself to small italic text\[IndentingNewLine]
  \usepackage{titlesec} % Allows customization of titles
  \renewcommand\thesection{\Roman{section}} % Roman numerals for the sections
  \renewcommand\thesubsection{\Roman{subsection}} % Roman numerals for subsections
  \titleformat{\section}[block]{\large\scshape\centering}{\thesection.}{1em}{} % Change the look of the section titles
  \titleformat{\subsection}[block]{\large}{\thesubsection.}{1em}{} % Change the look of the section titles
  \usepackage{fancyhdr} % Headers and footers
  \pagestyle{fancy} % All pages have headers and footers
  \fancyhead{} % Blank out the default header
  \fancyfoot{} % Blank out the default footer
  \fancyhead[C]{X-meeting $\bullet$ October 2019 $\bullet$ Campos do  Jord\~ao} % Custom header text
  \fancyfoot[RO,LE]{} % Custom footer text
  %----------------------------------------------------------------------------------------
  % TITLE SECTION
  %---------------------------------------------------------------------------------------- 
 
 \title{\vspace{-15mm}\fontsize{24pt}{10pt}\selectfont\textbf{ An integrated computational pipeline for inferring microbe-host interactions }} % Article title
  
  
  \author{ Tahila Andrighetti, Leila Gul, Tamas Korcsmaros, Padhmanand Sudhakar }
  
  \affil{ Earlham Institute }
  \vspace{-5mm}
  \date{}
  
  %---------------------------------------------------------------------------------------- 
  
  \begin{document}
  
  
  \maketitle % Insert title
  
  
  \thispagestyle{fancy} % All pages have headers and footers
  %----------------------------------------------------------------------------------------  
  % ABSTRACT
  
  %----------------------------------------------------------------------------------------  
  
  \begin{abstract}
  Microbiota-host interactions are inherent in the evolution of most organisms with both positive and negative impacts. Hence,  investigating host-microbiome interactions is crucial for understanding ecosystem dynamics,  as well as the metabolism and physiology of diverse organisms. One way to evaluate such interactions is to study how organisms such as bacteria interact with their hosts at a molecular level. By detecting interspecies protein-protein interactions,  it is possible to infer the host molecular mechanisms which are modulated by the bacterial proteins. However,  detecting such interactions by experimental techniques remains challenging from a time and cost perspective. A more viable alternative to studying microbiome-host interactions is to use computational tools to predict them. In this work,  we developed a pipeline by which it is possible to predict microbiome-host interactions and evaluate which molecular mechanisms in the host are potentially modulated by microbial proteins. With this end in sight,  our pipeline integrates multi-omics data,  such as metaproteomics which provides information about the composition of the proteins,  microbe-host protein-protein interaction prediction along with host multilayer molecular networks. As a use case,  we used a metaproteomics dataset which contains data (metaproteomics,  metagenomics,  host transcriptomics) from patients diagnosed with Crohn’s disease (CD) and those who are healthy. By selecting differentially expressed proteins between the two conditions,  we first predicted which bacterial proteins interact with human receptor proteins by using domain-domain and domain-motif interaction information from public databases. Then,  we selected autophagy genes as potential target nodes,  as autophagy is one of the known dysregulated cellular processes in CD. The next step consisted of compiling a signaling network which starts from bacterial protein-host receptor interactions,  and ultimately reaching the selected host target genes through protein-protein and transcriptional regulatory interactions. From the obtained network,   it was possible to identify putative molecular mechanisms by which bacterial proteins can modulate autophagy in the context of Crohn’s disease.
  
  Funding:  \\ 
  \end{abstract}
  \end{document} 