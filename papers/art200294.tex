
  \documentclass[twoside]{article}
  \usepackage[affil-it]{authblk}
  \usepackage{lipsum} % Package to generate dummy text throughout this template
  \usepackage{eurosym}
  \usepackage[sc]{mathpazo} % Use the Palatino font
  \usepackage[T1]{fontenc} % Use 8-bit encoding that has 256 glyphs
  \usepackage[utf8]{inputenc}
  \linespread{1.05} % Line spacing-Palatino needs more space between lines
  \usepackage{microtype} % Slightly tweak font spacing for aesthetics\[IndentingNewLine]
  \usepackage[hmarginratio=1:1,top=32mm,columnsep=20pt]{geometry} % Document margins
  \usepackage{multicol} % Used for the two-column layout of the document
  \usepackage[hang,small,labelfont=bf,up,textfont=it,up]{caption} % Custom captions under//above floats in tables or figures
  \usepackage{booktabs} % Horizontal rules in tables
  \usepackage{float} % Required for tables and figures in the multi-column environment-they need to be placed in specific locations with the[H] (e.g. \begin{table}[H])
  \usepackage{hyperref} % For hyperlinks in the PDF
  \usepackage{lettrine} % The lettrine is the first enlarged letter at the beginning of the text
  \usepackage{paralist} % Used for the compactitem environment which makes bullet points with less space between them
  \usepackage{abstract} % Allows abstract customization
  \renewcommand{\abstractnamefont}{\normalfont\bfseries} 
  %\renewcommand{\abstracttextfont}{\normalfont\small\itshape} % Set the abstract itself to small italic text\[IndentingNewLine]
  \usepackage{titlesec} % Allows customization of titles
  \renewcommand\thesection{\Roman{section}} % Roman numerals for the sections
  \renewcommand\thesubsection{\Roman{subsection}} % Roman numerals for subsections
  \titleformat{\section}[block]{\large\scshape\centering}{\thesection.}{1em}{} % Change the look of the section titles
  \titleformat{\subsection}[block]{\large}{\thesubsection.}{1em}{} % Change the look of the section titles
  \usepackage{fancyhdr} % Headers and footers
  \pagestyle{fancy} % All pages have headers and footers
  \fancyhead{} % Blank out the default header
  \fancyfoot{} % Blank out the default footer
  \fancyhead[C]{X-meeting $\bullet$ October 2019 $\bullet$ Campos do  Jord\~ao} % Custom header text
  \fancyfoot[RO,LE]{} % Custom footer text
  %----------------------------------------------------------------------------------------
  % TITLE SECTION
  %---------------------------------------------------------------------------------------- 
 
 \title{\vspace{-15mm}\fontsize{24pt}{10pt}\selectfont\textbf{ HOMOLOGY MODELING AND MOLECULAR DOCKING STUDIES OF ARYLALKYLAMINE N-ACETYLTRANSFERASE (aaNAT) of Aedes aegypti }} % Article title
  
  
  \author{ Maria Ang\'elica Bomfim Oliveira, Fabr\'{\i}cio Santos Barbosa, Tarcisio Silva Melo, Bruno S. Andrade }
  
  \affil{ Universidade Estadual do Sudoeste da Bahia,  Brazil. }
  \vspace{-5mm}
  \date{}
  
  %---------------------------------------------------------------------------------------- 
  
  \begin{document}
  
  
  \maketitle % Insert title
  
  
  \thispagestyle{fancy} % All pages have headers and footers
  %----------------------------------------------------------------------------------------  
  % ABSTRACT
  
  %----------------------------------------------------------------------------------------  
  
  \begin{abstract}
  According to World Health Organization (WHO) data,  about 700, 000 people die each year due to diseases transmitted by the Aedes aegypti mosquito: Dengue,  Chikungunya and ZIKA,  among other arboviruses. Additionally,  this mosquito is capable of reproducing in urban environments,  as well as act as vector,  transmit and replicate diferent type of virus,  and becoming a great problem of public health in many undeveloped countries. One way for controlling this vector is studying its metabolism,  and identifying important protein target which can be modeled and used for. The enzyme arylalkylamine N-acetyltransferase (aaNAT) is essential in the process of cuticle sclerotization and mosquito development. Therefore,  the aim of this work was to perform an homology modeling of the aaNAT,  as well as searching for bioactive molecules which can complex with this target in order to act as inhibitors. Using the Swiss Model Workspace (https://swissmodel.expasy.org/),  protein modeling results showed dopamine N-acetyltransferase protein (PDB code 3V8I) as the best template,  with 60.10\% identity and and 72\% of corverage. The protein model was validated with a QMEAN value of -0.23. A virtual screening apprach was used to find ligand compounds which can complex with aaNAT,  using ZINC database of natural and synthetic compounds. Autodock Vina calculations revealed several ligands with high affinity energy with aaNAT which can be proposed as new inseticides against A. aegypti. The best protein-ligand complexes will be subjected to molecular docking calculations for describing ligand behavior inside the active pocket for 50 nanosseconds of calculation.
  
  Funding: CAPES \\ 
  \end{abstract}
  \end{document} 