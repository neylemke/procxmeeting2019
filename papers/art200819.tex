
  \documentclass[twoside]{article}
  \usepackage[affil-it]{authblk}
  \usepackage{lipsum} % Package to generate dummy text throughout this template
  \usepackage{eurosym}
  \usepackage[sc]{mathpazo} % Use the Palatino font
  \usepackage[T1]{fontenc} % Use 8-bit encoding that has 256 glyphs
  \usepackage[utf8]{inputenc}
  \linespread{1.05} % Line spacing-Palatino needs more space between lines
  \usepackage{microtype} % Slightly tweak font spacing for aesthetics\[IndentingNewLine]
  \usepackage[hmarginratio=1:1,top=32mm,columnsep=20pt]{geometry} % Document margins
  \usepackage{multicol} % Used for the two-column layout of the document
  \usepackage[hang,small,labelfont=bf,up,textfont=it,up]{caption} % Custom captions under//above floats in tables or figures
  \usepackage{booktabs} % Horizontal rules in tables
  \usepackage{float} % Required for tables and figures in the multi-column environment-they need to be placed in specific locations with the[H] (e.g. \begin{table}[H])
  \usepackage{hyperref} % For hyperlinks in the PDF
  \usepackage{lettrine} % The lettrine is the first enlarged letter at the beginning of the text
  \usepackage{paralist} % Used for the compactitem environment which makes bullet points with less space between them
  \usepackage{abstract} % Allows abstract customization
  \renewcommand{\abstractnamefont}{\normalfont\bfseries} 
  %\renewcommand{\abstracttextfont}{\normalfont\small\itshape} % Set the abstract itself to small italic text\[IndentingNewLine]
  \usepackage{titlesec} % Allows customization of titles
  \renewcommand\thesection{\Roman{section}} % Roman numerals for the sections
  \renewcommand\thesubsection{\Roman{subsection}} % Roman numerals for subsections
  \titleformat{\section}[block]{\large\scshape\centering}{\thesection.}{1em}{} % Change the look of the section titles
  \titleformat{\subsection}[block]{\large}{\thesubsection.}{1em}{} % Change the look of the section titles
  \usepackage{fancyhdr} % Headers and footers
  \pagestyle{fancy} % All pages have headers and footers
  \fancyhead{} % Blank out the default header
  \fancyfoot{} % Blank out the default footer
  \fancyhead[C]{X-meeting $\bullet$ October 2019 $\bullet$ Campos do  Jord\~ao} % Custom header text
  \fancyfoot[RO,LE]{} % Custom footer text
  %----------------------------------------------------------------------------------------
  % TITLE SECTION
  %---------------------------------------------------------------------------------------- 
 
 \title{\vspace{-15mm}\fontsize{24pt}{10pt}\selectfont\textbf{ Semantic Similarity Integration for Gene Network Inference }} % Article title
  
  
  \author{ Roger Verzola Peres de Lima, F\'abio Fernandes da Rocha Vicente }
  
  \affil{ Federal University of Technology - Paran\'a }
  \vspace{-5mm}
  \date{}
  
  %---------------------------------------------------------------------------------------- 
  
  \begin{document}
  
  
  \maketitle % Insert title
  
  
  \thispagestyle{fancy} % All pages have headers and footers
  %----------------------------------------------------------------------------------------  
  % ABSTRACT
  
  %----------------------------------------------------------------------------------------  
  
  \begin{abstract}
  Genes are fundamental elements in the dynamic of biological systems. Finding out how genes interact with the dynamic of biological systems may foster not only a better understanding of living beings,  but also the possible genetic manipulations of said living beings with a specific aim. Therefore,  the inference of gene regulatory networks is of great importance. However,  an issue in that field is the small amount of samples available when compared to the amount of variables,  severely limiting the inference power of purely statistical methods. In this work a method that contours that difficulty by uniting both quantitative data and qualitative data is proposed.  Our method combines two types of data: gene expression and gene ontology. The criterion function calculates the mean conditional entropy over the normalized gene expression data and the GFD-Net over the genes’ ontology annotations. GFD-Net is a method that gives a numerical score to the functional dissimilarity of a gene network based on gene ontology. The proposed algorithm to use is the Sequential Forward Feature Selection (SFFS) due to its easy implementation and deterministic nature.  Therefore,  the proposed method is made of two parts: an algorithm that selects a suboptimal set of genes (called predictors) that may interact with the target gene; and a criterion function that said algorithm will use to determine which subset of genes the predictor set will be. Running said algorithm over every target gene enables us to form a gene regulatory network by creating an edge between each predictor set and their respective targets.  The method aims to use two distinct forms of evaluation unifying,  therefore,  both semantic and quantitative measures.
  
  Funding: CAPES,  CNPq,  Funda\c{c}\~ao Arauc\'aria \\ 
  \end{abstract}
  \end{document} 