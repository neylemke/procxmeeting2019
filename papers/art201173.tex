
  \documentclass[twoside]{article}
  \usepackage[affil-it]{authblk}
  \usepackage{lipsum} % Package to generate dummy text throughout this template
  \usepackage{eurosym}
  \usepackage[sc]{mathpazo} % Use the Palatino font
  \usepackage[T1]{fontenc} % Use 8-bit encoding that has 256 glyphs
  \usepackage[utf8]{inputenc}
  \linespread{1.05} % Line spacing-Palatino needs more space between lines
  \usepackage{microtype} % Slightly tweak font spacing for aesthetics\[IndentingNewLine]
  \usepackage[hmarginratio=1:1,top=32mm,columnsep=20pt]{geometry} % Document margins
  \usepackage{multicol} % Used for the two-column layout of the document
  \usepackage[hang,small,labelfont=bf,up,textfont=it,up]{caption} % Custom captions under//above floats in tables or figures
  \usepackage{booktabs} % Horizontal rules in tables
  \usepackage{float} % Required for tables and figures in the multi-column environment-they need to be placed in specific locations with the[H] (e.g. \begin{table}[H])
  \usepackage{hyperref} % For hyperlinks in the PDF
  \usepackage{lettrine} % The lettrine is the first enlarged letter at the beginning of the text
  \usepackage{paralist} % Used for the compactitem environment which makes bullet points with less space between them
  \usepackage{abstract} % Allows abstract customization
  \renewcommand{\abstractnamefont}{\normalfont\bfseries} 
  %\renewcommand{\abstracttextfont}{\normalfont\small\itshape} % Set the abstract itself to small italic text\[IndentingNewLine]
  \usepackage{titlesec} % Allows customization of titles
  \renewcommand\thesection{\Roman{section}} % Roman numerals for the sections
  \renewcommand\thesubsection{\Roman{subsection}} % Roman numerals for subsections
  \titleformat{\section}[block]{\large\scshape\centering}{\thesection.}{1em}{} % Change the look of the section titles
  \titleformat{\subsection}[block]{\large}{\thesubsection.}{1em}{} % Change the look of the section titles
  \usepackage{fancyhdr} % Headers and footers
  \pagestyle{fancy} % All pages have headers and footers
  \fancyhead{} % Blank out the default header
  \fancyfoot{} % Blank out the default footer
  \fancyhead[C]{X-meeting $\bullet$ October 2019 $\bullet$ Campos do  Jord\~ao} % Custom header text
  \fancyfoot[RO,LE]{} % Custom footer text
  %----------------------------------------------------------------------------------------
  % TITLE SECTION
  %---------------------------------------------------------------------------------------- 
 
 \title{\vspace{-15mm}\fontsize{24pt}{10pt}\selectfont\textbf{ Circulating miRNAs can affect the melanoma microenvironment and outcome }} % Article title
  
  
  \author{ J\'essica Gon\c{c}alves Vieira da Cruz, Marco Ant\^onio Marques Pretti, Natasha Jorge, Mart\'{\i}n Hern\'an Bonamino, Patricia Abr\~ao Possik, Mariana Boroni }
  
  \affil{ Laboratory of Functional Genomics and Bioinformatics,  Oswaldo Cruz Institute (IOC),  Oswaldo Cruz Foundation (Fiocruz),  Rio de Janeiro,  RJ,  Brazil. }
  \vspace{-5mm}
  \date{}
  
  %---------------------------------------------------------------------------------------- 
  
  \begin{document}
  
  
  \maketitle % Insert title
  
  
  \thispagestyle{fancy} % All pages have headers and footers
  %----------------------------------------------------------------------------------------  
  % ABSTRACT
  
  %----------------------------------------------------------------------------------------  
  
  \begin{abstract}
  Metastatic melanoma is an aggressive and deadly disease,  with high capacity for metastasis and resistance to treatment,  resulting in high number of patients dying within 5 years of diagnosis. Various tumors,  including melanoma,  can interact with their microenvironment and modulate it to enable disease progression,  presenting immune evasion characteristics and facilitating metastatic behaviour. To delve the impact of the crosstalk between melanoma cells and tumor microenvironment (TME) on patient’s outcome,  we accessed RNA-Seq data from 164 metastatic melanoma samples from The Cancer Genome Atlas to characterize their TME using the CIBERSORT. Next,  samples were separated into 3 groups by unsupervised hierarchical clustering analysis based on their TME profiles (Jaccard bootstrap mean: G1 = 0.56,  G2 = 0.75,  G3 = 0.84). The TME profile in each group was distinct,  with G1 enriched in na\"{\i}ve,  memory and plasma B cells and depleted in resting natural killer (NK) cells,  G2 enriched in T CD8 cells,  monocytes and M1 macrophages and G3 enriched in M0 macrophages and depleted in plasma cells,  T CD8 cells,  memory activated T CD4 cells,  follicular T helper cells,  activated NK cells,  monocytes,  and resting dendritic cells (p = 0.05,  Mann-Whitney test - MW). Overall survival of the groups was compared and G2 patients presented a significantly better prognosis than G3 (p = 0.01,  log-rank test,  Hazard Ratio (HR) = 0.49,  CI.95 = 0.28 ~ 0.85). To better understand the interplay between tumor and its TME,  we investigated  putative interactions between differentially expressed miRNA (miR) and target genes (mRNAs) in G3 compared to G2. We selected miR-targets pairs (MTP) that were predicted in at least one of the databases available in the multimiR package and that were highly negatively correlated to each other(r = -0.4 and p = 0.5),  ending up with a list of 139 MTP. We use igraph to represent the network of MTPs,  including additional information of gene expression levels,  impact on survival,  and possible origins of the miRNA. We found interactions that suggested inter- and intracellular regulations,  with tumor modulating gene expression on microenvironment cells and vice-versa. One interesting example is the MTP mir-149/NLRC5. The mir-149 is upregulated in G3 and does not impact patients’ survival. However,  downregulation of its target,  NLRC5,  has a negative impact on patients overall survival (p = 0.0047,  log-rank test,  HR = 0.46,  CI.95 = 0.27 ~ 0.8). NLRC5 is a transcription coactivator that regulates the expression of genes belonging to the antigen presentation pathway. We found many of its targets such as HLA-C,  TAP1 and B2M also downregulated in G3,  with significant impact on overall survival. Moreover,  mutations in the antigen processing and presenting pathway in G3 were associated with increased number of neoepitopes - new and potentially immunogenic peptides generated by mutations (p = 0.05,  MW). This can help explain why the immune infiltrate in G3 is so poor on effector cells. Our results point to a crosstalk between melanoma and the TME that can impact the cell types present within the microenvironment and the capacity of the tumor to evade immune surveillance,  favouring metastasis and a worse patient’s outcome. This knowledge can be used in the future for melanoma and treatment assessment by looking into circulating molecules that can inform on TME constitution.
  
  Funding:  \\ 
  \end{abstract}
  \end{document} 