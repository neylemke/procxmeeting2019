
  \documentclass[twoside]{article}
  \usepackage[affil-it]{authblk}
  \usepackage{lipsum} % Package to generate dummy text throughout this template
  \usepackage{eurosym}
  \usepackage[sc]{mathpazo} % Use the Palatino font
  \usepackage[T1]{fontenc} % Use 8-bit encoding that has 256 glyphs
  \usepackage[utf8]{inputenc}
  \linespread{1.05} % Line spacing-Palatino needs more space between lines
  \usepackage{microtype} % Slightly tweak font spacing for aesthetics\[IndentingNewLine]
  \usepackage[hmarginratio=1:1,top=32mm,columnsep=20pt]{geometry} % Document margins
  \usepackage{multicol} % Used for the two-column layout of the document
  \usepackage[hang,small,labelfont=bf,up,textfont=it,up]{caption} % Custom captions under//above floats in tables or figures
  \usepackage{booktabs} % Horizontal rules in tables
  \usepackage{float} % Required for tables and figures in the multi-column environment-they need to be placed in specific locations with the[H] (e.g. \begin{table}[H])
  \usepackage{hyperref} % For hyperlinks in the PDF
  \usepackage{lettrine} % The lettrine is the first enlarged letter at the beginning of the text
  \usepackage{paralist} % Used for the compactitem environment which makes bullet points with less space between them
  \usepackage{abstract} % Allows abstract customization
  \renewcommand{\abstractnamefont}{\normalfont\bfseries} 
  %\renewcommand{\abstracttextfont}{\normalfont\small\itshape} % Set the abstract itself to small italic text\[IndentingNewLine]
  \usepackage{titlesec} % Allows customization of titles
  \renewcommand\thesection{\Roman{section}} % Roman numerals for the sections
  \renewcommand\thesubsection{\Roman{subsection}} % Roman numerals for subsections
  \titleformat{\section}[block]{\large\scshape\centering}{\thesection.}{1em}{} % Change the look of the section titles
  \titleformat{\subsection}[block]{\large}{\thesubsection.}{1em}{} % Change the look of the section titles
  \usepackage{fancyhdr} % Headers and footers
  \pagestyle{fancy} % All pages have headers and footers
  \fancyhead{} % Blank out the default header
  \fancyfoot{} % Blank out the default footer
  \fancyhead[C]{X-meeting $\bullet$ October 2019 $\bullet$ Campos do  Jord\~ao} % Custom header text
  \fancyfoot[RO,LE]{} % Custom footer text
  %----------------------------------------------------------------------------------------
  % TITLE SECTION
  %---------------------------------------------------------------------------------------- 
 
 \title{\vspace{-15mm}\fontsize{24pt}{10pt}\selectfont\textbf{ Microbiomes of Velloziaceae from phosphorus-impoverished soils of the campos rupestres,  a biodiversity hotspot }} % Article title
  
  
  \author{ Ant\^onio Pedro de Castello Branco da Rocha Camargo, Rafael Soares Correa de Souza, Paulo Arruda, Marcelo Falsarella Carazzolle }
  
  \affil{ Universidade Estadual de Campinas }
  \vspace{-5mm}
  \date{}
  
  %---------------------------------------------------------------------------------------- 
  
  \begin{document}
  
  
  \maketitle % Insert title
  
  
  \thispagestyle{fancy} % All pages have headers and footers
  %----------------------------------------------------------------------------------------  
  % ABSTRACT
  
  %----------------------------------------------------------------------------------------  
  
  \begin{abstract}
  The rocky,  seasonally-dry and nutrient-impoverished soils of the Brazilian campos rupestres impose severe growth-limiting conditions on plants. Species of a dominant plant family,  Velloziaceae,  are highly specialized to low-nutrient conditions and seasonal water availability of this environment,  where phosphorus (P) is the key limiting nutrient. Despite plant-microbe associations playing critical roles in stressful ecosystems,  the contribution of these interactions in the campos rupestres remains poorly studied. We generated and investigated the first microbiome sequencing data of Velloziaceae spp. thriving in contrasting substrates of campos rupestres.
We assessed the microbiomes of Vellozia epidendroides,  which occupies shallow patches of soil,  and Barbacenia macrantha,  growing on exposed rocks. The prokaryotic and fungal profiles were assessed by rRNA barcode sequencing (16S V4 and ITS2) of epiphytic and endophytic compartments of roots,  stems,  leaves and surrounding soil/rocks. Through this data,  we found that there is a large quantity of as-yet-unknown microorganisms thriving in the campos rupestres environment. When contrasting the microbiomes of the two plants,  we observed major differences regarding the community composition,  diversity and colonization profiles. Interestingly,  we also noticed that there are several highly abundant microorganisms that associate with both V. epidendroides and B. Macrantha,  suggesting a shared core microbiome in this environment.
Shotgun sequencing of total DNA extracted from microbial samples of rhizosphere and substrate was performed to investigate the functional landscape of the campos rupestres microbiomes. The samples were individually assembled and annotated,  generating a median number of 9, 907 noncoding genes and 2, 544, 611 protein-coding genes. The comparison between metabolic profiles of communities associated with the substrates and the rhizospheres of the two plants revealed major functional differences between the two microbiomes.
We foresee that these data will contribute to decipher how the microbiome contributes to plant functioning in the campos rupestres,  and to unravel new strategies for improved crop productivity in stressful environments.
  
  Funding: FAPESP \\ 
  \end{abstract}
  \end{document} 