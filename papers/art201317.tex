
  \documentclass[twoside]{article}
  \usepackage[affil-it]{authblk}
  \usepackage{lipsum} % Package to generate dummy text throughout this template
  \usepackage{eurosym}
  \usepackage[sc]{mathpazo} % Use the Palatino font
  \usepackage[T1]{fontenc} % Use 8-bit encoding that has 256 glyphs
  \usepackage[utf8]{inputenc}
  \linespread{1.05} % Line spacing-Palatino needs more space between lines
  \usepackage{microtype} % Slightly tweak font spacing for aesthetics\[IndentingNewLine]
  \usepackage[hmarginratio=1:1,top=32mm,columnsep=20pt]{geometry} % Document margins
  \usepackage{multicol} % Used for the two-column layout of the document
  \usepackage[hang,small,labelfont=bf,up,textfont=it,up]{caption} % Custom captions under//above floats in tables or figures
  \usepackage{booktabs} % Horizontal rules in tables
  \usepackage{float} % Required for tables and figures in the multi-column environment-they need to be placed in specific locations with the[H] (e.g. \begin{table}[H])
  \usepackage{hyperref} % For hyperlinks in the PDF
  \usepackage{lettrine} % The lettrine is the first enlarged letter at the beginning of the text
  \usepackage{paralist} % Used for the compactitem environment which makes bullet points with less space between them
  \usepackage{abstract} % Allows abstract customization
  \renewcommand{\abstractnamefont}{\normalfont\bfseries} 
  %\renewcommand{\abstracttextfont}{\normalfont\small\itshape} % Set the abstract itself to small italic text\[IndentingNewLine]
  \usepackage{titlesec} % Allows customization of titles
  \renewcommand\thesection{\Roman{section}} % Roman numerals for the sections
  \renewcommand\thesubsection{\Roman{subsection}} % Roman numerals for subsections
  \titleformat{\section}[block]{\large\scshape\centering}{\thesection.}{1em}{} % Change the look of the section titles
  \titleformat{\subsection}[block]{\large}{\thesubsection.}{1em}{} % Change the look of the section titles
  \usepackage{fancyhdr} % Headers and footers
  \pagestyle{fancy} % All pages have headers and footers
  \fancyhead{} % Blank out the default header
  \fancyfoot{} % Blank out the default footer
  \fancyhead[C]{X-meeting $\bullet$ October 2019 $\bullet$ Campos do  Jord\~ao} % Custom header text
  \fancyfoot[RO,LE]{} % Custom footer text
  %----------------------------------------------------------------------------------------
  % TITLE SECTION
  %---------------------------------------------------------------------------------------- 
 
 \title{\vspace{-15mm}\fontsize{24pt}{10pt}\selectfont\textbf{ Prediction,  identification and characterization of genomic islands in Aeromonas spp. }} % Article title
  
  
  \author{ Antonio Camilo da Silva Filho, Camilla Reginatto De Pierri, Diogo de Jesus Soares Machado, Roberto Tadeu Raittz, Jeroniza Nunes  Marchaukoski, Cynthia Maria Telles Fadel-Picheth, Geraldo Picheth }
  
  \affil{ Federal University of Paran\'a }
  \vspace{-5mm}
  \date{}
  
  %---------------------------------------------------------------------------------------- 
  
  \begin{document}
  
  
  \maketitle % Insert title
  
  
  \thispagestyle{fancy} % All pages have headers and footers
  %----------------------------------------------------------------------------------------  
  % ABSTRACT
  
  %----------------------------------------------------------------------------------------  
  
  \begin{abstract}
  Aeromonas are pathogenic bacteria,  mostly aquatic and with wide environmental distribution. These bacteria can cause infections in humans,  varying in severity,  for example in cases of uncomplicated acute gastroenteritis,  where the disease does not pose serious health risks,  and septicemia,  which can be fatal. Virulence is the ability of a microorganism to cause disease in a host,  and several features of the Aeromonas genome have been proven to be associated with virulence. Even though virulence characteristics are similar between species,  there are not many studies that discuss the behavior of these genes in the Aeromonas genome in general. However,  it is already known that one of the main mechanisms that bacteria use to share genetic material is the genomic islands (GIs). In this perspective,  to understand and identify the main genes related to pathogenic potential and virulence mechanisms in Aeromonas strains,  this study aims to characterize and compare the gene content and respective function of Gls in Aeromonas,  as well as to analyze their GIs distribution between different species of this bacteria. For this,  complete genomes of 58 Aeromonas were obtained from the NCBI Database. GI prediction was performed using IslandViewer4 software,  which combines analyzes by comparative genomics and sequence composition; the two most used methodologies to identify these regions. To determine GI functions,  we used the ARDB,  CARD,  NDARO (antibiotic resistance),  Patric,  VFDB,  VICTORS (Virulence),  DrugBank,  TTD (drug targets) databases. Analysis of GI distribution among species was performed using the CD-HIT-2D cluster with a 70\% self-score similarity. To evaluate the content of complete genomes in all Aeromonas studied,  a phylogenetic tree was constructed using SWeeP method. The results showed that it is possible to determine the genetic diversity of these organisms and to characterize GIs according to their related functions and products using clustering and phylogeny. It was possible to identify the relationships between the origin of GIs (clinical / environmental / animal) in the total set of Aeromonas. Phylogenetic analysis complemented cluster analysis by showing that bacteria are misclassified,  such as Aeromonas hydrophila YL17 and Aeromonas hydrophila 4AK4 strains that are not hydrophila species.
  
  Funding: CAPES \\ 
  \end{abstract}
  \end{document} 