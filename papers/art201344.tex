
  \documentclass[twoside]{article}
  \usepackage[affil-it]{authblk}
  \usepackage{lipsum} % Package to generate dummy text throughout this template
  \usepackage{eurosym}
  \usepackage[sc]{mathpazo} % Use the Palatino font
  \usepackage[T1]{fontenc} % Use 8-bit encoding that has 256 glyphs
  \usepackage[utf8]{inputenc}
  \linespread{1.05} % Line spacing-Palatino needs more space between lines
  \usepackage{microtype} % Slightly tweak font spacing for aesthetics\[IndentingNewLine]
  \usepackage[hmarginratio=1:1,top=32mm,columnsep=20pt]{geometry} % Document margins
  \usepackage{multicol} % Used for the two-column layout of the document
  \usepackage[hang,small,labelfont=bf,up,textfont=it,up]{caption} % Custom captions under//above floats in tables or figures
  \usepackage{booktabs} % Horizontal rules in tables
  \usepackage{float} % Required for tables and figures in the multi-column environment-they need to be placed in specific locations with the[H] (e.g. \begin{table}[H])
  \usepackage{hyperref} % For hyperlinks in the PDF
  \usepackage{lettrine} % The lettrine is the first enlarged letter at the beginning of the text
  \usepackage{paralist} % Used for the compactitem environment which makes bullet points with less space between them
  \usepackage{abstract} % Allows abstract customization
  \renewcommand{\abstractnamefont}{\normalfont\bfseries} 
  %\renewcommand{\abstracttextfont}{\normalfont\small\itshape} % Set the abstract itself to small italic text\[IndentingNewLine]
  \usepackage{titlesec} % Allows customization of titles
  \renewcommand\thesection{\Roman{section}} % Roman numerals for the sections
  \renewcommand\thesubsection{\Roman{subsection}} % Roman numerals for subsections
  \titleformat{\section}[block]{\large\scshape\centering}{\thesection.}{1em}{} % Change the look of the section titles
  \titleformat{\subsection}[block]{\large}{\thesubsection.}{1em}{} % Change the look of the section titles
  \usepackage{fancyhdr} % Headers and footers
  \pagestyle{fancy} % All pages have headers and footers
  \fancyhead{} % Blank out the default header
  \fancyfoot{} % Blank out the default footer
  \fancyhead[C]{X-meeting $\bullet$ October 2019 $\bullet$ Campos do  Jord\~ao} % Custom header text
  \fancyfoot[RO,LE]{} % Custom footer text
  %----------------------------------------------------------------------------------------
  % TITLE SECTION
  %---------------------------------------------------------------------------------------- 
 
 \title{\vspace{-15mm}\fontsize{24pt}{10pt}\selectfont\textbf{ SEARCH AND CHARACTERIZATION OF NON-CODIFYING RNAs IN ISOLATES OF Bacillus thuringiensis (Bacillus cereus sensu lato) BY RNA SEQUENCING }} % Article title
  
  
  \author{ Viviane Aparecida Gobetti, freddy Eddinson Ninaja Zegarra, Laurival Ant\^onio Vilas Boas }
  
  \affil{ Universidade Tecnol\'ogica Federal do Paran\'a }
  \vspace{-5mm}
  \date{}
  
  %---------------------------------------------------------------------------------------- 
  
  \begin{document}
  
  
  \maketitle % Insert title
  
  
  \thispagestyle{fancy} % All pages have headers and footers
  %----------------------------------------------------------------------------------------  
  % ABSTRACT
  
  %----------------------------------------------------------------------------------------  
  
  \begin{abstract}
  Evidence shows a close similarity in the genomes of Bacillus thuringiensis,  Bacillus cereus and Bacillus anthracis bacteria belonging to the Bacillus cereus Sensu Lato group,  suggesting that they should be considered to be of the same species. Thus,  this research work aims to identify and characterize non-coding RNAs in Bacillus thuringiensis (Bt) strain 407,  assisting in the taxonomic understanding and understanding of virulence factors through the proposed methodology,  which adds search for homologous sequences in Rfam database. The adopted methodology used the complete genome of the Bt 407 strain of B. thuringiensis along with the total sequencing of Bt enriched with small RNAs extracted in two growth phases of the bacteria (end of log growth phase and beginning of stationary phase respectively). Subsequently,  genome sequences were extracted in highly expressed regions,  where possibly there would be a greater likelihood of finding candidates for ncRNAs. The extracted sequences were used together with the cmsearch tool of the INFERNAL package,  using the maximum inclusion limits of 0.01 e-value for candidate classification. At the end of the research,  a total of 894 candidates were identified,  with 355 significant candidates (according to inclusion limits) and 49 ncRNA families. The work allowed the search and identification of ncRNAs in the Bt 407 strain of B. thuringiensis. ,  demonstrating different composition profiles of ncRNAs,  with variations of candidates and sequences in both phases of growth of the bacteria,  contributing to a better understanding of the life cycle of this bacterium,  as well as the ncRNAs responsible for certain characteristics presented by this species.
  
  Funding:  \\ 
  \end{abstract}
  \end{document} 