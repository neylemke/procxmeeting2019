
  \documentclass[twoside]{article}
  \usepackage[affil-it]{authblk}
  \usepackage{lipsum} % Package to generate dummy text throughout this template
  \usepackage{eurosym}
  \usepackage[sc]{mathpazo} % Use the Palatino font
  \usepackage[T1]{fontenc} % Use 8-bit encoding that has 256 glyphs
  \usepackage[utf8]{inputenc}
  \linespread{1.05} % Line spacing-Palatino needs more space between lines
  \usepackage{microtype} % Slightly tweak font spacing for aesthetics\[IndentingNewLine]
  \usepackage[hmarginratio=1:1,top=32mm,columnsep=20pt]{geometry} % Document margins
  \usepackage{multicol} % Used for the two-column layout of the document
  \usepackage[hang,small,labelfont=bf,up,textfont=it,up]{caption} % Custom captions under//above floats in tables or figures
  \usepackage{booktabs} % Horizontal rules in tables
  \usepackage{float} % Required for tables and figures in the multi-column environment-they need to be placed in specific locations with the[H] (e.g. \begin{table}[H])
  \usepackage{hyperref} % For hyperlinks in the PDF
  \usepackage{lettrine} % The lettrine is the first enlarged letter at the beginning of the text
  \usepackage{paralist} % Used for the compactitem environment which makes bullet points with less space between them
  \usepackage{abstract} % Allows abstract customization
  \renewcommand{\abstractnamefont}{\normalfont\bfseries} 
  %\renewcommand{\abstracttextfont}{\normalfont\small\itshape} % Set the abstract itself to small italic text\[IndentingNewLine]
  \usepackage{titlesec} % Allows customization of titles
  \renewcommand\thesection{\Roman{section}} % Roman numerals for the sections
  \renewcommand\thesubsection{\Roman{subsection}} % Roman numerals for subsections
  \titleformat{\section}[block]{\large\scshape\centering}{\thesection.}{1em}{} % Change the look of the section titles
  \titleformat{\subsection}[block]{\large}{\thesubsection.}{1em}{} % Change the look of the section titles
  \usepackage{fancyhdr} % Headers and footers
  \pagestyle{fancy} % All pages have headers and footers
  \fancyhead{} % Blank out the default header
  \fancyfoot{} % Blank out the default footer
  \fancyhead[C]{X-meeting $\bullet$ October 2019 $\bullet$ Campos do  Jord\~ao} % Custom header text
  \fancyfoot[RO,LE]{} % Custom footer text
  %----------------------------------------------------------------------------------------
  % TITLE SECTION
  %---------------------------------------------------------------------------------------- 
 
 \title{\vspace{-15mm}\fontsize{24pt}{10pt}\selectfont\textbf{ Comparative genomics of the type four secretion system pumping ATPases VirD4/VirB4 from the FtsK–HerA superfamily reveals a new clade in the Candidate Phyla Radiation. }} % Article title
  
  
  \author{ Gianlucca Gon\c{c}alves Nicastro, Robson Francisco de Souza }
  
  \affil{  }
  \vspace{-5mm}
  \date{}
  
  %---------------------------------------------------------------------------------------- 
  
  \begin{document}
  
  
  \maketitle % Insert title
  
  
  \thispagestyle{fancy} % All pages have headers and footers
  %----------------------------------------------------------------------------------------  
  % ABSTRACT
  
  %----------------------------------------------------------------------------------------  
  
  \begin{abstract}
  Bacteria are unicellular organisms that participate in a diversity of biological interactions in the environment. One of the most important of these interactions is the transfer of genetic material between different organism. This mechanism is known as horizontal gene transfer and plays an important role in bacterial evolution and the emergence of antibiotic resistance. The type IV secretion systems (T4SS) are large multi-protein structures involved in the single-stranded DNA transfer. The components of this system can be divided into three main parts: the pump ATPases,  the elements that form the translocation channel and the proteins that form the pilus. T4SS genes are widely spread among prokaryotes,  including Archaea. Previous phylogenetic studies of the T4SS VirB4/D4 ATPases demonstrated a robust separation in eight large clades,  that in part reflects the structure of the cell envelope and were used as the basis of a phylogenetic classification of T4SS. Despite its broad covering,  such studies lack information about the most recent sequenced genomes,  as the DPANN archaea and the bacterial candidate phyla radiation (CPR). In this study,  we sought to expand the T4SS classification based on the VirD4/B4 phylogeny. In order to reliably identify all VirD4/B4 homologs,  we performed iterative sequence similarity searches against NCBI’s non-redundant database using representatives of the most closely related families within the FtsK–HerA superfamily,  including members of the VirD4,  VirB4,  HerA and FtsK families. After removal of redundant sequences,  careful phylogenetic analysis revealed that the HerA,  VirD4 and VirB4 proteins form well defined monophyletic groups. The VirD4 and VirB4 subtrees were found to contain clades similar to those identified in previous works,  but revealed the presence of a new group formed mostly by members of CPR. We also observed the presence of a new clade of VirD4 genes of Actinobacteria. Our classification thus reveals a new type of T4SS from CPR bacteria,  that could be related to a singular cell envelope structure within this clade.
  
  Funding: Fapesp,  Capes \\ 
  \end{abstract}
  \end{document} 