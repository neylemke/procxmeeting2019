
  \documentclass[twoside]{article}
  \usepackage[affil-it]{authblk}
  \usepackage{lipsum} % Package to generate dummy text throughout this template
  \usepackage{eurosym}
  \usepackage[sc]{mathpazo} % Use the Palatino font
  \usepackage[T1]{fontenc} % Use 8-bit encoding that has 256 glyphs
  \usepackage[utf8]{inputenc}
  \linespread{1.05} % Line spacing-Palatino needs more space between lines
  \usepackage{microtype} % Slightly tweak font spacing for aesthetics\[IndentingNewLine]
  \usepackage[hmarginratio=1:1,top=32mm,columnsep=20pt]{geometry} % Document margins
  \usepackage{multicol} % Used for the two-column layout of the document
  \usepackage[hang,small,labelfont=bf,up,textfont=it,up]{caption} % Custom captions under//above floats in tables or figures
  \usepackage{booktabs} % Horizontal rules in tables
  \usepackage{float} % Required for tables and figures in the multi-column environment-they need to be placed in specific locations with the[H] (e.g. \begin{table}[H])
  \usepackage{hyperref} % For hyperlinks in the PDF
  \usepackage{lettrine} % The lettrine is the first enlarged letter at the beginning of the text
  \usepackage{paralist} % Used for the compactitem environment which makes bullet points with less space between them
  \usepackage{abstract} % Allows abstract customization
  \renewcommand{\abstractnamefont}{\normalfont\bfseries} 
  %\renewcommand{\abstracttextfont}{\normalfont\small\itshape} % Set the abstract itself to small italic text\[IndentingNewLine]
  \usepackage{titlesec} % Allows customization of titles
  \renewcommand\thesection{\Roman{section}} % Roman numerals for the sections
  \renewcommand\thesubsection{\Roman{subsection}} % Roman numerals for subsections
  \titleformat{\section}[block]{\large\scshape\centering}{\thesection.}{1em}{} % Change the look of the section titles
  \titleformat{\subsection}[block]{\large}{\thesubsection.}{1em}{} % Change the look of the section titles
  \usepackage{fancyhdr} % Headers and footers
  \pagestyle{fancy} % All pages have headers and footers
  \fancyhead{} % Blank out the default header
  \fancyfoot{} % Blank out the default footer
  \fancyhead[C]{X-meeting $\bullet$ October 2019 $\bullet$ Campos do  Jord\~ao} % Custom header text
  \fancyfoot[RO,LE]{} % Custom footer text
  %----------------------------------------------------------------------------------------
  % TITLE SECTION
  %---------------------------------------------------------------------------------------- 
 
 \title{\vspace{-15mm}\fontsize{24pt}{10pt}\selectfont\textbf{ NLR genes in aquatic mammals and where to find them }} % Article title
  
  
  \author{ Maria Luiza Andreani, Mariana Freitas Nery }
  
  \affil{ Unicamp }
  \vspace{-5mm}
  \date{}
  
  %---------------------------------------------------------------------------------------- 
  
  \begin{document}
  
  
  \maketitle % Insert title
  
  
  \thispagestyle{fancy} % All pages have headers and footers
  %----------------------------------------------------------------------------------------  
  % ABSTRACT
  
  %----------------------------------------------------------------------------------------  
  
  \begin{abstract}
  The immune system relies on receptors to achieve a proper immune response. Pattern Recognition Receptors (PRRs) are the main receptors in the innate immune response. PRRs allow the detection of conserved pathogen-associated molecular patterns (PAMPs) and damage-associated molecular patterns (DAMPs). As a survival strategy,  pathogens may develop mechanisms to avoid detection by PRRs and,  therefore,  there is room for pathogen mediated selection. Changes on ecological environment may lead to new challenges to immunity system,  as composition of pathogens may vary. Accordingly,  here we analyzed an innate immune family of receptors in our own sequenced genomes of non-model species of aquatic mammals,  comparing recently diverged species of fluvial and marine mammals in cetaceans and sirenians. We focused on the NLR (Nod-Like Receptors) family of immune genes formed by a NACTH domain,  a C-Terminal LRR region and a CARD or PYD domain in the N-Terminal region. They are intracellular receptors that can detect signs of viruses,  bacteria and cellular damage. From this family we characterized the NOD1,  NOD2,  NLRC4 and NLRP3 genes in Sotalia guianensis,  Sotalia fluviatilis,  Trichechus manatus and Trichechus inunguis. We blasted the corresponding Tursiops truncatus genes on our sequenced genomes and investigated 10, 000 bp upstream and downstream searching for pseudogenes or duplicates as this family of immune receptors is known to vary on copy number along lineages. Moreover,  NLR receptors NLRP2,  NLRP4,  NLRP5,  NLRP7-9 and NLRP11-13 are located in the same chromosome in humans,  therefore we identified the flanking genes of the most upstream and downstream receptors in ungulates species and blasted them in the sequenced genomes,  extracting the region in between. Further,  we compared our findings to genes present on terrestrial mammals. Characterization of innate immune genes on non-model species may lead to a better understanding of molecular evolutionary dynamics of host-pathogen interactions and how they evolve in the wild.
  
  Funding: CAPES,  Instituto de Biologia \\ 
  \end{abstract}
  \end{document} 