
  \documentclass[twoside]{article}
  \usepackage[affil-it]{authblk}
  \usepackage{lipsum} % Package to generate dummy text throughout this template
  \usepackage{eurosym}
  \usepackage[sc]{mathpazo} % Use the Palatino font
  \usepackage[T1]{fontenc} % Use 8-bit encoding that has 256 glyphs
  \usepackage[utf8]{inputenc}
  \linespread{1.05} % Line spacing-Palatino needs more space between lines
  \usepackage{microtype} % Slightly tweak font spacing for aesthetics\[IndentingNewLine]
  \usepackage[hmarginratio=1:1,top=32mm,columnsep=20pt]{geometry} % Document margins
  \usepackage{multicol} % Used for the two-column layout of the document
  \usepackage[hang,small,labelfont=bf,up,textfont=it,up]{caption} % Custom captions under//above floats in tables or figures
  \usepackage{booktabs} % Horizontal rules in tables
  \usepackage{float} % Required for tables and figures in the multi-column environment-they need to be placed in specific locations with the[H] (e.g. \begin{table}[H])
  \usepackage{hyperref} % For hyperlinks in the PDF
  \usepackage{lettrine} % The lettrine is the first enlarged letter at the beginning of the text
  \usepackage{paralist} % Used for the compactitem environment which makes bullet points with less space between them
  \usepackage{abstract} % Allows abstract customization
  \renewcommand{\abstractnamefont}{\normalfont\bfseries} 
  %\renewcommand{\abstracttextfont}{\normalfont\small\itshape} % Set the abstract itself to small italic text\[IndentingNewLine]
  \usepackage{titlesec} % Allows customization of titles
  \renewcommand\thesection{\Roman{section}} % Roman numerals for the sections
  \renewcommand\thesubsection{\Roman{subsection}} % Roman numerals for subsections
  \titleformat{\section}[block]{\large\scshape\centering}{\thesection.}{1em}{} % Change the look of the section titles
  \titleformat{\subsection}[block]{\large}{\thesubsection.}{1em}{} % Change the look of the section titles
  \usepackage{fancyhdr} % Headers and footers
  \pagestyle{fancy} % All pages have headers and footers
  \fancyhead{} % Blank out the default header
  \fancyfoot{} % Blank out the default footer
  \fancyhead[C]{X-meeting $\bullet$ October 2019 $\bullet$ Campos do  Jord\~ao} % Custom header text
  \fancyfoot[RO,LE]{} % Custom footer text
  %----------------------------------------------------------------------------------------
  % TITLE SECTION
  %---------------------------------------------------------------------------------------- 
 
 \title{\vspace{-15mm}\fontsize{24pt}{10pt}\selectfont\textbf{ A bioinformatics approach to cancer vaccines prioritization based cancer-testis antigens in melanoma }} % Article title
  
  
  \author{ Andre Fonseca, ANA CAROLINA MIRANDA FERNANDES CO\^ELHO, Sandro Jose de Souza }
  
  \affil{ USP }
  \vspace{-5mm}
  \date{}
  
  %---------------------------------------------------------------------------------------- 
  
  \begin{document}
  
  
  \maketitle % Insert title
  
  
  \thispagestyle{fancy} % All pages have headers and footers
  %----------------------------------------------------------------------------------------  
  % ABSTRACT
  
  %----------------------------------------------------------------------------------------  
  
  \begin{abstract}
  Peptide-based vaccines are a promising approach to cancer immunotherapy. Nowadays,  several clinical trials using peptide-based vaccines have been carried in distinct tumor types,  such as colorectal,  lung cancer,  and melanoma. Traditionally,  these vaccines are synthesized based on tumor-associated antigens,  such as cancer-testis antigens (CTAs). The CTAs are a large family of tumor-associated antigens with expression restricted in normal tissues (testis) and expressed in a broad number of tumor types. In addition,  CTAs have been pointing out as important candidates due to low toxicity and high specificity. However,  several attempts to used CTAs as vaccines were considered a failure,  most likely due to a wrong candidate choice and/or absence of techniques to analyze the clinical risks. Consequently,  it remains as an opened challenge,  with two main bottlenecks: i) properly prioritization of novel candidates,  and then ii) how to combine these candidates effectively. Here,  we present a multiple CTA vaccine to approach melanoma patients. First,  we analyzed a large collection of CTAs in the TCGA cohort,  in which was filtered candidates based on individual contribution to good prognosis and high T cell CD8+ correlation. As a result,  were selected seven candidates,  including INSL3,  HSF5,  GTSF1L,  FAM209A,  GPR31,  SIRPD,  and HEATR9. Next,  we clustered patients into distinct groups based on the number of co-expressed CTAs. In a total,  four groups were obtained,  named as Red (>5 co-expressed CTAs),   Yellow (>2 to 4),  Blue (>1 to 2),  Grey (without expressed CTAs). Interestingly,  the survival ratio has a significant improvement based on the number of co-expressed CTAs. Furthermore,  it seems to be independent of other immune response drivers,  such as mutation load. Finally,  we evaluate the antigenic regions (epitopes) for each CTA protein. In short,  the epitopes predictions were carried out using netMHC algorithms,  considering a subset of high frequent HLA alleles based on population data. As a result,  40 and 27 high antigenicity regions were found to MHC I and II,  respectively. Interestingly,  a few peptides are associated with both CD4 and CD8 responses. In conclusion,  these findings can be an interesting resource for a cancer vaccine protocol for melanoma.
  
  Funding:  \\ 
  \end{abstract}
  \end{document} 