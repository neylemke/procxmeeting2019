
  \documentclass[twoside]{article}
  \usepackage[affil-it]{authblk}
  \usepackage{lipsum} % Package to generate dummy text throughout this template
  \usepackage{eurosym}
  \usepackage[sc]{mathpazo} % Use the Palatino font
  \usepackage[T1]{fontenc} % Use 8-bit encoding that has 256 glyphs
  \usepackage[utf8]{inputenc}
  \linespread{1.05} % Line spacing-Palatino needs more space between lines
  \usepackage{microtype} % Slightly tweak font spacing for aesthetics\[IndentingNewLine]
  \usepackage[hmarginratio=1:1,top=32mm,columnsep=20pt]{geometry} % Document margins
  \usepackage{multicol} % Used for the two-column layout of the document
  \usepackage[hang,small,labelfont=bf,up,textfont=it,up]{caption} % Custom captions under//above floats in tables or figures
  \usepackage{booktabs} % Horizontal rules in tables
  \usepackage{float} % Required for tables and figures in the multi-column environment-they need to be placed in specific locations with the[H] (e.g. \begin{table}[H])
  \usepackage{hyperref} % For hyperlinks in the PDF
  \usepackage{lettrine} % The lettrine is the first enlarged letter at the beginning of the text
  \usepackage{paralist} % Used for the compactitem environment which makes bullet points with less space between them
  \usepackage{abstract} % Allows abstract customization
  \renewcommand{\abstractnamefont}{\normalfont\bfseries} 
  %\renewcommand{\abstracttextfont}{\normalfont\small\itshape} % Set the abstract itself to small italic text\[IndentingNewLine]
  \usepackage{titlesec} % Allows customization of titles
  \renewcommand\thesection{\Roman{section}} % Roman numerals for the sections
  \renewcommand\thesubsection{\Roman{subsection}} % Roman numerals for subsections
  \titleformat{\section}[block]{\large\scshape\centering}{\thesection.}{1em}{} % Change the look of the section titles
  \titleformat{\subsection}[block]{\large}{\thesubsection.}{1em}{} % Change the look of the section titles
  \usepackage{fancyhdr} % Headers and footers
  \pagestyle{fancy} % All pages have headers and footers
  \fancyhead{} % Blank out the default header
  \fancyfoot{} % Blank out the default footer
  \fancyhead[C]{X-meeting $\bullet$ October 2019 $\bullet$ Campos do  Jord\~ao} % Custom header text
  \fancyfoot[RO,LE]{} % Custom footer text
  %----------------------------------------------------------------------------------------
  % TITLE SECTION
  %---------------------------------------------------------------------------------------- 
 
 \title{\vspace{-15mm}\fontsize{24pt}{10pt}\selectfont\textbf{ Diversity study of small open reading frames (sORFs) of healthy and in Alzheimer's Disease brain }} % Article title
  
  
  \author{ Saloe Bispo, Fabio Passetti }
  
  \affil{ Laboratory of Functional Genomics and Bioinformatics,  Oswaldo Cruz Institute (IOC),  Oswaldo Cruz Foundation (Fiocruz),  Rio de Janeiro,  RJ,  Brazil. Laboratory of Gene Expression Regulation,  Carlos Chagas Institute (ICC),  Oswaldo Cruz Foundation (Fiocruz),  Curitiba,  PR,  Brazil. }
  \vspace{-5mm}
  \date{}
  
  %---------------------------------------------------------------------------------------- 
  
  \begin{document}
  
  
  \maketitle % Insert title
  
  
  \thispagestyle{fancy} % All pages have headers and footers
  %----------------------------------------------------------------------------------------  
  % ABSTRACT
  
  %----------------------------------------------------------------------------------------  
  
  \begin{abstract}
  The aging of the world population is associated with the increased frequency of people diagnosed with dementias. These are responsible for the greatest burden of neurodegenerative diseases,  with Alzheimer’s representing approximately 60-70\% of dementia cases. Computational approaches that integrate genome,  transcriptome and proteome data have been developed by our research group to study human transcripts and their polypeptide products in an area known as proteogenomics. The discovery of small open reading frames (sORFs) in gene and protein databases called small ORF encoded polypeptides (sORF-encoded polypeptides) or SEPs has revealed a fundamental shortcoming in our knowledge of protein-coding genes. Some of these new sORFs have crucial biological roles in cells and organisms,  which motivates the search for new sORFs. In this study,  we are developing a proteogenomics approach for the identification of proteoforms in human and mouse focusing on SEPs. Thus,  predictions of SEPs will be incorporated into the human and murine proteoform repository SpliceProt maintained by our group. Currently,  we are using public shotgun proteomics data from healthy and AD affected brain samples,  searching for new SEPs expressed under such conditions. Preliminary data have shown the presence of previously undetected microproteins in proteomics data from AD,  derived from sORFs (lncRNAs and antisense).
  
  Funding: CNPq and ICC \\ 
  \end{abstract}
  \end{document} 