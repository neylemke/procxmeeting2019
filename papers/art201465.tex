
  \documentclass[twoside]{article}
  \usepackage[affil-it]{authblk}
  \usepackage{lipsum} % Package to generate dummy text throughout this template
  \usepackage{eurosym}
  \usepackage[sc]{mathpazo} % Use the Palatino font
  \usepackage[T1]{fontenc} % Use 8-bit encoding that has 256 glyphs
  \usepackage[utf8]{inputenc}
  \linespread{1.05} % Line spacing-Palatino needs more space between lines
  \usepackage{microtype} % Slightly tweak font spacing for aesthetics\[IndentingNewLine]
  \usepackage[hmarginratio=1:1,top=32mm,columnsep=20pt]{geometry} % Document margins
  \usepackage{multicol} % Used for the two-column layout of the document
  \usepackage[hang,small,labelfont=bf,up,textfont=it,up]{caption} % Custom captions under//above floats in tables or figures
  \usepackage{booktabs} % Horizontal rules in tables
  \usepackage{float} % Required for tables and figures in the multi-column environment-they need to be placed in specific locations with the[H] (e.g. \begin{table}[H])
  \usepackage{hyperref} % For hyperlinks in the PDF
  \usepackage{lettrine} % The lettrine is the first enlarged letter at the beginning of the text
  \usepackage{paralist} % Used for the compactitem environment which makes bullet points with less space between them
  \usepackage{abstract} % Allows abstract customization
  \renewcommand{\abstractnamefont}{\normalfont\bfseries} 
  %\renewcommand{\abstracttextfont}{\normalfont\small\itshape} % Set the abstract itself to small italic text\[IndentingNewLine]
  \usepackage{titlesec} % Allows customization of titles
  \renewcommand\thesection{\Roman{section}} % Roman numerals for the sections
  \renewcommand\thesubsection{\Roman{subsection}} % Roman numerals for subsections
  \titleformat{\section}[block]{\large\scshape\centering}{\thesection.}{1em}{} % Change the look of the section titles
  \titleformat{\subsection}[block]{\large}{\thesubsection.}{1em}{} % Change the look of the section titles
  \usepackage{fancyhdr} % Headers and footers
  \pagestyle{fancy} % All pages have headers and footers
  \fancyhead{} % Blank out the default header
  \fancyfoot{} % Blank out the default footer
  \fancyhead[C]{X-meeting $\bullet$ October 2019 $\bullet$ Campos do  Jord\~ao} % Custom header text
  \fancyfoot[RO,LE]{} % Custom footer text
  %----------------------------------------------------------------------------------------
  % TITLE SECTION
  %---------------------------------------------------------------------------------------- 
 
 \title{\vspace{-15mm}\fontsize{24pt}{10pt}\selectfont\textbf{ Impacts of retroelements in tumorigenesis }} % Article title
  
  
  \author{ Fernanda Orpinelli, Jos\'e Leonel Lemos Buzzo, Thiago Luiz Araujo Miller, Pedro A F Galante }
  
  \affil{ USP }
  \vspace{-5mm}
  \date{}
  
  %---------------------------------------------------------------------------------------- 
  
  \begin{document}
  
  
  \maketitle % Insert title
  
  
  \thispagestyle{fancy} % All pages have headers and footers
  %----------------------------------------------------------------------------------------  
  % ABSTRACT
  
  %----------------------------------------------------------------------------------------  
  
  \begin{abstract}
  Colorectal cancer (CRC) is the third most common cancer in the world,  with nearly 1 million new cases annually diagnosed. Of these,  about 50\% of the patients evolve to death mainly due to the development of metastatic (secondary) tumors originated from primary colorectal tumors. Thus,  studying genetic variations in secondary tumors is central to better understand tumor progression and to provide treatments that are more effective for CRC-affected patients. Here,  we are using whole genome sequencing (WGS) data - matched normal tissue,  primary and secondary tumor tissues from 5 CRC-affected patients - and bioinformatics methodologies to study genomic variation present CRC. Among genomic variations,  we are developing methods to identify and evaluate the functional role from those caused by LINE-1,  retrocopies of protein-coding genes and Alu elements,  these latter two unexplored genomic variations with great potential to be functional. Three of five patients (60\%) were stage IV when diagnosed. When comparing early-stage patients with late-stage in the progression of CRC,  we found that the number of (likely) novels LINE-1 insertions are (10x) higher in the late stages of the disease. However,  these numbers are not the same while looking for retrocopies of coding genes. Patients that were stage IV had on average 40 retrotransposition events,  while one patient stage II had 95 retrotransposition events from coding genes and the stage III patient had only three. Thereby,  our work aims to investigate two types of variations poorly studied,  mostly in secondary/metastatic tumors,  and contribute to an understanding of the frequency and importance of these variations in tumorigenesis and metastasis of a very relevant type of cancer,  colorectal cancer.
  
  Funding: FAPESP \\ 
  \end{abstract}
  \end{document} 