
  \documentclass[twoside]{article}
  \usepackage[affil-it]{authblk}
  \usepackage{lipsum} % Package to generate dummy text throughout this template
  \usepackage{eurosym}
  \usepackage[sc]{mathpazo} % Use the Palatino font
  \usepackage[T1]{fontenc} % Use 8-bit encoding that has 256 glyphs
  \usepackage[utf8]{inputenc}
  \linespread{1.05} % Line spacing-Palatino needs more space between lines
  \usepackage{microtype} % Slightly tweak font spacing for aesthetics\[IndentingNewLine]
  \usepackage[hmarginratio=1:1,top=32mm,columnsep=20pt]{geometry} % Document margins
  \usepackage{multicol} % Used for the two-column layout of the document
  \usepackage[hang,small,labelfont=bf,up,textfont=it,up]{caption} % Custom captions under//above floats in tables or figures
  \usepackage{booktabs} % Horizontal rules in tables
  \usepackage{float} % Required for tables and figures in the multi-column environment-they need to be placed in specific locations with the[H] (e.g. \begin{table}[H])
  \usepackage{hyperref} % For hyperlinks in the PDF
  \usepackage{lettrine} % The lettrine is the first enlarged letter at the beginning of the text
  \usepackage{paralist} % Used for the compactitem environment which makes bullet points with less space between them
  \usepackage{abstract} % Allows abstract customization
  \renewcommand{\abstractnamefont}{\normalfont\bfseries} 
  %\renewcommand{\abstracttextfont}{\normalfont\small\itshape} % Set the abstract itself to small italic text\[IndentingNewLine]
  \usepackage{titlesec} % Allows customization of titles
  \renewcommand\thesection{\Roman{section}} % Roman numerals for the sections
  \renewcommand\thesubsection{\Roman{subsection}} % Roman numerals for subsections
  \titleformat{\section}[block]{\large\scshape\centering}{\thesection.}{1em}{} % Change the look of the section titles
  \titleformat{\subsection}[block]{\large}{\thesubsection.}{1em}{} % Change the look of the section titles
  \usepackage{fancyhdr} % Headers and footers
  \pagestyle{fancy} % All pages have headers and footers
  \fancyhead{} % Blank out the default header
  \fancyfoot{} % Blank out the default footer
  \fancyhead[C]{X-meeting $\bullet$ October 2019 $\bullet$ Campos do  Jord\~ao} % Custom header text
  \fancyfoot[RO,LE]{} % Custom footer text
  %----------------------------------------------------------------------------------------
  % TITLE SECTION
  %---------------------------------------------------------------------------------------- 
 
 \title{\vspace{-15mm}\fontsize{24pt}{10pt}\selectfont\textbf{ Optimization of SmTGR inhibitors using a Fragment-Based Drug Design (FBDD) approach. }} % Article title
  
  
  \author{ Roc\'{\i}o Luc\'{\i}a Beatriz Riveros Maidana, Floriano Paes Silva Junior, Ana Carolina Ramos Guimar\~aes }
  
  \affil{ Laborat\'orio de Gen\^omica Funcional e Bioinform\'atica,  Instituto Oswaldo Cruz,  Funda\c{c}\~ao Oswaldo Cruz,  Rio de Janeiro,  Brazil }
  \vspace{-5mm}
  \date{}
  
  %---------------------------------------------------------------------------------------- 
  
  \begin{document}
  
  
  \maketitle % Insert title
  
  
  \thispagestyle{fancy} % All pages have headers and footers
  %----------------------------------------------------------------------------------------  
  % ABSTRACT
  
  %----------------------------------------------------------------------------------------  
  
  \begin{abstract}
  Schistosomiasis is a neglected tropical parasitic disease caused by trematodes of the genus Schistosoma and specifically S. mansoni in Brazil. Schistosomiasis is the most important human helminth infection in terms of morbidity and mortality. The disease occurs in areas with poor sanitation and approximately 1.6 millions individuals were infected with S. mansoni in Brazil.  Praziquantel is the unique drug employed for the treatment of the disease. Although the success of the treatment,  the concern about resistance is growing and the development of new drugs is urgent. Thioredoxin glutathione reductase of Schistosoma mansoni (SmTGR) is a validated drug target that plays a crucial role in the redox homeostasis of the parasite. A crystallographic screening was held to detect the ligation of small fragments to SmTGR. 32 fragments were found to be ligated around 8 sites of the protein. After the search of analogues molecules and the study of the site of activity of the fragments,  six fragments was found presenting an inhibitory activity against SmTGR acting in an allosteric site of the protein,  located in the NADPH site. In this work an in silico fragment-based drug design (FBDD) will be carry out to optimize the compounds with highest inhibitory activity in order to propose new drug candidates against S. mansoni. Cavity analisis will be held to study the binding properties of the binding site. Linking and growing approach will be apply to the optimization of the hits. Evaluation of synthetic accessibility and ADMET properties prediction will be performed. Finally,  the selected compounds will be object of docking and molecular dynamic studies.
  
  Funding: Coordena\c{c}\~ao de Aperfei\c{c}oamento de Pessoal de N\'{\i}vel Superior (CAPES) \\ 
  \end{abstract}
  \end{document} 