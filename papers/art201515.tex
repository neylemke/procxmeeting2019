
  \documentclass[twoside]{article}
  \usepackage[affil-it]{authblk}
  \usepackage{lipsum} % Package to generate dummy text throughout this template
  \usepackage{eurosym}
  \usepackage[sc]{mathpazo} % Use the Palatino font
  \usepackage[T1]{fontenc} % Use 8-bit encoding that has 256 glyphs
  \usepackage[utf8]{inputenc}
  \linespread{1.05} % Line spacing-Palatino needs more space between lines
  \usepackage{microtype} % Slightly tweak font spacing for aesthetics\[IndentingNewLine]
  \usepackage[hmarginratio=1:1,top=32mm,columnsep=20pt]{geometry} % Document margins
  \usepackage{multicol} % Used for the two-column layout of the document
  \usepackage[hang,small,labelfont=bf,up,textfont=it,up]{caption} % Custom captions under//above floats in tables or figures
  \usepackage{booktabs} % Horizontal rules in tables
  \usepackage{float} % Required for tables and figures in the multi-column environment-they need to be placed in specific locations with the[H] (e.g. \begin{table}[H])
  \usepackage{hyperref} % For hyperlinks in the PDF
  \usepackage{lettrine} % The lettrine is the first enlarged letter at the beginning of the text
  \usepackage{paralist} % Used for the compactitem environment which makes bullet points with less space between them
  \usepackage{abstract} % Allows abstract customization
  \renewcommand{\abstractnamefont}{\normalfont\bfseries} 
  %\renewcommand{\abstracttextfont}{\normalfont\small\itshape} % Set the abstract itself to small italic text\[IndentingNewLine]
  \usepackage{titlesec} % Allows customization of titles
  \renewcommand\thesection{\Roman{section}} % Roman numerals for the sections
  \renewcommand\thesubsection{\Roman{subsection}} % Roman numerals for subsections
  \titleformat{\section}[block]{\large\scshape\centering}{\thesection.}{1em}{} % Change the look of the section titles
  \titleformat{\subsection}[block]{\large}{\thesubsection.}{1em}{} % Change the look of the section titles
  \usepackage{fancyhdr} % Headers and footers
  \pagestyle{fancy} % All pages have headers and footers
  \fancyhead{} % Blank out the default header
  \fancyfoot{} % Blank out the default footer
  \fancyhead[C]{X-meeting $\bullet$ October 2019 $\bullet$ Campos do  Jord\~ao} % Custom header text
  \fancyfoot[RO,LE]{} % Custom footer text
  %----------------------------------------------------------------------------------------
  % TITLE SECTION
  %---------------------------------------------------------------------------------------- 
 
 \title{\vspace{-15mm}\fontsize{24pt}{10pt}\selectfont\textbf{ Network Creation and Comparison From MicroRNAs Extracted From Peripheral Blood Of Primigravidae Submitted Or Not To Psychosocial Intervention }} % Article title
  
  
  \author{ Rayssa Maria de Melo Wanderley Feitosa, Helena Brentani, Ariane Machado Lima, Gisele Rodrigues Gouveia }
  
  \affil{ Universidade de S\~ao Paulo }
  \vspace{-5mm}
  \date{}
  
  %---------------------------------------------------------------------------------------- 
  
  \begin{document}
  
  
  \maketitle % Insert title
  
  
  \thispagestyle{fancy} % All pages have headers and footers
  %----------------------------------------------------------------------------------------  
  % ABSTRACT
  
  %----------------------------------------------------------------------------------------  
  
  \begin{abstract}
  Environmental disturbance during the initial phases of human development,  especially the gestational period,  brings consequences that can last the offspring lifetime. Among these environmental problems,  the mother’s exposition to different stressors is correlated to a significant increase in the offspring’s risk of developing various adversities,  including cognition problems,  emotional reactivity,  impaired sociability and psychiatric disorders. The here proposed study is part of a randomized double-blind psicossocial intervention for primigravidae in socioeconomic vulnerability. Notwithstanding,  there is a lack of knowledge on the possible biological markers of the intervention. Taking this into account,  the present study has the goal to associate an important epigenetic factor,  the microRNA,  which has its expression associated to the period mentioned and disturbances along it,  with the intervention. To this end,  a comparison between two interaction networks microRNAs-mRNA will be performed. The complex networks are going to be created using the differentially expressed microRNAs from two gestational times. One network will be created using the differentially expressed microRNAs between the baseline (T0) and 30 weeks  of gestation (T1),  extracted from placental exosomes from peripheral blood of pregnants submitted to psychosocial intervention (cases),  and the other network from differentially expressed microRNAs from the same moments (T0 and T1),  collected from mothers not submitted to the intervention (controls). After the microRNA differential expression results from the RT-qPCR,  a target prediction will be executed,  using only experimentally validated targets,  and the two networks containing the miRNAs and its target genes are going to be created on Cytoscape. The databases for experimentally validated targets are going to be extracted from miRTarBase,  TarBase and miRwalk2.0. Different measurements collected from each network individually,  using global topological properties and graph entropy are going to be compared to obtain a unique global value that represents the graphical differences between the two networks created.
  
  Funding: Funda\c{c}\~ao Maria Cec\'{\i}lia Souto Vidigal,  Grand Challenges Canada,  FAPESP e CNPq \\ 
  \end{abstract}
  \end{document} 