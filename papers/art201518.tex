
  \documentclass[twoside]{article}
  \usepackage[affil-it]{authblk}
  \usepackage{lipsum} % Package to generate dummy text throughout this template
  \usepackage{eurosym}
  \usepackage[sc]{mathpazo} % Use the Palatino font
  \usepackage[T1]{fontenc} % Use 8-bit encoding that has 256 glyphs
  \usepackage[utf8]{inputenc}
  \linespread{1.05} % Line spacing-Palatino needs more space between lines
  \usepackage{microtype} % Slightly tweak font spacing for aesthetics\[IndentingNewLine]
  \usepackage[hmarginratio=1:1,top=32mm,columnsep=20pt]{geometry} % Document margins
  \usepackage{multicol} % Used for the two-column layout of the document
  \usepackage[hang,small,labelfont=bf,up,textfont=it,up]{caption} % Custom captions under//above floats in tables or figures
  \usepackage{booktabs} % Horizontal rules in tables
  \usepackage{float} % Required for tables and figures in the multi-column environment-they need to be placed in specific locations with the[H] (e.g. \begin{table}[H])
  \usepackage{hyperref} % For hyperlinks in the PDF
  \usepackage{lettrine} % The lettrine is the first enlarged letter at the beginning of the text
  \usepackage{paralist} % Used for the compactitem environment which makes bullet points with less space between them
  \usepackage{abstract} % Allows abstract customization
  \renewcommand{\abstractnamefont}{\normalfont\bfseries} 
  %\renewcommand{\abstracttextfont}{\normalfont\small\itshape} % Set the abstract itself to small italic text\[IndentingNewLine]
  \usepackage{titlesec} % Allows customization of titles
  \renewcommand\thesection{\Roman{section}} % Roman numerals for the sections
  \renewcommand\thesubsection{\Roman{subsection}} % Roman numerals for subsections
  \titleformat{\section}[block]{\large\scshape\centering}{\thesection.}{1em}{} % Change the look of the section titles
  \titleformat{\subsection}[block]{\large}{\thesubsection.}{1em}{} % Change the look of the section titles
  \usepackage{fancyhdr} % Headers and footers
  \pagestyle{fancy} % All pages have headers and footers
  \fancyhead{} % Blank out the default header
  \fancyfoot{} % Blank out the default footer
  \fancyhead[C]{X-meeting $\bullet$ October 2019 $\bullet$ Campos do  Jord\~ao} % Custom header text
  \fancyfoot[RO,LE]{} % Custom footer text
  %----------------------------------------------------------------------------------------
  % TITLE SECTION
  %---------------------------------------------------------------------------------------- 
 
 \title{\vspace{-15mm}\fontsize{24pt}{10pt}\selectfont\textbf{ Circular RNAs contribute to tumorigenesis and tumor progression in colorectal cancer }} % Article title
  
  
  \author{ Vanessa Galdeno Freitas, Pedro Alexandre Favoretto Galante, Paula Fontes Asprino }
  
  \affil{ Interunidades em Bioinform\'atica - IME USP / Instituto de Ensino e Pesquisa S\'{\i}rio Liban\^es - IEP HSL }
  \vspace{-5mm}
  \date{}
  
  %---------------------------------------------------------------------------------------- 
  
  \begin{document}
  
  
  \maketitle % Insert title
  
  
  \thispagestyle{fancy} % All pages have headers and footers
  %----------------------------------------------------------------------------------------  
  % ABSTRACT
  
  %----------------------------------------------------------------------------------------  
  
  \begin{abstract}
  Circular RNAs (circRNAs) are a new class of RNA that form covalently closed continuous loops. Like long noncoding RNAs,  the effective functions of circRNAs mainly depend on the characteristics of their sequences and structures and there is a growing body of functional roles assigned to circRNAs. For example,  circRNAs act as miRNA sponges indirectly modulating gene expression of protein-coding genes. Furthermore,  dysregulation of circRNAs expression have been associated to several human pathologies,  such as cancer. In terms of expression,  studies have reported hundreds of circRNAs as more abundant than their corresponding linear mRNAs not only in tissues,  but also in the blood. Colorectal cancer (CRC) is the third most commonly diagnosed cancer and the fourth leading cause of cancer-related deaths in the world. Studies in CRC cell lines and CRC tissues show an overall reduction in circRNA abundance compared to healthy tissue,  allowing CRC cells unexpected and uncontrolled proliferation,  for example. Here,  we have characterized and studied the expression profile of circRNA in CRC cell lines aiming to better understand the role of these RNAs in the tumorigenesis and cancer progression. First,  RNA sequencing (RNA-Seq) data from two commercial cell lines from the same primary and metastatic CRC patient (SW480 and SW620 respectively) were performed in an Illumina NextSeq platform. Next,  two methods of RNA sequencing library preparation were used: i) the standard protocol suggested by Illumina; ii) and an in-house protocol to improve the detection of circRNAs. Finally,  all RNA-seq data were aligned to the reference genome (GRCh38) and used as input to identify circRNAs through the CircExplorer algorithm and further in-house computation pipelines. RSEM and Kallisto methodologies were also used to generated the gene expression profile from all cell lines and library preparations. R package EdgeR was used for the normalization and for selecting the differentially expressed genes. First,  our results show that our in-house protocol of RNA sequencing library preparation detected 70\% more circRNAs than a standard preparation commonly used. Next,  we evaluated the number of circRNA expressed in the primary (SW480) and metastatic (SW620 ) cell lines. We found 4, 024 circRNAs differentially expressed (FDR < 0.05; log2fold change >2 or <-2),  1, 964 up regulated and 2, 060 down regulated circRNAs in primary tumor versus metastatic cell lines. By evaluating the cellular pathways of genes generating circRNAs,  we identify several candidates involved in tumorigenesis and cancer progression,  such as lysine degradation,  EGFR tyrosine kinase inhibitor resistance,  RNA transport,  proteoglycans in cancer,  focal adhesion,  regulation of actin cytoskeleton,  and adherens junction. In summary,  we believe that our work has produced novel and pivotal information to a better understanding of the functional role of circRNAs in origination and progression of colorectal tumors.
  
  Funding: CAPES \\ 
  \end{abstract}
  \end{document} 