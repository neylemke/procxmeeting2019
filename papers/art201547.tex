
  \documentclass[twoside]{article}
  \usepackage[affil-it]{authblk}
  \usepackage{lipsum} % Package to generate dummy text throughout this template
  \usepackage{eurosym}
  \usepackage[sc]{mathpazo} % Use the Palatino font
  \usepackage[T1]{fontenc} % Use 8-bit encoding that has 256 glyphs
  \usepackage[utf8]{inputenc}
  \linespread{1.05} % Line spacing-Palatino needs more space between lines
  \usepackage{microtype} % Slightly tweak font spacing for aesthetics\[IndentingNewLine]
  \usepackage[hmarginratio=1:1,top=32mm,columnsep=20pt]{geometry} % Document margins
  \usepackage{multicol} % Used for the two-column layout of the document
  \usepackage[hang,small,labelfont=bf,up,textfont=it,up]{caption} % Custom captions under//above floats in tables or figures
  \usepackage{booktabs} % Horizontal rules in tables
  \usepackage{float} % Required for tables and figures in the multi-column environment-they need to be placed in specific locations with the[H] (e.g. \begin{table}[H])
  \usepackage{hyperref} % For hyperlinks in the PDF
  \usepackage{lettrine} % The lettrine is the first enlarged letter at the beginning of the text
  \usepackage{paralist} % Used for the compactitem environment which makes bullet points with less space between them
  \usepackage{abstract} % Allows abstract customization
  \renewcommand{\abstractnamefont}{\normalfont\bfseries} 
  %\renewcommand{\abstracttextfont}{\normalfont\small\itshape} % Set the abstract itself to small italic text\[IndentingNewLine]
  \usepackage{titlesec} % Allows customization of titles
  \renewcommand\thesection{\Roman{section}} % Roman numerals for the sections
  \renewcommand\thesubsection{\Roman{subsection}} % Roman numerals for subsections
  \titleformat{\section}[block]{\large\scshape\centering}{\thesection.}{1em}{} % Change the look of the section titles
  \titleformat{\subsection}[block]{\large}{\thesubsection.}{1em}{} % Change the look of the section titles
  \usepackage{fancyhdr} % Headers and footers
  \pagestyle{fancy} % All pages have headers and footers
  \fancyhead{} % Blank out the default header
  \fancyfoot{} % Blank out the default footer
  \fancyhead[C]{X-meeting $\bullet$ October 2019 $\bullet$ Campos do  Jord\~ao} % Custom header text
  \fancyfoot[RO,LE]{} % Custom footer text
  %----------------------------------------------------------------------------------------
  % TITLE SECTION
  %---------------------------------------------------------------------------------------- 
 
 \title{\vspace{-15mm}\fontsize{24pt}{10pt}\selectfont\textbf{ Nitrogen catabolite repression in Trichophyton rubrum during the adaptive process to host molecules }} % Article title
  
  
  \author{ Ma\'{\i}ra Pompeu Martins, Pablo R. Sanches, Nilce M. Martinez-Rossi, Antonio Rossi }
  
  \affil{  }
  \vspace{-5mm}
  \date{}
  
  %---------------------------------------------------------------------------------------- 
  
  \begin{document}
  
  
  \maketitle % Insert title
  
  
  \thispagestyle{fancy} % All pages have headers and footers
  %----------------------------------------------------------------------------------------  
  % ABSTRACT
  
  %----------------------------------------------------------------------------------------  
  
  \begin{abstract}
  The environmental challenges imposed to fungal pathogens require a dynamic metabolic modulation,  resulting in the activation or repression of critical factors,  enabling the establishment and perpetuation of the infection in their host. Wherefore,  to conquer the different host microenvironments,  pathogens not only depend on virulence factors but also on metabolic flexibility,  dynamically responding to stressing conditions in the host. The dermatophyte Trichophyton rubrum is a pathogenic fungus adapted to degrade keratinized tissues as the skin and nails,  utilizing the resulting amino acids and peptides as the final carbon sources. During infection,  the proteolytic activity is associated with deamination reactions,  resulting in the accumulation of nitrogen mainly as ammonium ions,  actively secreted by the fungus through intensive ammonia production. The secreted ammonia raises the extracellular pH,  resulting in an alkalinized ambient,  acting as a supporting factor to pathogenicity. We evaluated T. rubrum interaction with keratin,  in a metabolic perspective,  providing information about gene modulation of the dermatophyte during early infection stage after shifting from glucose- to keratin-containing culture media,  in comparison to glucose as the carbon source. Analyzing T. rubrum transcriptome using high-throughput RNA-sequencing (RNA-seq) technology,  we identified the repression of essential genes related to nitrogen metabolism,  as the ammonium transporter MepA,  a glutamate synthase,  a proline-specific permease,  and a urease. These results suggest the activation of an alternative pathway for nitrogen assimilation in the tested conditions,  necessary for the fungus survival in the host. The gene expression profiling of the host-pathogen interaction highlights candidate genes involved in fungal adaptation and survival,  elucidating the machinery required to the establishment of the initial stages of the infection.
  
  Funding: FAPESP,  CNPq,  CAPES,  and FAEPA \\ 
  \end{abstract}
  \end{document} 