
  \documentclass[twoside]{article}
  \usepackage[affil-it]{authblk}
  \usepackage{lipsum} % Package to generate dummy text throughout this template
  \usepackage{eurosym}
  \usepackage[sc]{mathpazo} % Use the Palatino font
  \usepackage[T1]{fontenc} % Use 8-bit encoding that has 256 glyphs
  \usepackage[utf8]{inputenc}
  \linespread{1.05} % Line spacing-Palatino needs more space between lines
  \usepackage{microtype} % Slightly tweak font spacing for aesthetics\[IndentingNewLine]
  \usepackage[hmarginratio=1:1,top=32mm,columnsep=20pt]{geometry} % Document margins
  \usepackage{multicol} % Used for the two-column layout of the document
  \usepackage[hang,small,labelfont=bf,up,textfont=it,up]{caption} % Custom captions under//above floats in tables or figures
  \usepackage{booktabs} % Horizontal rules in tables
  \usepackage{float} % Required for tables and figures in the multi-column environment-they need to be placed in specific locations with the[H] (e.g. \begin{table}[H])
  \usepackage{hyperref} % For hyperlinks in the PDF
  \usepackage{lettrine} % The lettrine is the first enlarged letter at the beginning of the text
  \usepackage{paralist} % Used for the compactitem environment which makes bullet points with less space between them
  \usepackage{abstract} % Allows abstract customization
  \renewcommand{\abstractnamefont}{\normalfont\bfseries} 
  %\renewcommand{\abstracttextfont}{\normalfont\small\itshape} % Set the abstract itself to small italic text\[IndentingNewLine]
  \usepackage{titlesec} % Allows customization of titles
  \renewcommand\thesection{\Roman{section}} % Roman numerals for the sections
  \renewcommand\thesubsection{\Roman{subsection}} % Roman numerals for subsections
  \titleformat{\section}[block]{\large\scshape\centering}{\thesection.}{1em}{} % Change the look of the section titles
  \titleformat{\subsection}[block]{\large}{\thesubsection.}{1em}{} % Change the look of the section titles
  \usepackage{fancyhdr} % Headers and footers
  \pagestyle{fancy} % All pages have headers and footers
  \fancyhead{} % Blank out the default header
  \fancyfoot{} % Blank out the default footer
  \fancyhead[C]{X-meeting $\bullet$ October 2019 $\bullet$ Campos do  Jord\~ao} % Custom header text
  \fancyfoot[RO,LE]{} % Custom footer text
  %----------------------------------------------------------------------------------------
  % TITLE SECTION
  %---------------------------------------------------------------------------------------- 
 
 \title{\vspace{-15mm}\fontsize{24pt}{10pt}\selectfont\textbf{ Scientific Dissemination in Bioinformatics }} % Article title
  
  
  \author{ Luana Luiza Bastos, Raquel Melo Minardi }
  
  \affil{ Universidade Federal de Minas Gerais - UFMG }
  \vspace{-5mm}
  \date{}
  
  %---------------------------------------------------------------------------------------- 
  
  \begin{document}
  
  
  \maketitle % Insert title
  
  
  \thispagestyle{fancy} % All pages have headers and footers
  %----------------------------------------------------------------------------------------  
  % ABSTRACT
  
  %----------------------------------------------------------------------------------------  
  
  \begin{abstract}
  Computational science has greatly contributed to the generation and organization of biological data as well as the development of more sophisticated techniques for solving biological problems. From gene sequencing to molecular biology the machine work has facilitated the discovery of revolutionary informations. As knowledge is accumulated and the need for understanding of life processes has increased,  biology keeps its position as a central science with greater reach and impact on society. Bioinformatics has gained greater relevance since it is the interdisciplinary field of knowledge linking biology with computation. This new science is responsible for the increasingly contribution to the development of algorithms,  that are unfortunately still applied and studied by a small number of interested people. Life science (and related) researchers have high demand for the development and use of these algorithms as they effectively contribute to the quality of the data produced. However,  often the algorithms used are poorly understood and are applied without the proper domain of their operation and parameters. In addition,  much has been said about the importance of scientific dissemination as the general population is far out removed from the researches carried out at the university. In this context,  the aim of this project is to develop didactic content to make bioinformatics algorithms simpler to understand and make the processes involved in the development more transparent to the users of a respective software. Besides,  we would like to foster the creation of communities for research discussion between researcher and society on topics related to bioinformatics in social media,  as well as to produce teaching content and scientific dissemination about bioinformatics. For this,  a channel on the Youtube platform were created with the proposal to present videos related with bioinformatics informations. Also,  it has been created a web site in which are deposited basic texts and references used for the production of videos and where accounts on social medias platforms,  such as Instagram and Facebook,  are divulgated in which content about bioinformatics and scientific dissemination are deposited periodically. A research on bioinformatics was conducted with undergraduate and graduate students of the Federal University of Minas Gerais through an online based form,  about the most used and impacting algorithms in different areas of bioinformatics. In addition,  this work aimed to understand the demands presented by students and themes that would be of interest to video production. For the selected themes,  related content is produced for social media,  including videos and text production. Currently the channel on Youtube called OnlineBioinfo has about 60 videos posted and has over 850 subscribers with approximately 21, 621 views. In the future it is expected to increase the number of people reached by the content created,  besides to improve its quality by developing guidelines that facilitate learning and that are of interest to the community.
  
  Funding:  \\ 
  \end{abstract}
  \end{document} 