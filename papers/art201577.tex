
  \documentclass[twoside]{article}
  \usepackage[affil-it]{authblk}
  \usepackage{lipsum} % Package to generate dummy text throughout this template
  \usepackage{eurosym}
  \usepackage[sc]{mathpazo} % Use the Palatino font
  \usepackage[T1]{fontenc} % Use 8-bit encoding that has 256 glyphs
  \usepackage[utf8]{inputenc}
  \linespread{1.05} % Line spacing-Palatino needs more space between lines
  \usepackage{microtype} % Slightly tweak font spacing for aesthetics\[IndentingNewLine]
  \usepackage[hmarginratio=1:1,top=32mm,columnsep=20pt]{geometry} % Document margins
  \usepackage{multicol} % Used for the two-column layout of the document
  \usepackage[hang,small,labelfont=bf,up,textfont=it,up]{caption} % Custom captions under//above floats in tables or figures
  \usepackage{booktabs} % Horizontal rules in tables
  \usepackage{float} % Required for tables and figures in the multi-column environment-they need to be placed in specific locations with the[H] (e.g. \begin{table}[H])
  \usepackage{hyperref} % For hyperlinks in the PDF
  \usepackage{lettrine} % The lettrine is the first enlarged letter at the beginning of the text
  \usepackage{paralist} % Used for the compactitem environment which makes bullet points with less space between them
  \usepackage{abstract} % Allows abstract customization
  \renewcommand{\abstractnamefont}{\normalfont\bfseries} 
  %\renewcommand{\abstracttextfont}{\normalfont\small\itshape} % Set the abstract itself to small italic text\[IndentingNewLine]
  \usepackage{titlesec} % Allows customization of titles
  \renewcommand\thesection{\Roman{section}} % Roman numerals for the sections
  \renewcommand\thesubsection{\Roman{subsection}} % Roman numerals for subsections
  \titleformat{\section}[block]{\large\scshape\centering}{\thesection.}{1em}{} % Change the look of the section titles
  \titleformat{\subsection}[block]{\large}{\thesubsection.}{1em}{} % Change the look of the section titles
  \usepackage{fancyhdr} % Headers and footers
  \pagestyle{fancy} % All pages have headers and footers
  \fancyhead{} % Blank out the default header
  \fancyfoot{} % Blank out the default footer
  \fancyhead[C]{X-meeting $\bullet$ October 2019 $\bullet$ Campos do  Jord\~ao} % Custom header text
  \fancyfoot[RO,LE]{} % Custom footer text
  %----------------------------------------------------------------------------------------
  % TITLE SECTION
  %---------------------------------------------------------------------------------------- 
 
 \title{\vspace{-15mm}\fontsize{24pt}{10pt}\selectfont\textbf{ CodAn: predictive models for the characterization of mRNA Transcripts }} % Article title
  
  
  \author{ Pedro Gabriel Nachtigall, Andr\'e Yoshiaki Kashiwabara, Alan Durham }
  
  \affil{ Laborat\'orio Especial de Toxinologia Aplicada (LETA),  Instituto Butantan,  S\~ao Paulo,  Brazil }
  \vspace{-5mm}
  \date{}
  
  %---------------------------------------------------------------------------------------- 
  
  \begin{document}
  
  
  \maketitle % Insert title
  
  
  \thispagestyle{fancy} % All pages have headers and footers
  %----------------------------------------------------------------------------------------  
  % ABSTRACT
  
  %----------------------------------------------------------------------------------------  
  
  \begin{abstract}
  The complete characterization of the coding sequences (CDSs) and untranslated regions (UTRs) of transcripts is an essential step on transcriptome annotation and expression profile analysis. First,  it defines which proteins should be synthesized by the messenger RNAs and are part of the proteome of the organism. The incorrect characterization of CDSs can lead to the prediction of non-existent proteins. Wrong protein predictions can eventually compromise knowledge if annotation databases are populated with similar incorrect predictions made in different genomes. Also,  the correct identification of CDSs is important for the characterization of the UTR landscape,  whereas the 3’UTR and 5’UTR are known as important regulators of the mRNA fate and translate process. Here,  we present CodAn,  a new computational approach to predict CDS and UTR sequences directly from transcriptome sequences of any Eukaryote species,  such as RNAseq assembly data. CodAn can be applied to full or partial transcripts and presents a better performance predicting the whole CDS than other approaches. CodAn requires low computational resources and can be used on any standard desktop computers,  and,  for large jobs,  can use the parallel processing capabilities of large multi-core servers. The data generated by CodAn can be used to improve genome annotation and help further experiments focused on understanding the evolution and biology of CDS and UTR sequences.
  
  Funding: This study was financed in part by the Coordena\c{c}\~ao de Aperfei\c{c}oamento de Pessoal de N\'{\i}vel Superior - Brasil (CAPES) - Finance Code 001 - Process number 88887.177457/2018-00. \\ 
  \end{abstract}
  \end{document} 