
  \documentclass[twoside]{article}
  \usepackage[affil-it]{authblk}
  \usepackage{lipsum} % Package to generate dummy text throughout this template
  \usepackage{eurosym}
  \usepackage[sc]{mathpazo} % Use the Palatino font
  \usepackage[T1]{fontenc} % Use 8-bit encoding that has 256 glyphs
  \usepackage[utf8]{inputenc}
  \linespread{1.05} % Line spacing-Palatino needs more space between lines
  \usepackage{microtype} % Slightly tweak font spacing for aesthetics\[IndentingNewLine]
  \usepackage[hmarginratio=1:1,top=32mm,columnsep=20pt]{geometry} % Document margins
  \usepackage{multicol} % Used for the two-column layout of the document
  \usepackage[hang,small,labelfont=bf,up,textfont=it,up]{caption} % Custom captions under//above floats in tables or figures
  \usepackage{booktabs} % Horizontal rules in tables
  \usepackage{float} % Required for tables and figures in the multi-column environment-they need to be placed in specific locations with the[H] (e.g. \begin{table}[H])
  \usepackage{hyperref} % For hyperlinks in the PDF
  \usepackage{lettrine} % The lettrine is the first enlarged letter at the beginning of the text
  \usepackage{paralist} % Used for the compactitem environment which makes bullet points with less space between them
  \usepackage{abstract} % Allows abstract customization
  \renewcommand{\abstractnamefont}{\normalfont\bfseries} 
  %\renewcommand{\abstracttextfont}{\normalfont\small\itshape} % Set the abstract itself to small italic text\[IndentingNewLine]
  \usepackage{titlesec} % Allows customization of titles
  \renewcommand\thesection{\Roman{section}} % Roman numerals for the sections
  \renewcommand\thesubsection{\Roman{subsection}} % Roman numerals for subsections
  \titleformat{\section}[block]{\large\scshape\centering}{\thesection.}{1em}{} % Change the look of the section titles
  \titleformat{\subsection}[block]{\large}{\thesubsection.}{1em}{} % Change the look of the section titles
  \usepackage{fancyhdr} % Headers and footers
  \pagestyle{fancy} % All pages have headers and footers
  \fancyhead{} % Blank out the default header
  \fancyfoot{} % Blank out the default footer
  \fancyhead[C]{X-meeting $\bullet$ October 2019 $\bullet$ Campos do  Jord\~ao} % Custom header text
  \fancyfoot[RO,LE]{} % Custom footer text
  %----------------------------------------------------------------------------------------
  % TITLE SECTION
  %---------------------------------------------------------------------------------------- 
 
 \title{\vspace{-15mm}\fontsize{24pt}{10pt}\selectfont\textbf{ Phylogenetic analysis of the TRAFAC class and discovery of the first prokaryotic septin }} % Article title
  
  
  \author{ Guilherme Bastos Gomes, Robson Francisco de Souza }
  
  \affil{ USP }
  \vspace{-5mm}
  \date{}
  
  %---------------------------------------------------------------------------------------- 
  
  \begin{document}
  
  
  \maketitle % Insert title
  
  
  \thispagestyle{fancy} % All pages have headers and footers
  %----------------------------------------------------------------------------------------  
  % ABSTRACT
  
  %----------------------------------------------------------------------------------------  
  
  \begin{abstract}
  Septins are GTPases capable of assembling into hetero/homo-oligomers that  were first described as components of a ring structure at the bud neck formed during cell division in Saccharomyces cerevisiae cells. The name septins is due to the localization of this ring at the septal collar. Posterior studies have shown that septins interact with actin and microtubules,  are able to form filaments and can regulate membrane dynamics,  such as blebbing formation and flexibility. Given their role in membrane dynamics and its ability to form scaffolds,  septins were dimmed the fourth element of the cellular cytoskeleton,  together with actin,  tubulin and intermediate filaments. Although initially identified only in animals and fungi,  recent studies have shown a wider distribution of this protein family across many eukaryotic lineages,  such as red and brown algae. Interestingly,  although many related small gtpases are known from prokaryotes,  no septin homolog was ever found in any bacterial genome. To understand the origins and distribution of septin-related proteins,  we performed exhaustive searches for homologs of members of the TRAFAC (Translation factor-related GTPases) group,  a wider assembly of small gtpases that includes septins. By focusing on a paraphyletic group known as paraseptins,  we were able to recover all the closest family members of septins within the TRAFAC class,  including Elongation Factors,  OBG-like,  RsgA,  Era,  TrmE,  MnmE,  Rab,  Ras,  Dynamins,  GIMAPs,  Tocs and Septins. Sequence similarity clustering revealed closely related groups,  from which representative sequences were gathered and aligned for phylogenetic analysis. Several paralogs of the GIMAP family of paraseptins were found in groups of Teleostei,  what agrees with the hypothesis of a fourth wide genome duplication in this lineage. The diversification of the GTPases of immunity associated proteins were also analysed and this protein is spread among non-opisthokont Eukaryotes. This analysis also revealed a close relative of the septin group spread among several bacterial clades and closely related to a clade of the Chlamydomonas reinhardtii septins. This bacterial putative septin is fused with a domain of unknown function composed of one soluble region surrounded by two transmembrane regions before a cytoplasmic region and an N-terminal region also associated with several other GTPases. The discovery of a close relative of septins spread across many bacterial lineages can shed new light on the evolution and origin of this important group of proteins and give material for further experimental analysis.
  
  Funding: CAPES \\ 
  \end{abstract}
  \end{document} 