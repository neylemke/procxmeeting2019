
  \documentclass[twoside]{article}
  \usepackage[affil-it]{authblk}
  \usepackage{lipsum} % Package to generate dummy text throughout this template
  \usepackage{eurosym}
  \usepackage[sc]{mathpazo} % Use the Palatino font
  \usepackage[T1]{fontenc} % Use 8-bit encoding that has 256 glyphs
  \usepackage[utf8]{inputenc}
  \linespread{1.05} % Line spacing-Palatino needs more space between lines
  \usepackage{microtype} % Slightly tweak font spacing for aesthetics\[IndentingNewLine]
  \usepackage[hmarginratio=1:1,top=32mm,columnsep=20pt]{geometry} % Document margins
  \usepackage{multicol} % Used for the two-column layout of the document
  \usepackage[hang,small,labelfont=bf,up,textfont=it,up]{caption} % Custom captions under//above floats in tables or figures
  \usepackage{booktabs} % Horizontal rules in tables
  \usepackage{float} % Required for tables and figures in the multi-column environment-they need to be placed in specific locations with the[H] (e.g. \begin{table}[H])
  \usepackage{hyperref} % For hyperlinks in the PDF
  \usepackage{lettrine} % The lettrine is the first enlarged letter at the beginning of the text
  \usepackage{paralist} % Used for the compactitem environment which makes bullet points with less space between them
  \usepackage{abstract} % Allows abstract customization
  \renewcommand{\abstractnamefont}{\normalfont\bfseries} 
  %\renewcommand{\abstracttextfont}{\normalfont\small\itshape} % Set the abstract itself to small italic text\[IndentingNewLine]
  \usepackage{titlesec} % Allows customization of titles
  \renewcommand\thesection{\Roman{section}} % Roman numerals for the sections
  \renewcommand\thesubsection{\Roman{subsection}} % Roman numerals for subsections
  \titleformat{\section}[block]{\large\scshape\centering}{\thesection.}{1em}{} % Change the look of the section titles
  \titleformat{\subsection}[block]{\large}{\thesubsection.}{1em}{} % Change the look of the section titles
  \usepackage{fancyhdr} % Headers and footers
  \pagestyle{fancy} % All pages have headers and footers
  \fancyhead{} % Blank out the default header
  \fancyfoot{} % Blank out the default footer
  \fancyhead[C]{X-meeting $\bullet$ October 2019 $\bullet$ Campos do  Jord\~ao} % Custom header text
  \fancyfoot[RO,LE]{} % Custom footer text
  %----------------------------------------------------------------------------------------
  % TITLE SECTION
  %---------------------------------------------------------------------------------------- 
 
 \title{\vspace{-15mm}\fontsize{24pt}{10pt}\selectfont\textbf{ Genomic characterization of Lactobacillus delbrueckii CIDCA 133: a potential probiotic strain }} % Article title
  
  
  \author{ Rodrigo Profeta Silveira Santos, Lu\'{\i}s Cl\'audio Lima de Jesus, Marcus Vinicius Can\'ario Viana, Jana\'{\i}na Can\'ario Cerqueira, Mariana Martins Drumond, Pamela Mancha-Agresti, Bertram Brenig, Vasco A de C Azevedo }
  
  \affil{ University G\"ottingen }
  \vspace{-5mm}
  \date{}
  
  %---------------------------------------------------------------------------------------- 
  
  \begin{document}
  
  
  \maketitle % Insert title
  
  
  \thispagestyle{fancy} % All pages have headers and footers
  %----------------------------------------------------------------------------------------  
  % ABSTRACT
  
  %----------------------------------------------------------------------------------------  
  
  \begin{abstract}
  The probiotic potential of Lactobacillus delbrueckii CIDCA 133,  isolated from raw cow milk,  have been validated by in vitro studies that showed its resistance to high acid and bile concentrations,  the ability to inhibit the growth of pathogenic microorganisms such as enterohemorrhagic Escherichia coli,  resistance to antimicrobial peptides of erythrocyte cells,  and anti-inflammatory and immunomodulatory properties. However,  despite of these important traits,  little is known about the molecular mechanisms involved in these processes. In this study,  we performed a comprehensive analysis of its genome to better understand the molecular basis of its in-vitro-tested probiotic characteristics. The genome of L. delbrueckii CIDCA 133,  herein named CIDCA 133,  was sequenced using Illumina HiSeq platform and assembled using Spades. The assembly resulted in 70 contigs and a 6223 bp sized cryptic plasmid. The characterization of its genome confirmed that  CIDCA 133 belongs to the subspecies  lactis,  with  98.21\% of ANI compared to the type strain L. delbrueckii subsp. lactis DSM 20072.  Comparative genomics analyses were conducted including  CIDCA 133 and  64 publicly available L. delbrueckii genomes. Despite the expected clustering of the subspecies in the phylogenomic tree,  generated by the software PEPR,  divergent identifications were found for some L. delbrueckii subsp.  bulgaricus strains,  which grouped in the same clade with another L. delbrueckii subsp. lactis. Given the similarity between the subspecies,  our results suggest that some genomes have been misclassified in the GenBank. Hence,  greater care should be taken with their characterization,  using genomic analysis as part of this process. Preliminary functional analysis shows that  CIDCA 133 has 18 exclusive genes,  one of these (nagB) required for the Amino sugar and nucleotide sugar metabolism pathway. These genes might be related to the metabolic pathway associated with the singular probiotic characteristics described for this strain,  and thus,  more analyses will be performed in order to characterize them.
  
  Funding:  \\ 
  \end{abstract}
  \end{document} 