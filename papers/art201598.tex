
  \documentclass[twoside]{article}
  \usepackage[affil-it]{authblk}
  \usepackage{lipsum} % Package to generate dummy text throughout this template
  \usepackage{eurosym}
  \usepackage[sc]{mathpazo} % Use the Palatino font
  \usepackage[T1]{fontenc} % Use 8-bit encoding that has 256 glyphs
  \usepackage[utf8]{inputenc}
  \linespread{1.05} % Line spacing-Palatino needs more space between lines
  \usepackage{microtype} % Slightly tweak font spacing for aesthetics\[IndentingNewLine]
  \usepackage[hmarginratio=1:1,top=32mm,columnsep=20pt]{geometry} % Document margins
  \usepackage{multicol} % Used for the two-column layout of the document
  \usepackage[hang,small,labelfont=bf,up,textfont=it,up]{caption} % Custom captions under//above floats in tables or figures
  \usepackage{booktabs} % Horizontal rules in tables
  \usepackage{float} % Required for tables and figures in the multi-column environment-they need to be placed in specific locations with the[H] (e.g. \begin{table}[H])
  \usepackage{hyperref} % For hyperlinks in the PDF
  \usepackage{lettrine} % The lettrine is the first enlarged letter at the beginning of the text
  \usepackage{paralist} % Used for the compactitem environment which makes bullet points with less space between them
  \usepackage{abstract} % Allows abstract customization
  \renewcommand{\abstractnamefont}{\normalfont\bfseries} 
  %\renewcommand{\abstracttextfont}{\normalfont\small\itshape} % Set the abstract itself to small italic text\[IndentingNewLine]
  \usepackage{titlesec} % Allows customization of titles
  \renewcommand\thesection{\Roman{section}} % Roman numerals for the sections
  \renewcommand\thesubsection{\Roman{subsection}} % Roman numerals for subsections
  \titleformat{\section}[block]{\large\scshape\centering}{\thesection.}{1em}{} % Change the look of the section titles
  \titleformat{\subsection}[block]{\large}{\thesubsection.}{1em}{} % Change the look of the section titles
  \usepackage{fancyhdr} % Headers and footers
  \pagestyle{fancy} % All pages have headers and footers
  \fancyhead{} % Blank out the default header
  \fancyfoot{} % Blank out the default footer
  \fancyhead[C]{X-meeting $\bullet$ October 2019 $\bullet$ Campos do  Jord\~ao} % Custom header text
  \fancyfoot[RO,LE]{} % Custom footer text
  %----------------------------------------------------------------------------------------
  % TITLE SECTION
  %---------------------------------------------------------------------------------------- 
 
 \title{\vspace{-15mm}\fontsize{24pt}{10pt}\selectfont\textbf{ Characterizing the global virome of Apis mellifera using a small RNA-based approach }} % Article title
  
  
  \author{ Juliana Armache, Jo\~ao Paulo Pereira de Almeida, Jo\~ao Trindade Marques, Eric Roberto Guimar\~aes Rocha Aguiar }
  
  \affil{ Universidade Federal de Minas Gerais }
  \vspace{-5mm}
  \date{}
  
  %---------------------------------------------------------------------------------------- 
  
  \begin{document}
  
  
  \maketitle % Insert title
  
  
  \thispagestyle{fancy} % All pages have headers and footers
  %----------------------------------------------------------------------------------------  
  % ABSTRACT
  
  %----------------------------------------------------------------------------------------  
  
  \begin{abstract}
  Human activity is increasingly destabilizing ecosystems,  causing incalculable and often irreparable damage to biodiversity. In the floristics field,  the consequences of human action are already being observed globally,  generating a significant loss in native plants and impacting commercial agriculture. One of the factors that contribute to this scenario is the decreasing number of pollinating agents,  mainly insects,  driven by the increased use of pesticides that are harmful to those animals. Among insects,  bees are considered the main pollinators,  having a global distribution and attending to a wide spectrum of plants. Besides being sensitive to pesticides,  these animals are also susceptible to viral infections,  and these two factors combined have led to colony collapses with consequent economic loss. Therefore,  knowledge of bee virome can help the development of new strategies to control and prevent viral infections. This study aimed to analyze the collection of circulating viruses in different populations of  Apis mellifera  using a strategy based on small RNA sequencing data. Twenty-four libraries of A. mellifera small RNAs,  originated from South Africa,  USA,  China,  Netherlands and UK were chosen. First,  the libraries were pre-processed to remove sequencing adapters and to filter sequences with bad quality (Phred <20). The second step was to remove sequences that mapped to host genome (Apis mellifera) or known bacterial genomes. The remaining reads were subjected to a de novo assembly strategy using Velvet and SPAdes. The resulting assembled contigs were characterized by sequence similarity analysis against NT and NR databases using BLAST. Viral sequences were found in 17 out of 24 libraries,  and the results included some viruses known to cause high mortality rates in bee colonies,  such as  Varroa destructor virus  (VDV) and  Deformed wing virus  (DWV). From all the assembled contigs,  some of them showed a significant similarity  to known viruses at the nucleotide level,  suggesting they are likely new strains of these viruses. In other cases,  the similarity to viral sequences was limited to the aminoacid level,  which suggests these might be new viral species. Spatial analysis showed that DWV is infecting bees from all over the continents,  while other viruses are restricted to some regions. In addition,  the contigs that could belong to new viral species were restricted to libraries from Europe. Using this approach we were able to find in our samples viruses that could have both economic and ecological importance,  and possibly a new virus that infects Apis mellifera. Unraveling global virome of bees is an important tool to identify and monitor viruses that may cause harm to colonies. This is the first step to help prevent future outbreaks to avoid big economical and biodiversity losses.
  
  Funding: CNPq \\ 
  \end{abstract}
  \end{document} 