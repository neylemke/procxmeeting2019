
  \documentclass[twoside]{article}
  \usepackage[affil-it]{authblk}
  \usepackage{lipsum} % Package to generate dummy text throughout this template
  \usepackage{eurosym}
  \usepackage[sc]{mathpazo} % Use the Palatino font
  \usepackage[T1]{fontenc} % Use 8-bit encoding that has 256 glyphs
  \usepackage[utf8]{inputenc}
  \linespread{1.05} % Line spacing-Palatino needs more space between lines
  \usepackage{microtype} % Slightly tweak font spacing for aesthetics\[IndentingNewLine]
  \usepackage[hmarginratio=1:1,top=32mm,columnsep=20pt]{geometry} % Document margins
  \usepackage{multicol} % Used for the two-column layout of the document
  \usepackage[hang,small,labelfont=bf,up,textfont=it,up]{caption} % Custom captions under//above floats in tables or figures
  \usepackage{booktabs} % Horizontal rules in tables
  \usepackage{float} % Required for tables and figures in the multi-column environment-they need to be placed in specific locations with the[H] (e.g. \begin{table}[H])
  \usepackage{hyperref} % For hyperlinks in the PDF
  \usepackage{lettrine} % The lettrine is the first enlarged letter at the beginning of the text
  \usepackage{paralist} % Used for the compactitem environment which makes bullet points with less space between them
  \usepackage{abstract} % Allows abstract customization
  \renewcommand{\abstractnamefont}{\normalfont\bfseries} 
  %\renewcommand{\abstracttextfont}{\normalfont\small\itshape} % Set the abstract itself to small italic text\[IndentingNewLine]
  \usepackage{titlesec} % Allows customization of titles
  \renewcommand\thesection{\Roman{section}} % Roman numerals for the sections
  \renewcommand\thesubsection{\Roman{subsection}} % Roman numerals for subsections
  \titleformat{\section}[block]{\large\scshape\centering}{\thesection.}{1em}{} % Change the look of the section titles
  \titleformat{\subsection}[block]{\large}{\thesubsection.}{1em}{} % Change the look of the section titles
  \usepackage{fancyhdr} % Headers and footers
  \pagestyle{fancy} % All pages have headers and footers
  \fancyhead{} % Blank out the default header
  \fancyfoot{} % Blank out the default footer
  \fancyhead[C]{X-meeting $\bullet$ October 2019 $\bullet$ Campos do  Jord\~ao} % Custom header text
  \fancyfoot[RO,LE]{} % Custom footer text
  %----------------------------------------------------------------------------------------
  % TITLE SECTION
  %---------------------------------------------------------------------------------------- 
 
 \title{\vspace{-15mm}\fontsize{24pt}{10pt}\selectfont\textbf{ sideRETRO: structural variations intercurences discovery environment for retrocopies }} % Article title
  
  
  \author{ Jos\'e Leonel Lemos Buzzo, Thiago Luiz Araujo Miller, Pedro Alexandre Favoretto Galante }
  
  \affil{ USP }
  \vspace{-5mm}
  \date{}
  
  %---------------------------------------------------------------------------------------- 
  
  \begin{document}
  
  
  \maketitle % Insert title
  
  
  \thispagestyle{fancy} % All pages have headers and footers
  %----------------------------------------------------------------------------------------  
  % ABSTRACT
  
  %----------------------------------------------------------------------------------------  
  
  \begin{abstract}
  The understanding of Transposable elements' mechanisms are gaining a rising
focus on actual researches as key features of genomic structures and the
impacts of their dynamics are directly associated with many pathological
scenarios.
	It has been clearly shown the protagonic role exerted,  for example,  by
somatic retrotransposon insertions in tumorigenenic cases,  whether disrupting
promoters of critical protein coding genes,  or creating new splicing sites and
isoforms,  or even becoming expressed.
	Therefore,  accurate detection methods ought to be settled down for them, 
methods by which whole genomic assessments of the copy number variations of these
transposable elements would be feasible. However,  some quantitative difficulties
lies on this task concerning the ambiguity on mapping new inserts: actual
aligners frequently report their reads as belonging to the parenthal gene in the
reference genome. Corroborating this fact,  literature shows only a few examples
of bioinformatics tools available to this errand and,  even these,  cannot ensure
their accuracy because of a lack in false positive statistical controls. So,  to
get around this problem,  a candidate algorithm would need to (1) distinguish
ambiguous read mappings from their parenthals and other fixed copies of it, 
based on an annotated gene list; and (2) robustly learn to discern the false
positive cases using some simulation approach.
	Now focusing on retrocopies,  which are new insertions of processed protein
coding genes' mRNAs made by LINE retrotransposon machinery,  we present
sideRETRO,  a computational bioinformatis tool for polymorphic and somatic
retrocopies discovery,  in a genomic landscape. Provided with an accuracy
tuning simulation method suited for this task and a ambiguity solving engine
based on discordant reads mappings,  our tool was used to search for retrocopies
on ten whole genomes sequenced from healthy individuals with more than eighty
years old. It was chosen only healty individuals because of the need to compose
a non tumorigenic retrocopies profile when comparing to future pathological
samples.
	So,  assessed by a previous simulation batch for false positive filtering, 
our algorithm reached a 0.87 accuracy rate on a 5-fold cross validation. And, 
when used on the ten whole genomes with high coverage (~40x),  it discovered a
mean of seventeen retrocopies per genome,  which were further identified when
polymorphic or not and when cancer related or not.
  
  Funding: CNPq \\ 
  \end{abstract}
  \end{document} 