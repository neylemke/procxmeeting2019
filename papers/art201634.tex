
  \documentclass[twoside]{article}
  \usepackage[affil-it]{authblk}
  \usepackage{lipsum} % Package to generate dummy text throughout this template
  \usepackage{eurosym}
  \usepackage[sc]{mathpazo} % Use the Palatino font
  \usepackage[T1]{fontenc} % Use 8-bit encoding that has 256 glyphs
  \usepackage[utf8]{inputenc}
  \linespread{1.05} % Line spacing-Palatino needs more space between lines
  \usepackage{microtype} % Slightly tweak font spacing for aesthetics\[IndentingNewLine]
  \usepackage[hmarginratio=1:1,top=32mm,columnsep=20pt]{geometry} % Document margins
  \usepackage{multicol} % Used for the two-column layout of the document
  \usepackage[hang,small,labelfont=bf,up,textfont=it,up]{caption} % Custom captions under//above floats in tables or figures
  \usepackage{booktabs} % Horizontal rules in tables
  \usepackage{float} % Required for tables and figures in the multi-column environment-they need to be placed in specific locations with the[H] (e.g. \begin{table}[H])
  \usepackage{hyperref} % For hyperlinks in the PDF
  \usepackage{lettrine} % The lettrine is the first enlarged letter at the beginning of the text
  \usepackage{paralist} % Used for the compactitem environment which makes bullet points with less space between them
  \usepackage{abstract} % Allows abstract customization
  \renewcommand{\abstractnamefont}{\normalfont\bfseries} 
  %\renewcommand{\abstracttextfont}{\normalfont\small\itshape} % Set the abstract itself to small italic text\[IndentingNewLine]
  \usepackage{titlesec} % Allows customization of titles
  \renewcommand\thesection{\Roman{section}} % Roman numerals for the sections
  \renewcommand\thesubsection{\Roman{subsection}} % Roman numerals for subsections
  \titleformat{\section}[block]{\large\scshape\centering}{\thesection.}{1em}{} % Change the look of the section titles
  \titleformat{\subsection}[block]{\large}{\thesubsection.}{1em}{} % Change the look of the section titles
  \usepackage{fancyhdr} % Headers and footers
  \pagestyle{fancy} % All pages have headers and footers
  \fancyhead{} % Blank out the default header
  \fancyfoot{} % Blank out the default footer
  \fancyhead[C]{X-meeting $\bullet$ October 2019 $\bullet$ Campos do  Jord\~ao} % Custom header text
  \fancyfoot[RO,LE]{} % Custom footer text
  %----------------------------------------------------------------------------------------
  % TITLE SECTION
  %---------------------------------------------------------------------------------------- 
 
 \title{\vspace{-15mm}\fontsize{24pt}{10pt}\selectfont\textbf{ Whole-exome sequencing evaluation of BCG responsiveness in high-risk non-muscle invasive bladder cancer (NMIBC) }} % Article title
  
  
  \author{ Diogo A Bastos, Romulo L Mattedi, Rodrigo Araujo Sequeira Barreiro, Filipe Ferreira dos Santos, Vanessa Candiotti Buzatto, Cibele Masotti, Jussara M Souza, Mariana Zuliani Theodoro de Lima, Giulia W. Frigugliett, Carlos Dzik, Denis L Jardim, Rafael Coelho, Leopoldo A Ribeiro Filho, Mauricio D Cordeiro, William C Nahas, Evandro S de Mello, Roger Chammas, Luiz Fernando Lima Reis, Fabiana Bettoni, Pedro Galante, Anamaria A. Camargo }
  
  \affil{ Translacional Oncology Center,  ICESP }
  \vspace{-5mm}
  \date{}
  
  %---------------------------------------------------------------------------------------- 
  
  \begin{document}
  
  
  \maketitle % Insert title
  
  
  \thispagestyle{fancy} % All pages have headers and footers
  %----------------------------------------------------------------------------------------  
  % ABSTRACT
  
  %----------------------------------------------------------------------------------------  
  
  \begin{abstract}
  Non-Muscle Invasive Bladder Cancer (NMIBC) accounts for 70-80\% of the cases of bladder cancer. The gold-standard therapy for NMIBC are a transurethral resection of the lesion and an intravesical injection of Bacillus Calmette-Guerin (BCG). Immunotherapy using BCG are associated with the reduction of tumor recurrence and progression,  but only approximately 50\% of the patients benefit from this therapy and approximately 20\% of the BCG treated patients interrupt the therapy due to side effects. The underlying mechanisms associated with the response of BCG immunotherapy are not yet well understood and there is not an available biomarker for response. Here,  we sequenced the whole exome (WXS) from tumor of 35 (17 responsive,  BCG-R; and 18 unresponsive,  BCG-UR) high-risk NMIBC patients from Instituto do C\^ancer do Estado de S\~ao Paulo (ICESP) and performed a variant calling pipeline in order to find genomic variables associated with patient outcome. For the variant calling,  we used the GATK best practices for variants calling and thereafter we filtered out germline variants by the presence in populational variants database (ExAC and 1000 Genomes) and the recurrence in our cohort.  Our results show differences of tumor mutational burden (TMB) between BCG-R and BCG-UR (p-value = 0.045),  in which the low-TMB group showed a higher relapse-free survival than the high-TMB group (p-value = 0.0092). We also evaluated tumor heterogeneity by Mutant-Allele Tumor Heterogeneity (MATH) score,  however no statistical significance was found between BCG-R and BCG-UR. In the end,  we found an import result to non-muscle invasive bladder cancer patients: TMB as a potential predictive biomarker for BCG immunotherapy response.
  
  Funding: CNPq \\ 
  \end{abstract}
  \end{document} 