
  \documentclass[twoside]{article}
  \usepackage[affil-it]{authblk}
  \usepackage{lipsum} % Package to generate dummy text throughout this template
  \usepackage{eurosym}
  \usepackage[sc]{mathpazo} % Use the Palatino font
  \usepackage[T1]{fontenc} % Use 8-bit encoding that has 256 glyphs
  \usepackage[utf8]{inputenc}
  \linespread{1.05} % Line spacing-Palatino needs more space between lines
  \usepackage{microtype} % Slightly tweak font spacing for aesthetics\[IndentingNewLine]
  \usepackage[hmarginratio=1:1,top=32mm,columnsep=20pt]{geometry} % Document margins
  \usepackage{multicol} % Used for the two-column layout of the document
  \usepackage[hang,small,labelfont=bf,up,textfont=it,up]{caption} % Custom captions under//above floats in tables or figures
  \usepackage{booktabs} % Horizontal rules in tables
  \usepackage{float} % Required for tables and figures in the multi-column environment-they need to be placed in specific locations with the[H] (e.g. \begin{table}[H])
  \usepackage{hyperref} % For hyperlinks in the PDF
  \usepackage{lettrine} % The lettrine is the first enlarged letter at the beginning of the text
  \usepackage{paralist} % Used for the compactitem environment which makes bullet points with less space between them
  \usepackage{abstract} % Allows abstract customization
  \renewcommand{\abstractnamefont}{\normalfont\bfseries} 
  %\renewcommand{\abstracttextfont}{\normalfont\small\itshape} % Set the abstract itself to small italic text\[IndentingNewLine]
  \usepackage{titlesec} % Allows customization of titles
  \renewcommand\thesection{\Roman{section}} % Roman numerals for the sections
  \renewcommand\thesubsection{\Roman{subsection}} % Roman numerals for subsections
  \titleformat{\section}[block]{\large\scshape\centering}{\thesection.}{1em}{} % Change the look of the section titles
  \titleformat{\subsection}[block]{\large}{\thesubsection.}{1em}{} % Change the look of the section titles
  \usepackage{fancyhdr} % Headers and footers
  \pagestyle{fancy} % All pages have headers and footers
  \fancyhead{} % Blank out the default header
  \fancyfoot{} % Blank out the default footer
  \fancyhead[C]{X-meeting $\bullet$ October 2019 $\bullet$ Campos do  Jord\~ao} % Custom header text
  \fancyfoot[RO,LE]{} % Custom footer text
  %----------------------------------------------------------------------------------------
  % TITLE SECTION
  %---------------------------------------------------------------------------------------- 
 
 \title{\vspace{-15mm}\fontsize{24pt}{10pt}\selectfont\textbf{ Comparative genomics analysis and classification of the Lactobacillus casei species }} % Article title
  
  
  \author{ Rodrigo Bentes Kato, Diego Lucas Neres Rodrigues, Juan Luis Valdez Baez, Roselane Gon\c{c}alves dos Santos, Stephane Fraga de Oliveira Tosta, Alessandra Lima da Silva, anne cybelle pinto gomide, Alfonso Gala-Garcia, Francielly Rodrigues da Costa, Vasco Ariston de Carvalho Azevedo }
  
  \affil{ Universidade Federal de Minas Gerais }
  \vspace{-5mm}
  \date{}
  
  %---------------------------------------------------------------------------------------- 
  
  \begin{document}
  
  
  \maketitle % Insert title
  
  
  \thispagestyle{fancy} % All pages have headers and footers
  %----------------------------------------------------------------------------------------  
  % ABSTRACT
  
  %----------------------------------------------------------------------------------------  
  
  \begin{abstract}
  Lactobacillus taxon has been widely studied for its probiotic characteristics,  including immunomodulatory and metabolic properties. The World Health Organization (WHO) points out that the genus and species classification of any probiotic is crucial before its use,  genomic studies of lactic bacteria are of fundamental importance in the legal,  biological and safety aspects. Therefore,  gene blocks related to pathogenicity mechanisms should be studied to ensure the use of bacteria in the standards provided by WHO,  although the literature indicates that some are symbiosis mechanisms. The aim of the present study was to compare the genome sequence of six strains of Lactobacillus casei and to show possible divergence points capable of segregating them from the others in the genus. The six complete L. casei genomes were obtained from the platform (NCBI) and standardized for annotation using the PROKKA software. For the comparison was developed a pipeline based on the Average Nucleotide Identity (ANI),  performed on the JSpeciesWS platform,  and on the construction of a phylogenomic tree of individuals of the genus using the PATRIC platform. Our analysis of genomic plasticity was performed using the visualization tools GIPSy and BRIG. We obtained images showing significant divergent points between the genomes applied to the study,  as well as a low nucleotide identity. Islands of pathogenicity totally or partially shared among all genomes were detected. This last result suggests that the use of this bacterium is not safe to use as a probiotic. There is still great difficulty in classifying L. casei at the species level. As phylogenetically,  L. casei is still very confused within the genus as other family members,  which negatively influences its taxonomic identification which is crucial as a safety measure for its use in food.
  
  Funding: CNPq,  CAPES,  FAPEMIG \\ 
  \end{abstract}
  \end{document} 