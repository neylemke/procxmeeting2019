
  \documentclass[twoside]{article}
  \usepackage[affil-it]{authblk}
  \usepackage{lipsum} % Package to generate dummy text throughout this template
  \usepackage{eurosym}
  \usepackage[sc]{mathpazo} % Use the Palatino font
  \usepackage[T1]{fontenc} % Use 8-bit encoding that has 256 glyphs
  \usepackage[utf8]{inputenc}
  \linespread{1.05} % Line spacing-Palatino needs more space between lines
  \usepackage{microtype} % Slightly tweak font spacing for aesthetics\[IndentingNewLine]
  \usepackage[hmarginratio=1:1,top=32mm,columnsep=20pt]{geometry} % Document margins
  \usepackage{multicol} % Used for the two-column layout of the document
  \usepackage[hang,small,labelfont=bf,up,textfont=it,up]{caption} % Custom captions under//above floats in tables or figures
  \usepackage{booktabs} % Horizontal rules in tables
  \usepackage{float} % Required for tables and figures in the multi-column environment-they need to be placed in specific locations with the[H] (e.g. \begin{table}[H])
  \usepackage{hyperref} % For hyperlinks in the PDF
  \usepackage{lettrine} % The lettrine is the first enlarged letter at the beginning of the text
  \usepackage{paralist} % Used for the compactitem environment which makes bullet points with less space between them
  \usepackage{abstract} % Allows abstract customization
  \renewcommand{\abstractnamefont}{\normalfont\bfseries} 
  %\renewcommand{\abstracttextfont}{\normalfont\small\itshape} % Set the abstract itself to small italic text\[IndentingNewLine]
  \usepackage{titlesec} % Allows customization of titles
  \renewcommand\thesection{\Roman{section}} % Roman numerals for the sections
  \renewcommand\thesubsection{\Roman{subsection}} % Roman numerals for subsections
  \titleformat{\section}[block]{\large\scshape\centering}{\thesection.}{1em}{} % Change the look of the section titles
  \titleformat{\subsection}[block]{\large}{\thesubsection.}{1em}{} % Change the look of the section titles
  \usepackage{fancyhdr} % Headers and footers
  \pagestyle{fancy} % All pages have headers and footers
  \fancyhead{} % Blank out the default header
  \fancyfoot{} % Blank out the default footer
  \fancyhead[C]{X-meeting $\bullet$ October 2019 $\bullet$ Campos do  Jord\~ao} % Custom header text
  \fancyfoot[RO,LE]{} % Custom footer text
  %----------------------------------------------------------------------------------------
  % TITLE SECTION
  %---------------------------------------------------------------------------------------- 
 
 \title{\vspace{-15mm}\fontsize{24pt}{10pt}\selectfont\textbf{ Genomic and epidemiologycal analyses of Manhheimia haemolytica strains }} % Article title
  
  
  \author{ Raquel Enma Hurtado Castillo, Jana\'{\i}na Can\'ario Cerqueira, Rodrigo Profeta Silveira Santos, Marcus Vinicius Can\'ario Viana, Vasco A de C Azevedo }
  
  \affil{ UFMG }
  \vspace{-5mm}
  \date{}
  
  %---------------------------------------------------------------------------------------- 
  
  \begin{document}
  
  
  \maketitle % Insert title
  
  
  \thispagestyle{fancy} % All pages have headers and footers
  %----------------------------------------------------------------------------------------  
  % ABSTRACT
  
  %----------------------------------------------------------------------------------------  
  
  \begin{abstract}
  Mannheimia haemolytica is a gram-negative bacterium,  commensal and opportunistic pathogen,  and the primary agent of respiratory infections on ruminants. This bacterium causes bovine respiratory disease,  a disease that generates a great economic loss in the cattle industry. Some pathogenic strains are associated to a specific serotype and the presence of integrative conjugative elements (ICEs) containing multi-drug resistance genes. To characterize the genomic features,  pathogenesis,  and distribution of this specie,  we performed a genomic and epidemiological analyses of 113 M. haemolytica strains from diverse hosts,  mostly cattle,  and clinical or non-clinical status. The serotypes were classified as serotype 1 (48.67\%),  2 (31.85\%),  6 (13.27\%) and unknown (6.19\%). According to Multilocus Sequence Typing (MLST),  the strains were classified as  ST1 (61.94\%),  ST2 (30.08\%),  ST7 (2.26\%),  ST47 (1.13\%),  ST3 (1.13\%) and unknown (3.39\%). The pangenome was estimated as open. Phylogenetic analysis using whole-genome single nucleotide polymorphisms (SNPs) showed that most strains with the same serotype were clustered together. Principal Component Analysis (PCA) based on the accessory genes segregated all identified serotypes. The serotypes 1 and 6 are discretely segregated,  but all strains belong to sequence type ST1,  while all serotype 2 strains belong to sequence type ST2. Strains with unknown serotype did not form clusters. The integrative conjugative elements ICEPmu1 and ICEMh1 were present only in strains from USA,  in both clinical and non-clinical sample,  but not in all strains. The results suggest that the in silico-identified serotypes could be discriminated by SNP. In addition,  the serotype 2 could be differentiated from serotypes 1 and 6 by the accessory genome. The association of genomic features with clinical or non-clinical isolates,  and host species could not be evaluated due to lack of available data. Integrative conjugative elements were reported only in USA isolates,  but were not specific to clinical or non-clinical isolates. Virulence factor distribution across strains will be performed in further analyses. Our findings identified genomic features that could be associated to serotypes and geographical location that could help to develop strategies for surveillance,  control and prevention on respiratory infections by M. haemolytica.
  
  Funding:  \\ 
  \end{abstract}
  \end{document} 