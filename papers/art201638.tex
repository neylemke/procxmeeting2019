
  \documentclass[twoside]{article}
  \usepackage[affil-it]{authblk}
  \usepackage{lipsum} % Package to generate dummy text throughout this template
  \usepackage{eurosym}
  \usepackage[sc]{mathpazo} % Use the Palatino font
  \usepackage[T1]{fontenc} % Use 8-bit encoding that has 256 glyphs
  \usepackage[utf8]{inputenc}
  \linespread{1.05} % Line spacing-Palatino needs more space between lines
  \usepackage{microtype} % Slightly tweak font spacing for aesthetics\[IndentingNewLine]
  \usepackage[hmarginratio=1:1,top=32mm,columnsep=20pt]{geometry} % Document margins
  \usepackage{multicol} % Used for the two-column layout of the document
  \usepackage[hang,small,labelfont=bf,up,textfont=it,up]{caption} % Custom captions under//above floats in tables or figures
  \usepackage{booktabs} % Horizontal rules in tables
  \usepackage{float} % Required for tables and figures in the multi-column environment-they need to be placed in specific locations with the[H] (e.g. \begin{table}[H])
  \usepackage{hyperref} % For hyperlinks in the PDF
  \usepackage{lettrine} % The lettrine is the first enlarged letter at the beginning of the text
  \usepackage{paralist} % Used for the compactitem environment which makes bullet points with less space between them
  \usepackage{abstract} % Allows abstract customization
  \renewcommand{\abstractnamefont}{\normalfont\bfseries} 
  %\renewcommand{\abstracttextfont}{\normalfont\small\itshape} % Set the abstract itself to small italic text\[IndentingNewLine]
  \usepackage{titlesec} % Allows customization of titles
  \renewcommand\thesection{\Roman{section}} % Roman numerals for the sections
  \renewcommand\thesubsection{\Roman{subsection}} % Roman numerals for subsections
  \titleformat{\section}[block]{\large\scshape\centering}{\thesection.}{1em}{} % Change the look of the section titles
  \titleformat{\subsection}[block]{\large}{\thesubsection.}{1em}{} % Change the look of the section titles
  \usepackage{fancyhdr} % Headers and footers
  \pagestyle{fancy} % All pages have headers and footers
  \fancyhead{} % Blank out the default header
  \fancyfoot{} % Blank out the default footer
  \fancyhead[C]{X-meeting $\bullet$ October 2019 $\bullet$ Campos do  Jord\~ao} % Custom header text
  \fancyfoot[RO,LE]{} % Custom footer text
  %----------------------------------------------------------------------------------------
  % TITLE SECTION
  %---------------------------------------------------------------------------------------- 
 
 \title{\vspace{-15mm}\fontsize{24pt}{10pt}\selectfont\textbf{ Evolution of lignocellulose degradation characterizes the adaptation for heterotrophy to carbohydrates that appeared in Fungi }} % Article title
  
  
  \author{ Fen\'{\i}cia Brito, TETSU SAKAMOTO, Jos\'e Miguel Ortega }
  
  \affil{ Universidade Federal de Minas Gerais }
  \vspace{-5mm}
  \date{}
  
  %---------------------------------------------------------------------------------------- 
  
  \begin{document}
  
  
  \maketitle % Insert title
  
  
  \thispagestyle{fancy} % All pages have headers and footers
  %----------------------------------------------------------------------------------------  
  % ABSTRACT
  
  %----------------------------------------------------------------------------------------  
  
  \begin{abstract}
  Lignocellulose is the most abundant terrestrial biopolymer on Earth. Cellulose is a high molecular weight linear homopolymer of repeated units of glucose,  containing highly crystalline regions and less-ordered amorphous regions. Hemicellulose is a linear and branched heterogeneous polymer typically composed of five sugars: L-arabinose,  D-galactose,  D-glucose,  D-mannose and D-xylose,  and acetic,  glucuronic and ferulic acids. The backbone of hemicelluloses chains can be a homopolymer or a heteropolymer. Lignin is a very complex molecule constructed of phenylpropane units linked in a large three-dimensional structure. Lignin is closely bound to cellulose and hemicellulose and are particularly important in the formation of cell walls,  especially in wood and bark,  because they lend rigidity and do not rot easily. Fungal ability as decomposers of organic matter is widely known. When it comes to the degradation of lignocellulose fungi also stand out for their efficiency. Although some bacteria are known to decompose the lignocellulose polymer at some level,  we wondered which organisms have the essencial lignocellulosic enzymes and hence when the lignocellulose degradation has emerged. In this work we investigated 7, 957 proteomes from the UniProt Reference Proteomes database for the set of enzymes for lignocellulose degradation. Using software TaxOnTree to performed phylogenetic and taxonomic distribution analysis,  we were able to identify which organisms present which proteins and described their potential mechanism for decomposing the lignocellulose polymer. Some organisms can decompose the celulose fiber from its ends,  by using a set of two enzymes: Cellobiose dehydrogenase or Cellobiohydrolase and Beta-glucosidase. Besides fungi,  these enzymes were found in bacteria and in some metazoans. Taxonomic distribution suggests either a Lateral Gene Transfer (LGT) event or great deletions from an ancient ancestor whether the enzymes lack function. The two enzymes implicated in the breakdown of the xylan portion of hemicellulose,  Beta-xylanase and Beta-xylosidase,  were found in two orders of Halobacteria,  different bacteria phyla (e.g. Bacteroidetes,  Acidobacteria,  Proteobacteria and Actinobacteria),  in fungi and plants. The two enzymes for the breakdown of the manan backbone in hemicellulose,  Beta-mannosidase and Endo-1, 4-beta-mannosidase,  were found in fungi,  plants and a few bacteria phyla (Bacteroidetes,  Proteobacteria,  Actinobacteria and Chloroflexi). As for the lignin decomposition,  the main enzymes involved in these process,  ligninases MnP,  LP and VP in the Agaricomycetes class,  although some other peroxidases,  also found in bacteria,  can also participate of the lignin breakdown. Remarkably,  only fungi have the complete set of enzymes for decomposition of the entire polymer of celulose,  hemicelulose and lignin. These results suggests that either LGT was a driving force on spreading these enzymes in species of bacteria and metazoans,  or great event of deletions avoided the fixation of such activities originated in a very ancient ancestor. By inspecting the distribution,  we favor the possibility of the effective digestion have been originated in fungi with ocasional LGT events. It seems that the proper mechanisms for the breakdown of the cell wall components and wood rotting emerged in fungi,  simultaneously with their adaptation for the heterotrophy. The ability of degrading these polymers might have been important at the end of Carbonifera period for the cycling of fixed carbon,  and describes the characteristics of heterotrophy for carbohydrates that appeared in the clade of fungi.
  
  Funding: CAPES; Computational Biology Networks: Biologia Sist\^emica do C\^ancer,  BSC \\ 
  \end{abstract}
  \end{document} 