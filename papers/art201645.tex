
  \documentclass[twoside]{article}
  \usepackage[affil-it]{authblk}
  \usepackage{lipsum} % Package to generate dummy text throughout this template
  \usepackage{eurosym}
  \usepackage[sc]{mathpazo} % Use the Palatino font
  \usepackage[T1]{fontenc} % Use 8-bit encoding that has 256 glyphs
  \usepackage[utf8]{inputenc}
  \linespread{1.05} % Line spacing-Palatino needs more space between lines
  \usepackage{microtype} % Slightly tweak font spacing for aesthetics\[IndentingNewLine]
  \usepackage[hmarginratio=1:1,top=32mm,columnsep=20pt]{geometry} % Document margins
  \usepackage{multicol} % Used for the two-column layout of the document
  \usepackage[hang,small,labelfont=bf,up,textfont=it,up]{caption} % Custom captions under//above floats in tables or figures
  \usepackage{booktabs} % Horizontal rules in tables
  \usepackage{float} % Required for tables and figures in the multi-column environment-they need to be placed in specific locations with the[H] (e.g. \begin{table}[H])
  \usepackage{hyperref} % For hyperlinks in the PDF
  \usepackage{lettrine} % The lettrine is the first enlarged letter at the beginning of the text
  \usepackage{paralist} % Used for the compactitem environment which makes bullet points with less space between them
  \usepackage{abstract} % Allows abstract customization
  \renewcommand{\abstractnamefont}{\normalfont\bfseries} 
  %\renewcommand{\abstracttextfont}{\normalfont\small\itshape} % Set the abstract itself to small italic text\[IndentingNewLine]
  \usepackage{titlesec} % Allows customization of titles
  \renewcommand\thesection{\Roman{section}} % Roman numerals for the sections
  \renewcommand\thesubsection{\Roman{subsection}} % Roman numerals for subsections
  \titleformat{\section}[block]{\large\scshape\centering}{\thesection.}{1em}{} % Change the look of the section titles
  \titleformat{\subsection}[block]{\large}{\thesubsection.}{1em}{} % Change the look of the section titles
  \usepackage{fancyhdr} % Headers and footers
  \pagestyle{fancy} % All pages have headers and footers
  \fancyhead{} % Blank out the default header
  \fancyfoot{} % Blank out the default footer
  \fancyhead[C]{X-meeting $\bullet$ October 2019 $\bullet$ Campos do  Jord\~ao} % Custom header text
  \fancyfoot[RO,LE]{} % Custom footer text
  %----------------------------------------------------------------------------------------
  % TITLE SECTION
  %---------------------------------------------------------------------------------------- 
 
 \title{\vspace{-15mm}\fontsize{24pt}{10pt}\selectfont\textbf{ If menstruation is recent in evolution of great primates,  how ancient are the genes that control its periods? }} % Article title
  
  
  \author{ Andre Luiz Garcia de Oliveira, Arthur Pereira da Fonseca, Jos\'e Miguel Ortega }
  
  \affil{ Universidade Federal de Minas Gerais }
  \vspace{-5mm}
  \date{}
  
  %---------------------------------------------------------------------------------------- 
  
  \begin{document}
  
  
  \maketitle % Insert title
  
  
  \thispagestyle{fancy} % All pages have headers and footers
  %----------------------------------------------------------------------------------------  
  % ABSTRACT
  
  %----------------------------------------------------------------------------------------  
  
  \begin{abstract}
  Orthologues grouping provides a better knowledge of the sequences producing a certain function,  but allow us to verify the taxonomic distribution and,  after that,  infer the clade of origin of novel functions. With this in mind we set up to study the origin of genes implicated with menstruation. Menstruation is a rare event in mammals,  but it is an essential phase for the human reproductive cycle of some great primates in the ancient world. In them,  ovarian steroid hormones regulate endometrial function and human menstruation. In humans,  after ovulation,  the corpus luteum secretes high levels of progesterone to maintain endometrial receptivity if fertilization occurs. Without trophoblast implantation and decidualization,  the corpus luteum regresses,  causing a marked decline in circulating progesterone levels. This triggers a local inflammatory response in the endometrium involving leukocyte infiltration,  cytokine release,  edema,  and activation of matrix metalloproteinases (MMP-1 and MMP-3) and strong contractions caused by prostaglandins. The result is tissue rupture and the fall of the upper two-thirds of the endometrium,  the functional layer,  during the menstrual phase of the cycle. 
We conducted a study on the origin of the main proteins that act in endometrial degradation,  we can draw an evolutionary line to unravel the mystery surrounding this event. Using text mining,  the MedlineRanker and PESCADOR programs,  we have constructed a pathway for endometrial degradation comprising 28 genes. The protein sequences were used to build FastTree phylogenetic trees after multiple alignments with Muscle,  and the ladistic distribution was analyzed with the TaxOnTree web tool of our group.
The origin of the pathway requires the appearance of placentals,  with some parts appearing in mammals and one is restricted to the order Catarrhini. For example,  the LEFTY-2 gene,  an inhibitor that positively regulates the transcription of ECM-degrading MMPs,  originated in Catarrhini,  however,  the other member of his family that acts on the pathway,  LEFTY-1,  appeared already in Eutheria. According to this scenario,  Plasmines,  which also participate in ECM's degradation process,  emerged in Theria. However,  Plasminogen PLAU,  the plasmin precursor,  can only be converted by the PLAUR receptor when it emerged in Eutheria and its ligand,  which is plasminogen itself,  originated in Mammalia. In conclusion,  the pathway has a recent origin bias,  however the data we present do not map functions restricted to the family Hominidae of the great primates and more studies are needed.
  
  Funding: CAPES Computational Biology Networks: Biologia Sist\^emica do C\^ancer,  BSC. \\ 
  \end{abstract}
  \end{document} 