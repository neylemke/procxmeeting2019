
  \documentclass[twoside]{article}
  \usepackage[affil-it]{authblk}
  \usepackage{lipsum} % Package to generate dummy text throughout this template
  \usepackage{eurosym}
  \usepackage[sc]{mathpazo} % Use the Palatino font
  \usepackage[T1]{fontenc} % Use 8-bit encoding that has 256 glyphs
  \usepackage[utf8]{inputenc}
  \linespread{1.05} % Line spacing-Palatino needs more space between lines
  \usepackage{microtype} % Slightly tweak font spacing for aesthetics\[IndentingNewLine]
  \usepackage[hmarginratio=1:1,top=32mm,columnsep=20pt]{geometry} % Document margins
  \usepackage{multicol} % Used for the two-column layout of the document
  \usepackage[hang,small,labelfont=bf,up,textfont=it,up]{caption} % Custom captions under//above floats in tables or figures
  \usepackage{booktabs} % Horizontal rules in tables
  \usepackage{float} % Required for tables and figures in the multi-column environment-they need to be placed in specific locations with the[H] (e.g. \begin{table}[H])
  \usepackage{hyperref} % For hyperlinks in the PDF
  \usepackage{lettrine} % The lettrine is the first enlarged letter at the beginning of the text
  \usepackage{paralist} % Used for the compactitem environment which makes bullet points with less space between them
  \usepackage{abstract} % Allows abstract customization
  \renewcommand{\abstractnamefont}{\normalfont\bfseries} 
  %\renewcommand{\abstracttextfont}{\normalfont\small\itshape} % Set the abstract itself to small italic text\[IndentingNewLine]
  \usepackage{titlesec} % Allows customization of titles
  \renewcommand\thesection{\Roman{section}} % Roman numerals for the sections
  \renewcommand\thesubsection{\Roman{subsection}} % Roman numerals for subsections
  \titleformat{\section}[block]{\large\scshape\centering}{\thesection.}{1em}{} % Change the look of the section titles
  \titleformat{\subsection}[block]{\large}{\thesubsection.}{1em}{} % Change the look of the section titles
  \usepackage{fancyhdr} % Headers and footers
  \pagestyle{fancy} % All pages have headers and footers
  \fancyhead{} % Blank out the default header
  \fancyfoot{} % Blank out the default footer
  \fancyhead[C]{X-meeting $\bullet$ October 2019 $\bullet$ Campos do  Jord\~ao} % Custom header text
  \fancyfoot[RO,LE]{} % Custom footer text
  %----------------------------------------------------------------------------------------
  % TITLE SECTION
  %---------------------------------------------------------------------------------------- 
 
 \title{\vspace{-15mm}\fontsize{24pt}{10pt}\selectfont\textbf{ GENOMIC SURVEILLANCE OF ZIKA AND CHIKUNGUNYA VIRUS IN MINAS GERAIS,  BRAZIL }} % Article title
  
  
  \author{ FELIPE CAMPOS DE MELO IANI, Marta Giovanetti, Jaqueline Goes de Jesus, Talita \'Emile Ribeiro Adelino, Maira Alves Pereira, Joilson Xavier dos Santos Junior, Vagner de Souza Fonseca, Julien Theze, Ester Cerdeira Sabino, Marluce Aparecida Assun\c{c}\~ao Oliveira, Aristeu Mascarenhas da Fonseca, Flavia Salles, Nuno Rodrigues Faria, Luiz Carlos Junior Alcantara }
  
  \affil{ Laborat\'orio de Gen\'etica Celular e Molecular,  ICB,  Universidade Federal de Minas Gerais,  Belo Horizonte,  Minas Gerais,  Brazil }
  \vspace{-5mm}
  \date{}
  
  %---------------------------------------------------------------------------------------- 
  
  \begin{document}
  
  
  \maketitle % Insert title
  
  
  \thispagestyle{fancy} % All pages have headers and footers
  %----------------------------------------------------------------------------------------  
  % ABSTRACT
  
  %----------------------------------------------------------------------------------------  
  
  \begin{abstract}
  Recent emergencies of chikungunya (CHIKV) and zika (ZIKV) viruses have raised serious concerns due to the virus’ rapid dissemination into new geographic areas and the clinical features associated with infection. Using a combination of portable whole genome sequencing and molecular clock analyses,  we investigated the genomic diversity and epidemiological dynamics of CHIKV and ZIKV in different municipalities of Minas Gerais state (MG),  Southeast Brazil. MG had the first CHIKV confirmed case in September 2014 but was confirmed as an imported case from Venezuela. MG has since 2014 up to date,  a total of 31, 095 probable cases and of these,  275 CHIKV cases have been RT-qPCR laboratory-confirmed. The first Brazilian cases of autochthonous transmission of the ZIKV were confirmed in May 2015 in northeast Brazil. However,  the first laboratory confirmed case in MG was in February 2016. MG has since 2014 up to date,  a total of 15, 644 probable cases and of these,  1, 609 ZIKV cases have been RT-qPCR laboratory-confirmed.
We generate 14 CHIKV and 7 ZIKV near-complete genomes sequences of virus isolate obtained directly from clinical samples. Our phylogenetic reconstructions indicated the co-circulation of the CHIKV - East-Central-South-African (ECSA) lineage and the ZIKV Asian genotype in MG state. Time-measured phylogenetic analysis revealed the CHIKV ECSA lineage was introduced in Minas Gerais around November 2014 (95\% Bayesian credible interval: May 2014 to April 2015) and that the ZIKV – Asian genotype was introduced around July 2014 (95\% Bayesian credible interval: May to December 2014). Beside CHIKV circulation in the state has been quickly detected by the surveillance system,  our data indicate that ZIKV probably has circulated unnoticed for 18 months before the first confirmed laboratory case. These findings reinforce that continued genomic surveillance strategies are needed to assist in the monitoring and understanding of arbovirus epidemics,  which might help to attenuate the public health impact of infectious diseases.
  
  Funding: Fapemig,  CNPq,  Capes,  ZIKAlliance,  Royal Society and Wellcome Trust Sir Henry Dale Fellowship,  John Fell Research Fund,  ERC,  \\ 
  \end{abstract}
  \end{document} 