
  \documentclass[twoside]{article}
  \usepackage[affil-it]{authblk}
  \usepackage{lipsum} % Package to generate dummy text throughout this template
  \usepackage{eurosym}
  \usepackage[sc]{mathpazo} % Use the Palatino font
  \usepackage[T1]{fontenc} % Use 8-bit encoding that has 256 glyphs
  \usepackage[utf8]{inputenc}
  \linespread{1.05} % Line spacing-Palatino needs more space between lines
  \usepackage{microtype} % Slightly tweak font spacing for aesthetics\[IndentingNewLine]
  \usepackage[hmarginratio=1:1,top=32mm,columnsep=20pt]{geometry} % Document margins
  \usepackage{multicol} % Used for the two-column layout of the document
  \usepackage[hang,small,labelfont=bf,up,textfont=it,up]{caption} % Custom captions under//above floats in tables or figures
  \usepackage{booktabs} % Horizontal rules in tables
  \usepackage{float} % Required for tables and figures in the multi-column environment-they need to be placed in specific locations with the[H] (e.g. \begin{table}[H])
  \usepackage{hyperref} % For hyperlinks in the PDF
  \usepackage{lettrine} % The lettrine is the first enlarged letter at the beginning of the text
  \usepackage{paralist} % Used for the compactitem environment which makes bullet points with less space between them
  \usepackage{abstract} % Allows abstract customization
  \renewcommand{\abstractnamefont}{\normalfont\bfseries} 
  %\renewcommand{\abstracttextfont}{\normalfont\small\itshape} % Set the abstract itself to small italic text\[IndentingNewLine]
  \usepackage{titlesec} % Allows customization of titles
  \renewcommand\thesection{\Roman{section}} % Roman numerals for the sections
  \renewcommand\thesubsection{\Roman{subsection}} % Roman numerals for subsections
  \titleformat{\section}[block]{\large\scshape\centering}{\thesection.}{1em}{} % Change the look of the section titles
  \titleformat{\subsection}[block]{\large}{\thesubsection.}{1em}{} % Change the look of the section titles
  \usepackage{fancyhdr} % Headers and footers
  \pagestyle{fancy} % All pages have headers and footers
  \fancyhead{} % Blank out the default header
  \fancyfoot{} % Blank out the default footer
  \fancyhead[C]{X-meeting $\bullet$ October 2019 $\bullet$ Campos do  Jord\~ao} % Custom header text
  \fancyfoot[RO,LE]{} % Custom footer text
  %----------------------------------------------------------------------------------------
  % TITLE SECTION
  %---------------------------------------------------------------------------------------- 
 
 \title{\vspace{-15mm}\fontsize{24pt}{10pt}\selectfont\textbf{ MOLECULAR MODELING METHODS APPLIED TO THE STUDY OF Staphylococcus aureus TARGET PROTEINS }} % Article title
  
  
  \author{ William Mesquita da Costa, Levy Bueno Alves, Nelson Jos\'e Freitas da Silveira, Patr\'{\i}cia da Silva Antunes }
  
  \affil{ Universidade Federal de Alfenas }
  \vspace{-5mm}
  \date{}
  
  %---------------------------------------------------------------------------------------- 
  
  \begin{document}
  
  
  \maketitle % Insert title
  
  
  \thispagestyle{fancy} % All pages have headers and footers
  %----------------------------------------------------------------------------------------  
  % ABSTRACT
  
  %----------------------------------------------------------------------------------------  
  
  \begin{abstract}
  The application of computational methods in the rational planning of new drugs has made possible the construction of biological and chemical models that,  when subjected to specific software,  allow visualizing,  simulating and interpreting systems involved in protein-ligand interaction. From this point of view,  this work was carried out with the aim of using molecular modeling on target proteins of the Staphylococcus aureus,  in order to identify molecular patterns with pharmacodynamic characteristics and inhibitory action spectrum in three enzymes present in the pathogen peptidoglycan biosynthesis,  the following being: murD,  ddI  e uppP. From the primary sequence of murD and uppP proteins,  3D models were constructed with the help of the Modeller program. The murD and uppP sequences were obtained from the UNIPROT database,  with codes P0A090 and P67391,  respectively. The sequences were submitted to the PSI-BLAST algorithm in the PDB database for templates identification. Among the returned templates,  the crystallographic structures 3LK7 and 6CB2,  similar to murD and uppP proteins,  were selected,  respectively. The ddI protein was recovered from the PDB database under code 2I87 and the models were optimized. After model validation,  both were submitted to molecular docking using the Glide dock-XP program,  considering a virtual screening containing 1046 ligands for the identification of leading compounds. The best models generated by Modeller were murD\_0332 and uppP\_0703,  both presenting 93.8\% of amino acids residues in favorable regions on the Ramachandran plot. After optimization,  the models showed good stereochemical results. The molecular docking study revealed that aminoglycoside compounds can anchor at murD,  uppP and ddI binding sites simultaneously. From the structural information obtained in the virtual screening it was possible to design new molecular patterns considering ADMET pharmacokinetic aspects. Through the proposed molecular patterns it was possible to generate bioisosteres and,  through new molecular docking analysis,  to propose a new oral administration in silico drug.
  
  Funding:  \\ 
  \end{abstract}
  \end{document} 