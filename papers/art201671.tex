
  \documentclass[twoside]{article}
  \usepackage[affil-it]{authblk}
  \usepackage{lipsum} % Package to generate dummy text throughout this template
  \usepackage{eurosym}
  \usepackage[sc]{mathpazo} % Use the Palatino font
  \usepackage[T1]{fontenc} % Use 8-bit encoding that has 256 glyphs
  \usepackage[utf8]{inputenc}
  \linespread{1.05} % Line spacing-Palatino needs more space between lines
  \usepackage{microtype} % Slightly tweak font spacing for aesthetics\[IndentingNewLine]
  \usepackage[hmarginratio=1:1,top=32mm,columnsep=20pt]{geometry} % Document margins
  \usepackage{multicol} % Used for the two-column layout of the document
  \usepackage[hang,small,labelfont=bf,up,textfont=it,up]{caption} % Custom captions under//above floats in tables or figures
  \usepackage{booktabs} % Horizontal rules in tables
  \usepackage{float} % Required for tables and figures in the multi-column environment-they need to be placed in specific locations with the[H] (e.g. \begin{table}[H])
  \usepackage{hyperref} % For hyperlinks in the PDF
  \usepackage{lettrine} % The lettrine is the first enlarged letter at the beginning of the text
  \usepackage{paralist} % Used for the compactitem environment which makes bullet points with less space between them
  \usepackage{abstract} % Allows abstract customization
  \renewcommand{\abstractnamefont}{\normalfont\bfseries} 
  %\renewcommand{\abstracttextfont}{\normalfont\small\itshape} % Set the abstract itself to small italic text\[IndentingNewLine]
  \usepackage{titlesec} % Allows customization of titles
  \renewcommand\thesection{\Roman{section}} % Roman numerals for the sections
  \renewcommand\thesubsection{\Roman{subsection}} % Roman numerals for subsections
  \titleformat{\section}[block]{\large\scshape\centering}{\thesection.}{1em}{} % Change the look of the section titles
  \titleformat{\subsection}[block]{\large}{\thesubsection.}{1em}{} % Change the look of the section titles
  \usepackage{fancyhdr} % Headers and footers
  \pagestyle{fancy} % All pages have headers and footers
  \fancyhead{} % Blank out the default header
  \fancyfoot{} % Blank out the default footer
  \fancyhead[C]{X-meeting $\bullet$ October 2019 $\bullet$ Campos do  Jord\~ao} % Custom header text
  \fancyfoot[RO,LE]{} % Custom footer text
  %----------------------------------------------------------------------------------------
  % TITLE SECTION
  %---------------------------------------------------------------------------------------- 
 
 \title{\vspace{-15mm}\fontsize{24pt}{10pt}\selectfont\textbf{ Identification of intragenic retrocopies in chimaric transcripts in humans }} % Article title
  
  
  \author{ Rafael Luiz Vieira Mercuri, Helena Beatriz da Concei\c{c}\~ao, Pedro Alexandre Favoretto Galante }
  
  \affil{ Instituto de Ensino e Pesquisa,  Hospital S\'{\i}rio-Liban\^es }
  \vspace{-5mm}
  \date{}
  
  %---------------------------------------------------------------------------------------- 
  
  \begin{document}
  
  
  \maketitle % Insert title
  
  
  \thispagestyle{fancy} % All pages have headers and footers
  %----------------------------------------------------------------------------------------  
  % ABSTRACT
  
  %----------------------------------------------------------------------------------------  
  
  \begin{abstract}
  The evolution of a species occurs by the emergence of new genes. These can be originated through LINE1 (L1) mediated gene duplication,  a phenomenon often encountered in the human genome. Although sometimes these genes are not functional,  recent studies have shown that some retrocopies are transcribed and functional. Among all the retrocopies genes encoding a genome,  those inserted into the intronic regions of (other) coding genes deserve attention. Because they are in a gene region,  they may influence the transcription and post-transcriptional processing of the "host gene." It is currently known that there are retrocopies present in the human genome that are located in introns of coding genes. However,  very little is known about the influence and contribution of these retrocopies in relation to their host genes. The aim of this work was to elaborate a systematic study of the retrocopies inserted in introns and exons of human coding genes (intragenic). The RCPedia (https://www.bioinfo.mochsl.org.br/rcpedia/) was used as a database to identify intragenic retrocopies in humans. Subsequently a comparison was made with GTEx (https://gtexportal.org/) to verify the expression profile in 53 human tissues of host genes and their intragenic retrocopies. First,  we found 2499 intragenic retrocopies (990 in the same transcription strand and 1509 in the opposite strand to their host genes) and of these,  65\% (1630 retrocopies) had their expression confirmed by GTEx. Testis presented the highest number of expressed retrocopies and bladder the lowest. Interestingly,  ,  some of expressed retrocopies are located into genes involved in pathological pathways as Ras Homolog Family G (RHOG). Thus,  our results bring important knowledge and can contribute to a better understanding of the origin of new genes and genetic novelties.
  
  Funding: FAPESP \\ 
  \end{abstract}
  \end{document} 