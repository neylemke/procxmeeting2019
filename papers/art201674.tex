
  \documentclass[twoside]{article}
  \usepackage[affil-it]{authblk}
  \usepackage{lipsum} % Package to generate dummy text throughout this template
  \usepackage{eurosym}
  \usepackage[sc]{mathpazo} % Use the Palatino font
  \usepackage[T1]{fontenc} % Use 8-bit encoding that has 256 glyphs
  \usepackage[utf8]{inputenc}
  \linespread{1.05} % Line spacing-Palatino needs more space between lines
  \usepackage{microtype} % Slightly tweak font spacing for aesthetics\[IndentingNewLine]
  \usepackage[hmarginratio=1:1,top=32mm,columnsep=20pt]{geometry} % Document margins
  \usepackage{multicol} % Used for the two-column layout of the document
  \usepackage[hang,small,labelfont=bf,up,textfont=it,up]{caption} % Custom captions under//above floats in tables or figures
  \usepackage{booktabs} % Horizontal rules in tables
  \usepackage{float} % Required for tables and figures in the multi-column environment-they need to be placed in specific locations with the[H] (e.g. \begin{table}[H])
  \usepackage{hyperref} % For hyperlinks in the PDF
  \usepackage{lettrine} % The lettrine is the first enlarged letter at the beginning of the text
  \usepackage{paralist} % Used for the compactitem environment which makes bullet points with less space between them
  \usepackage{abstract} % Allows abstract customization
  \renewcommand{\abstractnamefont}{\normalfont\bfseries} 
  %\renewcommand{\abstracttextfont}{\normalfont\small\itshape} % Set the abstract itself to small italic text\[IndentingNewLine]
  \usepackage{titlesec} % Allows customization of titles
  \renewcommand\thesection{\Roman{section}} % Roman numerals for the sections
  \renewcommand\thesubsection{\Roman{subsection}} % Roman numerals for subsections
  \titleformat{\section}[block]{\large\scshape\centering}{\thesection.}{1em}{} % Change the look of the section titles
  \titleformat{\subsection}[block]{\large}{\thesubsection.}{1em}{} % Change the look of the section titles
  \usepackage{fancyhdr} % Headers and footers
  \pagestyle{fancy} % All pages have headers and footers
  \fancyhead{} % Blank out the default header
  \fancyfoot{} % Blank out the default footer
  \fancyhead[C]{X-meeting $\bullet$ October 2019 $\bullet$ Campos do  Jord\~ao} % Custom header text
  \fancyfoot[RO,LE]{} % Custom footer text
  %----------------------------------------------------------------------------------------
  % TITLE SECTION
  %---------------------------------------------------------------------------------------- 
 
 \title{\vspace{-15mm}\fontsize{24pt}{10pt}\selectfont\textbf{ Regulatory elements of carbon metabolism in sugarcane }} % Article title
  
  
  \author{ Al\'{\i}cia L.de Melo, Alan Durham, Glaucia Souza Mendes }
  
  \affil{  }
  \vspace{-5mm}
  \date{}
  
  %---------------------------------------------------------------------------------------- 
  
  \begin{document}
  
  
  \maketitle % Insert title
  
  
  \thispagestyle{fancy} % All pages have headers and footers
  %----------------------------------------------------------------------------------------  
  % ABSTRACT
  
  %----------------------------------------------------------------------------------------  
  
  \begin{abstract}
  The increase in the access to renewable energy sources contributes to the global efforts to reduce greenhouse gas emissions. Among these sources,  we highlight the cellulosic bioethanol technology,  which is generated from plant parts,  mainly sugarcane. Knowledge of the components of secondary cell wall metabolism regulation networks will allow the construction of biotechnological tools for the development of energy cane,  which has more fiber and biomass,  by changing the carbon partition of sugar to biomass. Synthetic promoters are a potential tool for the creation of technology that can be applied in the production of transgenic plants with the desired biomass characteristics or by the use of gene editing methodologies such as CRISPR-CAS. Thus,  the characterization of the architecture of the promoters of the target genes involved in the carbon metabolism regulatory networks is essential to provide tools for transgene technology and gene editing. In the present work,  a gene from the sugarcane SP80-3280 genome involved in the starch and sucrose metabolic pathway was selected for the initial testing of the analysis methodologies. An important aspect of motif discovery is the delimitation of the sequences that will be used to search for motifs. This delimitation was based on the location of the transcription start site,  which was determined by three different approaches. Only one of these approaches yielded positive results for the detection of representative motifs. Based on that,  it was possible to identify six potential transcription factor binding sites for this gene. The next steps involve the comparison of these detected motifs to the ones already described in the literature,  as well as applying the methodology to a great number of genes at the same time.
  
  Funding: Funda\c{c}\~ao de Amparo \`a Pesquisa do Estado de S\~ao Paulo \\ 
  \end{abstract}
  \end{document} 