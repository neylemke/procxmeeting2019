
  \documentclass[twoside]{article}
  \usepackage[affil-it]{authblk}
  \usepackage{lipsum} % Package to generate dummy text throughout this template
  \usepackage{eurosym}
  \usepackage[sc]{mathpazo} % Use the Palatino font
  \usepackage[T1]{fontenc} % Use 8-bit encoding that has 256 glyphs
  \usepackage[utf8]{inputenc}
  \linespread{1.05} % Line spacing-Palatino needs more space between lines
  \usepackage{microtype} % Slightly tweak font spacing for aesthetics\[IndentingNewLine]
  \usepackage[hmarginratio=1:1,top=32mm,columnsep=20pt]{geometry} % Document margins
  \usepackage{multicol} % Used for the two-column layout of the document
  \usepackage[hang,small,labelfont=bf,up,textfont=it,up]{caption} % Custom captions under//above floats in tables or figures
  \usepackage{booktabs} % Horizontal rules in tables
  \usepackage{float} % Required for tables and figures in the multi-column environment-they need to be placed in specific locations with the[H] (e.g. \begin{table}[H])
  \usepackage{hyperref} % For hyperlinks in the PDF
  \usepackage{lettrine} % The lettrine is the first enlarged letter at the beginning of the text
  \usepackage{paralist} % Used for the compactitem environment which makes bullet points with less space between them
  \usepackage{abstract} % Allows abstract customization
  \renewcommand{\abstractnamefont}{\normalfont\bfseries} 
  %\renewcommand{\abstracttextfont}{\normalfont\small\itshape} % Set the abstract itself to small italic text\[IndentingNewLine]
  \usepackage{titlesec} % Allows customization of titles
  \renewcommand\thesection{\Roman{section}} % Roman numerals for the sections
  \renewcommand\thesubsection{\Roman{subsection}} % Roman numerals for subsections
  \titleformat{\section}[block]{\large\scshape\centering}{\thesection.}{1em}{} % Change the look of the section titles
  \titleformat{\subsection}[block]{\large}{\thesubsection.}{1em}{} % Change the look of the section titles
  \usepackage{fancyhdr} % Headers and footers
  \pagestyle{fancy} % All pages have headers and footers
  \fancyhead{} % Blank out the default header
  \fancyfoot{} % Blank out the default footer
  \fancyhead[C]{X-meeting $\bullet$ October 2019 $\bullet$ Campos do  Jord\~ao} % Custom header text
  \fancyfoot[RO,LE]{} % Custom footer text
  %----------------------------------------------------------------------------------------
  % TITLE SECTION
  %---------------------------------------------------------------------------------------- 
 
 \title{\vspace{-15mm}\fontsize{24pt}{10pt}\selectfont\textbf{ The exon-Junction Complex Proteins MAGOH and MAGOHB are pro-tumorigenic factors in glioblastoma }} % Article title
  
  
  \author{ Fabiana Marcelino Meliso, Wei-Qing Li, Andr\'e Luiz V. Savio, Bruna R. Correa, Mei Qiao, Pedro A F Galante, Luiz O. Penalva }
  
  \affil{ Hospital Sirio Liban\^es }
  \vspace{-5mm}
  \date{}
  
  %---------------------------------------------------------------------------------------- 
  
  \begin{document}
  
  
  \maketitle % Insert title
  
  
  \thispagestyle{fancy} % All pages have headers and footers
  %----------------------------------------------------------------------------------------  
  % ABSTRACT
  
  %----------------------------------------------------------------------------------------  
  
  \begin{abstract}
  Glioblastoma multiforme (GBM) is the most aggressive tumor of the central nervous system. In spite of advances in science and medicine,  the average life expectancy remains about 18 months after diagnosis and the current treatment are becoming obsolete. Therefore,  the search for more effective GBM therapeutic targets is urgently needed. To look for new therapeutics targets,  here we investigated,  through computation tools and next-generation sequencing data,  the role of exon junction complex (EJC) components MAGOH/B,  key genes of the post-transcriptional gene regulation mechanism,  in low-grade gliomas and GBM. Our results show that a higher expression of MAGOH/B is: i) positively correlated to more aggressive gliomas; ii) significantly related to lower overall survival of GBM patients and iii) with GBM showing worst responses to treatments. We also find that the knockdown of MAGOH/B decreases the GBM cell lines viability and proliferation,  but increase their apoptosis. Additionally,  we find that MAGOH/B knockdown changes the expression of genes associated with splicing,  RNA transport,  translation,  and cell cycle affected,  suggesting an auto-regulation. Interestingly,  genes that were alternatively spliced by MAGOH/B  KD  were  linked to  RNA  stability/processing/metabolism,   DNA  repair,  and stress response,  Gene Ontology  pathways commonly deregulated in cancer. Furthermore,  we have shown MAGOH/B KD reaches RS exons and leads to stop-codon gain and frame change in genes commonly deregulated in GBM. In summary,  we believe that MAGOH/B are key genes involved in GBM,  which would be investigated as new markers for the disease or novel targets for therapy in the near future.
  
  Funding: PNPD/CAPES,  Hospital Sirio Liban\^es \\ 
  \end{abstract}
  \end{document} 