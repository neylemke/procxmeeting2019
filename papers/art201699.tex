
  \documentclass[twoside]{article}
  \usepackage[affil-it]{authblk}
  \usepackage{lipsum} % Package to generate dummy text throughout this template
  \usepackage{eurosym}
  \usepackage[sc]{mathpazo} % Use the Palatino font
  \usepackage[T1]{fontenc} % Use 8-bit encoding that has 256 glyphs
  \usepackage[utf8]{inputenc}
  \linespread{1.05} % Line spacing-Palatino needs more space between lines
  \usepackage{microtype} % Slightly tweak font spacing for aesthetics\[IndentingNewLine]
  \usepackage[hmarginratio=1:1,top=32mm,columnsep=20pt]{geometry} % Document margins
  \usepackage{multicol} % Used for the two-column layout of the document
  \usepackage[hang,small,labelfont=bf,up,textfont=it,up]{caption} % Custom captions under//above floats in tables or figures
  \usepackage{booktabs} % Horizontal rules in tables
  \usepackage{float} % Required for tables and figures in the multi-column environment-they need to be placed in specific locations with the[H] (e.g. \begin{table}[H])
  \usepackage{hyperref} % For hyperlinks in the PDF
  \usepackage{lettrine} % The lettrine is the first enlarged letter at the beginning of the text
  \usepackage{paralist} % Used for the compactitem environment which makes bullet points with less space between them
  \usepackage{abstract} % Allows abstract customization
  \renewcommand{\abstractnamefont}{\normalfont\bfseries} 
  %\renewcommand{\abstracttextfont}{\normalfont\small\itshape} % Set the abstract itself to small italic text\[IndentingNewLine]
  \usepackage{titlesec} % Allows customization of titles
  \renewcommand\thesection{\Roman{section}} % Roman numerals for the sections
  \renewcommand\thesubsection{\Roman{subsection}} % Roman numerals for subsections
  \titleformat{\section}[block]{\large\scshape\centering}{\thesection.}{1em}{} % Change the look of the section titles
  \titleformat{\subsection}[block]{\large}{\thesubsection.}{1em}{} % Change the look of the section titles
  \usepackage{fancyhdr} % Headers and footers
  \pagestyle{fancy} % All pages have headers and footers
  \fancyhead{} % Blank out the default header
  \fancyfoot{} % Blank out the default footer
  \fancyhead[C]{X-meeting $\bullet$ October 2019 $\bullet$ Campos do  Jord\~ao} % Custom header text
  \fancyfoot[RO,LE]{} % Custom footer text
  %----------------------------------------------------------------------------------------
  % TITLE SECTION
  %---------------------------------------------------------------------------------------- 
 
 \title{\vspace{-15mm}\fontsize{24pt}{10pt}\selectfont\textbf{ Can we predict protein essentiality based on their pyshico-chemical features ? }} % Article title
  
  
  \author{ Mauricio Lopes Casagrande, Ney Lemke, Marcio Luis Acencio }
  
  \affil{  }
  \vspace{-5mm}
  \date{}
  
  %---------------------------------------------------------------------------------------- 
  
  \begin{document}
  
  
  \maketitle % Insert title
  
  
  \thispagestyle{fancy} % All pages have headers and footers
  %----------------------------------------------------------------------------------------  
  % ABSTRACT
  
  %----------------------------------------------------------------------------------------  
  
  \begin{abstract}
  The way genes organize and behave during the life of a living being hints that there may be a set of genes essential to life and reproduction,  acknowledged as essential genes. These genes show high evolutionary conservation rate when compared to genomic media and stay relatively preserved through time,  making it possible to find homology even on distant species. The process of discovering essential genes in vitro requires gene expression modulation techniques such as single-gene knockout,  RNA interference,  conditional knockout or CRISPR,  and has always been an extensive,  laborious effort. Nowadays it is possible to shorten this effort by computational,  ortogenetic or phylogenetic approaches,  but any method able to shorten this effort even more would be welcome. This work intends to present an in silico method to aid prediction of gene essentiality based only on physico-chemical features of the proteins synthesised by a gene. We applied machine learning techniques to verify if there really is some kind of relation between these features and gene essentiality and to evaluate the predicting capability of two algorithms using two distinct organisms. The knowledge about the set of essential genes of a certain organism could help improving our comprehension about the mechanisms related to the genetic base needed to sustain life,  signaling new effective antimicrobial drug targets,  increasing our knowledge about synthetic life or even guiding genetic therapy.
  
  Funding:  \\ 
  \end{abstract}
  \end{document} 