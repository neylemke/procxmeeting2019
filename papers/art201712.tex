
  \documentclass[twoside]{article}
  \usepackage[affil-it]{authblk}
  \usepackage{lipsum} % Package to generate dummy text throughout this template
  \usepackage{eurosym}
  \usepackage[sc]{mathpazo} % Use the Palatino font
  \usepackage[T1]{fontenc} % Use 8-bit encoding that has 256 glyphs
  \usepackage[utf8]{inputenc}
  \linespread{1.05} % Line spacing-Palatino needs more space between lines
  \usepackage{microtype} % Slightly tweak font spacing for aesthetics\[IndentingNewLine]
  \usepackage[hmarginratio=1:1,top=32mm,columnsep=20pt]{geometry} % Document margins
  \usepackage{multicol} % Used for the two-column layout of the document
  \usepackage[hang,small,labelfont=bf,up,textfont=it,up]{caption} % Custom captions under//above floats in tables or figures
  \usepackage{booktabs} % Horizontal rules in tables
  \usepackage{float} % Required for tables and figures in the multi-column environment-they need to be placed in specific locations with the[H] (e.g. \begin{table}[H])
  \usepackage{hyperref} % For hyperlinks in the PDF
  \usepackage{lettrine} % The lettrine is the first enlarged letter at the beginning of the text
  \usepackage{paralist} % Used for the compactitem environment which makes bullet points with less space between them
  \usepackage{abstract} % Allows abstract customization
  \renewcommand{\abstractnamefont}{\normalfont\bfseries} 
  %\renewcommand{\abstracttextfont}{\normalfont\small\itshape} % Set the abstract itself to small italic text\[IndentingNewLine]
  \usepackage{titlesec} % Allows customization of titles
  \renewcommand\thesection{\Roman{section}} % Roman numerals for the sections
  \renewcommand\thesubsection{\Roman{subsection}} % Roman numerals for subsections
  \titleformat{\section}[block]{\large\scshape\centering}{\thesection.}{1em}{} % Change the look of the section titles
  \titleformat{\subsection}[block]{\large}{\thesubsection.}{1em}{} % Change the look of the section titles
  \usepackage{fancyhdr} % Headers and footers
  \pagestyle{fancy} % All pages have headers and footers
  \fancyhead{} % Blank out the default header
  \fancyfoot{} % Blank out the default footer
  \fancyhead[C]{X-meeting $\bullet$ October 2019 $\bullet$ Campos do  Jord\~ao} % Custom header text
  \fancyfoot[RO,LE]{} % Custom footer text
  %----------------------------------------------------------------------------------------
  % TITLE SECTION
  %---------------------------------------------------------------------------------------- 
 
 \title{\vspace{-15mm}\fontsize{24pt}{10pt}\selectfont\textbf{ Comparative genomics of Acinetobacter baumannii strains }} % Article title
  
  
  \author{ Diego Lucas Neres Rodrigues, Raquel Enma Hurtado Castillo, Daniella Camargo Costa, anne cybelle pinto gomide, Vasco A de C Azevedo, Francielly Rodrigues da Costa, Flavia Figueira Aburjaile }
  
  \affil{ UFMG }
  \vspace{-5mm}
  \date{}
  
  %---------------------------------------------------------------------------------------- 
  
  \begin{document}
  
  
  \maketitle % Insert title
  
  
  \thispagestyle{fancy} % All pages have headers and footers
  %----------------------------------------------------------------------------------------  
  % ABSTRACT
  
  %----------------------------------------------------------------------------------------  
  
  \begin{abstract}
  Acinetobacter baumannii is an important opportunistic pathogen causing meningitis,  bacteremia,  pneumonia,  erysipelas and genitourinary tract infections. In recent years,  this etiologic agent has acquired multiple mechanisms of resistance to a diverse range of antimicrobials. Its ability to survive in different environments combined with its resistance to drugs makes it extremely difficult to treat patients suffering from infections associated with this pathogen. In this context,  this research aims to elucidate in silico analysis from the genome of different strains of A. baumannii to elucidate some mechanisms of resistance related to multiple drug efflux pumps,  as well as,  to search for genomic resistance islands among 122 A. baumannii strains. By means of the in silico comparison of fifteen different strains of A. baumannii,  it was possible to observe a phylogenetic approximation of the A. baumannii individuals,  with the presence of few polymorphic points in the conserved 16S rRNA gene,  as well as the presence and metabolic analysis of 14 proteins related to efflux pumps in each of the selected strains,  all of which were evaluated after automatic annotation and manual curation,  where the results maintained the functionality of the proteins with reliability. In conclusion,  there is great phylogenetic proximity between the A. baumannii strains studied. However,  regarding comparative genomics and sequence annotation,  there are differences between the products generated automatically and manually,  showing possible points of polymorphism. In addition,  the search for metabolic islands revealed the presence of resistance islands that may elucidate the prolonged survival of this bacterial species.
  
  Funding: CAPES,  CNPq e FAPEMIG \\ 
  \end{abstract}
  \end{document} 