
  \documentclass[twoside]{article}
  \usepackage[affil-it]{authblk}
  \usepackage{lipsum} % Package to generate dummy text throughout this template
  \usepackage{eurosym}
  \usepackage[sc]{mathpazo} % Use the Palatino font
  \usepackage[T1]{fontenc} % Use 8-bit encoding that has 256 glyphs
  \usepackage[utf8]{inputenc}
  \linespread{1.05} % Line spacing-Palatino needs more space between lines
  \usepackage{microtype} % Slightly tweak font spacing for aesthetics\[IndentingNewLine]
  \usepackage[hmarginratio=1:1,top=32mm,columnsep=20pt]{geometry} % Document margins
  \usepackage{multicol} % Used for the two-column layout of the document
  \usepackage[hang,small,labelfont=bf,up,textfont=it,up]{caption} % Custom captions under//above floats in tables or figures
  \usepackage{booktabs} % Horizontal rules in tables
  \usepackage{float} % Required for tables and figures in the multi-column environment-they need to be placed in specific locations with the[H] (e.g. \begin{table}[H])
  \usepackage{hyperref} % For hyperlinks in the PDF
  \usepackage{lettrine} % The lettrine is the first enlarged letter at the beginning of the text
  \usepackage{paralist} % Used for the compactitem environment which makes bullet points with less space between them
  \usepackage{abstract} % Allows abstract customization
  \renewcommand{\abstractnamefont}{\normalfont\bfseries} 
  %\renewcommand{\abstracttextfont}{\normalfont\small\itshape} % Set the abstract itself to small italic text\[IndentingNewLine]
  \usepackage{titlesec} % Allows customization of titles
  \renewcommand\thesection{\Roman{section}} % Roman numerals for the sections
  \renewcommand\thesubsection{\Roman{subsection}} % Roman numerals for subsections
  \titleformat{\section}[block]{\large\scshape\centering}{\thesection.}{1em}{} % Change the look of the section titles
  \titleformat{\subsection}[block]{\large}{\thesubsection.}{1em}{} % Change the look of the section titles
  \usepackage{fancyhdr} % Headers and footers
  \pagestyle{fancy} % All pages have headers and footers
  \fancyhead{} % Blank out the default header
  \fancyfoot{} % Blank out the default footer
  \fancyhead[C]{X-meeting $\bullet$ October 2019 $\bullet$ Campos do  Jord\~ao} % Custom header text
  \fancyfoot[RO,LE]{} % Custom footer text
  %----------------------------------------------------------------------------------------
  % TITLE SECTION
  %---------------------------------------------------------------------------------------- 
 
 \title{\vspace{-15mm}\fontsize{24pt}{10pt}\selectfont\textbf{ THE ROLE OF TUMOR HLA IN NON-MUSCLE INVASIVE BLADDER CANCER RESPONSE TO BCG IMMUNOTHERAPY }} % Article title
  
  
  \author{ Ramon Torreglosa do Carmo, Giulia Wada Friguglietti, Diogo Bastos, Vitor Rezende da Costa Aguiar, Fabiana Bettoni, Diogo Meyer, Anamaria A. Camargo, Cibele Masotti }
  
  \affil{  }
  \vspace{-5mm}
  \date{}
  
  %---------------------------------------------------------------------------------------- 
  
  \begin{document}
  
  
  \maketitle % Insert title
  
  
  \thispagestyle{fancy} % All pages have headers and footers
  %----------------------------------------------------------------------------------------  
  % ABSTRACT
  
  %----------------------------------------------------------------------------------------  
  
  \begin{abstract}
  The choice treatment for non-muscle invasive bladder cancer (NMIBC) is the complete transurethral resection of the tumor but followed by adjuvant immunotherapy with intravesical BCG (attenuated Bacillus Calmette-Gu\'erin) instillations in high-risk cases of recurrence or progression. Immunotherapy significantly decreases the risk of disease recurrence because it eliminates tumor cells by stimulating the patient's immune response,  capable of stimulating lymphocytes that supposedly recognize tumor antigens or the bacillus internalized by the tumor cells. Many patients do not respond to BCG treatment: 30-40\% relapse and 10-25\% develop muscle-invasive forms. There are no predictive biomarkers of BCG response implemented in clinical practice,  and,  in the context of immunotherapy with immune checkpoint inhibitors,  there is a growing body of evidence pointing to predictive response factors related to the antigen presentation mechanism. HLA class-I molecules are directly involved in the presentation of neoantigens and,  therefore,  are also associated with tumor immune evasion mechanisms,  including HLA somatic mutations,  large deletions or transcriptional silencing. HLA polymorphism can also modify the response to immunotherapy,  as it has been recently shown for melanoma and lung cancer: individuals with tumors bearing certain HLA-A and B supertypes or heterozygous for all HLA loci have lower risks of disease relapse and progression. In this work,  we HLA-typed 35 primary tumors of NMIBC-BCG treated patients (17 responsive and 18 unresponsive to treatment) from exomes using Optitype. To estimate copy number variation (CNV) from exomes,  we used CODEX. We investigated whether loss of diversity in HLA loci (homozygosity in at least HLA-A,  B or C) or the presence B62 supertype was associated with survival rates. We did not observe a significant correlation between loss of diversity and relapse-free survival (RFS; Log-rank test p=0.52),  neither progression-free survival (PFS; Log-rank test p=0.099),  but individuals with tumors heterozygous for all HLA loci had longer RFS and PFS. We observed four tumors segregating the B62 supertype,  but it was not significantly enriched in the unresponsive group (3 resistant: 1 sensitive,  Fisher’s exact test p=0.6041).  We further evaluated if CNVs overlapping HLA-A,  B and C could interfere in BCG response. We observed 1,  4,  and 1 deletions and 1,  3,  and 3 duplications overlapping HLA-A,  B,  and C coding regions,  respectively,  in 7 tumors. There was no correlation between response to treatment and CNVs overlapping HLA (Fisher’s exact test p=1). Besides not finding a significant association with treatment response,  which is attributed to lack of power for most of our analysis,  it is of note that an unresponsive HLA-B homozygous tumor also had an overlapping deletion within the gene coding region,  suggesting that the antigen presentation may be compromised in this case. This is the first evaluation of HLA variation in the context of NMIBC treatment,  and we believed that an extended sample size may uncover the role of HLA in BCG-response.
  
  Funding: Capes \\ 
  \end{abstract}
  \end{document} 