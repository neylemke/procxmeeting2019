
  \documentclass[twoside]{article}
  \usepackage[affil-it]{authblk}
  \usepackage{lipsum} % Package to generate dummy text throughout this template
  \usepackage{eurosym}
  \usepackage[sc]{mathpazo} % Use the Palatino font
  \usepackage[T1]{fontenc} % Use 8-bit encoding that has 256 glyphs
  \usepackage[utf8]{inputenc}
  \linespread{1.05} % Line spacing-Palatino needs more space between lines
  \usepackage{microtype} % Slightly tweak font spacing for aesthetics\[IndentingNewLine]
  \usepackage[hmarginratio=1:1,top=32mm,columnsep=20pt]{geometry} % Document margins
  \usepackage{multicol} % Used for the two-column layout of the document
  \usepackage[hang,small,labelfont=bf,up,textfont=it,up]{caption} % Custom captions under//above floats in tables or figures
  \usepackage{booktabs} % Horizontal rules in tables
  \usepackage{float} % Required for tables and figures in the multi-column environment-they need to be placed in specific locations with the[H] (e.g. \begin{table}[H])
  \usepackage{hyperref} % For hyperlinks in the PDF
  \usepackage{lettrine} % The lettrine is the first enlarged letter at the beginning of the text
  \usepackage{paralist} % Used for the compactitem environment which makes bullet points with less space between them
  \usepackage{abstract} % Allows abstract customization
  \renewcommand{\abstractnamefont}{\normalfont\bfseries} 
  %\renewcommand{\abstracttextfont}{\normalfont\small\itshape} % Set the abstract itself to small italic text\[IndentingNewLine]
  \usepackage{titlesec} % Allows customization of titles
  \renewcommand\thesection{\Roman{section}} % Roman numerals for the sections
  \renewcommand\thesubsection{\Roman{subsection}} % Roman numerals for subsections
  \titleformat{\section}[block]{\large\scshape\centering}{\thesection.}{1em}{} % Change the look of the section titles
  \titleformat{\subsection}[block]{\large}{\thesubsection.}{1em}{} % Change the look of the section titles
  \usepackage{fancyhdr} % Headers and footers
  \pagestyle{fancy} % All pages have headers and footers
  \fancyhead{} % Blank out the default header
  \fancyfoot{} % Blank out the default footer
  \fancyhead[C]{X-meeting $\bullet$ October 2019 $\bullet$ Campos do  Jord\~ao} % Custom header text
  \fancyfoot[RO,LE]{} % Custom footer text
  %----------------------------------------------------------------------------------------
  % TITLE SECTION
  %---------------------------------------------------------------------------------------- 
 
 \title{\vspace{-15mm}\fontsize{24pt}{10pt}\selectfont\textbf{ Cradle-loop barrel in Leptospira and novel GAF fusion proteins }} % Article title
  
  
  \author{ Rodolfo Alvarenga Ribeiro, Daniela Valdivieso, Cristiane Rodrigues Guzzo Carvalho, Robson Francisco de Souza }
  
  \affil{ Institute of Biomedical Sciences,  USP }
  \vspace{-5mm}
  \date{}
  
  %---------------------------------------------------------------------------------------- 
  
  \begin{document}
  
  
  \maketitle % Insert title
  
  
  \thispagestyle{fancy} % All pages have headers and footers
  %----------------------------------------------------------------------------------------  
  % ABSTRACT
  
  %----------------------------------------------------------------------------------------  
  
  \begin{abstract}
  Leptospirosis is an infectious disease of high incidence in tropical regions,  caused by bacteria of the genus Leptospira. The second bacterial messenger,  c-di-GMP,  acts on different signaling pathways that result in the regulation of virulence,  mobility and biofilm formation that may be related to the infectious process. The protein encoded by the LIC\_11920 gene shows DUF1577 and PilZ domains(YcgR-like and PilZ),  and is a recognised as a member of the cradle-loop barrel fold,  which comprehends a set of protein families that act as sensor and/or flagella structure as well as type 6 secretion system proteins,  such as PilZ and YcgR. PilZ is an intracellular c-di-GMP sensor whose performance has already been related to the regulation of resistance or pathogenicity in organisms such as Borrelia,  but little is known about the involvement of PilZ homologues,  including LIC\_11920 ,  on the c-di-GMP-mediated signaling pathways in Leptospira interrogans serovar Copenhageni. The DUF1577 has an unusual GAF domain in fusion with a YcgR and a PilZ domains,  which could be a recent autapomorphy in the leptospiral clade. Such fusions have pointed to have relationship with diversity-generating retroelements,  which could have an important role in Leptospira evolution. We intend to characterize the c-di-GMP-mediated signaling pathways in L. interrogans from the structural and functional analysis of the LIC\_11920 protein,  as well as to clarify the classification of the fold,  in order to place the DUF1577.  Along with our in-silico strategies we intend to evaluate the structure of LIC\_11920 in the search for a target of pharmacological intervention in the treatment of leptospirosis.
  
  Funding: CAPES 51/2013 CAPES 88887.357176/2019-00 \\ 
  \end{abstract}
  \end{document} 