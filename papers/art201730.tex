
  \documentclass[twoside]{article}
  \usepackage[affil-it]{authblk}
  \usepackage{lipsum} % Package to generate dummy text throughout this template
  \usepackage{eurosym}
  \usepackage[sc]{mathpazo} % Use the Palatino font
  \usepackage[T1]{fontenc} % Use 8-bit encoding that has 256 glyphs
  \usepackage[utf8]{inputenc}
  \linespread{1.05} % Line spacing-Palatino needs more space between lines
  \usepackage{microtype} % Slightly tweak font spacing for aesthetics\[IndentingNewLine]
  \usepackage[hmarginratio=1:1,top=32mm,columnsep=20pt]{geometry} % Document margins
  \usepackage{multicol} % Used for the two-column layout of the document
  \usepackage[hang,small,labelfont=bf,up,textfont=it,up]{caption} % Custom captions under//above floats in tables or figures
  \usepackage{booktabs} % Horizontal rules in tables
  \usepackage{float} % Required for tables and figures in the multi-column environment-they need to be placed in specific locations with the[H] (e.g. \begin{table}[H])
  \usepackage{hyperref} % For hyperlinks in the PDF
  \usepackage{lettrine} % The lettrine is the first enlarged letter at the beginning of the text
  \usepackage{paralist} % Used for the compactitem environment which makes bullet points with less space between them
  \usepackage{abstract} % Allows abstract customization
  \renewcommand{\abstractnamefont}{\normalfont\bfseries} 
  %\renewcommand{\abstracttextfont}{\normalfont\small\itshape} % Set the abstract itself to small italic text\[IndentingNewLine]
  \usepackage{titlesec} % Allows customization of titles
  \renewcommand\thesection{\Roman{section}} % Roman numerals for the sections
  \renewcommand\thesubsection{\Roman{subsection}} % Roman numerals for subsections
  \titleformat{\section}[block]{\large\scshape\centering}{\thesection.}{1em}{} % Change the look of the section titles
  \titleformat{\subsection}[block]{\large}{\thesubsection.}{1em}{} % Change the look of the section titles
  \usepackage{fancyhdr} % Headers and footers
  \pagestyle{fancy} % All pages have headers and footers
  \fancyhead{} % Blank out the default header
  \fancyfoot{} % Blank out the default footer
  \fancyhead[C]{X-meeting $\bullet$ October 2019 $\bullet$ Campos do  Jord\~ao} % Custom header text
  \fancyfoot[RO,LE]{} % Custom footer text
  %----------------------------------------------------------------------------------------
  % TITLE SECTION
  %---------------------------------------------------------------------------------------- 
 
 \title{\vspace{-15mm}\fontsize{24pt}{10pt}\selectfont\textbf{ Characterization of Xop effectors in Xanthomonas citri subsp. malvacearum }} % Article title
  
  
  \author{ MANUELA  CORREIA  DION\'ISIO, Juan Carlos Ariute, Ana Maria Benko-Iseppon, Flavia Figueira Aburjaile }
  
  \affil{ UFPE }
  \vspace{-5mm}
  \date{}
  
  %---------------------------------------------------------------------------------------- 
  
  \begin{document}
  
  
  \maketitle % Insert title
  
  
  \thispagestyle{fancy} % All pages have headers and footers
  %----------------------------------------------------------------------------------------  
  % ABSTRACT
  
  %----------------------------------------------------------------------------------------  
  
  \begin{abstract}
  Cotton (Gossypium spp.) is a fiber cash crop extensively sown throughout the world. At the Northeast of Brazil,  for example,  cotton yields are highly valuable for export. However,  in the last years,  the producers have been challenged by the maintenance costs of the plant. Once that diseases and pests significantly reduce cotton yield,  and a phytosanitary control must be implemented. In plant pathogenic bacteria,  type III effectors (T3Es) play a crucial role in pathogenicity,  mainly against Gram-negative bacteria. Bacterial blight of cotton is incited by Xanthomonas citri subsp. malvacearum,  a disease responsible for large losses of cultivars. Chemicals products used to treat the disease present low efficacy and pollute the environment. Therefore,  targeting the group of effectors like Xops (Xanthomonas outer proteins) constitutes a significant approach for the generation of more resistant plants. In this context,  this study aims to identify and characterize Xops genes in genomes of X. citri subsp. malvacearum. Therefore,  we selected seven genomic sequences of X. citri subsp. malvacearum in GenBank (NCBI database). Afterward,  an automatic annotation was carried out using RAST (Rapid Annotation using Subsystem Technology) tool followed by manual curation. Further,  we intend to obtain the Xops involved in the pathogenicity of cotton. During the automatic annotation (RAST),  we found some plasmids and transposable elements,  which explains their high genetic variability. Also,  we suggested that virulence genes are implicated in plant-pathogen interaction with effectors type III (T3Es). Thus,  we identified effectors like xopC,  xopJ,  xopAG,  AvrBs3,  AvrXa10 and PthXol in X. citri pv. malvacearum in the 7 genomes and we related to their pathogenicity. In our studies,  we show the presence of Xop effectors,  as well as their dynamic organization in the genomes. Furthermore,  the analysis of comparative genomes by Mauve software presented deletions,  inversions and insertions,  which suggest the genetic variability in these strains. Hence,  the characterization of these proteins may subsequently be used,  as targets for the control of phytosanitary problems in cotton the Brazilian Northeast.
  
  Funding: CAPES,  CNPq,  FACEPE. \\ 
  \end{abstract}
  \end{document} 