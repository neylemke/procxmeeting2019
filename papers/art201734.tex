
  \documentclass[twoside]{article}
  \usepackage[affil-it]{authblk}
  \usepackage{lipsum} % Package to generate dummy text throughout this template
  \usepackage{eurosym}
  \usepackage[sc]{mathpazo} % Use the Palatino font
  \usepackage[T1]{fontenc} % Use 8-bit encoding that has 256 glyphs
  \usepackage[utf8]{inputenc}
  \linespread{1.05} % Line spacing-Palatino needs more space between lines
  \usepackage{microtype} % Slightly tweak font spacing for aesthetics\[IndentingNewLine]
  \usepackage[hmarginratio=1:1,top=32mm,columnsep=20pt]{geometry} % Document margins
  \usepackage{multicol} % Used for the two-column layout of the document
  \usepackage[hang,small,labelfont=bf,up,textfont=it,up]{caption} % Custom captions under//above floats in tables or figures
  \usepackage{booktabs} % Horizontal rules in tables
  \usepackage{float} % Required for tables and figures in the multi-column environment-they need to be placed in specific locations with the[H] (e.g. \begin{table}[H])
  \usepackage{hyperref} % For hyperlinks in the PDF
  \usepackage{lettrine} % The lettrine is the first enlarged letter at the beginning of the text
  \usepackage{paralist} % Used for the compactitem environment which makes bullet points with less space between them
  \usepackage{abstract} % Allows abstract customization
  \renewcommand{\abstractnamefont}{\normalfont\bfseries} 
  %\renewcommand{\abstracttextfont}{\normalfont\small\itshape} % Set the abstract itself to small italic text\[IndentingNewLine]
  \usepackage{titlesec} % Allows customization of titles
  \renewcommand\thesection{\Roman{section}} % Roman numerals for the sections
  \renewcommand\thesubsection{\Roman{subsection}} % Roman numerals for subsections
  \titleformat{\section}[block]{\large\scshape\centering}{\thesection.}{1em}{} % Change the look of the section titles
  \titleformat{\subsection}[block]{\large}{\thesubsection.}{1em}{} % Change the look of the section titles
  \usepackage{fancyhdr} % Headers and footers
  \pagestyle{fancy} % All pages have headers and footers
  \fancyhead{} % Blank out the default header
  \fancyfoot{} % Blank out the default footer
  \fancyhead[C]{X-meeting $\bullet$ October 2019 $\bullet$ Campos do  Jord\~ao} % Custom header text
  \fancyfoot[RO,LE]{} % Custom footer text
  %----------------------------------------------------------------------------------------
  % TITLE SECTION
  %---------------------------------------------------------------------------------------- 
 
 \title{\vspace{-15mm}\fontsize{24pt}{10pt}\selectfont\textbf{ A study of AGN inference with Tsallis Entropy }} % Article title
  
  
  \author{ Cassio Henrique dos Santos Amador, Fabr\'{\i}cio Martins Lopes }
  
  \affil{ Universidade Tecnol\'ogica Federal do Paran\'a }
  \vspace{-5mm}
  \date{}
  
  %---------------------------------------------------------------------------------------- 
  
  \begin{document}
  
  
  \maketitle % Insert title
  
  
  \thispagestyle{fancy} % All pages have headers and footers
  %----------------------------------------------------------------------------------------  
  % ABSTRACT
  
  %----------------------------------------------------------------------------------------  
  
  \begin{abstract}
  In the field of gene networks,  network inference is an open problem. This inference is handicapped by the low number of samples and the great network complexity (number of genes and even more interactions between them). These networks can be modeled using Probabilistic Boolean Networks,  where the gene relationships are modeled as Boolean operators,  and these operators can depend on one,  two or more genes. Besides,  the network structure can be inferred if the correct criterion function is chosen,  which can be the information entropy,  for instance. Previous work has shown that Tsallis entropy as a criterion function is more accurate to infer network structures than traditional Shannon entropy,  if one can find the best non-extensive parameter. This parameter is related to the interaction distance between elements,  the complexity of this interaction,  and the possible absence of probability configurations between the elements,  in this case,  the genes. The present work investigates artificial gene networks (AGN) to analyze the reasons for this entropy to be more efficient,  and what is the effect of a network structure topology on the non-extensive parameter used for its inference. The use of artificial gene networks allow a controlled environment,  and the analysis with a larger number of samples than experimental results could allow for. Different genes with various number of links in the network,  from different scale-free networks were studied. We also analyzed the relationship between these different genes and the inference of the network as a whole,  and the results are compared with inferences obtained from the Shanon entropy.
  
  Funding: UTFPR \\ 
  \end{abstract}
  \end{document} 