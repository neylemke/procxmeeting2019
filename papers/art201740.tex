
  \documentclass[twoside]{article}
  \usepackage[affil-it]{authblk}
  \usepackage{lipsum} % Package to generate dummy text throughout this template
  \usepackage{eurosym}
  \usepackage[sc]{mathpazo} % Use the Palatino font
  \usepackage[T1]{fontenc} % Use 8-bit encoding that has 256 glyphs
  \usepackage[utf8]{inputenc}
  \linespread{1.05} % Line spacing-Palatino needs more space between lines
  \usepackage{microtype} % Slightly tweak font spacing for aesthetics\[IndentingNewLine]
  \usepackage[hmarginratio=1:1,top=32mm,columnsep=20pt]{geometry} % Document margins
  \usepackage{multicol} % Used for the two-column layout of the document
  \usepackage[hang,small,labelfont=bf,up,textfont=it,up]{caption} % Custom captions under//above floats in tables or figures
  \usepackage{booktabs} % Horizontal rules in tables
  \usepackage{float} % Required for tables and figures in the multi-column environment-they need to be placed in specific locations with the[H] (e.g. \begin{table}[H])
  \usepackage{hyperref} % For hyperlinks in the PDF
  \usepackage{lettrine} % The lettrine is the first enlarged letter at the beginning of the text
  \usepackage{paralist} % Used for the compactitem environment which makes bullet points with less space between them
  \usepackage{abstract} % Allows abstract customization
  \renewcommand{\abstractnamefont}{\normalfont\bfseries} 
  %\renewcommand{\abstracttextfont}{\normalfont\small\itshape} % Set the abstract itself to small italic text\[IndentingNewLine]
  \usepackage{titlesec} % Allows customization of titles
  \renewcommand\thesection{\Roman{section}} % Roman numerals for the sections
  \renewcommand\thesubsection{\Roman{subsection}} % Roman numerals for subsections
  \titleformat{\section}[block]{\large\scshape\centering}{\thesection.}{1em}{} % Change the look of the section titles
  \titleformat{\subsection}[block]{\large}{\thesubsection.}{1em}{} % Change the look of the section titles
  \usepackage{fancyhdr} % Headers and footers
  \pagestyle{fancy} % All pages have headers and footers
  \fancyhead{} % Blank out the default header
  \fancyfoot{} % Blank out the default footer
  \fancyhead[C]{X-meeting $\bullet$ October 2019 $\bullet$ Campos do  Jord\~ao} % Custom header text
  \fancyfoot[RO,LE]{} % Custom footer text
  %----------------------------------------------------------------------------------------
  % TITLE SECTION
  %---------------------------------------------------------------------------------------- 
 
 \title{\vspace{-15mm}\fontsize{24pt}{10pt}\selectfont\textbf{ Computational Approach of NSCLC markers applied to drug design against Pd-l1 and Homology Modeling by Tusc2 (FUS1) }} % Article title
  
  
  \author{ Patr\'{\i}cia da Silva Antunes, Nelson Jos\'e Freitas da Silveira, Levy Bueno Alves, Thiago Castilho Elias, M\'arcia Paranho Veloso, William Mesquita da Costa }
  
  \affil{  }
  \vspace{-5mm}
  \date{}
  
  %---------------------------------------------------------------------------------------- 
  
  \begin{document}
  
  
  \maketitle % Insert title
  
  
  \thispagestyle{fancy} % All pages have headers and footers
  %----------------------------------------------------------------------------------------  
  % ABSTRACT
  
  %----------------------------------------------------------------------------------------  
  
  \begin{abstract}
  The treatment of cancer is part of public health policy in Brazil. The needs for the development of new drugs again different mechanisms of tumor cell survival,  makes computational simulation essential to predict the interaction of molecules with selected targets. Non-small cell lung cancer (NSCLC) is one of the most aggressive types of cancer,  with low survival rate and high metastatic capacity. In this work,  we selected two NSCLC markers,  Pd-l1 and Tusc2. Pd-l1 is a Pd-1 binding protein that it is normally expressed in cells of the immune system and highly expressed in tumor cells as an escape mechanism for cell death. On the other hand,  Tusc2 is a tumor suppressor candidate,  and its low NSCLC expression is related to the regulation of Pd-l1. Currently,  drugs that block the Pd-l / Pd-l1 checkpoint are antibodies that rely on protein engineering and cause immunogenicity problems. Therefore,  in the first study,  we used the strategy to develop small anti-Pd-l1 molecules that block the Pd-1 / Pd-l1 interaction. In the second study,  as there isnt a reported structure of Tusc2,  this work focused in the homology modeling of its 3D structure that will help in further studies with this marker. In the first study,  we also simulate the interaction of active compounds from the herbal medicine Euphorbia tirucalli linneau,  popularly known as Aveloz and used against cancer in the popular medicine. To perform molecular docking,  we used the Induced Fit Docking protocol (IFD) as implemented in the Schrodinger suite. The pharmacokinetic properties of the molecules with the best-performing docking score (GScore) with Pd-l1 were determined using the QikProt included in the suit. The best candidate to inhibit the Pd-1 shows a docking score of -12.461 kcal/mol which represent a good result for small inhibitory molecules. Therefore,  this molecule is expected to be a potential drug against Pd-1. In the second study,  we used the MODELLER software to build 3D models and the Refine Loops protocol for loop optimization of Tusc2 structure. We obtained a 3D model of Tusc2 with 94.3\% stereochemical quality. These results help in directing the in vitro and in vivo experiments needed to construct new drugs against NSCLC.
  
  Funding: Federal University of Alfenas \\ 
  \end{abstract}
  \end{document} 