
  \documentclass[twoside]{article}
  \usepackage[affil-it]{authblk}
  \usepackage{lipsum} % Package to generate dummy text throughout this template
  \usepackage{eurosym}
  \usepackage[sc]{mathpazo} % Use the Palatino font
  \usepackage[T1]{fontenc} % Use 8-bit encoding that has 256 glyphs
  \usepackage[utf8]{inputenc}
  \linespread{1.05} % Line spacing-Palatino needs more space between lines
  \usepackage{microtype} % Slightly tweak font spacing for aesthetics\[IndentingNewLine]
  \usepackage[hmarginratio=1:1,top=32mm,columnsep=20pt]{geometry} % Document margins
  \usepackage{multicol} % Used for the two-column layout of the document
  \usepackage[hang,small,labelfont=bf,up,textfont=it,up]{caption} % Custom captions under//above floats in tables or figures
  \usepackage{booktabs} % Horizontal rules in tables
  \usepackage{float} % Required for tables and figures in the multi-column environment-they need to be placed in specific locations with the[H] (e.g. \begin{table}[H])
  \usepackage{hyperref} % For hyperlinks in the PDF
  \usepackage{lettrine} % The lettrine is the first enlarged letter at the beginning of the text
  \usepackage{paralist} % Used for the compactitem environment which makes bullet points with less space between them
  \usepackage{abstract} % Allows abstract customization
  \renewcommand{\abstractnamefont}{\normalfont\bfseries} 
  %\renewcommand{\abstracttextfont}{\normalfont\small\itshape} % Set the abstract itself to small italic text\[IndentingNewLine]
  \usepackage{titlesec} % Allows customization of titles
  \renewcommand\thesection{\Roman{section}} % Roman numerals for the sections
  \renewcommand\thesubsection{\Roman{subsection}} % Roman numerals for subsections
  \titleformat{\section}[block]{\large\scshape\centering}{\thesection.}{1em}{} % Change the look of the section titles
  \titleformat{\subsection}[block]{\large}{\thesubsection.}{1em}{} % Change the look of the section titles
  \usepackage{fancyhdr} % Headers and footers
  \pagestyle{fancy} % All pages have headers and footers
  \fancyhead{} % Blank out the default header
  \fancyfoot{} % Blank out the default footer
  \fancyhead[C]{X-meeting $\bullet$ October 2019 $\bullet$ Campos do  Jord\~ao} % Custom header text
  \fancyfoot[RO,LE]{} % Custom footer text
  %----------------------------------------------------------------------------------------
  % TITLE SECTION
  %---------------------------------------------------------------------------------------- 
 
 \title{\vspace{-15mm}\fontsize{24pt}{10pt}\selectfont\textbf{ PFMutStats: A new method for describing missense mutations by Annotations,  Conservation,  Coevolution,  Interactions and Structural Feature }} % Article title
  
  
  \author{ Marcelo Querino Lima Afonso, Lucas Bleicher }
  
  \affil{ Universidade Federal de Minas Gerais }
  \vspace{-5mm}
  \date{}
  
  %---------------------------------------------------------------------------------------- 
  
  \begin{document}
  
  
  \maketitle % Insert title
  
  
  \thispagestyle{fancy} % All pages have headers and footers
  %----------------------------------------------------------------------------------------  
  % ABSTRACT
  
  %----------------------------------------------------------------------------------------  
  
  \begin{abstract}
  Predicting the functional effect of missense mutations in proteins is essential for understanding the mechanisms of several diseases,  and for the rational design of proteins for various applications. In this work we aimed to develop a highly integrative methodology for analysing the possible effects of a given mutation in the functions exerted by a protein.
This work is based on analysing Multiple Sequence Alignments of Protein Domains from the Pfam database. Several mutation descriptors are already implemented in a Web Application such as: amino acid frequency before and after mutation,  position conservation,  UniProt Protein Families annotations,  residue interactions established before and after a mutation,  centrality metrics related to the residue interaction network of a protein,  residue depth and secondary structure assignment,  and Ramachandran Distribution analysis before and after the mutations. Various dynamic plots were developed in the D3.js library in order to illustrate and interpret the descriptive results. For the Ramachandran Distribution analysis a Database of reference structures was assembled.
All Pfam Protein Family alignments from the UniProt database were downloaded and parsed for calculating and storing the amino acid frequency and column conservation by the Jensen-Shannon entropy. Residue depth was calculated using the latest theoretical model for maximum accessibility as a reference and assignment to the surface or protein interior was based on calculating two vectors of the amino acids frequencies at either site using different assignment thresholds and looking for maximum divergence between these vectors. Residue Interactions Networks were generated by the RING Software and a given chain rank in graph centrality metrics (Betweenness,  Closeness and Degree) were calculated and weighted based on the interactions energies. Other features have been planned and should be implemented in the following year.
This study was financed in part by the Coordena\c{c}\~ao de Aperfei\c{c}oamento de Pessoal de N\'{\i}vel Superior - Brasil (CAPES).
  
  Funding: Coordena\c{c}\~ao de Aperfei\c{c}oamento de Pessoal de N\'{\i}vel Superior - Brasil (CAPES). \\ 
  \end{abstract}
  \end{document} 