
  \documentclass[twoside]{article}
  \usepackage[affil-it]{authblk}
  \usepackage{lipsum} % Package to generate dummy text throughout this template
  \usepackage{eurosym}
  \usepackage[sc]{mathpazo} % Use the Palatino font
  \usepackage[T1]{fontenc} % Use 8-bit encoding that has 256 glyphs
  \usepackage[utf8]{inputenc}
  \linespread{1.05} % Line spacing-Palatino needs more space between lines
  \usepackage{microtype} % Slightly tweak font spacing for aesthetics\[IndentingNewLine]
  \usepackage[hmarginratio=1:1,top=32mm,columnsep=20pt]{geometry} % Document margins
  \usepackage{multicol} % Used for the two-column layout of the document
  \usepackage[hang,small,labelfont=bf,up,textfont=it,up]{caption} % Custom captions under//above floats in tables or figures
  \usepackage{booktabs} % Horizontal rules in tables
  \usepackage{float} % Required for tables and figures in the multi-column environment-they need to be placed in specific locations with the[H] (e.g. \begin{table}[H])
  \usepackage{hyperref} % For hyperlinks in the PDF
  \usepackage{lettrine} % The lettrine is the first enlarged letter at the beginning of the text
  \usepackage{paralist} % Used for the compactitem environment which makes bullet points with less space between them
  \usepackage{abstract} % Allows abstract customization
  \renewcommand{\abstractnamefont}{\normalfont\bfseries} 
  %\renewcommand{\abstracttextfont}{\normalfont\small\itshape} % Set the abstract itself to small italic text\[IndentingNewLine]
  \usepackage{titlesec} % Allows customization of titles
  \renewcommand\thesection{\Roman{section}} % Roman numerals for the sections
  \renewcommand\thesubsection{\Roman{subsection}} % Roman numerals for subsections
  \titleformat{\section}[block]{\large\scshape\centering}{\thesection.}{1em}{} % Change the look of the section titles
  \titleformat{\subsection}[block]{\large}{\thesubsection.}{1em}{} % Change the look of the section titles
  \usepackage{fancyhdr} % Headers and footers
  \pagestyle{fancy} % All pages have headers and footers
  \fancyhead{} % Blank out the default header
  \fancyfoot{} % Blank out the default footer
  \fancyhead[C]{X-meeting $\bullet$ October 2019 $\bullet$ Campos do  Jord\~ao} % Custom header text
  \fancyfoot[RO,LE]{} % Custom footer text
  %----------------------------------------------------------------------------------------
  % TITLE SECTION
  %---------------------------------------------------------------------------------------- 
 
 \title{\vspace{-15mm}\fontsize{24pt}{10pt}\selectfont\textbf{ Interaction Between TNF and SVMPs of PI Class: Molecular Modeling and Docking at a Glance }} % Article title
  
  
  \author{ Luana Luiza Bastos, Let\'{\i}cia Xavier Silva Cant\~ao, Leandro Liborio da Silva Matos, Raquel Melo Minardi }
  
  \affil{ Universidade Federal de Minas Gerais - UFMG }
  \vspace{-5mm}
  \date{}
  
  %---------------------------------------------------------------------------------------- 
  
  \begin{document}
  
  
  \maketitle % Insert title
  
  
  \thispagestyle{fancy} % All pages have headers and footers
  %----------------------------------------------------------------------------------------  
  % ABSTRACT
  
  %----------------------------------------------------------------------------------------  
  
  \begin{abstract}
  TNF (Tumor Necrosis Factor) produced mainly by monocytes,  is a proinflammatory cytokine that plays an important role in modulating inflammatory responses and host defense mechanisms. At increased levels such cytosine is closely related to degenerative diseases such as rheumatoid arthritis. Metalloproteases are enzymes characterized by their zinc catalytic structure at their active site,   snake venom metalloproteases (SVMPs) are responsible for inducing haemorrhage and disturbances in the prey blood coagulation cascade. However,  certain SVMPs do not have hemorrhagic activity (PI class SVMPs),  thus having effects such as inhibition of platelet aggregation,  apoptosis induction and pro anti-inflammatory activities. Previous studies carried out in silico and in vitro with the enzyme BmooMP-alpha-I isolated from snake venom Bothrops moojeni  a  metalloprotease class PI have demonstrated its ability to directly inhibit TNF in immunopathologies like colitis.The aim of this study is to perform in silico studies to elucidate the interaction between tumor necrosis factor and metalloproteases class PI present  in snake venom. For this will  be carried out protein-protein interaction dockings of TNF and SVMPs,  structural alignments and molecular modeling will be performed. As result,  it is expected to be obtained Ligand Root Mean Square Deviation LRMS = 5.0 or Interactive Root Mean Square Deviation IRMS = 2.0 $\AA$.  At the end of this project we hope to contribute to the understanding of the interaction between TNF and SVMPS class PI and their therapeutic applications.
  
  Funding:  \\ 
  \end{abstract}
  \end{document} 