
  \documentclass[twoside]{article}
  \usepackage[affil-it]{authblk}
  \usepackage{lipsum} % Package to generate dummy text throughout this template
  \usepackage{eurosym}
  \usepackage[sc]{mathpazo} % Use the Palatino font
  \usepackage[T1]{fontenc} % Use 8-bit encoding that has 256 glyphs
  \usepackage[utf8]{inputenc}
  \linespread{1.05} % Line spacing-Palatino needs more space between lines
  \usepackage{microtype} % Slightly tweak font spacing for aesthetics\[IndentingNewLine]
  \usepackage[hmarginratio=1:1,top=32mm,columnsep=20pt]{geometry} % Document margins
  \usepackage{multicol} % Used for the two-column layout of the document
  \usepackage[hang,small,labelfont=bf,up,textfont=it,up]{caption} % Custom captions under//above floats in tables or figures
  \usepackage{booktabs} % Horizontal rules in tables
  \usepackage{float} % Required for tables and figures in the multi-column environment-they need to be placed in specific locations with the[H] (e.g. \begin{table}[H])
  \usepackage{hyperref} % For hyperlinks in the PDF
  \usepackage{lettrine} % The lettrine is the first enlarged letter at the beginning of the text
  \usepackage{paralist} % Used for the compactitem environment which makes bullet points with less space between them
  \usepackage{abstract} % Allows abstract customization
  \renewcommand{\abstractnamefont}{\normalfont\bfseries} 
  %\renewcommand{\abstracttextfont}{\normalfont\small\itshape} % Set the abstract itself to small italic text\[IndentingNewLine]
  \usepackage{titlesec} % Allows customization of titles
  \renewcommand\thesection{\Roman{section}} % Roman numerals for the sections
  \renewcommand\thesubsection{\Roman{subsection}} % Roman numerals for subsections
  \titleformat{\section}[block]{\large\scshape\centering}{\thesection.}{1em}{} % Change the look of the section titles
  \titleformat{\subsection}[block]{\large}{\thesubsection.}{1em}{} % Change the look of the section titles
  \usepackage{fancyhdr} % Headers and footers
  \pagestyle{fancy} % All pages have headers and footers
  \fancyhead{} % Blank out the default header
  \fancyfoot{} % Blank out the default footer
  \fancyhead[C]{X-meeting $\bullet$ October 2019 $\bullet$ Campos do  Jord\~ao} % Custom header text
  \fancyfoot[RO,LE]{} % Custom footer text
  %----------------------------------------------------------------------------------------
  % TITLE SECTION
  %---------------------------------------------------------------------------------------- 
 
 \title{\vspace{-15mm}\fontsize{24pt}{10pt}\selectfont\textbf{ IN SILICO ANALYSIS OF THE CONSERVATION OF LEUCISM-RELATED GENES IN VERTEBRATES }} % Article title
  
  
  \author{ Let\'{\i}cia Xavier Silva Cant\~ao, Raquel Melo Minardi, Fabiana Alves }
  
  \affil{ Universidade Federal de Minas Gerais - UFMG }
  \vspace{-5mm}
  \date{}
  
  %---------------------------------------------------------------------------------------- 
  
  \begin{document}
  
  
  \maketitle % Insert title
  
  
  \thispagestyle{fancy} % All pages have headers and footers
  %----------------------------------------------------------------------------------------  
  % ABSTRACT
  
  %----------------------------------------------------------------------------------------  
  
  \begin{abstract}
  Leucism is an anomaly of the pigmentation of the skin of animals and manifests itself as the total or partial loss of the natural color of the species,  and can affect parts of or the entire body of an individual. This change is caused by gene mutation,  or by changes in expression in the some genes related to melanin synthesis. Related to this anomaly are the genes EDN3,  EDNRB,  KIT,  MITF,  PAX3 and SOX10,  responsible for the migration and differentiation of melanocytes; this genes are highly conserved  and indicate that they have some important function for the survival of organisms. The aim of the present study was to analyze in silico the conservation of Leucism-related genes in vertebrates and to evaluate the phylogenetic relationship of the same. The NCBI database was used to obtain vertebrate mRNA sequences. Then,  global and local alignment was performed  the significance of the data by E-value. Phylogenetic analysis was based on the construction of a Bayesian inference phylogenetic tree. Among Leucism-related genes,  only EDN3 was significantly conserved.Possibly EDN3 is the main candidate gene for the leucism induction. It was found that the phylogeny of mammals selected (after alignments) for tree construction did not allow a well resolved relationship. While for the order Rodentia the relations corroborate with phylogenies already found in the literature. Birds and mammals were grouped into distinct groups.
  
  Funding:  \\ 
  \end{abstract}
  \end{document} 