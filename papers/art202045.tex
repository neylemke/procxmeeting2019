
  \documentclass[twoside]{article}
  \usepackage[affil-it]{authblk}
  \usepackage{lipsum} % Package to generate dummy text throughout this template
  \usepackage{eurosym}
  \usepackage[sc]{mathpazo} % Use the Palatino font
  \usepackage[T1]{fontenc} % Use 8-bit encoding that has 256 glyphs
  \usepackage[utf8]{inputenc}
  \linespread{1.05} % Line spacing-Palatino needs more space between lines
  \usepackage{microtype} % Slightly tweak font spacing for aesthetics\[IndentingNewLine]
  \usepackage[hmarginratio=1:1,top=32mm,columnsep=20pt]{geometry} % Document margins
  \usepackage{multicol} % Used for the two-column layout of the document
  \usepackage[hang,small,labelfont=bf,up,textfont=it,up]{caption} % Custom captions under//above floats in tables or figures
  \usepackage{booktabs} % Horizontal rules in tables
  \usepackage{float} % Required for tables and figures in the multi-column environment-they need to be placed in specific locations with the[H] (e.g. \begin{table}[H])
  \usepackage{hyperref} % For hyperlinks in the PDF
  \usepackage{lettrine} % The lettrine is the first enlarged letter at the beginning of the text
  \usepackage{paralist} % Used for the compactitem environment which makes bullet points with less space between them
  \usepackage{abstract} % Allows abstract customization
  \renewcommand{\abstractnamefont}{\normalfont\bfseries} 
  %\renewcommand{\abstracttextfont}{\normalfont\small\itshape} % Set the abstract itself to small italic text\[IndentingNewLine]
  \usepackage{titlesec} % Allows customization of titles
  \renewcommand\thesection{\Roman{section}} % Roman numerals for the sections
  \renewcommand\thesubsection{\Roman{subsection}} % Roman numerals for subsections
  \titleformat{\section}[block]{\large\scshape\centering}{\thesection.}{1em}{} % Change the look of the section titles
  \titleformat{\subsection}[block]{\large}{\thesubsection.}{1em}{} % Change the look of the section titles
  \usepackage{fancyhdr} % Headers and footers
  \pagestyle{fancy} % All pages have headers and footers
  \fancyhead{} % Blank out the default header
  \fancyfoot{} % Blank out the default footer
  \fancyhead[C]{X-meeting $\bullet$ October 2019 $\bullet$ Campos do  Jord\~ao} % Custom header text
  \fancyfoot[RO,LE]{} % Custom footer text
  %----------------------------------------------------------------------------------------
  % TITLE SECTION
  %---------------------------------------------------------------------------------------- 
 
 \title{\vspace{-15mm}\fontsize{24pt}{10pt}\selectfont\textbf{ The Death Is Red: Analysis of the Predicted Secretome of Aspergillus welwitschiae,  with Emphasis in Pathogenicity and Carbohydrate Metabolism }} % Article title
  
  
  \author{ Gabriel Quintanilha Peixoto, Daniel Silva Ara\'ujo, Rodrigo Bentes Kato, Paula Luize Camargos Fonseca, Luiz Marcelo Ribeiro Tom\'e, F\'abio Malcher Miranda, Rommel Thiago Juc\'a Ramos, Bertram Brenig, Vasco A de C Azevedo, Fernanda Badotti, Eric Roberto Guimar\~aes Rocha Aguiar, Arist\'oteles G\'oes Neto }
  
  \affil{ University G\"ottingen }
  \vspace{-5mm}
  \date{}
  
  %---------------------------------------------------------------------------------------- 
  
  \begin{document}
  
  
  \maketitle % Insert title
  
  
  \thispagestyle{fancy} % All pages have headers and footers
  %----------------------------------------------------------------------------------------  
  % ABSTRACT
  
  %----------------------------------------------------------------------------------------  
  
  \begin{abstract}
  In 2018 Aspergillus welwitschiae was described as the causing agent of the bole rot disease of sisal (Agave sisalana) rather than Aspergillus niger as previously thought. A. welwitschiae is a cryptic species of the A. niger/welwitschiae clade. Since then,  we have sought to understand the mechanisms of pathogenicity of this fungus by sequencing the genome of two strains isolated from infected stem tissues of A. sisalana,   and subsequent analysis of the gene content of these fungi. The genomes,  sequenced in Illumina HiSeq 2500 platform,  were assembled de novo and annotated. Obtained genome size in CCMB663 was ~34.8 Mbp containing 12, 549 protein-coding genes,  while CCMB674 genome is ~32.2 Mbp long,  containing 11, 809 protein-coding genes. The resulting gene prediction models were used for analyzing secretomes,  in which different software was applied as filters to select the putative secreted proteins that were subsequently associated with Gene Ontology (GO) terms. Our results show that most of the associated terms describe functions associated with the degradation of proteins,  lipids,  and especially carbohydrates and that the relative abundance of these terms is similar between different isolates. From this set of proteins predicted to be secreted,  possible effector proteins were identified,  which are involved in the invasion and colonization of the infected plant. Most of the identified putative effector proteins have no similarity with known effectors,  and in those in which was possible to identify conserved domains,  essential and very important genes associated to virulence in plants are included,  including genes that act to silence the immune response or to stimulate it,  causing cell death in plant tissues. Secretory and cytosolic plant cell wall degrading enzymes were also analyzed,  describing a great diversity of genes encoding for carbohydrate-active enzymes,  some of which specialized in the degradation of carbohydrates present in large amounts in the tissues of sisal stem. Altogether,  our results improved the understanding of the Aspergillus welwitschiae x Agave sisalana pathosystem,  and also identified new effectors of interest in the fungus-plant interaction.
  
  Funding:  \\ 
  \end{abstract}
  \end{document} 