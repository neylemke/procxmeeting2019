
  \documentclass[twoside]{article}
  \usepackage[affil-it]{authblk}
  \usepackage{lipsum} % Package to generate dummy text throughout this template
  \usepackage{eurosym}
  \usepackage[sc]{mathpazo} % Use the Palatino font
  \usepackage[T1]{fontenc} % Use 8-bit encoding that has 256 glyphs
  \usepackage[utf8]{inputenc}
  \linespread{1.05} % Line spacing-Palatino needs more space between lines
  \usepackage{microtype} % Slightly tweak font spacing for aesthetics\[IndentingNewLine]
  \usepackage[hmarginratio=1:1,top=32mm,columnsep=20pt]{geometry} % Document margins
  \usepackage{multicol} % Used for the two-column layout of the document
  \usepackage[hang,small,labelfont=bf,up,textfont=it,up]{caption} % Custom captions under//above floats in tables or figures
  \usepackage{booktabs} % Horizontal rules in tables
  \usepackage{float} % Required for tables and figures in the multi-column environment-they need to be placed in specific locations with the[H] (e.g. \begin{table}[H])
  \usepackage{hyperref} % For hyperlinks in the PDF
  \usepackage{lettrine} % The lettrine is the first enlarged letter at the beginning of the text
  \usepackage{paralist} % Used for the compactitem environment which makes bullet points with less space between them
  \usepackage{abstract} % Allows abstract customization
  \renewcommand{\abstractnamefont}{\normalfont\bfseries} 
  %\renewcommand{\abstracttextfont}{\normalfont\small\itshape} % Set the abstract itself to small italic text\[IndentingNewLine]
  \usepackage{titlesec} % Allows customization of titles
  \renewcommand\thesection{\Roman{section}} % Roman numerals for the sections
  \renewcommand\thesubsection{\Roman{subsection}} % Roman numerals for subsections
  \titleformat{\section}[block]{\large\scshape\centering}{\thesection.}{1em}{} % Change the look of the section titles
  \titleformat{\subsection}[block]{\large}{\thesubsection.}{1em}{} % Change the look of the section titles
  \usepackage{fancyhdr} % Headers and footers
  \pagestyle{fancy} % All pages have headers and footers
  \fancyhead{} % Blank out the default header
  \fancyfoot{} % Blank out the default footer
  \fancyhead[C]{X-meeting $\bullet$ October 2019 $\bullet$ Campos do  Jord\~ao} % Custom header text
  \fancyfoot[RO,LE]{} % Custom footer text
  %----------------------------------------------------------------------------------------
  % TITLE SECTION
  %---------------------------------------------------------------------------------------- 
 
 \title{\vspace{-15mm}\fontsize{24pt}{10pt}\selectfont\textbf{ piRNAs expression profiles: Estimates and insights found in four human tumor tissues }} % Article title
  
  
  \author{ Ricardo Piuco, Pedro A F Galante }
  
  \affil{  }
  \vspace{-5mm}
  \date{}
  
  %---------------------------------------------------------------------------------------- 
  
  \begin{document}
  
  
  \maketitle % Insert title
  
  
  \thispagestyle{fancy} % All pages have headers and footers
  %----------------------------------------------------------------------------------------  
  % ABSTRACT
  
  %----------------------------------------------------------------------------------------  
  
  \begin{abstract}
  Small RNAs plays several roles regulating gene expression,  particularly,  PIWI-interacting RNA (piRNA),  ranging from approximately 21 to 35 nucleotides act suppressing transposable elements and coding mRNAs. Association of these RNAs with piwi argonauts proteins forming the silencing complex as well as their functional interactions have been discovered in recent years. However,  it is still mostly restricted to germlines. Knowing that alterations in this complex are related to increased proliferation and invasion of cancer cells,  we seek to improve the quantification of piRNAs and identify expression profiles in somatic and tumor context. First,  we obtained data from 2359 small RNA sequencing studies from four tissues: breast,  colon,  prostate and pancreas (1252,  472,  149 and 486 samples,  respectively). Using piRNAdb.org database,  we found an average of 170 piRNAs expressed in tumor samples and 130 in normal tissues. Compared to the total of 27700 known human piRNAs,  we noticed that only a small part is present,  however we observed distinct expression profiles between the analyzed conditions,  providing evidence to further refine the evaluated parameters and benefit the entire research field. In parallel,  we sought to evaluate predicted targets for these piRNAs that might has an oncogenic role or that could be used as biomarkers. For example,  NEFL,  target of hsa-piR-28400,  has been studied in association with microRNAs and cell proliferation in glioblastoma and breast tumors. Then,  to expand the knowledge of piRNA profiles,  we grouped the samples according to their tumor size,  lymph node and metastasis characteristics. This restriction allowed the observation of piRNAs that could be attributed to each of the situations mentioned,  for example,  there is a greater amount of piRNAs expressed in T3-classified breast tumor (5 piRNAs) compared to T2 samples (1). Different results from those found when evaluating normal versus tumor samples,  where more piRNAs are expressed in healthy tissue,  as an example,  we found 37 piRNAs more expressed in normal colon tissue compared to tumor (11 piRNAs). At the end of our observations we believe that even when comparing samples according to their broader characteristic we identified distinct piRNA expression profiles and by further investigating the sample classifications and predicted targets information we provide good parameters to be used by the scientific community. Results that will be the basis for further studies and/or refinement to early diagnosis,  disease progression or even the development of more accurate treatments for patients of these and other tumors.
  
  Funding:  \\ 
  \end{abstract}
  \end{document} 