
  \documentclass[twoside]{article}
  \usepackage[affil-it]{authblk}
  \usepackage{lipsum} % Package to generate dummy text throughout this template
  \usepackage{eurosym}
  \usepackage[sc]{mathpazo} % Use the Palatino font
  \usepackage[T1]{fontenc} % Use 8-bit encoding that has 256 glyphs
  \usepackage[utf8]{inputenc}
  \linespread{1.05} % Line spacing-Palatino needs more space between lines
  \usepackage{microtype} % Slightly tweak font spacing for aesthetics\[IndentingNewLine]
  \usepackage[hmarginratio=1:1,top=32mm,columnsep=20pt]{geometry} % Document margins
  \usepackage{multicol} % Used for the two-column layout of the document
  \usepackage[hang,small,labelfont=bf,up,textfont=it,up]{caption} % Custom captions under//above floats in tables or figures
  \usepackage{booktabs} % Horizontal rules in tables
  \usepackage{float} % Required for tables and figures in the multi-column environment-they need to be placed in specific locations with the[H] (e.g. \begin{table}[H])
  \usepackage{hyperref} % For hyperlinks in the PDF
  \usepackage{lettrine} % The lettrine is the first enlarged letter at the beginning of the text
  \usepackage{paralist} % Used for the compactitem environment which makes bullet points with less space between them
  \usepackage{abstract} % Allows abstract customization
  \renewcommand{\abstractnamefont}{\normalfont\bfseries} 
  %\renewcommand{\abstracttextfont}{\normalfont\small\itshape} % Set the abstract itself to small italic text\[IndentingNewLine]
  \usepackage{titlesec} % Allows customization of titles
  \renewcommand\thesection{\Roman{section}} % Roman numerals for the sections
  \renewcommand\thesubsection{\Roman{subsection}} % Roman numerals for subsections
  \titleformat{\section}[block]{\large\scshape\centering}{\thesection.}{1em}{} % Change the look of the section titles
  \titleformat{\subsection}[block]{\large}{\thesubsection.}{1em}{} % Change the look of the section titles
  \usepackage{fancyhdr} % Headers and footers
  \pagestyle{fancy} % All pages have headers and footers
  \fancyhead{} % Blank out the default header
  \fancyfoot{} % Blank out the default footer
  \fancyhead[C]{X-meeting $\bullet$ October 2019 $\bullet$ Campos do  Jord\~ao} % Custom header text
  \fancyfoot[RO,LE]{} % Custom footer text
  %----------------------------------------------------------------------------------------
  % TITLE SECTION
  %---------------------------------------------------------------------------------------- 
 
 \title{\vspace{-15mm}\fontsize{24pt}{10pt}\selectfont\textbf{ Identification of non-homologous isofunctional and species- specific enzymes in Mycobacterium abscessus as possible therapeutic targets }} % Article title
  
  
  \author{ Fernanda Cristina Medeiros de Oliveira, Philip N Suffys, Solange Alves Vinhas, Mois\'es Palaci, Pedro Henrique Campanini C\^andido, Elizabeth Andrade Marques, Tania Folescu, Rafael Silva Duarte, Marcos Paulo Catanho de Souza, Ana Carolina Ramos Guimar\~aes }
  
  \affil{ Funda\c{c}\~ao Oswaldo Cruz,  Instituto Oswaldo Cruz,  Laborat\'orio de Gen\'etica Molecular de Microrganismos,  Rio de Janeiro,  Brazil }
  \vspace{-5mm}
  \date{}
  
  %---------------------------------------------------------------------------------------- 
  
  \begin{document}
  
  
  \maketitle % Insert title
  
  
  \thispagestyle{fancy} % All pages have headers and footers
  %----------------------------------------------------------------------------------------  
  % ABSTRACT
  
  %----------------------------------------------------------------------------------------  
  
  \begin{abstract}
  The genus Mycobacterium is composed of several pathogenic species. Many organisms belonging to this group are responsible for diseases,  such as tuberculosis,  leprosy and other serious infections. Mycobacterium abscessus (MABSC) is responsible for severe infections that are becoming common all over the world. Outbreaks caused by MABSC have already been reported in several countries,  including Brazil. Treatment of infections caused by MABSC remains ineffective due to antibiotic resistance,  justifying the search for potential new therapeutic targets. The identification of non-homologous isofunctional (analogous) and species-specific (taxonomically restricted) enzymes in these parasites compared to the human host might be a relevant approach to disclose potential new therapeutic targets,  as we already demonstrated in previous works. Although catalysing the same reaction,  non-homologous isofunctional enzymes have distinct evolutionary origin as well as significantly distinct fold topologies and three-dimensional structures. On the other hand,  species-specific enzymes are those found only in one organism in relation to others,  therefore comprising the most evident targets. In this work,  eleven M. abscessus genomes were annotated with three distinct tools to identify and characterize encoded proteins with predicted enzymatic activity: RAST,  Argot2.5,  and BLASTKOALA. A computational pipeline developed by our group (AnEnPi) was used to predict genes encoding putative non-homologous isofunctional enzymes in enzymatic activities shared between those mycobacteria and the human host,  as well as species-specific enzymes encoded in M. abscessus genomes. We identified events of evolutionary convergence in 11 enzymatic activities comprising 11 metabolic pathways,  shared by M. abscessus and H. sapiens,  involving 153 enzymes in total. Five enzymatic activities (EC 5.3.3.2 - terpenoid biosynthesis,  EC 2.3.1.1 - arginine biosynthesis,  EC 2.7.7.4 - purine metabolism,  EC 4.1.1.19 - arginine and proline metabolism,  and EC 4.2.2.1 - nitrogen metabolism) are candidates to be further investigated as new therapeutic targets for drug development against M. abscessus. On the other hand,  1, 592 species-specific enzymes (compared to the human host),  shared among the eleven isolates of M. abscessus,  were also identified in this process,  belonging to 105 distinct enzymatic activities acting on 96 metabolic pathways of M. abscessus. The predicted non-homologous isofunctional instances were confirmed based on structural folding and protein signatures assigned to the implicated enzymes. Mapping these non-homologous isofunctional and species-specific enzymes in M. abscessus metabolism revealed potential new therapeutic targets to control infections caused by MABSC.
  
  Funding: CAPES,  PAPES-FIOCRUZ,  CNPq,  FAPERJ,  Plataforma de Bioinform\'atica Fiocruz RPT04A/RJ \\ 
  \end{abstract}
  \end{document} 