
  \documentclass[twoside]{article}
  \usepackage[affil-it]{authblk}
  \usepackage{lipsum} % Package to generate dummy text throughout this template
  \usepackage{eurosym}
  \usepackage[sc]{mathpazo} % Use the Palatino font
  \usepackage[T1]{fontenc} % Use 8-bit encoding that has 256 glyphs
  \usepackage[utf8]{inputenc}
  \linespread{1.05} % Line spacing-Palatino needs more space between lines
  \usepackage{microtype} % Slightly tweak font spacing for aesthetics\[IndentingNewLine]
  \usepackage[hmarginratio=1:1,top=32mm,columnsep=20pt]{geometry} % Document margins
  \usepackage{multicol} % Used for the two-column layout of the document
  \usepackage[hang,small,labelfont=bf,up,textfont=it,up]{caption} % Custom captions under//above floats in tables or figures
  \usepackage{booktabs} % Horizontal rules in tables
  \usepackage{float} % Required for tables and figures in the multi-column environment-they need to be placed in specific locations with the[H] (e.g. \begin{table}[H])
  \usepackage{hyperref} % For hyperlinks in the PDF
  \usepackage{lettrine} % The lettrine is the first enlarged letter at the beginning of the text
  \usepackage{paralist} % Used for the compactitem environment which makes bullet points with less space between them
  \usepackage{abstract} % Allows abstract customization
  \renewcommand{\abstractnamefont}{\normalfont\bfseries} 
  %\renewcommand{\abstracttextfont}{\normalfont\small\itshape} % Set the abstract itself to small italic text\[IndentingNewLine]
  \usepackage{titlesec} % Allows customization of titles
  \renewcommand\thesection{\Roman{section}} % Roman numerals for the sections
  \renewcommand\thesubsection{\Roman{subsection}} % Roman numerals for subsections
  \titleformat{\section}[block]{\large\scshape\centering}{\thesection.}{1em}{} % Change the look of the section titles
  \titleformat{\subsection}[block]{\large}{\thesubsection.}{1em}{} % Change the look of the section titles
  \usepackage{fancyhdr} % Headers and footers
  \pagestyle{fancy} % All pages have headers and footers
  \fancyhead{} % Blank out the default header
  \fancyfoot{} % Blank out the default footer
  \fancyhead[C]{X-meeting $\bullet$ October 2019 $\bullet$ Campos do  Jord\~ao} % Custom header text
  \fancyfoot[RO,LE]{} % Custom footer text
  %----------------------------------------------------------------------------------------
  % TITLE SECTION
  %---------------------------------------------------------------------------------------- 
 
 \title{\vspace{-15mm}\fontsize{24pt}{10pt}\selectfont\textbf{ A DATABASE OF GLUCOSE-TOLERANT $\beta$-GLUCOSIDASES }} % Article title
  
  
  \author{ Leandro Liborio da Silva Matos, Diego C\'esar Batista Mariano, Naiara Pantuza, Luana Luiza Bastos, Let\'{\i}cia Xavier Silva Cant\~ao, Raquel Melo Minardi }
  
  \affil{ Universidade Federal de Minas Gerais - UFMG }
  \vspace{-5mm}
  \date{}
  
  %---------------------------------------------------------------------------------------- 
  
  \begin{document}
  
  
  \maketitle % Insert title
  
  
  \thispagestyle{fancy} % All pages have headers and footers
  %----------------------------------------------------------------------------------------  
  % ABSTRACT
  
  %----------------------------------------------------------------------------------------  
  
  \begin{abstract}
  $\beta$-glucosidase (BGL) is an important enzyme for the production process of lignocellulosic biofuels. It is responsible for cellulose degradation in the glucose used in the fermentation process. However,  the literature has described that most BGLs are inhibited by glucose. BGL has been classified into several subfamilies of glycosyl hydrolases,  such as GH1,  GH3,  GH5,  GH9,  GH30,  and GH116. Studies have shown members of family GH1 are more efficient for cellulose degradation in biofuel due to their higher resistance to product inhibition. Recently,  a database of glucose-tolerant $\beta$-glucosidases,  called Glutantbase,  has been proposed for understanding the impact of point mutations in distinct structures. Glutantbase contains data of multiple alignments of GH1 enzymes obtained from Clustal Omega,  catalytic residues,  secondary structure,  and extrapolated mutations for more than 3, 500 GH1 $\beta$-glucosidases sequences extracted from UniProt that were modeled by reference using MODELLER. However,  many aspects of some mutations have not been established. For instance,  the mutations E96K and M416I in Bacillus polymyxa $\beta$-glucosidase,  that have been reported to thermostability. A computational way to try to understand better the impact of these mutations in BGL structure is through coevolution networks. In this study,  we proposed to use coevolution networks to try to understand better the impact of mutations. To determine the coevolution network,  we can use the PFSTATS tool. In addition,  graph modeling can be used to try to understand the importance of correlated residues with many other residues. Finally,  alanine scanning can be used to obtain the importance of the residues highly statistical coupling. We hope that the results obtained in this work may shed light on a new generation of second-generation biofuels. Glutantbase is available at <http://bioinfo.dcc.ufmg.br/glutantbase/>.
  
  Funding:  \\ 
  \end{abstract}
  \end{document} 