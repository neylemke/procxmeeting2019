
  \documentclass[twoside]{article}
  \usepackage[affil-it]{authblk}
  \usepackage{lipsum} % Package to generate dummy text throughout this template
  \usepackage{eurosym}
  \usepackage[sc]{mathpazo} % Use the Palatino font
  \usepackage[T1]{fontenc} % Use 8-bit encoding that has 256 glyphs
  \usepackage[utf8]{inputenc}
  \linespread{1.05} % Line spacing-Palatino needs more space between lines
  \usepackage{microtype} % Slightly tweak font spacing for aesthetics\[IndentingNewLine]
  \usepackage[hmarginratio=1:1,top=32mm,columnsep=20pt]{geometry} % Document margins
  \usepackage{multicol} % Used for the two-column layout of the document
  \usepackage[hang,small,labelfont=bf,up,textfont=it,up]{caption} % Custom captions under//above floats in tables or figures
  \usepackage{booktabs} % Horizontal rules in tables
  \usepackage{float} % Required for tables and figures in the multi-column environment-they need to be placed in specific locations with the[H] (e.g. \begin{table}[H])
  \usepackage{hyperref} % For hyperlinks in the PDF
  \usepackage{lettrine} % The lettrine is the first enlarged letter at the beginning of the text
  \usepackage{paralist} % Used for the compactitem environment which makes bullet points with less space between them
  \usepackage{abstract} % Allows abstract customization
  \renewcommand{\abstractnamefont}{\normalfont\bfseries} 
  %\renewcommand{\abstracttextfont}{\normalfont\small\itshape} % Set the abstract itself to small italic text\[IndentingNewLine]
  \usepackage{titlesec} % Allows customization of titles
  \renewcommand\thesection{\Roman{section}} % Roman numerals for the sections
  \renewcommand\thesubsection{\Roman{subsection}} % Roman numerals for subsections
  \titleformat{\section}[block]{\large\scshape\centering}{\thesection.}{1em}{} % Change the look of the section titles
  \titleformat{\subsection}[block]{\large}{\thesubsection.}{1em}{} % Change the look of the section titles
  \usepackage{fancyhdr} % Headers and footers
  \pagestyle{fancy} % All pages have headers and footers
  \fancyhead{} % Blank out the default header
  \fancyfoot{} % Blank out the default footer
  \fancyhead[C]{X-meeting $\bullet$ October 2019 $\bullet$ Campos do  Jord\~ao} % Custom header text
  \fancyfoot[RO,LE]{} % Custom footer text
  %----------------------------------------------------------------------------------------
  % TITLE SECTION
  %---------------------------------------------------------------------------------------- 
 
 \title{\vspace{-15mm}\fontsize{24pt}{10pt}\selectfont\textbf{ Proteomic approach for the evaluation of oxide nitric dependent pathways during Leishmania major infection }} % Article title
  
  
  \author{ Adriene Yumi Ishimoto, Luiza A. Castro-Jorge, Dario Sim\~oes Zamboni }
  
  \affil{ Universidade de S\~ao Paulo }
  \vspace{-5mm}
  \date{}
  
  %---------------------------------------------------------------------------------------- 
  
  \begin{document}
  
  
  \maketitle % Insert title
  
  
  \thispagestyle{fancy} % All pages have headers and footers
  %----------------------------------------------------------------------------------------  
  % ABSTRACT
  
  %----------------------------------------------------------------------------------------  
  
  \begin{abstract}
  Leishmaniasis is a tropical and subtropical endemic disease caused by parasites of the genus Leishmania. The disease clinical manifestations depend on the infecting Leishmania species and the host immune response,  and can be classified into two types: tegumentary and visceral. Macrophages represent the primary line of defense against infection,  and nitric oxide (NO) production are one of the major mechanisms involved in eliminating parasites. In order to identify proteins involved in the parasite control through the nitric oxide pathway,  we performed a proteomic analysis. Bone marrow derived macrophages (BMDMs) from wild type and NOS2-deficient C57BL/6J mice were infected with Leishmania major,  and protein levels were quantified by mass spectrometry. The analysis was based on statistical calculations available in packages and functions of the R language,  such as t.test function,  and limma and ROTS packages. The differentially expressed proteins (DEPs) were defined as those that obtained p-value less than 0.05. Gsr,  Arg1,  F13a1,  Pcna,  Plin3 and Cd36 proteins were more differentially expressed in the context of Leishmania infection. Through pathway enrichment analysis using packages such as clusterProfiler,  ReactomePA,  WebGestalt and EGSEA,  we identified activated pathways related to the regulation of adaptive immune response,  immune effector process and leukocyte mediated immunity. Cd14,  Cd36,  Arg1,  Ctsl,  Dctn2,  Rab7 e Arf1 were selected as acting in the regulation of the identified pathways by protein interaction analysis. The reliability of the detection of differentially expressed proteins was increased when using different approaches to statistical analysis,  mainly when compared to traditional methods. The evaluation of the protein interaction network allowed to identify important proteins of modulated pathways during the infection,  improving the understanding of infection control in the absence of nitric oxide.
  
  Funding: Universidade de S\~ao Paulo \\ 
  \end{abstract}
  \end{document} 