
  \documentclass[twoside]{article}
  \usepackage[affil-it]{authblk}
  \usepackage{lipsum} % Package to generate dummy text throughout this template
  \usepackage{eurosym}
  \usepackage[sc]{mathpazo} % Use the Palatino font
  \usepackage[T1]{fontenc} % Use 8-bit encoding that has 256 glyphs
  \usepackage[utf8]{inputenc}
  \linespread{1.05} % Line spacing-Palatino needs more space between lines
  \usepackage{microtype} % Slightly tweak font spacing for aesthetics\[IndentingNewLine]
  \usepackage[hmarginratio=1:1,top=32mm,columnsep=20pt]{geometry} % Document margins
  \usepackage{multicol} % Used for the two-column layout of the document
  \usepackage[hang,small,labelfont=bf,up,textfont=it,up]{caption} % Custom captions under//above floats in tables or figures
  \usepackage{booktabs} % Horizontal rules in tables
  \usepackage{float} % Required for tables and figures in the multi-column environment-they need to be placed in specific locations with the[H] (e.g. \begin{table}[H])
  \usepackage{hyperref} % For hyperlinks in the PDF
  \usepackage{lettrine} % The lettrine is the first enlarged letter at the beginning of the text
  \usepackage{paralist} % Used for the compactitem environment which makes bullet points with less space between them
  \usepackage{abstract} % Allows abstract customization
  \renewcommand{\abstractnamefont}{\normalfont\bfseries} 
  %\renewcommand{\abstracttextfont}{\normalfont\small\itshape} % Set the abstract itself to small italic text\[IndentingNewLine]
  \usepackage{titlesec} % Allows customization of titles
  \renewcommand\thesection{\Roman{section}} % Roman numerals for the sections
  \renewcommand\thesubsection{\Roman{subsection}} % Roman numerals for subsections
  \titleformat{\section}[block]{\large\scshape\centering}{\thesection.}{1em}{} % Change the look of the section titles
  \titleformat{\subsection}[block]{\large}{\thesubsection.}{1em}{} % Change the look of the section titles
  \usepackage{fancyhdr} % Headers and footers
  \pagestyle{fancy} % All pages have headers and footers
  \fancyhead{} % Blank out the default header
  \fancyfoot{} % Blank out the default footer
  \fancyhead[C]{X-meeting $\bullet$ October 2019 $\bullet$ Campos do  Jord\~ao} % Custom header text
  \fancyfoot[RO,LE]{} % Custom footer text
  %----------------------------------------------------------------------------------------
  % TITLE SECTION
  %---------------------------------------------------------------------------------------- 
 
 \title{\vspace{-15mm}\fontsize{24pt}{10pt}\selectfont\textbf{ Accurate Identification of Hosts from Environmental Viruses Using Deep Learning Networks and High Level Features }} % Article title
  
  
  \author{ Deyvid Amgarten, Bruno Iha, Aline Maria da Silva, Jo\~ao Carlos Setubal }
  
  \affil{ Laborat\'orio de T\'ecnicas Especiais,  Hospital Israelita Albert Einstein }
  \vspace{-5mm}
  \date{}
  
  %---------------------------------------------------------------------------------------- 
  
  \begin{document}
  
  
  \maketitle % Insert title
  
  
  \thispagestyle{fancy} % All pages have headers and footers
  %----------------------------------------------------------------------------------------  
  % ABSTRACT
  
  %----------------------------------------------------------------------------------------  
  
  \begin{abstract}
  Microbial genomics has been experiencing expressive changes in the last decade,  mostly due to improvements in environmental sampling and sequencing techniques generally known as metagenomics. Thousands of new complete virus genomes are made available every year,  but experimental characterization has not kept pace. In particular,  information about the host of viruses whose genomes have been sequenced is commonly lacking. This is one of the most essential information needed for cultivation,  isolation and many other microbiological characterization techniques. Here we present a toolkit called vHULK (Viral Host Unveiling Kit),  which predicts taxonomic and biological attributes of a virus’ host. Our tool receives complete or high quality virus draft genomes as input,  and provides as output: 1. Probability scores of predicted host genus and/or species; 2. Tables of cross-probabilities among possible hosts; and 3. information theory measurements and a rank of informative features about virus-host relationships. Our methodology is based on feature extraction of virus’ genomes with known hosts leading to matrices containing thousands of features. After feature extraction and feature normalization,  matrices were used to train a multilayer perceptron deep neural network classifier. Performance was measured by assessing validation and training curves,  as well as by measurement in batch test sets. For a multiclass problem of 62 possible bacterial host genus,  vHULK presented an average accuracy of 98\% in the test set. When development is finished,  vHULK will be freely available through user-friendly python scripts at a Github repository.
  
  Funding: This work had the support from CAPES,  CNPq and FAPESP research funding agencies \\ 
  \end{abstract}
  \end{document} 