
  \documentclass[twoside]{article}
  \usepackage[affil-it]{authblk}
  \usepackage{lipsum} % Package to generate dummy text throughout this template
  \usepackage{eurosym}
  \usepackage[sc]{mathpazo} % Use the Palatino font
  \usepackage[T1]{fontenc} % Use 8-bit encoding that has 256 glyphs
  \usepackage[utf8]{inputenc}
  \linespread{1.05} % Line spacing-Palatino needs more space between lines
  \usepackage{microtype} % Slightly tweak font spacing for aesthetics\[IndentingNewLine]
  \usepackage[hmarginratio=1:1,top=32mm,columnsep=20pt]{geometry} % Document margins
  \usepackage{multicol} % Used for the two-column layout of the document
  \usepackage[hang,small,labelfont=bf,up,textfont=it,up]{caption} % Custom captions under//above floats in tables or figures
  \usepackage{booktabs} % Horizontal rules in tables
  \usepackage{float} % Required for tables and figures in the multi-column environment-they need to be placed in specific locations with the[H] (e.g. \begin{table}[H])
  \usepackage{hyperref} % For hyperlinks in the PDF
  \usepackage{lettrine} % The lettrine is the first enlarged letter at the beginning of the text
  \usepackage{paralist} % Used for the compactitem environment which makes bullet points with less space between them
  \usepackage{abstract} % Allows abstract customization
  \renewcommand{\abstractnamefont}{\normalfont\bfseries} 
  %\renewcommand{\abstracttextfont}{\normalfont\small\itshape} % Set the abstract itself to small italic text\[IndentingNewLine]
  \usepackage{titlesec} % Allows customization of titles
  \renewcommand\thesection{\Roman{section}} % Roman numerals for the sections
  \renewcommand\thesubsection{\Roman{subsection}} % Roman numerals for subsections
  \titleformat{\section}[block]{\large\scshape\centering}{\thesection.}{1em}{} % Change the look of the section titles
  \titleformat{\subsection}[block]{\large}{\thesubsection.}{1em}{} % Change the look of the section titles
  \usepackage{fancyhdr} % Headers and footers
  \pagestyle{fancy} % All pages have headers and footers
  \fancyhead{} % Blank out the default header
  \fancyfoot{} % Blank out the default footer
  \fancyhead[C]{X-meeting $\bullet$ October 2019 $\bullet$ Campos do  Jord\~ao} % Custom header text
  \fancyfoot[RO,LE]{} % Custom footer text
  %----------------------------------------------------------------------------------------
  % TITLE SECTION
  %---------------------------------------------------------------------------------------- 
 
 \title{\vspace{-15mm}\fontsize{24pt}{10pt}\selectfont\textbf{ Precise Identification and Genome Recovery of Viral Pathogens Through a Non-Specific Target Virome Clinical Test }} % Article title
  
  
  \author{ Deyvid Amgarten, Fernanda Malta, Murilo Castro Cervato, Nair Hideko Muto, Pedro Sebe, Jo\~ao Renato Rebello Pinho }
  
  \affil{ Laborat\'orio de T\'ecnicas Especiais,  Hospital Israelita Albert Einstein }
  \vspace{-5mm}
  \date{}
  
  %---------------------------------------------------------------------------------------- 
  
  \begin{document}
  
  
  \maketitle % Insert title
  
  
  \thispagestyle{fancy} % All pages have headers and footers
  %----------------------------------------------------------------------------------------  
  % ABSTRACT
  
  %----------------------------------------------------------------------------------------  
  
  \begin{abstract}
  Massive parallel sequencing techniques radically changed the diagnostic workflow by providing a quick and powerful tool for clinical diagnosis and precision medicine. The basis of rare diseases as well as clinically relevant mutations underlying some types of cancers have been precisely diagnosed and characterized using NGS techniques. In a similar way,  etiological causes of infectious diseases,  such as encephalitis,  arboviroses and hepatitis have been precisely identified. In this work,  we report the validation process of a new metagenomic protocol for unbiased identification of viral pathogens in clinical samples. The test is based on a two-step cDNA random amplification followed by Illumina sequencing. Data generated is used as input to a custom bioinformatic pipeline especially developed for delivering fast,  accurate and clinician-friendly diagnosis. The validation process was performed according to College of American Pathologists (CAP) guidelines,  evaluating accuracy,  intra and inter-assay reproducibility,  and limit of detection. Preliminary results show accuracy rates of 100\% compared with standard PCR tests,  total agreement among different assays and 104 virions/mL as limit of detection (similar to PCR limits). Moreover,  we were able to recover high quality data to characterize almost complete or whole genomes from different viral pathogens with mean depth of coverage ranging from 100x to 8000x. Recovered genomes were also used to identify viral genotype and drug resistance variants,  as well as to perform phylogenetics analysis. Altogether,  these results show a multi-functional test to potentially replace several gold-standard molecular diagnosis techniques such as qualitative PCR and Sanger sequencing method in a unique protocol. We emphasize that the virome technique developed in this work is not limited to previous knowledge of the pathogen,  allowing early detection of outbreaks due to novel pathogens and playing a key role in public health surveillance.
  
  Funding:  \\ 
  \end{abstract}
  \end{document} 