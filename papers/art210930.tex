
  \documentclass[twoside]{article}
  \usepackage[affil-it]{authblk}
  \usepackage{lipsum} % Package to generate dummy text throughout this template
  \usepackage{eurosym}
  \usepackage[sc]{mathpazo} % Use the Palatino font
  \usepackage[T1]{fontenc} % Use 8-bit encoding that has 256 glyphs
  \usepackage[utf8]{inputenc}
  \linespread{1.05} % Line spacing-Palatino needs more space between lines
  \usepackage{microtype} % Slightly tweak font spacing for aesthetics\[IndentingNewLine]
  \usepackage[hmarginratio=1:1,top=32mm,columnsep=20pt]{geometry} % Document margins
  \usepackage{multicol} % Used for the two-column layout of the document
  \usepackage[hang,small,labelfont=bf,up,textfont=it,up]{caption} % Custom captions under//above floats in tables or figures
  \usepackage{booktabs} % Horizontal rules in tables
  \usepackage{float} % Required for tables and figures in the multi-column environment-they need to be placed in specific locations with the[H] (e.g. \begin{table}[H])
  \usepackage{hyperref} % For hyperlinks in the PDF
  \usepackage{lettrine} % The lettrine is the first enlarged letter at the beginning of the text
  \usepackage{paralist} % Used for the compactitem environment which makes bullet points with less space between them
  \usepackage{abstract} % Allows abstract customization
  \renewcommand{\abstractnamefont}{\normalfont\bfseries} 
  %\renewcommand{\abstracttextfont}{\normalfont\small\itshape} % Set the abstract itself to small italic text\[IndentingNewLine]
  \usepackage{titlesec} % Allows customization of titles
  \renewcommand\thesection{\Roman{section}} % Roman numerals for the sections
  \renewcommand\thesubsection{\Roman{subsection}} % Roman numerals for subsections
  \titleformat{\section}[block]{\large\scshape\centering}{\thesection.}{1em}{} % Change the look of the section titles
  \titleformat{\subsection}[block]{\large}{\thesubsection.}{1em}{} % Change the look of the section titles
  \usepackage{fancyhdr} % Headers and footers
  \pagestyle{fancy} % All pages have headers and footers
  \fancyhead{} % Blank out the default header
  \fancyfoot{} % Blank out the default footer
  \fancyhead[C]{X-meeting $\bullet$ October 2019 $\bullet$ Campos do  Jord\~ao} % Custom header text
  \fancyfoot[RO,LE]{} % Custom footer text
  %----------------------------------------------------------------------------------------
  % TITLE SECTION
  %---------------------------------------------------------------------------------------- 
 
 \title{\vspace{-15mm}\fontsize{24pt}{10pt}\selectfont\textbf{ Introns and homing endonucleases shape mitochondrial genomes of fungal species from Hypocreales order (Acomycota) }} % Article title
  
  
  \author{ paula  luize camargos fonseca, Ruth Barros, Daniel Silva Ara\'ujo, Dener Eduardo Bortolini, Gabriel Quintanilha Peixoto, Vasco Ariston de Carvalho Azevedo, Bertram Brenig, Luiz Eduardo Vieira Del Bem, Fernanda Badotti, Arist\'oteles G\'oes-Neto, Eric Roberto Guimar\~aes Rocha Aguiar }
  
  \affil{ University G\"ottingen }
  \vspace{-5mm}
  \date{}
  
  %---------------------------------------------------------------------------------------- 
  
  \begin{document}
  
  
  \maketitle % Insert title
  
  
  \thispagestyle{fancy} % All pages have headers and footers
  %----------------------------------------------------------------------------------------  
  % ABSTRACT
  
  %----------------------------------------------------------------------------------------  
  
  \begin{abstract}
  The order Hypocreales is composed of ubiquitous and ecologically diverse fungi classified in saprobes,  biotrophs and pathogens of other species. One of the main genera is Trichoderma,  which are used as biocontrol and biofertilizers agents for plant growth. Due to the variety of ecological functions presented by species from Hypocreales order,  comparative genomics is an important tool to understand the differences observed in the fitness of these organisms. The mitochondrial genome (mtDNA) play an important role,  providing energy to the cells and regulating processes related to immune response. However,  although its importance,  the mechanisms that shape fungal mtDNA still poorly understood. To better understand mechanisms involved in the variability and evolution of mitochondrial genomes we investigated fungal species from the Hypocreales order. First,  we sequenced and annotated T. harzianum mitochondrial genome,  which was compared to others 34 mtDNAs species that were publicly available. Comparative analysis revealed the considerable elasticity mtDNAs,  with length ranging from 24, 565 to 103, 844 pb. Although the size variation observed in mitochondrial genomes,  gene copy number,  size and structure of coding elements were highly conserved,  suggesting that differences is likely on non-coding regions. Among the elements classified as non-coding regions,  introns and homing endonucleases genes (HEGs) were the main contributors to the size variations. 267 out of 332 identified introns showed sequence similarity between species. The most fragmented genes (rrnL and cox1) exhibited the highest frequency of HEGs within intronic regions. In the genes with the lowest frequency of fragmentation (rrns and atp8),  HEGs and introns were absent. We also investigated the possible transference of mitochondrial genes to the nuclear genome (NUMT). The gene nad5 was the most widespread in the nuclear genomes. In contrast,  the genes atp8,  atp9 and cox3,  events of transference were not identified. Since the genes atp8,  atp9 and cox3 are unique to all mtDNA evaluated,  they were used to construct a time-scaled phylogenetic tree to estimate the origin of the order based on mitochondria information and to determine whether the presence of fragmented genes,  introns,  HEGs and NUMTs were related to time divergence. However,  a weak association was found,  indicating that other mechanisms could be responsible for the abundance of introns and HEGs. Altogether,  our results indicate that HEGs and introns play an important role on the shaping of mitochondrial genomes,  whether on fragmentation,  duplication or transference of genes to the nuclear genome.
  
  Funding: Conselho Nacional de Desenvolvimento Cient\'{\i}fico e Tecnol\'ogico \\ 
  \end{abstract}
  \end{document} 