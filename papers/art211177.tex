
  \documentclass[twoside]{article}
  \usepackage[affil-it]{authblk}
  \usepackage{lipsum} % Package to generate dummy text throughout this template
  \usepackage{eurosym}
  \usepackage[sc]{mathpazo} % Use the Palatino font
  \usepackage[T1]{fontenc} % Use 8-bit encoding that has 256 glyphs
  \usepackage[utf8]{inputenc}
  \linespread{1.05} % Line spacing-Palatino needs more space between lines
  \usepackage{microtype} % Slightly tweak font spacing for aesthetics\[IndentingNewLine]
  \usepackage[hmarginratio=1:1,top=32mm,columnsep=20pt]{geometry} % Document margins
  \usepackage{multicol} % Used for the two-column layout of the document
  \usepackage[hang,small,labelfont=bf,up,textfont=it,up]{caption} % Custom captions under//above floats in tables or figures
  \usepackage{booktabs} % Horizontal rules in tables
  \usepackage{float} % Required for tables and figures in the multi-column environment-they need to be placed in specific locations with the[H] (e.g. \begin{table}[H])
  \usepackage{hyperref} % For hyperlinks in the PDF
  \usepackage{lettrine} % The lettrine is the first enlarged letter at the beginning of the text
  \usepackage{paralist} % Used for the compactitem environment which makes bullet points with less space between them
  \usepackage{abstract} % Allows abstract customization
  \renewcommand{\abstractnamefont}{\normalfont\bfseries} 
  %\renewcommand{\abstracttextfont}{\normalfont\small\itshape} % Set the abstract itself to small italic text\[IndentingNewLine]
  \usepackage{titlesec} % Allows customization of titles
  \renewcommand\thesection{\Roman{section}} % Roman numerals for the sections
  \renewcommand\thesubsection{\Roman{subsection}} % Roman numerals for subsections
  \titleformat{\section}[block]{\large\scshape\centering}{\thesection.}{1em}{} % Change the look of the section titles
  \titleformat{\subsection}[block]{\large}{\thesubsection.}{1em}{} % Change the look of the section titles
  \usepackage{fancyhdr} % Headers and footers
  \pagestyle{fancy} % All pages have headers and footers
  \fancyhead{} % Blank out the default header
  \fancyfoot{} % Blank out the default footer
  \fancyhead[C]{X-meeting $\bullet$ October 2019 $\bullet$ Campos do  Jord\~ao} % Custom header text
  \fancyfoot[RO,LE]{} % Custom footer text
  %----------------------------------------------------------------------------------------
  % TITLE SECTION
  %---------------------------------------------------------------------------------------- 
 
 \title{\vspace{-15mm}\fontsize{24pt}{10pt}\selectfont\textbf{ Mitogenome data reveals strong differentiation among the isolated populations of Heliconius hermathena: a white sand ecosystem specialist. }} % Article title
  
  
  \author{ Pedro de Gusm\~ao Ribeiro, Renato Rogner Ramos, Darli Massardo, Marilia Lion, Marcio Zikan Cardoso, Marcus Kronforst, Andr\'e Victor Lucci Freitas, Marcelo Mendes Brand\~ao, Karina Lucas da Silva Brand\~ao }
  
  \affil{ UNICAMP }
  \vspace{-5mm}
  \date{}
  
  %---------------------------------------------------------------------------------------- 
  
  \begin{document}
  
  
  \maketitle % Insert title
  
  
  \thispagestyle{fancy} % All pages have headers and footers
  %----------------------------------------------------------------------------------------  
  % ABSTRACT
  
  %----------------------------------------------------------------------------------------  
  
  \begin{abstract}
  Cycles of forest retraction and expansion during the Pleistocene presumably played a crucial role in the diversification of neotropical species by the formation of isolated forest refugia. Similarly,  these cycles generated the strongly isolated pattern of the Amazonian white sand ecosystems: non-forest habitats with white sandy soils surrounded by forest matrix. As previously inferred by ecological and morphological data,  such processes may have led to the isolation and diversification of subpopulations of H. hermathena,  a butterfly endemic to these ecosystems with seven subspecies identified by their color patterns. Nevertheless,  the genetic differentiation and structure among H. hermathena subpopulations are still unknown. We sequenced the mitogenomes of 71 individuals across six of the seven subespecies of H. hermathena from eitgh different localities. We then performed bayesian phylogenetic inference and population structure analyses in order to analyze patterns of differentiation among H. hermathena subpopulations and their phylogenetic relationships. Most of the analysis were performed in user-friendly platforms such as Galaxy Project and Geneoius 10,  which implements most of the state-of-the-art bioinformatics programs. Currently,  we are accessing the divergence times among these subpopulations to infer the specific mechanisms regarding the group’s phylogeography and evolution. We show that two populations with equal wing color pattern (H. h. sheppardi) sampled from two different localities exhibit high genetic divergence and population structure. Conversely,  a pair of phenotipically divergent subspecies (H. h. vereatta and H. h. duckeii) from two near sample sites in Faro,  are genetically similar and have lower fixation index when compared to other sample localities. Furthermore,  we found highly distinct haplogroups among H. hermathena subpopulations,  with each haplogroup strongly structured in its own sampling locality. Our results suggest that the fragmented pattern of the white sand ecosystems may have actually played an important role in the formation and maintenance of differentiated populations and subspecies of H. hermathena. For this butterfly,  the forest function as a barrier for free gene flow among its current populations.
  
  Funding:  \\ 
  \end{abstract}
  \end{document} 