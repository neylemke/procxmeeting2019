
  \documentclass[twoside]{article}
  \usepackage[affil-it]{authblk}
  \usepackage{lipsum} % Package to generate dummy text throughout this template
  \usepackage{eurosym}
  \usepackage[sc]{mathpazo} % Use the Palatino font
  \usepackage[T1]{fontenc} % Use 8-bit encoding that has 256 glyphs
  \usepackage[utf8]{inputenc}
  \linespread{1.05} % Line spacing-Palatino needs more space between lines
  \usepackage{microtype} % Slightly tweak font spacing for aesthetics\[IndentingNewLine]
  \usepackage[hmarginratio=1:1,top=32mm,columnsep=20pt]{geometry} % Document margins
  \usepackage{multicol} % Used for the two-column layout of the document
  \usepackage[hang,small,labelfont=bf,up,textfont=it,up]{caption} % Custom captions under//above floats in tables or figures
  \usepackage{booktabs} % Horizontal rules in tables
  \usepackage{float} % Required for tables and figures in the multi-column environment-they need to be placed in specific locations with the[H] (e.g. \begin{table}[H])
  \usepackage{hyperref} % For hyperlinks in the PDF
  \usepackage{lettrine} % The lettrine is the first enlarged letter at the beginning of the text
  \usepackage{paralist} % Used for the compactitem environment which makes bullet points with less space between them
  \usepackage{abstract} % Allows abstract customization
  \renewcommand{\abstractnamefont}{\normalfont\bfseries} 
  %\renewcommand{\abstracttextfont}{\normalfont\small\itshape} % Set the abstract itself to small italic text\[IndentingNewLine]
  \usepackage{titlesec} % Allows customization of titles
  \renewcommand\thesection{\Roman{section}} % Roman numerals for the sections
  \renewcommand\thesubsection{\Roman{subsection}} % Roman numerals for subsections
  \titleformat{\section}[block]{\large\scshape\centering}{\thesection.}{1em}{} % Change the look of the section titles
  \titleformat{\subsection}[block]{\large}{\thesubsection.}{1em}{} % Change the look of the section titles
  \usepackage{fancyhdr} % Headers and footers
  \pagestyle{fancy} % All pages have headers and footers
  \fancyhead{} % Blank out the default header
  \fancyfoot{} % Blank out the default footer
  \fancyhead[C]{X-meeting $\bullet$ October 2019 $\bullet$ Campos do  Jord\~ao} % Custom header text
  \fancyfoot[RO,LE]{} % Custom footer text
  %----------------------------------------------------------------------------------------
  % TITLE SECTION
  %---------------------------------------------------------------------------------------- 
 
 \title{\vspace{-15mm}\fontsize{24pt}{10pt}\selectfont\textbf{ A STRUCTURAL AND EVOLUTIVE APPROACH ON NUCLEOTIDE EXCISION REPAIR IN EUKARYOTES }} % Article title
  
  
  \author{ Rayana dos Santos Feltrin, ANA L\'UCIA ANVERSA SEGATTO, Tiago Antonio de Souza, Andr\'e Passaglia Schuch }
  
  \affil{ Universidade Federal de Santa Maria }
  \vspace{-5mm}
  \date{}
  
  %---------------------------------------------------------------------------------------- 
  
  \begin{document}
  
  
  \maketitle % Insert title
  
  
  \thispagestyle{fancy} % All pages have headers and footers
  %----------------------------------------------------------------------------------------  
  % ABSTRACT
  
  %----------------------------------------------------------------------------------------  
  
  \begin{abstract}
  Among several DNA repair mechanisms developed throughout evolution,  the nucleotide excision repair (NER) is the most versatile repair pathway as it removes a wide range of structurally unrelated lesions that distort the double-helix. Given its importance in the maintenance of genome integrity,  it appeared earlier in the evolution of species. However,  few current studies involving NER have an evolutive approach. Moreover,  the availability of a large amount of data on genome databases makes possible to retrieve sequences of NER components from many species. Therefore,  we performed a homology assessment for ten main proteins from the NER pathway in 13 eukaryotic organisms (Mus musculus,  Gallus gallus,  Alligator mississippiensis,  Xenopus laevis,  Danio rerio,  Caenorhabditis elegans,  Drosophila melogaster,  Arabidopsis thaliana,  Trypanosoma cruzi,  Giardia lamblia,  Saccharomyces cerevisiae,  and Schizosaccharomyces pombe). To do so,  we used human reference sequences as queries to perform a tblastn search against each genome,  aiming to obtain the nucleotide and respective protein sequences from each species. In order to get the best orthologue candidates,  we applied four criteria as sequence filters: e-value,  percent identity,  protein domains in common,  and protein sequence length. To identify the conserved protein domains,  we submitted the sequences to Simple Modular Architecture Research Tool (SMART),  considering Pfam database. The protein sequences from each NER component were aligned on MEGA 7 software by MUSCLE,  and then the most conserved regions were obtained with Gblocks 0.91b. To set the best evolutionary models,  the resulting alignments were evaluated on ProtTest 3.4.2. Thus,  an estimate of the evolutive distance of the retrieved sequences in relation to the human ones was conducted on MEGA 7. We also run a phylogenetic reconstruction for each NER protein by using the Neighbor-Joining algorithm,  considering the substitution model JTT+G,  1000 bootstrap replicates,  and pairwise deletion with the same software. Additional analyses were carried out to assess the number of introns of the genes by using Gene Structure Display Server 2.0 (GSDS 2.0),  as well as gene sizes,  relying on information from GenBank. In general,  our results show that most of NER components analyzed have homologs in the eukaryotic species,  but some of them were not detected using our criteria. We also observed that although several variations are found in the gene structures,  the protein structure has only small changes. Furthermore,  this work suggests further research to better investigate the real biological efficiency of the NER pathway in some eukaryotic organisms.
  
  Funding: CAPES and CNPq \\ 
  \end{abstract}
  \end{document} 