
  \documentclass[twoside]{article}
  \usepackage[affil-it]{authblk}
  \usepackage{lipsum} % Package to generate dummy text throughout this template
  \usepackage{eurosym}
  \usepackage[sc]{mathpazo} % Use the Palatino font
  \usepackage[T1]{fontenc} % Use 8-bit encoding that has 256 glyphs
  \usepackage[utf8]{inputenc}
  \linespread{1.05} % Line spacing-Palatino needs more space between lines
  \usepackage{microtype} % Slightly tweak font spacing for aesthetics\[IndentingNewLine]
  \usepackage[hmarginratio=1:1,top=32mm,columnsep=20pt]{geometry} % Document margins
  \usepackage{multicol} % Used for the two-column layout of the document
  \usepackage[hang,small,labelfont=bf,up,textfont=it,up]{caption} % Custom captions under//above floats in tables or figures
  \usepackage{booktabs} % Horizontal rules in tables
  \usepackage{float} % Required for tables and figures in the multi-column environment-they need to be placed in specific locations with the[H] (e.g. \begin{table}[H])
  \usepackage{hyperref} % For hyperlinks in the PDF
  \usepackage{lettrine} % The lettrine is the first enlarged letter at the beginning of the text
  \usepackage{paralist} % Used for the compactitem environment which makes bullet points with less space between them
  \usepackage{abstract} % Allows abstract customization
  \renewcommand{\abstractnamefont}{\normalfont\bfseries} 
  %\renewcommand{\abstracttextfont}{\normalfont\small\itshape} % Set the abstract itself to small italic text\[IndentingNewLine]
  \usepackage{titlesec} % Allows customization of titles
  \renewcommand\thesection{\Roman{section}} % Roman numerals for the sections
  \renewcommand\thesubsection{\Roman{subsection}} % Roman numerals for subsections
  \titleformat{\section}[block]{\large\scshape\centering}{\thesection.}{1em}{} % Change the look of the section titles
  \titleformat{\subsection}[block]{\large}{\thesubsection.}{1em}{} % Change the look of the section titles
  \usepackage{fancyhdr} % Headers and footers
  \pagestyle{fancy} % All pages have headers and footers
  \fancyhead{} % Blank out the default header
  \fancyfoot{} % Blank out the default footer
  \fancyhead[C]{X-meeting $\bullet$ October 2019 $\bullet$ Campos do  Jord\~ao} % Custom header text
  \fancyfoot[RO,LE]{} % Custom footer text
  %----------------------------------------------------------------------------------------
  % TITLE SECTION
  %---------------------------------------------------------------------------------------- 
 
 \title{\vspace{-15mm}\fontsize{24pt}{10pt}\selectfont\textbf{ Classification,  diversity and structural analysis of Ammonium Transporters }} % Article title
  
  
  \author{ Gilberto Hideo Kaihami, Aureliano Coelho Proen\c{c}a Guedes, Gabriel S\'anchez Hueck, Gianlucca Gon\c{c}alves Nicastro, Robson Francisco de Souza }
  
  \affil{  }
  \vspace{-5mm}
  \date{}
  
  %---------------------------------------------------------------------------------------- 
  
  \begin{document}
  
  
  \maketitle % Insert title
  
  
  \thispagestyle{fancy} % All pages have headers and footers
  %----------------------------------------------------------------------------------------  
  % ABSTRACT
  
  %----------------------------------------------------------------------------------------  
  
  \begin{abstract}
  Nitrogen is an essential building block for many biological macromolecules. Its transportation through the cell membrane occurs in the protonated state (NH4+) or in the reduced form (NH3) and is mediated by proteins belonging to the AMT/MEP/Rh superfamily. Although the importance of this transporter family has long been recognized,  the evolutionary relationships of several experimentally characterized homologs and the transport mechanism remain obscure. The current literature recognizes most members of the AMT and Rh families from eukaryotes,  while homologs of MEP are found across all domains of life. To better understand the distribution and evolution of this superfamily,  we systematically searched for homologs in the NCBI’s protein database and inferred phylogenies for both the superfamily and each of its member families. Our results confidently assign to the AMT family several bacterial and archaeal homologs and revealed a well defined monophyletic group,  comprising several instances of fusions of AMT homologs to signal transduction output domains,  such as kinases,  nucleotidyl cyclases,  chemotaxis proteins and DNA binding domains. These fusions are typical of gram-negative bacteria. Gram-positive bacteria and Archaea often encode fusions to the regulatory protein P-II that belong to two other,  unrelated,  clades. Recent experimental work has shown that an AMT domain fused to a histidine kinase,  isolated from a gram-negative bacteria,  lost its transporter activity and acts as a sensor for extracellular levels of ammonium. Analysis of the structure of the histidine kinase’s sensor domain suggests the signal transduction fusion,  and the associated functional shift,  may have been favored by the localization of the NH4+ ion in AMT-like transporters,  where it occupies a non-canonical pore and is stabilized by residues distinct from those observed in the MEP and Rh proteins. Based on these results,  we suggest a novel mechanism for transport and sensing of ammonium for the members of the AMT family and highlight experimental evidence that supports the hypothesis that all members of the AMT family fused to signal transduction domains lack its transport activity and function only as a sensor.
  
  Funding: Capes,  Fapesp \\ 
  \end{abstract}
  \end{document} 