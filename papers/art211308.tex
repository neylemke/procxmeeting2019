
  \documentclass[twoside]{article}
  \usepackage[affil-it]{authblk}
  \usepackage{lipsum} % Package to generate dummy text throughout this template
  \usepackage{eurosym}
  \usepackage[sc]{mathpazo} % Use the Palatino font
  \usepackage[T1]{fontenc} % Use 8-bit encoding that has 256 glyphs
  \usepackage[utf8]{inputenc}
  \linespread{1.05} % Line spacing-Palatino needs more space between lines
  \usepackage{microtype} % Slightly tweak font spacing for aesthetics\[IndentingNewLine]
  \usepackage[hmarginratio=1:1,top=32mm,columnsep=20pt]{geometry} % Document margins
  \usepackage{multicol} % Used for the two-column layout of the document
  \usepackage[hang,small,labelfont=bf,up,textfont=it,up]{caption} % Custom captions under//above floats in tables or figures
  \usepackage{booktabs} % Horizontal rules in tables
  \usepackage{float} % Required for tables and figures in the multi-column environment-they need to be placed in specific locations with the[H] (e.g. \begin{table}[H])
  \usepackage{hyperref} % For hyperlinks in the PDF
  \usepackage{lettrine} % The lettrine is the first enlarged letter at the beginning of the text
  \usepackage{paralist} % Used for the compactitem environment which makes bullet points with less space between them
  \usepackage{abstract} % Allows abstract customization
  \renewcommand{\abstractnamefont}{\normalfont\bfseries} 
  %\renewcommand{\abstracttextfont}{\normalfont\small\itshape} % Set the abstract itself to small italic text\[IndentingNewLine]
  \usepackage{titlesec} % Allows customization of titles
  \renewcommand\thesection{\Roman{section}} % Roman numerals for the sections
  \renewcommand\thesubsection{\Roman{subsection}} % Roman numerals for subsections
  \titleformat{\section}[block]{\large\scshape\centering}{\thesection.}{1em}{} % Change the look of the section titles
  \titleformat{\subsection}[block]{\large}{\thesubsection.}{1em}{} % Change the look of the section titles
  \usepackage{fancyhdr} % Headers and footers
  \pagestyle{fancy} % All pages have headers and footers
  \fancyhead{} % Blank out the default header
  \fancyfoot{} % Blank out the default footer
  \fancyhead[C]{X-meeting $\bullet$ October 2019 $\bullet$ Campos do  Jord\~ao} % Custom header text
  \fancyfoot[RO,LE]{} % Custom footer text
  %----------------------------------------------------------------------------------------
  % TITLE SECTION
  %---------------------------------------------------------------------------------------- 
 
 \title{\vspace{-15mm}\fontsize{24pt}{10pt}\selectfont\textbf{ Investigating the genomics of the evolution of sociality in Hymenoptera }} % Article title
  
  
  \author{ Maycon Douglas de Oliveira, Jos\'e Eust\'aquio dos Santos J\'unior, Francisco Pereira Lobo }
  
  \affil{ Universidade Federal de Minas Gerais }
  \vspace{-5mm}
  \date{}
  
  %---------------------------------------------------------------------------------------- 
  
  \begin{document}
  
  
  \maketitle % Insert title
  
  
  \thispagestyle{fancy} % All pages have headers and footers
  %----------------------------------------------------------------------------------------  
  % ABSTRACT
  
  %----------------------------------------------------------------------------------------  
  
  \begin{abstract}
  Different forms of social behavior arose across metazoans,  varying from simple aggregates to caste-differentiated insect colonies. Wasps,  ants,  and bees,  insects of the Order Hymenoptera,  have a wide variation in sociality,  ranging from solitary individuals to colonies with hundreds of millions of individuals and the most complex form of social behavior,  eusociality,  defined by traits such as reproduction-based division of labor and cooperative brood care. One of the possible ways of annotating genes is by associating those with their biological functions,  which are descriptive terms of the possible roles of that gene’s products in biological systems. In this work,  using the order of magnitude of individuals per colony (IPC) as a proxy for society complexity,  we aimed at surveying high-quality Hymenoptera genomes for biological functions whose frequencies are significantly associated with the increase of colony size,  searching for possible genomic components that may contribute to the emergence of this complex phenotype. After a thorough literature and database review,  we selected 39 Hymenoptera species that have both high quality non-redundant proteomes (defined as the sets of the longest coding sequences per locus with an expected content of single-copy orthologs > 90\%) and estimated IPC values. Each proteome was de novo annotated using InterProScan to predict protein domains using the Pfam database. Using this data,  we searched for biological functions whose frequencies across genomes are significantly associated with IPC values. For significance,  we took into account the multiple hypothesis scenarios for both Pearson's correlation and phylogeny-aware linear models,  requiring corrected p-values < 0.1 for both tests. From the set of 4520 distinct biological functions,  11 were found to have a significant positive correlation. Some of them are proposed to play important roles in eusocial insects,  such as GO:0016575 (histone deacetylation) and GO:0004407 (histone deacetylase activity). Epigenetic transcription control plays an important role in eusocial insects,  regulating caste differentiation in Hymenoptera like Apis mellifera. Other biological functions and protein domains found to have correlations with higher degrees of sociality remain to be characterized in this context,  and are interesting candidates to be subject of future research. We look forward to keep investigating those results and possibly shed new light on the genomics of eusociality in Hymenoptera.
  
  Funding: PPG Gen\'etica - ICB - UFMG \\ 
  \end{abstract}
  \end{document} 