
  \documentclass[twoside]{article}
  \usepackage[affil-it]{authblk}
  \usepackage{lipsum} % Package to generate dummy text throughout this template
  \usepackage{eurosym}
  \usepackage[sc]{mathpazo} % Use the Palatino font
  \usepackage[T1]{fontenc} % Use 8-bit encoding that has 256 glyphs
  \usepackage[utf8]{inputenc}
  \linespread{1.05} % Line spacing-Palatino needs more space between lines
  \usepackage{microtype} % Slightly tweak font spacing for aesthetics\[IndentingNewLine]
  \usepackage[hmarginratio=1:1,top=32mm,columnsep=20pt]{geometry} % Document margins
  \usepackage{multicol} % Used for the two-column layout of the document
  \usepackage[hang,small,labelfont=bf,up,textfont=it,up]{caption} % Custom captions under//above floats in tables or figures
  \usepackage{booktabs} % Horizontal rules in tables
  \usepackage{float} % Required for tables and figures in the multi-column environment-they need to be placed in specific locations with the[H] (e.g. \begin{table}[H])
  \usepackage{hyperref} % For hyperlinks in the PDF
  \usepackage{lettrine} % The lettrine is the first enlarged letter at the beginning of the text
  \usepackage{paralist} % Used for the compactitem environment which makes bullet points with less space between them
  \usepackage{abstract} % Allows abstract customization
  \renewcommand{\abstractnamefont}{\normalfont\bfseries} 
  %\renewcommand{\abstracttextfont}{\normalfont\small\itshape} % Set the abstract itself to small italic text\[IndentingNewLine]
  \usepackage{titlesec} % Allows customization of titles
  \renewcommand\thesection{\Roman{section}} % Roman numerals for the sections
  \renewcommand\thesubsection{\Roman{subsection}} % Roman numerals for subsections
  \titleformat{\section}[block]{\large\scshape\centering}{\thesection.}{1em}{} % Change the look of the section titles
  \titleformat{\subsection}[block]{\large}{\thesubsection.}{1em}{} % Change the look of the section titles
  \usepackage{fancyhdr} % Headers and footers
  \pagestyle{fancy} % All pages have headers and footers
  \fancyhead{} % Blank out the default header
  \fancyfoot{} % Blank out the default footer
  \fancyhead[C]{X-meeting $\bullet$ October 2019 $\bullet$ Campos do  Jord\~ao} % Custom header text
  \fancyfoot[RO,LE]{} % Custom footer text
  %----------------------------------------------------------------------------------------
  % TITLE SECTION
  %---------------------------------------------------------------------------------------- 
 
 \title{\vspace{-15mm}\fontsize{24pt}{10pt}\selectfont\textbf{ Novel HMG-CoA reductase inhibitors development by integrating dyslipidemic patients’ genetic studies and molecular modelling }} % Article title
  
  
  \author{ Glaucio Monteiro Ferreira, Victor Fernandes de Oliveira, Thales Kronenberger, Ros\'ario Dominguez Crespo Hirata, Mario Hiroyuki Hirata, Fausto Feres }
  
  \affil{ Laboratory of Molecular Research in Cardiology (LIMC),  Dante Pazzanese Institute of Cardiology,  S\~ao Paulo,  Brazil. }
  \vspace{-5mm}
  \date{}
  
  %---------------------------------------------------------------------------------------- 
  
  \begin{document}
  
  
  \maketitle % Insert title
  
  
  \thispagestyle{fancy} % All pages have headers and footers
  %----------------------------------------------------------------------------------------  
  % ABSTRACT
  
  %----------------------------------------------------------------------------------------  
  
  \begin{abstract}
  Dyslipidemias are a group of functional disease caused by any alteration in lipid metabolism,  resulting modifications in plasma of lipoproteins. The most important lipoprotein related with high risk to develop atherosclerosis is low-density lipoproteins,  that is treated mainly by statins,  the first-choice pharmacological therapy,  a potent inhibitor of 3-hydroxy-3-methyl-glutaryl-coenzyme A reductase (HMGCR). Previous study we identified functional variants of HMGCR,  whose structural conformations modify the molecular interactions with HMG-CoA/NADPH/statins,  thus contributing to the treatment failure. This data can explain in part a decreased treatment efficiency. This study aims to shed a light on the HMGCR-resistance phenomena from a structural point of view. Therewith,  in order to obtain new HMGCR-inhibiting drugs,  the use of strategies rational drug planning integrating genetic knowledge,  pharmaceutical chemistry,  pharmacology,  biochemistry and molecular modelling could contribute to reduce the current pandemic scenario of cardiovascular diseases and reduce public costs. 3D structures of the proteins containing HMGCR variants generated by homology modeling method. The HMGCR crystallographic structure that used was deposited in the Protein Data Bank (PDB: 1HWK),  which was used to identify important regions of the active site,  disorganized regions and conserved regions of the HMGCR,  the global alignment was carried out to detect these regions. In order to validate the model,  the quality of the model evaluated in comparison to the crystallographic structure by Ramachandran plots. Variants were identified in HMGCR gene in our sequencing database at the Dante Pazzanese Institute of Cardiology (IDPC) and compared with the literature (IDPC: rs12916; rs59083; rs5909 and rs17238554; rs5908; rs2241402 and rs3846662). It is noteworthy that only rs5908 (Ile638Val) is exonic and missense variant. However,  its clinical significance remains to be fully characterized. Ultimately,  the understanding,  at the molecular level,  of these mutations and their functional impact on the statin interaction. For instance,  the rs5908 polymorphism,  which is contained in the interface between subunits HMGCR,  may have further structural implications since it is opposite to the statin binding site. Patients who present this variant demonstrate clinical data of cholesterol fractions between CT: 250 mg/dL; HDL: 50 mg/dL e LDL: 210 mg/dL (sequencing database at IDCP). The results will overall contribute to the scientific knowledge in the area of drug discovery as well as in the field of pharmacogenomics,  highlighting the importance of understanding the genetic mechanisms and integrating information for the discovery of new drugs,  not only for the treatment of dyslipidemias but also for the treatment of other diseases.
  
  Funding: FAPESP project n$^o$ 2019/06172-4 \\ 
  \end{abstract}
  \end{document} 