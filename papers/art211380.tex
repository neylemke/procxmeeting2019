
  \documentclass[twoside]{article}
  \usepackage[affil-it]{authblk}
  \usepackage{lipsum} % Package to generate dummy text throughout this template
  \usepackage{eurosym}
  \usepackage[sc]{mathpazo} % Use the Palatino font
  \usepackage[T1]{fontenc} % Use 8-bit encoding that has 256 glyphs
  \usepackage[utf8]{inputenc}
  \linespread{1.05} % Line spacing-Palatino needs more space between lines
  \usepackage{microtype} % Slightly tweak font spacing for aesthetics\[IndentingNewLine]
  \usepackage[hmarginratio=1:1,top=32mm,columnsep=20pt]{geometry} % Document margins
  \usepackage{multicol} % Used for the two-column layout of the document
  \usepackage[hang,small,labelfont=bf,up,textfont=it,up]{caption} % Custom captions under//above floats in tables or figures
  \usepackage{booktabs} % Horizontal rules in tables
  \usepackage{float} % Required for tables and figures in the multi-column environment-they need to be placed in specific locations with the[H] (e.g. \begin{table}[H])
  \usepackage{hyperref} % For hyperlinks in the PDF
  \usepackage{lettrine} % The lettrine is the first enlarged letter at the beginning of the text
  \usepackage{paralist} % Used for the compactitem environment which makes bullet points with less space between them
  \usepackage{abstract} % Allows abstract customization
  \renewcommand{\abstractnamefont}{\normalfont\bfseries} 
  %\renewcommand{\abstracttextfont}{\normalfont\small\itshape} % Set the abstract itself to small italic text\[IndentingNewLine]
  \usepackage{titlesec} % Allows customization of titles
  \renewcommand\thesection{\Roman{section}} % Roman numerals for the sections
  \renewcommand\thesubsection{\Roman{subsection}} % Roman numerals for subsections
  \titleformat{\section}[block]{\large\scshape\centering}{\thesection.}{1em}{} % Change the look of the section titles
  \titleformat{\subsection}[block]{\large}{\thesubsection.}{1em}{} % Change the look of the section titles
  \usepackage{fancyhdr} % Headers and footers
  \pagestyle{fancy} % All pages have headers and footers
  \fancyhead{} % Blank out the default header
  \fancyfoot{} % Blank out the default footer
  \fancyhead[C]{X-meeting $\bullet$ October 2019 $\bullet$ Campos do  Jord\~ao} % Custom header text
  \fancyfoot[RO,LE]{} % Custom footer text
  %----------------------------------------------------------------------------------------
  % TITLE SECTION
  %---------------------------------------------------------------------------------------- 
 
 \title{\vspace{-15mm}\fontsize{24pt}{10pt}\selectfont\textbf{ PCSK9 three-dimensional reconstruction by homology modelling and new LDL receptor interaction regions }} % Article title
  
  
  \author{ Vitor Galv\~ao Lopes, Victor Fernandes de Oliveira, Thales Kronenberger, Mario Hiroyuki Hirata, Ros\'ario Dominguez Crespo Hirata, Glaucio Monteiro Ferreira }
  
  \affil{ Laboratory of Molecular Biology applied to Diagnostic (LBMAD),  Department of Pharmacy,  Faculty of Pharmaceutical Sciences,  University of S\~ao Paulo,  S\~ao Paulo,  SP,  Brazil. }
  \vspace{-5mm}
  \date{}
  
  %---------------------------------------------------------------------------------------- 
  
  \begin{document}
  
  
  \maketitle % Insert title
  
  
  \thispagestyle{fancy} % All pages have headers and footers
  %----------------------------------------------------------------------------------------  
  % ABSTRACT
  
  %----------------------------------------------------------------------------------------  
  
  \begin{abstract}
  Dyslipidemias are a group of functional disease caused by any alteration in lipid metabolism,  resulting modifications in plasma of lipoproteins. The most important lipoprotein related with high risk to develop atherosclerosis is low-density lipoproteins,  that is treated mainly by statins,  the first-choice pharmacological therapy,  a potent inhibitor of 3-hydroxy-3-methyl-glutaryl-coenzyme A reductase (HMGCR). In this way,  proprotein convertase subtilisin/kexin type 9 (PCSK9) is an enzyme that cleaves low-density lipoprotein (LDL) receptor and therefore it controls cholesterol homeostasis. Some specific mutations cause gain-in-fuction of PCSK9,  which result in increased cholesterol levels in blood. In this way,  the PCSK9 has been used as a target to develop cholesterol lowering terapies. This work aimed to build a tridimensional model of Proprotein convertase subtilisin/kexin type 9 (PCSK9) protein to study the interaction regions with the LDL receptor,  an important molecule in cholesterol homeostasis. The PCSK9 3D structure (Proprotein convertase subtilisin/kexin type 9) were constructed by homology modelling method. In this way,  was used the crystallized structure deposited in Protein Data Bank (PDB: 2P4E) to identify the catalytic site regions of interest,  between disorganized or conserved regions of PCSK9. The global alignment was validated by Z-score and Ramachandran graphics. In the 3D model of PCSK9,  it was possible to identify regions had not been elucidated in the crystallographic structure (Figure 1). The regions reconstructed by the homology modeling method are mainly found in the PCSK9 catalytic domain (in yellow),  where interactions with the LDL receptor occur. The C-terminal domain (in green) is responsible for the structural stability of PCSK9. The pre-domain (in red) is the self-cleaved region by PCSK9 itself and is not part of the tertiary structure that interacts with the LDL receptor. In the catalytic domain,  amino acids (Asp168,  Glu169,  Tyr170,  Gln171,  Pro172,  Pro173 and Asp174) that are not found in the PDB:2P4E crystallographic structure can be observed,  possibly due to the limitation of the crystallography method in predicting very labile regions. The Ramachandram graph presented 0.1\% (Gly68) of amino acids outliers,  demonstrating robustness of the constructed model. Three-dimensional reconstruction makes it possible to understand the structure of PCSK9 and to identify new regions of interaction with the LDL receptor that may be important for the development of new drugs.
  
  Funding: FAPESP project n$^o$ 2019/06172-4 \\ 
  \end{abstract}
  \end{document} 