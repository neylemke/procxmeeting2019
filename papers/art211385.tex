
  \documentclass[twoside]{article}
  \usepackage[affil-it]{authblk}
  \usepackage{lipsum} % Package to generate dummy text throughout this template
  \usepackage{eurosym}
  \usepackage[sc]{mathpazo} % Use the Palatino font
  \usepackage[T1]{fontenc} % Use 8-bit encoding that has 256 glyphs
  \usepackage[utf8]{inputenc}
  \linespread{1.05} % Line spacing-Palatino needs more space between lines
  \usepackage{microtype} % Slightly tweak font spacing for aesthetics\[IndentingNewLine]
  \usepackage[hmarginratio=1:1,top=32mm,columnsep=20pt]{geometry} % Document margins
  \usepackage{multicol} % Used for the two-column layout of the document
  \usepackage[hang,small,labelfont=bf,up,textfont=it,up]{caption} % Custom captions under//above floats in tables or figures
  \usepackage{booktabs} % Horizontal rules in tables
  \usepackage{float} % Required for tables and figures in the multi-column environment-they need to be placed in specific locations with the[H] (e.g. \begin{table}[H])
  \usepackage{hyperref} % For hyperlinks in the PDF
  \usepackage{lettrine} % The lettrine is the first enlarged letter at the beginning of the text
  \usepackage{paralist} % Used for the compactitem environment which makes bullet points with less space between them
  \usepackage{abstract} % Allows abstract customization
  \renewcommand{\abstractnamefont}{\normalfont\bfseries} 
  %\renewcommand{\abstracttextfont}{\normalfont\small\itshape} % Set the abstract itself to small italic text\[IndentingNewLine]
  \usepackage{titlesec} % Allows customization of titles
  \renewcommand\thesection{\Roman{section}} % Roman numerals for the sections
  \renewcommand\thesubsection{\Roman{subsection}} % Roman numerals for subsections
  \titleformat{\section}[block]{\large\scshape\centering}{\thesection.}{1em}{} % Change the look of the section titles
  \titleformat{\subsection}[block]{\large}{\thesubsection.}{1em}{} % Change the look of the section titles
  \usepackage{fancyhdr} % Headers and footers
  \pagestyle{fancy} % All pages have headers and footers
  \fancyhead{} % Blank out the default header
  \fancyfoot{} % Blank out the default footer
  \fancyhead[C]{X-meeting $\bullet$ October 2019 $\bullet$ Campos do  Jord\~ao} % Custom header text
  \fancyfoot[RO,LE]{} % Custom footer text
  %----------------------------------------------------------------------------------------
  % TITLE SECTION
  %---------------------------------------------------------------------------------------- 
 
 \title{\vspace{-15mm}\fontsize{24pt}{10pt}\selectfont\textbf{ Machine learning for predicting chemical-protein relations using graph embeddings }} % Article title
  
  
  \author{ Daniel Viana, Raquel Melo Minardi, Adriano Alonso Veloso }
  
  \affil{ UFMG }
  \vspace{-5mm}
  \date{}
  
  %---------------------------------------------------------------------------------------- 
  
  \begin{document}
  
  
  \maketitle % Insert title
  
  
  \thispagestyle{fancy} % All pages have headers and footers
  %----------------------------------------------------------------------------------------  
  % ABSTRACT
  
  %----------------------------------------------------------------------------------------  
  
  \begin{abstract}
  The volume of biological data available in many repositories is vast and increases almost exponentially. Among this universe of data,  we can highlight the relation between proteins and ligands. These relations may be the key to understanding biological processes,  drug metabolism,  drug design and repositioning,  and industrial protein optimization. However,  the large amount of data makes the process of analyzing and obtaining them manually a hard and toilsome process,  hence making the use of in silico methods indispensable. One of the computational approaches that can be used in this case is graph modeling,  where it is possible to represent proteins and ligands as nodes and the relationship between them as edges. So,  this work aims to propose a new database of chemical compounds and genomic products graph-based from an unstructured corpus manually curated and to suggest a method capable of predicting a relationship between them through machine learning techniques. For this,  we use the corpus chemical-protein interactions available on BioCreative VI Challenge. In order to predict relationships,  we first use the Neighborhood Based Node Embeddings (NBNE) algorithm,  an unsupervised method capable of generating node embeddings for graphs. Thus we produce a dataset containing the embeddings that represent nodes of the graph that contain a real relation,  class one. For class zero,  we generate false relationships randomly. To create prediction models,  we use the machine learning algorithms: Decision Tree,  Random Forest,  and SVM as classifiers. Among the experiments performed,  the method obtained the best result was SVM. Given the above,  the base created is a way to provide and organize unstructured data of relationships between genomic products and chemical compounds and can be used as a query for possible relationships between them. Finally,  the proposed method demonstrated to be efficient to suggest new relationships amid graph components through machine learning.
  
  Funding: CAPES \\ 
  \end{abstract}
  \end{document} 