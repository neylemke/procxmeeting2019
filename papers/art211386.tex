
  \documentclass[twoside]{article}
  \usepackage[affil-it]{authblk}
  \usepackage{lipsum} % Package to generate dummy text throughout this template
  \usepackage{eurosym}
  \usepackage[sc]{mathpazo} % Use the Palatino font
  \usepackage[T1]{fontenc} % Use 8-bit encoding that has 256 glyphs
  \usepackage[utf8]{inputenc}
  \linespread{1.05} % Line spacing-Palatino needs more space between lines
  \usepackage{microtype} % Slightly tweak font spacing for aesthetics\[IndentingNewLine]
  \usepackage[hmarginratio=1:1,top=32mm,columnsep=20pt]{geometry} % Document margins
  \usepackage{multicol} % Used for the two-column layout of the document
  \usepackage[hang,small,labelfont=bf,up,textfont=it,up]{caption} % Custom captions under//above floats in tables or figures
  \usepackage{booktabs} % Horizontal rules in tables
  \usepackage{float} % Required for tables and figures in the multi-column environment-they need to be placed in specific locations with the[H] (e.g. \begin{table}[H])
  \usepackage{hyperref} % For hyperlinks in the PDF
  \usepackage{lettrine} % The lettrine is the first enlarged letter at the beginning of the text
  \usepackage{paralist} % Used for the compactitem environment which makes bullet points with less space between them
  \usepackage{abstract} % Allows abstract customization
  \renewcommand{\abstractnamefont}{\normalfont\bfseries} 
  %\renewcommand{\abstracttextfont}{\normalfont\small\itshape} % Set the abstract itself to small italic text\[IndentingNewLine]
  \usepackage{titlesec} % Allows customization of titles
  \renewcommand\thesection{\Roman{section}} % Roman numerals for the sections
  \renewcommand\thesubsection{\Roman{subsection}} % Roman numerals for subsections
  \titleformat{\section}[block]{\large\scshape\centering}{\thesection.}{1em}{} % Change the look of the section titles
  \titleformat{\subsection}[block]{\large}{\thesubsection.}{1em}{} % Change the look of the section titles
  \usepackage{fancyhdr} % Headers and footers
  \pagestyle{fancy} % All pages have headers and footers
  \fancyhead{} % Blank out the default header
  \fancyfoot{} % Blank out the default footer
  \fancyhead[C]{X-meeting $\bullet$ October 2019 $\bullet$ Campos do  Jord\~ao} % Custom header text
  \fancyfoot[RO,LE]{} % Custom footer text
  %----------------------------------------------------------------------------------------
  % TITLE SECTION
  %---------------------------------------------------------------------------------------- 
 
 \title{\vspace{-15mm}\fontsize{24pt}{10pt}\selectfont\textbf{ Studies of LDL receptor activity in patients with familial hypercholesterolemia }} % Article title
  
  
  \author{ Thais Kristini Almendros Afonso, Victor Fernandes de Oliveira, Glaucio Monteiro Ferreira, J\'essica Bassani Borges, Gisele Medeiros Bastos, Profa. Dra. Tania Cristina Pithon-Curi, Renata Gorj\~ao, Rui Curi, Ros\'ario Dominguez Crespo Hirata, Mario Hiroyuki Hirata }
  
  \affil{ Laboratory of Molecular Research in Cardiology (LIMC),  Dante Pazzanese Institute of Cardiology,  S\~ao Paulo,  Brazil }
  \vspace{-5mm}
  \date{}
  
  %---------------------------------------------------------------------------------------- 
  
  \begin{document}
  
  
  \maketitle % Insert title
  
  
  \thispagestyle{fancy} % All pages have headers and footers
  %----------------------------------------------------------------------------------------  
  % ABSTRACT
  
  %----------------------------------------------------------------------------------------  
  
  \begin{abstract}
  The primary and more frequent alteration of the Familial Hypercholesterolemia (FH) affects the LDL receptor gene (LDLR),  with more than 1600 mutations described. These alterations result in the compromise of decreased LDL removal and consequently their accumulation in plasma. Nowadays,  genomic ultrasequencing technology generated high throughput results,  thus,  it is necessary to evaluate in vitro functional activity to make an association with disease. The aim of the present study was to analyze LDL receptor functional activity,  in mononuclear cell collected from patients that were identified new variants in LDLR gene that showed correlation with clinical manifestation of hypercholesterolemia after in silico analysis of protein structure. To meet this goal,  the T lymphocytes from the thirty FH carriers were isolated from the peripheral blood,  cultured and challenged for the expression of LDL receptors,  incubated with labeled LDL for binding assessment and internalization by the cells of each patient. The LDLR variants were selected after exome sequencing of a panel of 61 genes related to cholesterol homeostasis using the MiSeq platform System (Illumina). The in silico analyses were performed through a specific pipeline for variant calling and annotation. Initially,  these sequences were aligned to the reference human genome (GRCh38) using BWA. Afterwards,  GATK 4.0 was used for base quality score recalibration and variant calling to identify high-quality SNVs. Finally,  annotations and functional effect predictions for the SNVs were performed by PolyPhen-2,  SIFT,  PROVEAN and ANNOVAR. In order to elucidate the internalization of LDL particles,  in silico studies were performed to evaluate the functional impact of seven LDL receptor variants with APOB. For this,  3D models were built by homology modeling to obtain a predictive structure. Thus,  Protein-protein docking was performed on the server ClusPro 2.0,  using a standard set up,  between LDLr (PDB: 1n7d) and APOB (PDB: 1lsh). In this context,  in vitro study it was possible to note an increase in both the mean fluorescence of binding and binding and internalization in relation to the amount of LDLr on the cell surface. Corroborating to in vitro studies and molecular docking,  variant p.Gly592Glu (rs137929307) suggest an increase complex formation LDLr-APOB (LDLr: 31.5\% and LDL endocytosed: 50.7\%) while p.Asp601His (rs753707206) suggest less affinity between LDLr and APOB (LDLr: 98\% and LDL endocytized: 37.4\%). Therefore,  understand related mechanisms can measure the functional impact correlated with the pathogenicity of LDLr variants in HF.
  
  Funding: FAPESP - Project: 2018/11917-6 \\ 
  \end{abstract}
  \end{document} 