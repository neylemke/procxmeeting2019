
  \documentclass[twoside]{article}
  \usepackage[affil-it]{authblk}
  \usepackage{lipsum} % Package to generate dummy text throughout this template
  \usepackage{eurosym}
  \usepackage[sc]{mathpazo} % Use the Palatino font
  \usepackage[T1]{fontenc} % Use 8-bit encoding that has 256 glyphs
  \usepackage[utf8]{inputenc}
  \linespread{1.05} % Line spacing-Palatino needs more space between lines
  \usepackage{microtype} % Slightly tweak font spacing for aesthetics\[IndentingNewLine]
  \usepackage[hmarginratio=1:1,top=32mm,columnsep=20pt]{geometry} % Document margins
  \usepackage{multicol} % Used for the two-column layout of the document
  \usepackage[hang,small,labelfont=bf,up,textfont=it,up]{caption} % Custom captions under//above floats in tables or figures
  \usepackage{booktabs} % Horizontal rules in tables
  \usepackage{float} % Required for tables and figures in the multi-column environment-they need to be placed in specific locations with the[H] (e.g. \begin{table}[H])
  \usepackage{hyperref} % For hyperlinks in the PDF
  \usepackage{lettrine} % The lettrine is the first enlarged letter at the beginning of the text
  \usepackage{paralist} % Used for the compactitem environment which makes bullet points with less space between them
  \usepackage{abstract} % Allows abstract customization
  \renewcommand{\abstractnamefont}{\normalfont\bfseries} 
  %\renewcommand{\abstracttextfont}{\normalfont\small\itshape} % Set the abstract itself to small italic text\[IndentingNewLine]
  \usepackage{titlesec} % Allows customization of titles
  \renewcommand\thesection{\Roman{section}} % Roman numerals for the sections
  \renewcommand\thesubsection{\Roman{subsection}} % Roman numerals for subsections
  \titleformat{\section}[block]{\large\scshape\centering}{\thesection.}{1em}{} % Change the look of the section titles
  \titleformat{\subsection}[block]{\large}{\thesubsection.}{1em}{} % Change the look of the section titles
  \usepackage{fancyhdr} % Headers and footers
  \pagestyle{fancy} % All pages have headers and footers
  \fancyhead{} % Blank out the default header
  \fancyfoot{} % Blank out the default footer
  \fancyhead[C]{X-meeting $\bullet$ October 2019 $\bullet$ Campos do  Jord\~ao} % Custom header text
  \fancyfoot[RO,LE]{} % Custom footer text
  %----------------------------------------------------------------------------------------
  % TITLE SECTION
  %---------------------------------------------------------------------------------------- 
 
 \title{\vspace{-15mm}\fontsize{24pt}{10pt}\selectfont\textbf{ Neuroblastoma Meta-Analysis for Gene Characterization of INSS Stages }} % Article title
  
  
  \author{ Andr\'e Luiz Molan, Jos\'e Luiz Rybarczyk Filho }
  
  \affil{ Instituto de Bioci\^encias de Botucatu - UNESP }
  \vspace{-5mm}
  \date{}
  
  %---------------------------------------------------------------------------------------- 
  
  \begin{document}
  
  
  \maketitle % Insert title
  
  
  \thispagestyle{fancy} % All pages have headers and footers
  %----------------------------------------------------------------------------------------  
  % ABSTRACT
  
  %----------------------------------------------------------------------------------------  
  
  \begin{abstract}
  Neuroblastoma is an extracranial solid tumor,  very heterogeneous and with highly predictive clinical
behavior. It mainly affects individuals under 15 years old and it is classified according to the International
Neuroblastoma Staging System (INSS) and the International Neuroblastoma Pathology Classification
(INPC). With Next Generation Sequencing (NGS) technologies,  performing gene expression profiling of
tumors has become more common,  especially through RNA-seq. The amount of data generated, 
however,  is large. Thus,  the application of meta-analysis and functional enrichment techniques become
indispensable for a more effective study. In this paper,  based on the RNA-seq gene expression profile of
498 patients,  we performed a meta-analysis and a functional enrichment analysis searching for
significant gene groups in order to characterize the 5 major tumor stages according to the INSS stage
classification. The data were grouped according to these stages and the meta-analysis was performed in
the programming environment R with WGCNA package and its function metaAnalysis,  which uses the
Stouffer method and generates p-values for each of the genes present in the samples. These p-values
were corrected by FDR (False Discovery Rate) with the p.adjust function of the Stats R package, 
generating a q-value. Only q-values lower than 0.05 were considered to be significant. Functional
enrichment analysis was done by ADAM R package,  comparing,  two by two,  each of the tumor stages
(10 comparisons). For each comparison,  genes were regrouped according to their functions based on
Gene Ontology (biological processes,  molecular functions and cellular components) and pathways from
the KEGG repository. For each functional group,  q-values were calculated for gene diversity and gene
activity. Significant genes obtained with meta-analysis were related to significant groups (only groups
with q-value lower than 0.05) obtained by functional enrichment. We found 5163 significant genes in
meta-analysis. By relating these genes to the most important functional groups,  we noticed an increase
in gene and functional specificity proportional to the considered tumor stage. An example of this can be
observed when comparing stages 1 and 2 and 1 and 4 for gene activity and biological processes. In the
first comparison (1 and 2),  we observed 3631 functional groups and 1164 genes. However,  in the second
(1 and 4) we noticed 189 groups and 330 significant genes related to them.
  
  Funding: CAPES \\ 
  \end{abstract}
  \end{document} 