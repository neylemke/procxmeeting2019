
  \documentclass[twoside]{article}
  \usepackage[affil-it]{authblk}
  \usepackage{lipsum} % Package to generate dummy text throughout this template
  \usepackage{eurosym}
  \usepackage[sc]{mathpazo} % Use the Palatino font
  \usepackage[T1]{fontenc} % Use 8-bit encoding that has 256 glyphs
  \usepackage[utf8]{inputenc}
  \linespread{1.05} % Line spacing-Palatino needs more space between lines
  \usepackage{microtype} % Slightly tweak font spacing for aesthetics\[IndentingNewLine]
  \usepackage[hmarginratio=1:1,top=32mm,columnsep=20pt]{geometry} % Document margins
  \usepackage{multicol} % Used for the two-column layout of the document
  \usepackage[hang,small,labelfont=bf,up,textfont=it,up]{caption} % Custom captions under//above floats in tables or figures
  \usepackage{booktabs} % Horizontal rules in tables
  \usepackage{float} % Required for tables and figures in the multi-column environment-they need to be placed in specific locations with the[H] (e.g. \begin{table}[H])
  \usepackage{hyperref} % For hyperlinks in the PDF
  \usepackage{lettrine} % The lettrine is the first enlarged letter at the beginning of the text
  \usepackage{paralist} % Used for the compactitem environment which makes bullet points with less space between them
  \usepackage{abstract} % Allows abstract customization
  \renewcommand{\abstractnamefont}{\normalfont\bfseries} 
  %\renewcommand{\abstracttextfont}{\normalfont\small\itshape} % Set the abstract itself to small italic text\[IndentingNewLine]
  \usepackage{titlesec} % Allows customization of titles
  \renewcommand\thesection{\Roman{section}} % Roman numerals for the sections
  \renewcommand\thesubsection{\Roman{subsection}} % Roman numerals for subsections
  \titleformat{\section}[block]{\large\scshape\centering}{\thesection.}{1em}{} % Change the look of the section titles
  \titleformat{\subsection}[block]{\large}{\thesubsection.}{1em}{} % Change the look of the section titles
  \usepackage{fancyhdr} % Headers and footers
  \pagestyle{fancy} % All pages have headers and footers
  \fancyhead{} % Blank out the default header
  \fancyfoot{} % Blank out the default footer
  \fancyhead[C]{X-meeting $\bullet$ October 2019 $\bullet$ Campos do  Jord\~ao} % Custom header text
  \fancyfoot[RO,LE]{} % Custom footer text
  %----------------------------------------------------------------------------------------
  % TITLE SECTION
  %---------------------------------------------------------------------------------------- 
 
 \title{\vspace{-15mm}\fontsize{24pt}{10pt}\selectfont\textbf{ An In Silico approach for the identification of vaccine and drug targets against Mycoplasma genitalium,  causative agent of sexually transmitted pelvic inflammatory disease (PID) }} % Article title
  
  
  \author{ Arun Kumar Jaiswal, Wylerson Nogueira, sandeep tiwari, Rommel Thiago Juc\'a Ramos, Vasco A de C Azevedo, Siomar de Castro Soares }
  
  \affil{ Universidade Federal do Tri\^angulo Mineiro }
  \vspace{-5mm}
  \date{}
  
  %---------------------------------------------------------------------------------------- 
  
  \begin{document}
  
  
  \maketitle % Insert title
  
  
  \thispagestyle{fancy} % All pages have headers and footers
  %----------------------------------------------------------------------------------------  
  % ABSTRACT
  
  %----------------------------------------------------------------------------------------  
  
  \begin{abstract}
  Mycoplasma genitalium is a sexually transmitted pathogen characterized as a pleiomorphic,  flask shaped,  slow growing and obligate intracellular bacterium. It is one of the STI (sexually transmitted infections) pathogens associated with non-gonococcal urethritis in men and several inflammatory reproductive tract syndromes in women such as cervicitis,  pelvic inflammatory disease (PID) and infertility. Some studies have reported the infection of M. genitalium as a cause for infertility and adverse pregnancy outcomes such as preterm labor. Currently,  treatment of most M. genitalium infections occurs mainly in the context of syndromic management for urethritis,  cervicitis,  and PID,  owing to the lack of diagnostic test availability. With the advent of new high-throughput sequencing technologies and the rise of genomic data,  scientists are able to use computational methods to identify new targets,  which are time and cost effective in compare to classical approaches. Reverse vaccinology (RV) and subtractive genomics are conventional and popular approach in the post-genomic era for the prompt identification of novel vaccine and drug targets. In this study,  the prediction of putative vaccine and drug targets against Mycoplasma genitalium,  using reverse vaccinology and subtractive genomics is carried out. We used 10 strains of Mycoplasma genitalium for comparison. Briefly,  we used a combined reverse vaccinology and subtractive genomics approach and identified 12 putative antigenic proteins as vaccine targets and 7 drug targets. Furthermore,  the molecular docking analysis was performed with 5000 antimicrobial natural compounds downloaded from ZInc database. The drug-like natural compounds showed  the most  favored binding affinity against predicted drug targets,  which can be a candidate therapeutic target in the future against M. genitalium. We hypothesize that these identified therapeutic targets and antimicrobial drugs could be considered for prophylaxis of M. genitalium and hence should be subjected to further experimental validations.
  
  Funding: CAPES,  FAPEMIG,  CNPq \\ 
  \end{abstract}
  \end{document} 