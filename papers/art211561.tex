
  \documentclass[twoside]{article}
  \usepackage[affil-it]{authblk}
  \usepackage{lipsum} % Package to generate dummy text throughout this template
  \usepackage{eurosym}
  \usepackage[sc]{mathpazo} % Use the Palatino font
  \usepackage[T1]{fontenc} % Use 8-bit encoding that has 256 glyphs
  \usepackage[utf8]{inputenc}
  \linespread{1.05} % Line spacing-Palatino needs more space between lines
  \usepackage{microtype} % Slightly tweak font spacing for aesthetics\[IndentingNewLine]
  \usepackage[hmarginratio=1:1,top=32mm,columnsep=20pt]{geometry} % Document margins
  \usepackage{multicol} % Used for the two-column layout of the document
  \usepackage[hang,small,labelfont=bf,up,textfont=it,up]{caption} % Custom captions under//above floats in tables or figures
  \usepackage{booktabs} % Horizontal rules in tables
  \usepackage{float} % Required for tables and figures in the multi-column environment-they need to be placed in specific locations with the[H] (e.g. \begin{table}[H])
  \usepackage{hyperref} % For hyperlinks in the PDF
  \usepackage{lettrine} % The lettrine is the first enlarged letter at the beginning of the text
  \usepackage{paralist} % Used for the compactitem environment which makes bullet points with less space between them
  \usepackage{abstract} % Allows abstract customization
  \renewcommand{\abstractnamefont}{\normalfont\bfseries} 
  %\renewcommand{\abstracttextfont}{\normalfont\small\itshape} % Set the abstract itself to small italic text\[IndentingNewLine]
  \usepackage{titlesec} % Allows customization of titles
  \renewcommand\thesection{\Roman{section}} % Roman numerals for the sections
  \renewcommand\thesubsection{\Roman{subsection}} % Roman numerals for subsections
  \titleformat{\section}[block]{\large\scshape\centering}{\thesection.}{1em}{} % Change the look of the section titles
  \titleformat{\subsection}[block]{\large}{\thesubsection.}{1em}{} % Change the look of the section titles
  \usepackage{fancyhdr} % Headers and footers
  \pagestyle{fancy} % All pages have headers and footers
  \fancyhead{} % Blank out the default header
  \fancyfoot{} % Blank out the default footer
  \fancyhead[C]{X-meeting $\bullet$ October 2019 $\bullet$ Campos do  Jord\~ao} % Custom header text
  \fancyfoot[RO,LE]{} % Custom footer text
  %----------------------------------------------------------------------------------------
  % TITLE SECTION
  %---------------------------------------------------------------------------------------- 
 
 \title{\vspace{-15mm}\fontsize{24pt}{10pt}\selectfont\textbf{ Comparative genomics in the search for Antifreeze Proteins in Metschnikowia australis }} % Article title
  
  
  \author{ Heron Hil\'ario, Thiago Mafra Batista, Carlos Augusto Rosa, Luiz Henrique Rosa, Gl\'oria Regina Franco }
  
  \affil{ Universidade Federal do Sul da Bahia }
  \vspace{-5mm}
  \date{}
  
  %---------------------------------------------------------------------------------------- 
  
  \begin{document}
  
  
  \maketitle % Insert title
  
  
  \thispagestyle{fancy} % All pages have headers and footers
  %----------------------------------------------------------------------------------------  
  % ABSTRACT
  
  %----------------------------------------------------------------------------------------  
  
  \begin{abstract}
  Metschnikowia australis is a marine yeast confined to the Antarctic costal region. Conversely,  its most related species,  Metschnikowia bicuspidata,  is worldwide distributed,  but rarely retrieved from Antarctica’s seas. Besides global adaptations to life at low temperatures,  many psychrophilic organisms,  from bacteria to vertebrates,  have independently evolved protein coding genes that directly interfere with the ice formation process. These genes are collectively called Antifreeze Proteins (AFPs),  consisting of a diverse class that have inumerous biotechnological applications,  from frozen food industry to organ preservation for transplants. As polyphiletic,  AFP genes are harder to find with traditional sequence similarity search tools in less characterized taxa,  as Metschnikowia. One M. australis specimen was isolated by the MycoAntar project and,  due to its ability to survive freezing,  it was selected for further studies by our group. We have sequenced and assembled the M. australis genome,  for further genomic investigation. As M. bicuspidata genome was already available,  we devised an strategy to predict Open Reading Frames (ORFs) on both genomes and select those exclusive to M. australis,  which could potentially code new AFPs. The 249 ORFs exclusive to australis were then submmited to three machine learning based AFPs classifiers,  in order to select the most promissing candidates for in vitro expression analysis. Primers were designed for 17 selected candidates,  and these are now being used to probe both yeasts transcriptome after freezing stress. The candidate ORFs that are proven to be expressed will be further characterized. Genomic deletion and heterologous expression techniques will be used to confirm their relation to freezeing survival.
  
  Funding: Capes \\ 
  \end{abstract}
  \end{document} 