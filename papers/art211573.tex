
  \documentclass[twoside]{article}
  \usepackage[affil-it]{authblk}
  \usepackage{lipsum} % Package to generate dummy text throughout this template
  \usepackage{eurosym}
  \usepackage[sc]{mathpazo} % Use the Palatino font
  \usepackage[T1]{fontenc} % Use 8-bit encoding that has 256 glyphs
  \usepackage[utf8]{inputenc}
  \linespread{1.05} % Line spacing-Palatino needs more space between lines
  \usepackage{microtype} % Slightly tweak font spacing for aesthetics\[IndentingNewLine]
  \usepackage[hmarginratio=1:1,top=32mm,columnsep=20pt]{geometry} % Document margins
  \usepackage{multicol} % Used for the two-column layout of the document
  \usepackage[hang,small,labelfont=bf,up,textfont=it,up]{caption} % Custom captions under//above floats in tables or figures
  \usepackage{booktabs} % Horizontal rules in tables
  \usepackage{float} % Required for tables and figures in the multi-column environment-they need to be placed in specific locations with the[H] (e.g. \begin{table}[H])
  \usepackage{hyperref} % For hyperlinks in the PDF
  \usepackage{lettrine} % The lettrine is the first enlarged letter at the beginning of the text
  \usepackage{paralist} % Used for the compactitem environment which makes bullet points with less space between them
  \usepackage{abstract} % Allows abstract customization
  \renewcommand{\abstractnamefont}{\normalfont\bfseries} 
  %\renewcommand{\abstracttextfont}{\normalfont\small\itshape} % Set the abstract itself to small italic text\[IndentingNewLine]
  \usepackage{titlesec} % Allows customization of titles
  \renewcommand\thesection{\Roman{section}} % Roman numerals for the sections
  \renewcommand\thesubsection{\Roman{subsection}} % Roman numerals for subsections
  \titleformat{\section}[block]{\large\scshape\centering}{\thesection.}{1em}{} % Change the look of the section titles
  \titleformat{\subsection}[block]{\large}{\thesubsection.}{1em}{} % Change the look of the section titles
  \usepackage{fancyhdr} % Headers and footers
  \pagestyle{fancy} % All pages have headers and footers
  \fancyhead{} % Blank out the default header
  \fancyfoot{} % Blank out the default footer
  \fancyhead[C]{X-meeting $\bullet$ October 2019 $\bullet$ Campos do  Jord\~ao} % Custom header text
  \fancyfoot[RO,LE]{} % Custom footer text
  %----------------------------------------------------------------------------------------
  % TITLE SECTION
  %---------------------------------------------------------------------------------------- 
 
 \title{\vspace{-15mm}\fontsize{24pt}{10pt}\selectfont\textbf{ Role of DIMBOA in the fall-armyworm strain diversification inferred with transcriptome differential co-expression }} % Article title
  
  
  \author{ Karina Lucas da Silva Brand\~ao, Natalia Faraj Murad, Aline Peruchi, Celso Omoto, Antonio Figueira, Marcelo Mendes Brand\~ao }
  
  \affil{ Unicamp }
  \vspace{-5mm}
  \date{}
  
  %---------------------------------------------------------------------------------------- 
  
  \begin{document}
  
  
  \maketitle % Insert title
  
  
  \thispagestyle{fancy} % All pages have headers and footers
  %----------------------------------------------------------------------------------------  
  % ABSTRACT
  
  %----------------------------------------------------------------------------------------  
  
  \begin{abstract}
  Which factors induce speciation in insect lineages? That is undoubtedly one of the most fruitful topics of study,  with branches that encompass since the origin of biodiversity itself as far as agriculture production and pest control. Secondary compounds from plants have a major role on this process. Benzoxazinoids (BXDs),  or hydroxamic acids,  are one of the main secondary compounds found in several cereals,  providing plant defense against herbivorous insects and pathogens. The main BXD found in corn is the toxic aglucone DIMBOA (2, 4-dihydroxy-7-methoxy-1, 4-benzoxazin-3-one),  present also in wheat and rye,  although it is absent in rice. DIMBOA elicites variable responses from noctuid larvae of different species of Spodoptera,  including increased consumption and growth rates,  and antifeedant effect. On S. frugiperda,  the fall-armyworm (FAW),  however,  DIMBOA acts as a feeding stimulant,  and improves larvae growth at low concentration. The FAW is a widespread polyphagous moth and the principal pest of corn in South America. The species is distinguished into two host plant-related strains,  the corn strain (CS) that feeds preferentially on corn,  and the rice strain (RS),  that is found generally on rice. To test DIMBOA role in the lineage differentiation of S. frugiperda we reared both CS and RS larvae in artificial diet,  either enriched with DIMBOA or without this compound,  and evaluated differential co-expression in the transcriptome of both midgut and fat body larval tissues. This approach can point to genes related to DIMBOA response in the FAW and help to clarify its role in the strain differentiation. The 10-day-old larvae of RS reared on experimental DIMBOA-enriched diet grew less than CS larvae under the same condition. Enriched peptidases,  reductases and transferases,  all them involved in DIMBOA detoxification,  showed differences in expression between the two strains. The higher expression of glucosyltransferases in RS larvae midgut compared to CS larvae may be related to a higher activity of glucosylation to detoxify DIMBOA. Glucosyltransferases present in the CS larvae can be more specific than the ones present in the RS,  and for this reason CS larvae are more efficient than RS ones to detoxify DIMBOA,  what is translated in higher larval growing of the former.
  
  Funding: Fapesp (grants \#2012/16266-7; 2011/00417-3) \\ 
  \end{abstract}
  \end{document} 