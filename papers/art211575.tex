
  \documentclass[twoside]{article}
  \usepackage[affil-it]{authblk}
  \usepackage{lipsum} % Package to generate dummy text throughout this template
  \usepackage{eurosym}
  \usepackage[sc]{mathpazo} % Use the Palatino font
  \usepackage[T1]{fontenc} % Use 8-bit encoding that has 256 glyphs
  \usepackage[utf8]{inputenc}
  \linespread{1.05} % Line spacing-Palatino needs more space between lines
  \usepackage{microtype} % Slightly tweak font spacing for aesthetics\[IndentingNewLine]
  \usepackage[hmarginratio=1:1,top=32mm,columnsep=20pt]{geometry} % Document margins
  \usepackage{multicol} % Used for the two-column layout of the document
  \usepackage[hang,small,labelfont=bf,up,textfont=it,up]{caption} % Custom captions under//above floats in tables or figures
  \usepackage{booktabs} % Horizontal rules in tables
  \usepackage{float} % Required for tables and figures in the multi-column environment-they need to be placed in specific locations with the[H] (e.g. \begin{table}[H])
  \usepackage{hyperref} % For hyperlinks in the PDF
  \usepackage{lettrine} % The lettrine is the first enlarged letter at the beginning of the text
  \usepackage{paralist} % Used for the compactitem environment which makes bullet points with less space between them
  \usepackage{abstract} % Allows abstract customization
  \renewcommand{\abstractnamefont}{\normalfont\bfseries} 
  %\renewcommand{\abstracttextfont}{\normalfont\small\itshape} % Set the abstract itself to small italic text\[IndentingNewLine]
  \usepackage{titlesec} % Allows customization of titles
  \renewcommand\thesection{\Roman{section}} % Roman numerals for the sections
  \renewcommand\thesubsection{\Roman{subsection}} % Roman numerals for subsections
  \titleformat{\section}[block]{\large\scshape\centering}{\thesection.}{1em}{} % Change the look of the section titles
  \titleformat{\subsection}[block]{\large}{\thesubsection.}{1em}{} % Change the look of the section titles
  \usepackage{fancyhdr} % Headers and footers
  \pagestyle{fancy} % All pages have headers and footers
  \fancyhead{} % Blank out the default header
  \fancyfoot{} % Blank out the default footer
  \fancyhead[C]{X-meeting $\bullet$ October 2019 $\bullet$ Campos do  Jord\~ao} % Custom header text
  \fancyfoot[RO,LE]{} % Custom footer text
  %----------------------------------------------------------------------------------------
  % TITLE SECTION
  %---------------------------------------------------------------------------------------- 
 
 \title{\vspace{-15mm}\fontsize{24pt}{10pt}\selectfont\textbf{ Comparative mitogenomics of Sugiyamaella species,  yeasts of biotechnological importance }} % Article title
  
  
  \author{ Paula Silva Matos, Heron Hil\'ario, Rennan Garcias Moreira, Carlos Augusto Rosa, Thiago Mafra Batista, Gl\'oria Regina Franco }
  
  \affil{ Universidade Federal do Sul da Bahia }
  \vspace{-5mm}
  \date{}
  
  %---------------------------------------------------------------------------------------- 
  
  \begin{document}
  
  
  \maketitle % Insert title
  
  
  \thispagestyle{fancy} % All pages have headers and footers
  %----------------------------------------------------------------------------------------  
  % ABSTRACT
  
  %----------------------------------------------------------------------------------------  
  
  \begin{abstract}
  Microorganisms are widely used in the industry to produce several compounds. In the bioethanol production the most commonly used yeast is Saccharomyces cerevisiae,  which converts sugars,  mainly from sugarcane,  into ethanol. However,  this yeast is unable to degrade some plant polymers such as lignin to fermentable sugars,  which prevents full use of the raw material. Some yeasts,  like Sugiyamaella xylanicola,  are able to metabolize these polymers,  being promising candidates for second-generation ethanol production. These capabilities are conferred by enzymes encoded in the nuclear genomes of these microorganisms. However,  phylogenetical information about fermenting yeasts can be obtained by sequencing their mitochondrial genomes,  since they are small,  ease to assemble and contain informative gene sequences. In this study we sequenced,  assembled and annotated the mitogenome of the yeast S. xylanicola UFMG-CM-Y1884T,  collected from decaying woods in the Cara\c{c}a Mountains of Minas Gerais. The complete mitogenome of S. xylanicola was characterized and compared with previously published S. lignohabitans CBS 10342 and S. cerevisiae mitogenomes. The Sugiyamaella mitogenomes are 28 and 48 Kb long,  and shorter than the 85 Kb S. cerevisae mitogenome. A total of 26 and 46 mitochondrial tRNAs were annotated for S. xylanicola and S. lignohabitans,  respectively,  which consist of more than the minimal set of 24 tRNAs required for translation,  reported for S. cerevisiae. The Sugiyamaella mitochondrial genomes encode a complete respiration system,  including NADH dehydrogenase complex I. In S. cerevisiae and several other yeasts,  complex I is missing,  which leads to a reduction of the respiratory energy yield. Further,  we intend to investigate other aspects of mitochondrial genomes in order to discover additional possible Sugiyamaella advantages over the more commonly used yeasts for ethanol production.
  
  Funding: Pronex/FAPEMIG \\ 
  \end{abstract}
  \end{document} 