
  \documentclass[twoside]{article}
  \usepackage[affil-it]{authblk}
  \usepackage{lipsum} % Package to generate dummy text throughout this template
  \usepackage{eurosym}
  \usepackage[sc]{mathpazo} % Use the Palatino font
  \usepackage[T1]{fontenc} % Use 8-bit encoding that has 256 glyphs
  \usepackage[utf8]{inputenc}
  \linespread{1.05} % Line spacing-Palatino needs more space between lines
  \usepackage{microtype} % Slightly tweak font spacing for aesthetics\[IndentingNewLine]
  \usepackage[hmarginratio=1:1,top=32mm,columnsep=20pt]{geometry} % Document margins
  \usepackage{multicol} % Used for the two-column layout of the document
  \usepackage[hang,small,labelfont=bf,up,textfont=it,up]{caption} % Custom captions under//above floats in tables or figures
  \usepackage{booktabs} % Horizontal rules in tables
  \usepackage{float} % Required for tables and figures in the multi-column environment-they need to be placed in specific locations with the[H] (e.g. \begin{table}[H])
  \usepackage{hyperref} % For hyperlinks in the PDF
  \usepackage{lettrine} % The lettrine is the first enlarged letter at the beginning of the text
  \usepackage{paralist} % Used for the compactitem environment which makes bullet points with less space between them
  \usepackage{abstract} % Allows abstract customization
  \renewcommand{\abstractnamefont}{\normalfont\bfseries} 
  %\renewcommand{\abstracttextfont}{\normalfont\small\itshape} % Set the abstract itself to small italic text\[IndentingNewLine]
  \usepackage{titlesec} % Allows customization of titles
  \renewcommand\thesection{\Roman{section}} % Roman numerals for the sections
  \renewcommand\thesubsection{\Roman{subsection}} % Roman numerals for subsections
  \titleformat{\section}[block]{\large\scshape\centering}{\thesection.}{1em}{} % Change the look of the section titles
  \titleformat{\subsection}[block]{\large}{\thesubsection.}{1em}{} % Change the look of the section titles
  \usepackage{fancyhdr} % Headers and footers
  \pagestyle{fancy} % All pages have headers and footers
  \fancyhead{} % Blank out the default header
  \fancyfoot{} % Blank out the default footer
  \fancyhead[C]{X-meeting $\bullet$ October 2019 $\bullet$ Campos do  Jord\~ao} % Custom header text
  \fancyfoot[RO,LE]{} % Custom footer text
  %----------------------------------------------------------------------------------------
  % TITLE SECTION
  %---------------------------------------------------------------------------------------- 
 
 \title{\vspace{-15mm}\fontsize{24pt}{10pt}\selectfont\textbf{ Impact of differentially alternative spliced transcripts on proteome of mice infected with different strains of Trypanosoma cruzi }} % Article title
  
  
  \author{ Nayara Toledo, Raphael Tavares da Silva, Tiago Bruno Rezende de Castro, Carlos Renato Machado, Andr\'ea Mara Macedo, Mariana Fioramonte, Daniel Martins de Souza, Gl\'oria Regina Franco }
  
  \affil{ Universidade Federal de Minas Gerais }
  \vspace{-5mm}
  \date{}
  
  %---------------------------------------------------------------------------------------- 
  
  \begin{document}
  
  
  \maketitle % Insert title
  
  
  \thispagestyle{fancy} % All pages have headers and footers
  %----------------------------------------------------------------------------------------  
  % ABSTRACT
  
  %----------------------------------------------------------------------------------------  
  
  \begin{abstract}
  Since the description of Chagas disease,  caused by the protozoan parasite Trypanosoma cruzi,  the reasons for the different clinical manifestations of the disease in humans have yet to be completely revealed. Our group has previously shown that different strains of T. cruzi (JG- T.cruzi II and Col1.7G2-T. cruzi I) had a differential tissue tropism in BALB/c mice upon infection. Evidences that the genetic background of different mice lineages is contributing for changes in the differential tissue distribution of T. cruzi during infection were also found. Studies on differential gene expression,  aiming at elucidating which host genes could be modulated by distinct parasite strains,  were conducted using JG,  Col1.7G2 or an equivalent mixture of both strains. Alternative splicing is a regulatory mechanism of gene expression in which different exons and introns of the same pre-mRNA may be skipped or retained to produce distinct mature mRNAs. In recent years,  this mechanism has been shown to be a major source of cell-specific proteomic variation in mammalians. Thus,  the purpose of the present study is to integrate mass spectrometry-derived proteomic data from BALB/c infected hearts with the same T. cruzi strains and the above-mentioned RNA-Seq data. We performed quantification of gene expression at the transcript level and its all distinct isoforms. Comparing Col1.7G2 infected mice with control group,  we identified a total of 594 differentially expressed transcripts with false discovery rate less than 0.05,  including 543 upregulated and 51 downregulated. Comparing JG with control group,  a total of 901 transcripts were considered differentially expressed,  including 256 upregulated and 645 downregulated. Functional enrichment analysis showed Col1.7G2 induced a higher inflammatory response while JG exhibit a weaker activation of immune response genes. Furthermore,  JG-infected mice showed a notable reduction in expression of genes responsible for energetic metabolism,  mitochondrial oxidative phosphorylation,  and protein synthesis. Splicing events were frequent,  including an increase in the number of skipped exons,  increase in introns retention and increased in usage of alternate 5’ and 3’ splice sites. Our future steps include to correlate of these results with proteoforms identified by mass spectrometry.
  
  Funding: FAPEMIG,  CAPES \\ 
  \end{abstract}
  \end{document} 