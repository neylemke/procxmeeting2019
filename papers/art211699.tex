
  \documentclass[twoside]{article}
  \usepackage[affil-it]{authblk}
  \usepackage{lipsum} % Package to generate dummy text throughout this template
  \usepackage{eurosym}
  \usepackage[sc]{mathpazo} % Use the Palatino font
  \usepackage[T1]{fontenc} % Use 8-bit encoding that has 256 glyphs
  \usepackage[utf8]{inputenc}
  \linespread{1.05} % Line spacing-Palatino needs more space between lines
  \usepackage{microtype} % Slightly tweak font spacing for aesthetics\[IndentingNewLine]
  \usepackage[hmarginratio=1:1,top=32mm,columnsep=20pt]{geometry} % Document margins
  \usepackage{multicol} % Used for the two-column layout of the document
  \usepackage[hang,small,labelfont=bf,up,textfont=it,up]{caption} % Custom captions under//above floats in tables or figures
  \usepackage{booktabs} % Horizontal rules in tables
  \usepackage{float} % Required for tables and figures in the multi-column environment-they need to be placed in specific locations with the[H] (e.g. \begin{table}[H])
  \usepackage{hyperref} % For hyperlinks in the PDF
  \usepackage{lettrine} % The lettrine is the first enlarged letter at the beginning of the text
  \usepackage{paralist} % Used for the compactitem environment which makes bullet points with less space between them
  \usepackage{abstract} % Allows abstract customization
  \renewcommand{\abstractnamefont}{\normalfont\bfseries} 
  %\renewcommand{\abstracttextfont}{\normalfont\small\itshape} % Set the abstract itself to small italic text\[IndentingNewLine]
  \usepackage{titlesec} % Allows customization of titles
  \renewcommand\thesection{\Roman{section}} % Roman numerals for the sections
  \renewcommand\thesubsection{\Roman{subsection}} % Roman numerals for subsections
  \titleformat{\section}[block]{\large\scshape\centering}{\thesection.}{1em}{} % Change the look of the section titles
  \titleformat{\subsection}[block]{\large}{\thesubsection.}{1em}{} % Change the look of the section titles
  \usepackage{fancyhdr} % Headers and footers
  \pagestyle{fancy} % All pages have headers and footers
  \fancyhead{} % Blank out the default header
  \fancyfoot{} % Blank out the default footer
  \fancyhead[C]{X-meeting $\bullet$ October 2019 $\bullet$ Campos do  Jord\~ao} % Custom header text
  \fancyfoot[RO,LE]{} % Custom footer text
  %----------------------------------------------------------------------------------------
  % TITLE SECTION
  %---------------------------------------------------------------------------------------- 
 
 \title{\vspace{-15mm}\fontsize{24pt}{10pt}\selectfont\textbf{ AmpFLow: a containerized pipeline to assist in Reproducible and Replicable Microbiome research }} % Article title
  
  
  \author{ David Aciole Barbosa, Fabiano Menegidio, Yara Natercia Lima Faustino de Maria, Rafael dos Santos Gon\c{c}alves, Marcos Vinicius Yano, Regina Costa de Oliveira, Daniela L. Jabes, Luiz R. Nunes }
  
  \affil{ UMC }
  \vspace{-5mm}
  \date{}
  
  %---------------------------------------------------------------------------------------- 
  
  \begin{document}
  
  
  \maketitle % Insert title
  
  
  \thispagestyle{fancy} % All pages have headers and footers
  %----------------------------------------------------------------------------------------  
  % ABSTRACT
  
  %----------------------------------------------------------------------------------------  
  
  \begin{abstract}
  The increasing number of studies aimed at evaluating microorganism populations is providing vast and detailed information about a large variety of environments. Such microbiome data represent one of the most promising approaches currently available in biological sciences,  with possible applications in industry,  agriculture and health. However,  technical obstacles still must be overcome,  before results from microbiome analyses can be fully incorporated into innovative technologies. In fact,  scientists around the world claim that many fields of research are currently affected by a reproducibility challenge,  due to difficulties in obtaining all details and resources necessary to reproduce the same experiment/analysis in different laboratories. Microbiome analyses seem to be particularly affected by such reproducibility crisis and the lack of proper standardization for the complex bioinformatics procedures,  inherent to such analyses,  seems to be one of the main reasons for such problems. A new trend to solve this issue is Docker,  a system-independent technology which made possible the packaging of complete software environments by creating additional layers of operating system level virtualization abstraction bundled in the so called containers. In this sense,  we present AmpFlow,  a containerized pipeline designed to promote reproducible and replicable microbiome analyses. AmpFlow was built in Docker,  using a set of simple,  yet effective scripts,  to deploy tools available from Qiime,  which perform: (i) quality checking; (ii) pre-processing and (iii) processing of raw bacterial and fungal sequencing data,  creating reproducible OTU tables that are ready to be used in different post-processing platforms,  such as the online and R versions of Microbiome Analyst,  as well as the Galaxy version of LEfSe.
  
  Funding: FAPESP,  CAPES,  CNPQ \\ 
  \end{abstract}
  \end{document} 