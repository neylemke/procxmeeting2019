
  \documentclass[twoside]{article}
  \usepackage[affil-it]{authblk}
  \usepackage{lipsum} % Package to generate dummy text throughout this template
  \usepackage{eurosym}
  \usepackage[sc]{mathpazo} % Use the Palatino font
  \usepackage[T1]{fontenc} % Use 8-bit encoding that has 256 glyphs
  \usepackage[utf8]{inputenc}
  \linespread{1.05} % Line spacing-Palatino needs more space between lines
  \usepackage{microtype} % Slightly tweak font spacing for aesthetics\[IndentingNewLine]
  \usepackage[hmarginratio=1:1,top=32mm,columnsep=20pt]{geometry} % Document margins
  \usepackage{multicol} % Used for the two-column layout of the document
  \usepackage[hang,small,labelfont=bf,up,textfont=it,up]{caption} % Custom captions under//above floats in tables or figures
  \usepackage{booktabs} % Horizontal rules in tables
  \usepackage{float} % Required for tables and figures in the multi-column environment-they need to be placed in specific locations with the[H] (e.g. \begin{table}[H])
  \usepackage{hyperref} % For hyperlinks in the PDF
  \usepackage{lettrine} % The lettrine is the first enlarged letter at the beginning of the text
  \usepackage{paralist} % Used for the compactitem environment which makes bullet points with less space between them
  \usepackage{abstract} % Allows abstract customization
  \renewcommand{\abstractnamefont}{\normalfont\bfseries} 
  %\renewcommand{\abstracttextfont}{\normalfont\small\itshape} % Set the abstract itself to small italic text\[IndentingNewLine]
  \usepackage{titlesec} % Allows customization of titles
  \renewcommand\thesection{\Roman{section}} % Roman numerals for the sections
  \renewcommand\thesubsection{\Roman{subsection}} % Roman numerals for subsections
  \titleformat{\section}[block]{\large\scshape\centering}{\thesection.}{1em}{} % Change the look of the section titles
  \titleformat{\subsection}[block]{\large}{\thesubsection.}{1em}{} % Change the look of the section titles
  \usepackage{fancyhdr} % Headers and footers
  \pagestyle{fancy} % All pages have headers and footers
  \fancyhead{} % Blank out the default header
  \fancyfoot{} % Blank out the default footer
  \fancyhead[C]{X-meeting $\bullet$ October 2019 $\bullet$ Campos do  Jord\~ao} % Custom header text
  \fancyfoot[RO,LE]{} % Custom footer text
  %----------------------------------------------------------------------------------------
  % TITLE SECTION
  %---------------------------------------------------------------------------------------- 
 
 \title{\vspace{-15mm}\fontsize{24pt}{10pt}\selectfont\textbf{ Positive selection evidences on Moniliophthora PR-1 genes suggest evolution towards pathogenicity role }} % Article title
  
  
  \author{ Adrielle Ayumi de Vasconcelos, Renata Baroni, Paulo M. Tokimatu Filho, Paulo J. P. L. Teixeira, Marcelo Falsarella Carazzolle, Gon\c{c}alo Amarante Guimar\~aes Pereira, Juliana Jos\'e }
  
  \affil{ Institute of Biology; UNICAMP; Brazil }
  \vspace{-5mm}
  \date{}
  
  %---------------------------------------------------------------------------------------- 
  
  \begin{document}
  
  
  \maketitle % Insert title
  
  
  \thispagestyle{fancy} % All pages have headers and footers
  %----------------------------------------------------------------------------------------  
  % ABSTRACT
  
  %----------------------------------------------------------------------------------------  
  
  \begin{abstract}
  Moniliophthora perniciosa is a basidiomycete fungus with three known biotypes,  being the C-biotype the one that  infects Theobroma cacao,  causing witches broom disease (WBD).The arrival of WBD in cocoa plantations in Bahia led to enormous economic and social damage. In order to aid the development of forms of pathogen control,  the understanding of the plant defense mechanisms and virulence factors of M. perniciosa is necessary. Pathogenesis-related 1 (PR-1) proteins,  which belong to the SCP/TAPS or CAP superfamily,  are widespread markers of the induced defense response in plants against pathogens. Interestingly,  M. perniciosa’s genome contains 11 PR-1-like genes (named MpPR-1a to k),  many of them being highly up-regulated genes during WBD. In this study,  we carried out the evolutionary analysis of M. perniciosa’s PR-1 genes,  also searching for evidence of positive selection shaping those proteins. We used putative PR-1 gene families identified across the genomes of 22 Moniliophthora isolates (18 M. perniciosa and 4 M. roreri) for inference of the gene phylogenetic history within Moniliophthora and also for phylogenetic inference among orthologous PR-1 identified from other 17 species from Agaricales order. These analysis revealed that PR-1c,  a highly expressed gene during WBD,  is possibly exclusive to C-biotype isolates and is a recent paralog of PR-1j. Besides,  the five most recent PR-1 genes in the gene phylogenetic tree are the ones expressed during infection (f, g, i, k, h). The phylogenetic inference with Agaricales PR-1 genes revealed that PR-1a,  b,  d,  and j,  were the gene families with most orthologous from the various species,  while PR-1i,  g and k are only represented in Moniliophthora isolates. PR-1a, b and d are ubiquitously expressed during M. perniciosa mycelium stages and PR-1j is expressed in basidiomata,  suggesting that these proteins might have a role for fungi basal metabolism. Evolutionary models of positive selection were tested in all Moniliophthora PR-1 gene families using the dN/dS ratio with codeml package of PAML4. While branch-sites model using C-biotype branches as foreground did not detect evidence of positive selection,  testing for sites model detected signals of sites under positive selection in four PR-1 families (PR-1f,  g,  h,  i),  which are strongly expressed during WBD. These proteins might have evolved from non-pathogenic PR-1 proteins and became advantageous for the pathogen’s success in host infection during Moniliophthora recent evolution,  revealing important proteins and codon-targets for the pathogenicity of M. perniciosa in cocoa.
  
  Funding: Supported by FAPESP (2017/13015-7) \\ 
  \end{abstract}
  \end{document} 