
  \documentclass[twoside]{article}
  \usepackage[affil-it]{authblk}
  \usepackage{lipsum} % Package to generate dummy text throughout this template
  \usepackage{eurosym}
  \usepackage[sc]{mathpazo} % Use the Palatino font
  \usepackage[T1]{fontenc} % Use 8-bit encoding that has 256 glyphs
  \usepackage[utf8]{inputenc}
  \linespread{1.05} % Line spacing-Palatino needs more space between lines
  \usepackage{microtype} % Slightly tweak font spacing for aesthetics\[IndentingNewLine]
  \usepackage[hmarginratio=1:1,top=32mm,columnsep=20pt]{geometry} % Document margins
  \usepackage{multicol} % Used for the two-column layout of the document
  \usepackage[hang,small,labelfont=bf,up,textfont=it,up]{caption} % Custom captions under//above floats in tables or figures
  \usepackage{booktabs} % Horizontal rules in tables
  \usepackage{float} % Required for tables and figures in the multi-column environment-they need to be placed in specific locations with the[H] (e.g. \begin{table}[H])
  \usepackage{hyperref} % For hyperlinks in the PDF
  \usepackage{lettrine} % The lettrine is the first enlarged letter at the beginning of the text
  \usepackage{paralist} % Used for the compactitem environment which makes bullet points with less space between them
  \usepackage{abstract} % Allows abstract customization
  \renewcommand{\abstractnamefont}{\normalfont\bfseries} 
  %\renewcommand{\abstracttextfont}{\normalfont\small\itshape} % Set the abstract itself to small italic text\[IndentingNewLine]
  \usepackage{titlesec} % Allows customization of titles
  \renewcommand\thesection{\Roman{section}} % Roman numerals for the sections
  \renewcommand\thesubsection{\Roman{subsection}} % Roman numerals for subsections
  \titleformat{\section}[block]{\large\scshape\centering}{\thesection.}{1em}{} % Change the look of the section titles
  \titleformat{\subsection}[block]{\large}{\thesubsection.}{1em}{} % Change the look of the section titles
  \usepackage{fancyhdr} % Headers and footers
  \pagestyle{fancy} % All pages have headers and footers
  \fancyhead{} % Blank out the default header
  \fancyfoot{} % Blank out the default footer
  \fancyhead[C]{X-meeting $\bullet$ October 2019 $\bullet$ Campos do  Jord\~ao} % Custom header text
  \fancyfoot[RO,LE]{} % Custom footer text
  %----------------------------------------------------------------------------------------
  % TITLE SECTION
  %---------------------------------------------------------------------------------------- 
 
 \title{\vspace{-15mm}\fontsize{24pt}{10pt}\selectfont\textbf{ Genome-wide Association Studies Reveal Candidate Genes Important for the Interaction of Bacillus pumilus with Arabidopsis thaliana }} % Article title
  
  
  \author{ Marina Soneghett Cotta, Fernanda do Amaral, Leonardo Magalh\~aes Cruz, Fabio de Oliveira Pedrosa, Emanuel Maltempi de Souza, Tadashi Yokoyama, Gary Stacey }
  
  \affil{ University of Missouri }
  \vspace{-5mm}
  \date{}
  
  %---------------------------------------------------------------------------------------- 
  
  \begin{document}
  
  
  \maketitle % Insert title
  
  
  \thispagestyle{fancy} % All pages have headers and footers
  %----------------------------------------------------------------------------------------  
  % ABSTRACT
  
  %----------------------------------------------------------------------------------------  
  
  \begin{abstract}
  The plant growth promoting bacterium (PGPB) Bacillus pumilus is a nitrogen fixing and a gibberellin producer that increases the nitrogen content and shoot length and surface in plants. In addition,  this PGPB has the capability of improving plant growth under drought and saline conditions. The strain Bacillus pumilus TUAT-1 can increase rice’s roots and biomass and the content of nitrogen and chlorophyll. In this study,  through genome-wide association study (GWAS),  we evaluated the interaction between TUAT-1 and Arabidopsis thaliana. In order to do that,  288 A. thaliana accessions were screened for root architecture traits: main root length (MRL),  number of lateral roots (NLR),  branched zone (BZ),  total root length (TRL) and lateral root length (LRL). GWAS accelerate mixed model analysis was performed for the 5 traits within 288 ecotypes. Several ecotypes responded significantly to TUAT-1 inoculation: MRL (52.7\%),  NLR (14.2\%),  BZ (8.3\%),  TRL (21.2\%) and LRL (19.1\%). Many inoculated ecotypes were significantly affected in more than one trait. However,  only one ecotype showed a significant difference between control and inoculated plants for all the traits. While some of the ecotypes either did not respond or responded positively for growth,  a few ecotypes showed inhibition of root growth upon inoculation. Significant single nucleotide polymorphisms (SNPs) were detected in all the traits evaluated. 34\% of the significant SNPs were associated with more than one trait and,  1 SNP was associated with 4 different traits. Causative SNPs were selected according to missense or nonsense alterations and produced a list of candidate genes related to hormone production,  defense response,  root development,  autophagy,  and fatty acid metabolism. Candidate genes were very likely associated with the interaction between TUAT-1 and Arabidopsis and the plant growth promotion. In this study,  we validated previous reported Bacillus spp. and its plant interaction and growth promotion genes and highlight the potential genes involved in these mechanisms. The results of this work showed that some of the root architecture characteristics are genetic separable traits associated with the plant growth. We suggest that plant-bacteria interaction and the plant growth promotion are quantitative and multigene traits. This knowledge expands our understanding of the functional mechanisms driving the plant growth promotion by PGPB.
  
  Funding: Supported by INCT-FBN,  CNPq,  Funda\c{c}\~ao Arauc\'aria and CAPES \\ 
  \end{abstract}
  \end{document} 