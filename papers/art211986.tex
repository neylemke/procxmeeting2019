
  \documentclass[twoside]{article}
  \usepackage[affil-it]{authblk}
  \usepackage{lipsum} % Package to generate dummy text throughout this template
  \usepackage{eurosym}
  \usepackage[sc]{mathpazo} % Use the Palatino font
  \usepackage[T1]{fontenc} % Use 8-bit encoding that has 256 glyphs
  \usepackage[utf8]{inputenc}
  \linespread{1.05} % Line spacing-Palatino needs more space between lines
  \usepackage{microtype} % Slightly tweak font spacing for aesthetics\[IndentingNewLine]
  \usepackage[hmarginratio=1:1,top=32mm,columnsep=20pt]{geometry} % Document margins
  \usepackage{multicol} % Used for the two-column layout of the document
  \usepackage[hang,small,labelfont=bf,up,textfont=it,up]{caption} % Custom captions under//above floats in tables or figures
  \usepackage{booktabs} % Horizontal rules in tables
  \usepackage{float} % Required for tables and figures in the multi-column environment-they need to be placed in specific locations with the[H] (e.g. \begin{table}[H])
  \usepackage{hyperref} % For hyperlinks in the PDF
  \usepackage{lettrine} % The lettrine is the first enlarged letter at the beginning of the text
  \usepackage{paralist} % Used for the compactitem environment which makes bullet points with less space between them
  \usepackage{abstract} % Allows abstract customization
  \renewcommand{\abstractnamefont}{\normalfont\bfseries} 
  %\renewcommand{\abstracttextfont}{\normalfont\small\itshape} % Set the abstract itself to small italic text\[IndentingNewLine]
  \usepackage{titlesec} % Allows customization of titles
  \renewcommand\thesection{\Roman{section}} % Roman numerals for the sections
  \renewcommand\thesubsection{\Roman{subsection}} % Roman numerals for subsections
  \titleformat{\section}[block]{\large\scshape\centering}{\thesection.}{1em}{} % Change the look of the section titles
  \titleformat{\subsection}[block]{\large}{\thesubsection.}{1em}{} % Change the look of the section titles
  \usepackage{fancyhdr} % Headers and footers
  \pagestyle{fancy} % All pages have headers and footers
  \fancyhead{} % Blank out the default header
  \fancyfoot{} % Blank out the default footer
  \fancyhead[C]{X-meeting $\bullet$ October 2019 $\bullet$ Campos do  Jord\~ao} % Custom header text
  \fancyfoot[RO,LE]{} % Custom footer text
  %----------------------------------------------------------------------------------------
  % TITLE SECTION
  %---------------------------------------------------------------------------------------- 
 
 \title{\vspace{-15mm}\fontsize{24pt}{10pt}\selectfont\textbf{ Functional genomics of the Rhipicephalus microplus tick infection process by Metarhizium anisopliae: unraveling the mechanisms of host-pathogen infection }} % Article title
  
  
  \author{ Mateus Martins Frasnelli, Ana Trindade Wink, Claudia Elizabeth Thompson }
  
  \affil{ UFCSPA }
  \vspace{-5mm}
  \date{}
  
  %---------------------------------------------------------------------------------------- 
  
  \begin{document}
  
  
  \maketitle % Insert title
  
  
  \thispagestyle{fancy} % All pages have headers and footers
  %----------------------------------------------------------------------------------------  
  % ABSTRACT
  
  %----------------------------------------------------------------------------------------  
  
  \begin{abstract}
  Metarhizium anisopliae is an entomopathogenic fungus that causes infections in several arthropod species. The biological control of parasitic arthropods such as the tick Rhipicephalus microplus has great economic and sanitary interest,  since they bring losses in Brazilian livestock. Several acaricides are available in the market,  but they leave residues in meat,  milk and derivatives. Sustainable alternatives have been studied and one of the best-known models is Metarhizium anisopliae. However,  the development of an efficient control method is hard because it is yet not possible to completely understand their infection mechanisms. This project aims to help in the understanding of the evolutionary history of these organisms and pathogenicity related genes.
The research was performed considering the pathogenicity,  size,  assembly status,  GC content of the Metarhizium genomes deposited in public databases. We identified ortholog gene groups in the 12 species found using OrthoFinder. These sequences were aligned with PRANK software and,  subsequently,  a supermatrix was constructed from multiple alignments with SCaFos software. Finally,  distance and probabilistic methods of phylogenetic reconstruction were applied with the MEGA software,  and these processes were documented and used in the building of a pipeline.
We selected 5, 509 groups of orthologous genes,  from which we obtained a phylogenomic tree with high statistical support. A selected tree depicts an evolutionary relationship between the twelve genomes of the genus Metarhizium,  corroborating some of the data available in the literature. We were able to identify genes shared among different species and those specific to each organism. This has allowed the establishment of the relationships among Metarhizium species and the identification of genes related to pathogenicity.
  
  Funding: FAPERGS \\ 
  \end{abstract}
  \end{document} 