
  \documentclass[twoside]{article}
  \usepackage[affil-it]{authblk}
  \usepackage{lipsum} % Package to generate dummy text throughout this template
  \usepackage{eurosym}
  \usepackage[sc]{mathpazo} % Use the Palatino font
  \usepackage[T1]{fontenc} % Use 8-bit encoding that has 256 glyphs
  \usepackage[utf8]{inputenc}
  \linespread{1.05} % Line spacing-Palatino needs more space between lines
  \usepackage{microtype} % Slightly tweak font spacing for aesthetics\[IndentingNewLine]
  \usepackage[hmarginratio=1:1,top=32mm,columnsep=20pt]{geometry} % Document margins
  \usepackage{multicol} % Used for the two-column layout of the document
  \usepackage[hang,small,labelfont=bf,up,textfont=it,up]{caption} % Custom captions under//above floats in tables or figures
  \usepackage{booktabs} % Horizontal rules in tables
  \usepackage{float} % Required for tables and figures in the multi-column environment-they need to be placed in specific locations with the[H] (e.g. \begin{table}[H])
  \usepackage{hyperref} % For hyperlinks in the PDF
  \usepackage{lettrine} % The lettrine is the first enlarged letter at the beginning of the text
  \usepackage{paralist} % Used for the compactitem environment which makes bullet points with less space between them
  \usepackage{abstract} % Allows abstract customization
  \renewcommand{\abstractnamefont}{\normalfont\bfseries} 
  %\renewcommand{\abstracttextfont}{\normalfont\small\itshape} % Set the abstract itself to small italic text\[IndentingNewLine]
  \usepackage{titlesec} % Allows customization of titles
  \renewcommand\thesection{\Roman{section}} % Roman numerals for the sections
  \renewcommand\thesubsection{\Roman{subsection}} % Roman numerals for subsections
  \titleformat{\section}[block]{\large\scshape\centering}{\thesection.}{1em}{} % Change the look of the section titles
  \titleformat{\subsection}[block]{\large}{\thesubsection.}{1em}{} % Change the look of the section titles
  \usepackage{fancyhdr} % Headers and footers
  \pagestyle{fancy} % All pages have headers and footers
  \fancyhead{} % Blank out the default header
  \fancyfoot{} % Blank out the default footer
  \fancyhead[C]{X-meeting $\bullet$ October 2019 $\bullet$ Campos do  Jord\~ao} % Custom header text
  \fancyfoot[RO,LE]{} % Custom footer text
  %----------------------------------------------------------------------------------------
  % TITLE SECTION
  %---------------------------------------------------------------------------------------- 
 
 \title{\vspace{-15mm}\fontsize{24pt}{10pt}\selectfont\textbf{ The origin of the stoma was during the Paleozoic era,  as ancient genes were co-opted by the stoma system }} % Article title
  
  
  \author{ Beatriz Moura Kfoury de Castro, Tetsu Sakamoto, Jos\'e Miguel Ortega }
  
  \affil{ Universidade Federal de Minas Gerais }
  \vspace{-5mm}
  \date{}
  
  %---------------------------------------------------------------------------------------- 
  
  \begin{document}
  
  
  \maketitle % Insert title
  
  
  \thispagestyle{fancy} % All pages have headers and footers
  %----------------------------------------------------------------------------------------  
  % ABSTRACT
  
  %----------------------------------------------------------------------------------------  
  
  \begin{abstract}
  Stomata are typically found in plant leaves but can also be found in some stems. These structures comprise specialized cells known as guard cells surrounding stoma that function to open and close the stomatal pores,  allowing plant to take in carbon dioxide,  which is needed for photosynthesis. They also help to reduce water loss by closing the structure when conditions are hot or dry. In Arabidopsis,  cell lineage undergoes a series of cell divisions and cell-state transitions to produce a stoma. An evolutionarily ancient group that are known as basic helix-loop-helix (bHLH) transcription factors,  specify cellular identity in both plants and animals. These transcription factors regulate the transition and differentiation events through the lineage of plants. Some genes associated with regulating stomatal differentiation are also associated with hormonal and environmental stress responses. To retrieve genes associated with stomata,  scientific papers were collected describing stomatal development in Arabidopsis thaliana. These genes were analyzed and selected manually. After this selection,  a regulatory pathway was constructed with manual curation. To analyze the evolutionary origin of genes in the pathway,  their sequences were recovered from the UniProtKB database and their orthologs from other species were recovered using TaxOnTree software produced by our group. To determine the lowest common ancestor (LCA) between species with representative proteins,  we used the cluster of orthologs for inferring the clade of origin of the gene. As a result,  we were able to characterize the origin of 52 genes linked to the stoma system in Arabidopsis thaliana. Inference from gene origin showed that most of the genes are related to cell division and hormone and enviromental signaling and they appeared in the more ancient clades of the evolutionary history of plants,  approximately during the early Mesoproterozoic (1800-1300 Ma),  in the Eukaryota clade. Notably,  most of the genes had originated during early Paleozoic era (540-480 Ma) between a middle Cambrian – Early Ordovocian interval,  in the Embriophyta clade. These genes are mainly linked to the differentiation of stomatal tissue. Other genes originated more recently in plant evolution,  such as Magnoliphyta and Mesangiospermae clades,  during the middle Jurassic (175 Ma). Our data showed that the main genes involved in controlling stoma formation probably originated in Embryophyta,  during the conquest of the terrestrial environment by plants. The stoma structure controls the gas exchange by the plant and it is directly related to essential processes for plant survival,  such as respiration,  transpiration and photosynthesis. Although several genes originated more anciently in the early Mesoproterozoic,  these genes were probably later co-opted to form and control the stoma system,  as they linked an environmental and hormonal response role,  since during this period there was a large accumulation of carbon dioxide in atmosphere. Linking the networks that control stomatal development might improve our understanding of evolutionary history of plants and exemplifies how the response to the environment is important for understanding the origin of some current structures and their functions.
  
  Funding: CAPES Computational Biology Networks: Biologia Sist\^emica do C\^ancer,  BSC. UFMG \\ 
  \end{abstract}
  \end{document} 