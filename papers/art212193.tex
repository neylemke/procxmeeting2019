
  \documentclass[twoside]{article}
  \usepackage[affil-it]{authblk}
  \usepackage{lipsum} % Package to generate dummy text throughout this template
  \usepackage{eurosym}
  \usepackage[sc]{mathpazo} % Use the Palatino font
  \usepackage[T1]{fontenc} % Use 8-bit encoding that has 256 glyphs
  \usepackage[utf8]{inputenc}
  \linespread{1.05} % Line spacing-Palatino needs more space between lines
  \usepackage{microtype} % Slightly tweak font spacing for aesthetics\[IndentingNewLine]
  \usepackage[hmarginratio=1:1,top=32mm,columnsep=20pt]{geometry} % Document margins
  \usepackage{multicol} % Used for the two-column layout of the document
  \usepackage[hang,small,labelfont=bf,up,textfont=it,up]{caption} % Custom captions under//above floats in tables or figures
  \usepackage{booktabs} % Horizontal rules in tables
  \usepackage{float} % Required for tables and figures in the multi-column environment-they need to be placed in specific locations with the[H] (e.g. \begin{table}[H])
  \usepackage{hyperref} % For hyperlinks in the PDF
  \usepackage{lettrine} % The lettrine is the first enlarged letter at the beginning of the text
  \usepackage{paralist} % Used for the compactitem environment which makes bullet points with less space between them
  \usepackage{abstract} % Allows abstract customization
  \renewcommand{\abstractnamefont}{\normalfont\bfseries} 
  %\renewcommand{\abstracttextfont}{\normalfont\small\itshape} % Set the abstract itself to small italic text\[IndentingNewLine]
  \usepackage{titlesec} % Allows customization of titles
  \renewcommand\thesection{\Roman{section}} % Roman numerals for the sections
  \renewcommand\thesubsection{\Roman{subsection}} % Roman numerals for subsections
  \titleformat{\section}[block]{\large\scshape\centering}{\thesection.}{1em}{} % Change the look of the section titles
  \titleformat{\subsection}[block]{\large}{\thesubsection.}{1em}{} % Change the look of the section titles
  \usepackage{fancyhdr} % Headers and footers
  \pagestyle{fancy} % All pages have headers and footers
  \fancyhead{} % Blank out the default header
  \fancyfoot{} % Blank out the default footer
  \fancyhead[C]{X-meeting $\bullet$ October 2019 $\bullet$ Campos do  Jord\~ao} % Custom header text
  \fancyfoot[RO,LE]{} % Custom footer text
  %----------------------------------------------------------------------------------------
  % TITLE SECTION
  %---------------------------------------------------------------------------------------- 
 
 \title{\vspace{-15mm}\fontsize{24pt}{10pt}\selectfont\textbf{ lncRNAs modulated in response to metformin treatment }} % Article title
  
  
  \author{ Lucio Rezende Queiroz, Izabela Mamede Costa Andrade da Concei\c{c}\~ao, Marcelo Rizzatti Luizon, Gl\'oria Regina Franco }
  
  \affil{ Universidade Federal de Minas Gerais }
  \vspace{-5mm}
  \date{}
  
  %---------------------------------------------------------------------------------------- 
  
  \begin{document}
  
  
  \maketitle % Insert title
  
  
  \thispagestyle{fancy} % All pages have headers and footers
  %----------------------------------------------------------------------------------------  
  % ABSTRACT
  
  %----------------------------------------------------------------------------------------  
  
  \begin{abstract}
  Metformin is among the most widely prescribed drugs. It is used as first-line therapy for type 2 diabetes (T2D) and prescribed for numerous other diseases including cancer. Despite recent advances,  the molecular mechanisms underlying metformin action are not fully understood and the role of long noncoding RNAs (lncRNAs) are yet to be associated with metformin response. Using high throughput RNA sequencing,  we have analyzed the transcriptional diversity associated with the metformin treatment in human liver cells and identified in a transcriptomic-wide manner hundreds of differentially expressed genes affected by metformin,  including lncRNAs. Co-expression networks and guilt-by-association functional analysis allowed us to identify several novel relations between lncRNAs and genes with known function associated with metformin response.
These include NEAT1,  a lncRNA never before associated with gluconeogenesis repression upon metformin response and a potential regulator to known genes that are associated with noncoding RNA metabolic processes,  RNA processing,  cytoskeleton organization,  transcription activation,  and ATP binding. Moreover,  network analyses showed that SPATA41,  MIR222HG,  LINC02348,  DUBR,  LINC00324,  and MIR122HG,  collectively those lncRNAs are related to the regulation of transcription,  response to nutrients,  acute inflammatory response and are potentially attached to already known and novel pathways in the antidiabetic and anticancer effects of metformin. 
These results suggest that several pathways are regulated by lncRNAs in response to metformin. Our work opens new perspectives on the mechanisms by which lncRNAs that are activated due to metformin treatment are involved in the regulation and control of molecular pathophysiological mechanisms altered in the diseases where metformin is prescribed and thus provides novel candidates for T2D and other diseases treatment.
  
  Funding: CAPES – Coordena\c{c}\~ao de Aperfei\c{c}oamento de Pessoal de N\'{\i}vel Superior; CNPq – Conselho Nacional de Desenvolvimento Cient\'{\i}fico e Tecnol\'ogico \\ 
  \end{abstract}
  \end{document} 