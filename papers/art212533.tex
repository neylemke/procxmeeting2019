
  \documentclass[twoside]{article}
  \usepackage[affil-it]{authblk}
  \usepackage{lipsum} % Package to generate dummy text throughout this template
  \usepackage{eurosym}
  \usepackage[sc]{mathpazo} % Use the Palatino font
  \usepackage[T1]{fontenc} % Use 8-bit encoding that has 256 glyphs
  \usepackage[utf8]{inputenc}
  \linespread{1.05} % Line spacing-Palatino needs more space between lines
  \usepackage{microtype} % Slightly tweak font spacing for aesthetics\[IndentingNewLine]
  \usepackage[hmarginratio=1:1,top=32mm,columnsep=20pt]{geometry} % Document margins
  \usepackage{multicol} % Used for the two-column layout of the document
  \usepackage[hang,small,labelfont=bf,up,textfont=it,up]{caption} % Custom captions under//above floats in tables or figures
  \usepackage{booktabs} % Horizontal rules in tables
  \usepackage{float} % Required for tables and figures in the multi-column environment-they need to be placed in specific locations with the[H] (e.g. \begin{table}[H])
  \usepackage{hyperref} % For hyperlinks in the PDF
  \usepackage{lettrine} % The lettrine is the first enlarged letter at the beginning of the text
  \usepackage{paralist} % Used for the compactitem environment which makes bullet points with less space between them
  \usepackage{abstract} % Allows abstract customization
  \renewcommand{\abstractnamefont}{\normalfont\bfseries} 
  %\renewcommand{\abstracttextfont}{\normalfont\small\itshape} % Set the abstract itself to small italic text\[IndentingNewLine]
  \usepackage{titlesec} % Allows customization of titles
  \renewcommand\thesection{\Roman{section}} % Roman numerals for the sections
  \renewcommand\thesubsection{\Roman{subsection}} % Roman numerals for subsections
  \titleformat{\section}[block]{\large\scshape\centering}{\thesection.}{1em}{} % Change the look of the section titles
  \titleformat{\subsection}[block]{\large}{\thesubsection.}{1em}{} % Change the look of the section titles
  \usepackage{fancyhdr} % Headers and footers
  \pagestyle{fancy} % All pages have headers and footers
  \fancyhead{} % Blank out the default header
  \fancyfoot{} % Blank out the default footer
  \fancyhead[C]{X-meeting $\bullet$ October 2019 $\bullet$ Campos do  Jord\~ao} % Custom header text
  \fancyfoot[RO,LE]{} % Custom footer text
  %----------------------------------------------------------------------------------------
  % TITLE SECTION
  %---------------------------------------------------------------------------------------- 
 
 \title{\vspace{-15mm}\fontsize{24pt}{10pt}\selectfont\textbf{ IDENTIFICATION OF FUNGI IN A BRASILIAN PAINT OF THE 20th CENTURY }} % Article title
  
  
  \author{ Valqu\'{\i}ria de Oliveira Silva, paula  luize camargos fonseca, Maria  Aparecida de Resende Stoianoff, Arist\'oteles G\'oes Neto }
  
  \affil{ Universidade Federal de Minas Gerais }
  \vspace{-5mm}
  \date{}
  
  %---------------------------------------------------------------------------------------- 
  
  \begin{document}
  
  
  \maketitle % Insert title
  
  
  \thispagestyle{fancy} % All pages have headers and footers
  %----------------------------------------------------------------------------------------  
  % ABSTRACT
  
  %----------------------------------------------------------------------------------------  
  
  \begin{abstract}
  The cultural heritage is our present time baggage and the heritage left to future generations. These cultural objects such paintings,  sculptures,  scrolls,  archeologica sites are subject to several mechanismis of deterioration. These include microbial deterioration caused by fungi and bacteria that causes irreparable damage to cultural heritage. This work consisted in the isolation and identification of the fungi responsible for the deterioration in a twentieth-century Brazilian easel painting by italian-brazilian artist Lorenzato. The material for the isolation of the fungi was collected from the original work by means of sterile swabs in various regions of the pictorial surface between areas with fungal colonization and areas without apparent colonization. The samples were diluted in saline solution (0.85\%). And by the serial dilution method,  100$\mu$L aliquots were obtained from the  10-1,  10-2,  and 10-3 dilutions that were seeded on Potato Dextrose Agar (BDA) by the spread plate method. The isolated fungi were purified and observed at macroscopic and microscopic level for identification at genera level. We obtained 9 colonies of morphologically distinct fungi belonging to the following genera 2 Aspergillus,  4 Penicillium,  2 Hypocrea and 1 Nigrospora. Molecular characterization of the isolates was performed by extraction of fungal DNAs,  PCR,  and sequencing of the ITS4 and ITS5 regions of each sample. The results were processed in the Geneious software. They were analyzed on the Blast website for comparison and identification by similarity analysis with the NCBI Genbank database (nr) (the sequences deposited with the sequences obtained in this work were compared). The deteriorogenic species Aspergillus sydowii,  Penicillium crysogenum,  Hypocrea lixii and Nigrospora sphaerica were identified. The obtained results contribute for the understanding of the biodeteriogenic agents and also for the analysis of the conservation state of the studied object,  allowing the adoption of mitigating measures in the scope of the preventive and curative conservation collaborating for the preservation of Lorenzato painting.
  
  Funding:  \\ 
  \end{abstract}
  \end{document} 