
  \documentclass[twoside]{article}
  \usepackage[affil-it]{authblk}
  \usepackage{lipsum} % Package to generate dummy text throughout this template
  \usepackage{eurosym}
  \usepackage[sc]{mathpazo} % Use the Palatino font
  \usepackage[T1]{fontenc} % Use 8-bit encoding that has 256 glyphs
  \usepackage[utf8]{inputenc}
  \linespread{1.05} % Line spacing-Palatino needs more space between lines
  \usepackage{microtype} % Slightly tweak font spacing for aesthetics\[IndentingNewLine]
  \usepackage[hmarginratio=1:1,top=32mm,columnsep=20pt]{geometry} % Document margins
  \usepackage{multicol} % Used for the two-column layout of the document
  \usepackage[hang,small,labelfont=bf,up,textfont=it,up]{caption} % Custom captions under//above floats in tables or figures
  \usepackage{booktabs} % Horizontal rules in tables
  \usepackage{float} % Required for tables and figures in the multi-column environment-they need to be placed in specific locations with the[H] (e.g. \begin{table}[H])
  \usepackage{hyperref} % For hyperlinks in the PDF
  \usepackage{lettrine} % The lettrine is the first enlarged letter at the beginning of the text
  \usepackage{paralist} % Used for the compactitem environment which makes bullet points with less space between them
  \usepackage{abstract} % Allows abstract customization
  \renewcommand{\abstractnamefont}{\normalfont\bfseries} 
  %\renewcommand{\abstracttextfont}{\normalfont\small\itshape} % Set the abstract itself to small italic text\[IndentingNewLine]
  \usepackage{titlesec} % Allows customization of titles
  \renewcommand\thesection{\Roman{section}} % Roman numerals for the sections
  \renewcommand\thesubsection{\Roman{subsection}} % Roman numerals for subsections
  \titleformat{\section}[block]{\large\scshape\centering}{\thesection.}{1em}{} % Change the look of the section titles
  \titleformat{\subsection}[block]{\large}{\thesubsection.}{1em}{} % Change the look of the section titles
  \usepackage{fancyhdr} % Headers and footers
  \pagestyle{fancy} % All pages have headers and footers
  \fancyhead{} % Blank out the default header
  \fancyfoot{} % Blank out the default footer
  \fancyhead[C]{X-meeting $\bullet$ October 2019 $\bullet$ Campos do  Jord\~ao} % Custom header text
  \fancyfoot[RO,LE]{} % Custom footer text
  %----------------------------------------------------------------------------------------
  % TITLE SECTION
  %---------------------------------------------------------------------------------------- 
 
 \title{\vspace{-15mm}\fontsize{24pt}{10pt}\selectfont\textbf{ THE PAINTING I LIVE: FUNGI IDENTIFIED ON PICTORIAL SURFACE }} % Article title
  
  
  \author{ Valqu\'{\i}ria de Oliveira Silva, Paula Luize Camargos Fonseca, Luiz Marcelo Ribeiro Tom\'e, Arist\'oteles G\'oes Neto }
  
  \affil{ Universidade Federal de Minas Gerais }
  \vspace{-5mm}
  \date{}
  
  %---------------------------------------------------------------------------------------- 
  
  \begin{document}
  
  
  \maketitle % Insert title
  
  
  \thispagestyle{fancy} % All pages have headers and footers
  %----------------------------------------------------------------------------------------  
  % ABSTRACT
  
  %----------------------------------------------------------------------------------------  
  
  \begin{abstract}
  The cultural heritage in their miscellaneous  forms of artistic manifestation are subject to several mechanisms of chemical,  physical,  and biological deterioration. These biological mechanisms include microbial deterioration caused by fungi and bacteria. These cultural heritage refer to memory,  cultural identity of a society,  and represent the legacy left to future generations. The present study consisted in the identification of deteriogenic fungi present in the surface of the painting / panel of PORTINARI entitled “FREVO”. This painting is very important because it was the last work of the italian-brazilian prepared by the artist before his death,  and it is part of the Pampulha landscape Complex in BH / MG / Brazil,  consisting of a set of assets listed by UNESCO as a cultural heritage of humanity. Samples were collected on the pictorial surface of the original work by means of sterile swabs in various regions between areas that had fungal colonization,  and areas that did not have colonization. The samples were diluted in saline solution (0.85\%),  and by the serial dilution method,  100$\mu$L aliquots were obtained from the 10-1,  10-2,  and 10-3 dilutions that were seeded on Potato Dextrose Agar (BDA) by the spread plate method. The isolated fungi were purified and observed at macroscopic and microscopic level for identification at genus level. Molecular characterization of the isolates was performed by extraction of fungal DNAs,  PCR,  and sequencing of the nrDNA ITS region of each sample. The results were processed in the Geneious software. They were analyzed on the Blast site for comparison and identification by similarity analysis with the NCBI Genbank database (nr). We found 14 fungal species from 8 different families: 6 Aspergillaceae from the following genera (4 Aspergillus and 2 Penicillium),  2 Didymellaceae whose genera are (Epicoccum and Didymella),  1 Apiosporaceae (Arthinium),  1 Hypocreaceae (Trichoderma),  1 Calosphaeriaceae (Pleurostoma),  1 Chaetomiaceae (Chaetomium),  1 Mytilinidiacea (Mytilinidion),  and 1 Sporidiobolaceae (Rhodotorula). The obtained results contribute for the knowledge of the biodeteriogenic agents in the case studied the filamentous fungal and yeasts,  and also for the analysis of the conservation state of the fine work,  thus,  allowing the adoption of mitigating measures in the scope of the preventive and curative conservation,  collaborating for the preservation of the PORTINARI fine work.
  
  Funding:  \\ 
  \end{abstract}
  \end{document} 