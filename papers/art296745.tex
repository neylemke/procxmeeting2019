
  \documentclass[twoside]{article}
  \usepackage[affil-it]{authblk}
  \usepackage{lipsum} % Package to generate dummy text throughout this template
  \usepackage{eurosym}
  \usepackage[sc]{mathpazo} % Use the Palatino font
  \usepackage[T1]{fontenc} % Use 8-bit encoding that has 256 glyphs
  \usepackage[utf8]{inputenc}
  \linespread{1.05} % Line spacing-Palatino needs more space between lines
  \usepackage{microtype} % Slightly tweak font spacing for aesthetics\[IndentingNewLine]
  \usepackage[hmarginratio=1:1,top=32mm,columnsep=20pt]{geometry} % Document margins
  \usepackage{multicol} % Used for the two-column layout of the document
  \usepackage[hang,small,labelfont=bf,up,textfont=it,up]{caption} % Custom captions under//above floats in tables or figures
  \usepackage{booktabs} % Horizontal rules in tables
  \usepackage{float} % Required for tables and figures in the multi-column environment-they need to be placed in specific locations with the[H] (e.g. \begin{table}[H])
  \usepackage{hyperref} % For hyperlinks in the PDF
  \usepackage{lettrine} % The lettrine is the first enlarged letter at the beginning of the text
  \usepackage{paralist} % Used for the compactitem environment which makes bullet points with less space between them
  \usepackage{abstract} % Allows abstract customization
  \renewcommand{\abstractnamefont}{\normalfont\bfseries} 
  %\renewcommand{\abstracttextfont}{\normalfont\small\itshape} % Set the abstract itself to small italic text\[IndentingNewLine]
  \usepackage{titlesec} % Allows customization of titles
  \renewcommand\thesection{\Roman{section}} % Roman numerals for the sections
  \renewcommand\thesubsection{\Roman{subsection}} % Roman numerals for subsections
  \titleformat{\section}[block]{\large\scshape\centering}{\thesection.}{1em}{} % Change the look of the section titles
  \titleformat{\subsection}[block]{\large}{\thesubsection.}{1em}{} % Change the look of the section titles
  \usepackage{fancyhdr} % Headers and footers
  \pagestyle{fancy} % All pages have headers and footers
  \fancyhead{} % Blank out the default header
  \fancyfoot{} % Blank out the default footer
  \fancyhead[C]{X-meeting eXperience $\bullet$ November 2020} % Custom header text
  \fancyfoot[RO,LE]{} % Custom footer text
  %----------------------------------------------------------------------------------------
  % TITLE SECTION
  %---------------------------------------------------------------------------------------- 
 
 \title{\vspace{-15mm}\fontsize{24pt}{10pt}\selectfont\textbf{ The extracellular vesicles produced by the probiotic Propionibacterium freudenreichii in different conditions share a core proteome that includes immunomodulatory proteins }} % Article title
  
  
  \author{ Brenda Silva Rosa da Luz,  Fillipe Luiz Rosa do Carmo,  Gwenaël Jan,  Yves Le Loir,  Vasco A de C Azevedo,  Eric Guedon,  Vin\'{\i}cius de Rezende Rodovalho }
  
  \affil{ INAPG- Fran\c{c}a,  UNIVERSIDADE FEDERAL DE MINAS GERAIS,  UNIVERSIDADE FEDERAL DE MINAS GERAIS }
  \vspace{-5mm}
  \date{}
  
  %---------------------------------------------------------------------------------------- 
  
  \begin{document}
  
  
  \maketitle % Insert title
  
  
  \thispagestyle{fancy} % All pages have headers and footers
  %----------------------------------------------------------------------------------------  
  % ABSTRACT
  
  %----------------------------------------------------------------------------------------  
  
  \begin{abstract}
  The probiotic properties of several organisms have been consistently associated to surface or secreted factors,  including specific proteins that mediate interactions with the host. One of such organisms is Propionibacterium freudenreichii,  a Gram-positive dairy bacterium that has been long used in the production of cheese,  vitamin B12 and organic acids. We recently described that the extracellular vesicles (EVs) produced by this bacterium contain proteins that participate on their health-promoting roles. EVs are spherical nanostructures,  delimited by a lipid bilayer and containing several molecules in their interior. In the case of P. freudenreichii,  EVs contained surface-layer proteins previously associated to immunomodulation,  such as surface-layer protein B (SlpB) and E (SlpE). In addition,  EVs exerted an anti-inflammatory effect in vitro,  with the mitigation of NF-?B activation and IL-8 release by intestinal epithelial cells. In our first study,  EVs were purified by size-exclusion chromatography (SEC) from cultures in milk ultrafiltrate (UF). However,  it remained unclear if EVs characteristics and content would be similar in other growth or purification conditions. Therefore,  we analyzed EVs derived from P. freudenreichii cultures in yeast extract-lactate (YEL),  in addition to UF medium. Moreover,  we employed ultracentrifugation (UC) in a sucrose density gradient as a purification method,  in addition to SEC. Proteomics characterization was performed by NanoLC-ESI-MS/MS analysis. We found that,  considering all 4 conditions,  EVs share a core proteome of 308 proteins,  that included the immunomodulatory proteins SlpB and SlpE. The detection of immunomodulatory proteins in EVs obtained from different conditions opens up the possibility of EVs yield and activity optimization,  while retaining or improving immunomodulatory properties. Moreover,  functional enrichment analysis showed that the core proteome was mainly associated to energy and carbon metabolism,  ribosomal structure and biogenesis,  quorum sensing,  protein export and peptidoglycan biosynthesis. This indicates that some functional roles are conserved among different conditions of EVs obtention. The availability of this core proteome will allow a comprehensive analysis of the vesicular content of EVs produced by this probiotic strain and the identification of general traits of the proteins that are exported inside EVs. Overall,  this study contributes with a robust proteomic characterization of EVs produced by a relevant probiotic strain of P. freudenreichii.
  
  Funding:   \\
  \href{http://ab3c.org.br/xpress_pres2020/xmxp2020-296745.html}{Link to Video:}

  \end{abstract}
   
  \end{document} 