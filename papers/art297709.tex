
  \documentclass[twoside]{article}
  \usepackage[affil-it]{authblk}
  \usepackage{lipsum} % Package to generate dummy text throughout this template
  \usepackage{eurosym}
  \usepackage[sc]{mathpazo} % Use the Palatino font
  \usepackage[T1]{fontenc} % Use 8-bit encoding that has 256 glyphs
  \usepackage[utf8]{inputenc}
  \linespread{1.05} % Line spacing-Palatino needs more space between lines
  \usepackage{microtype} % Slightly tweak font spacing for aesthetics\[IndentingNewLine]
  \usepackage[hmarginratio=1:1,top=32mm,columnsep=20pt]{geometry} % Document margins
  \usepackage{multicol} % Used for the two-column layout of the document
  \usepackage[hang,small,labelfont=bf,up,textfont=it,up]{caption} % Custom captions under//above floats in tables or figures
  \usepackage{booktabs} % Horizontal rules in tables
  \usepackage{float} % Required for tables and figures in the multi-column environment-they need to be placed in specific locations with the[H] (e.g. \begin{table}[H])
  \usepackage{hyperref} % For hyperlinks in the PDF
  \usepackage{lettrine} % The lettrine is the first enlarged letter at the beginning of the text
  \usepackage{paralist} % Used for the compactitem environment which makes bullet points with less space between them
  \usepackage{abstract} % Allows abstract customization
  \renewcommand{\abstractnamefont}{\normalfont\bfseries} 
  %\renewcommand{\abstracttextfont}{\normalfont\small\itshape} % Set the abstract itself to small italic text\[IndentingNewLine]
  \usepackage{titlesec} % Allows customization of titles
  \renewcommand\thesection{\Roman{section}} % Roman numerals for the sections
  \renewcommand\thesubsection{\Roman{subsection}} % Roman numerals for subsections
  \titleformat{\section}[block]{\large\scshape\centering}{\thesection.}{1em}{} % Change the look of the section titles
  \titleformat{\subsection}[block]{\large}{\thesubsection.}{1em}{} % Change the look of the section titles
  \usepackage{fancyhdr} % Headers and footers
  \pagestyle{fancy} % All pages have headers and footers
  \fancyhead{} % Blank out the default header
  \fancyfoot{} % Blank out the default footer
  \fancyhead[C]{X-meeting eXperience $\bullet$ November 2020} % Custom header text
  \fancyfoot[RO,LE]{} % Custom footer text
  %----------------------------------------------------------------------------------------
  % TITLE SECTION
  %---------------------------------------------------------------------------------------- 
 
 \title{\vspace{-15mm}\fontsize{24pt}{10pt}\selectfont\textbf{ DNA methylation and functional annotation of CpG-SNPs in the TIMP3 locus associated with TIMP3 levels and preeclampsia }} % Article title
  
  
  \author{ Daniela Alves Pereira,  Natalia Duarte Linhares,  Marcelo Rizzatti Luizon,  Juliana de Oliveira Cruz }
  
  \affil{ Ufmg- Universidade Federal de Minas Gerais,  UNIVERSIDADE FEDERAL DE MINAS GERAIS }
  \vspace{-5mm}
  \date{}
  
  %---------------------------------------------------------------------------------------- 
  
  \begin{document}
  
  
  \maketitle % Insert title
  
  
  \thispagestyle{fancy} % All pages have headers and footers
  %----------------------------------------------------------------------------------------  
  % ABSTRACT
  
  %----------------------------------------------------------------------------------------  
  
  \begin{abstract}
  Preeclampsia (PE) is defined by hypertension after 20 weeks of gestation. Matrix metalloproteinases are endopeptidases involved in the extracellular matrix remodeling and trophoblast invasion during placentation,  and their activities are regulated by endogenous tissue inhibitors of metalloproteinases (TIMPs). DNA methylation (DNAm) is an essential epigenetic mark and changes in DNAm are involved in the pathogenesis of PE. The promoter of TIMP3 gene was found to be hypomethylated,  and the TIMP3 gene upregulated in placentas from PE,  which confirmed its relevance in the etiology of PE. Notably,  the correlation of SNPs located at CpG sites (CpG-SNPs) with allele-specific methylation in PE is unknown. In this study,  we searched for CpG islands in the TIMP3 locus and focused on CpG-SNPs putatively associated with differential DNAm within active regulatory elements associated with TIMP3 levels and/or with PE. We selected SNPs in the TIMP3 locus in association studies with PE and in the Catalog of Published GWAS. The CpG islands were predicted using software MethPrimer and the pairwise linkage disequilibrium (LD) among the selected SNPs was calculated using Haploview. In silico characterization of DNAm and regulatory elements nearby TIMP3 were performed in the UCSC Genome Browser. We selected 13 SNPs in the TIMP3 locus. This region includes seven CpG islands,  two in the TIMP3 promoter region. The TIMP3 locus contain 12 regulatory elements. Notably,  a 5.8kb promoter/enhancer segment targets TIMP3 and the other two genes,  and overlaps with the CpG-SNPs rs140495494,  rs58394026,  and rs5749511,  and the SNP rs9619311. ChIP-seq data for the histone mark H3K4me3 confirmed the presence of a potential promoter element in this region. Moreover,  11 out of the 13 selected SNPs are considered as CpG-SNPs. Nine out of the 11 were CpG-SNPs in the presence of the major allele,  and two in the presence of the minor allele. In both cases,  the CpG sites are lost in the presence of the alternative allele. We found higher pairwise LD between the SNPs rs2097326 and rs2413151,  and the rs9619311 and rs135025 in the European population. The in silico analysis of DNAm showed that the TIMP3 promoter region was methylated. Specifically,  the cg15004938 and cg07972276 sites interrogate the CpG-SNPs rs140495494 and rs5749511,  respectively,  which are methylated. Our findings suggest that CpG-SNPs may affect the epigenetic control of TIMP3 expression,  and may help to guide functional studies to elucidate the clinical role of TIMP3 expression and TIMP3 levels in PE and other diseases,  including cancer.
  
  Funding:   \\
  \href{http://ab3c.org.br/xpress_pres2020/xmxp2020-297709.html}{Link to Video:}

  \end{abstract}
   
  \end{document} 