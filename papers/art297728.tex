
  \documentclass[twoside]{article}
  \usepackage[affil-it]{authblk}
  \usepackage{lipsum} % Package to generate dummy text throughout this template
  \usepackage{eurosym}
  \usepackage[sc]{mathpazo} % Use the Palatino font
  \usepackage[T1]{fontenc} % Use 8-bit encoding that has 256 glyphs
  \usepackage[utf8]{inputenc}
  \linespread{1.05} % Line spacing-Palatino needs more space between lines
  \usepackage{microtype} % Slightly tweak font spacing for aesthetics\[IndentingNewLine]
  \usepackage[hmarginratio=1:1,top=32mm,columnsep=20pt]{geometry} % Document margins
  \usepackage{multicol} % Used for the two-column layout of the document
  \usepackage[hang,small,labelfont=bf,up,textfont=it,up]{caption} % Custom captions under//above floats in tables or figures
  \usepackage{booktabs} % Horizontal rules in tables
  \usepackage{float} % Required for tables and figures in the multi-column environment-they need to be placed in specific locations with the[H] (e.g. \begin{table}[H])
  \usepackage{hyperref} % For hyperlinks in the PDF
  \usepackage{lettrine} % The lettrine is the first enlarged letter at the beginning of the text
  \usepackage{paralist} % Used for the compactitem environment which makes bullet points with less space between them
  \usepackage{abstract} % Allows abstract customization
  \renewcommand{\abstractnamefont}{\normalfont\bfseries} 
  %\renewcommand{\abstracttextfont}{\normalfont\small\itshape} % Set the abstract itself to small italic text\[IndentingNewLine]
  \usepackage{titlesec} % Allows customization of titles
  \renewcommand\thesection{\Roman{section}} % Roman numerals for the sections
  \renewcommand\thesubsection{\Roman{subsection}} % Roman numerals for subsections
  \titleformat{\section}[block]{\large\scshape\centering}{\thesection.}{1em}{} % Change the look of the section titles
  \titleformat{\subsection}[block]{\large}{\thesubsection.}{1em}{} % Change the look of the section titles
  \usepackage{fancyhdr} % Headers and footers
  \pagestyle{fancy} % All pages have headers and footers
  \fancyhead{} % Blank out the default header
  \fancyfoot{} % Blank out the default footer
  \fancyhead[C]{X-meeting eXperience $\bullet$ November 2020} % Custom header text
  \fancyfoot[RO,LE]{} % Custom footer text
  %----------------------------------------------------------------------------------------
  % TITLE SECTION
  %---------------------------------------------------------------------------------------- 
 
 \title{\vspace{-15mm}\fontsize{24pt}{10pt}\selectfont\textbf{ Resistome profile of Acinetobacter baumannii }} % Article title
  
  
  \author{ Roselane Gon\c{c}alves dos Santos,  Flavia Figueira Aburjaile,  Vasco A de C Azevedo,  Francielly Rodrigues da Costa,  Diego Lucas Neres Rodrigues }
  
  \affil{ INAPG- Fran\c{c}a,  IOC/Fiocruz,  UFMG,  UNIVERSIDADE FEDERAL DE MINAS GERAIS }
  \vspace{-5mm}
  \date{}
  
  %---------------------------------------------------------------------------------------- 
  
  \begin{document}
  
  
  \maketitle % Insert title
  
  
  \thispagestyle{fancy} % All pages have headers and footers
  %----------------------------------------------------------------------------------------  
  % ABSTRACT
  
  %----------------------------------------------------------------------------------------  
  
  \begin{abstract}
  Acinetobacter baumannii is an important nosocomial pathogen. This Gram-negative bacterium causes several diseases,  such as pneumonia,  bacteremia,  meningitis,  osteomyelitis,  and erysipelas. It is also a pathogen highly known for its resistance to antimicrobials and its ability to survive in intensive care units assisted ventilation devices. Its characteristics make this pathogen an essential model for studies of resistance to antimicrobials. Besides,  in 2017 the World Health Organization announced that A. baumannii was a priority due to their exacerbated resistance to antimicrobials,  mainly of the class of $\beta$-lactams. In this context,  this work objective is to present the predicted resistance gene repertoire of A. baumannii. For this purpose,  the public genomes of 206 strains of this species were selected and evaluated by Comprehensive Antibiotic Resistance,  Antibacterial Biocide and Metal Resistance genes databases. These data are curated according to classes of antimicrobials. As main results,  we obtained a robust resistome composed of 131 genes related to the enzymatic inactivation of the antimicrobial compound and 26 genes encoding putative efflux pumps. We emphasize that the highlighted genes adeK,  adeJ,  adeI,  adeF,  adeG,  adeL,  adeN,  abeM,  and ampC were identified in all A. baumannii strains. The drug resistance in hospital environments is associated with AmpC $\beta$-lactamases in this pathogen,  requiring intensive monitoring. On average,  each strain showed 26 resistance genes,  except for the SDF strain,  which presented 12 genes,  and the AYE and 2008S11-069 strains that have 38 resistance genes. In conclusion,  this pathogen can be used as a good model of bacterial resistance for directing future studies aimed at therapeutic targets.
  
  Funding:   \\
  \href{http://ab3c.org.br/xpress_pres2020/xmxp2020-297728.html}{Link to Video:}

  \end{abstract}
   
  \end{document} 