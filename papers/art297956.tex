
  \documentclass[twoside]{article}
  \usepackage[affil-it]{authblk}
  \usepackage{lipsum} % Package to generate dummy text throughout this template
  \usepackage{eurosym}
  \usepackage[sc]{mathpazo} % Use the Palatino font
  \usepackage[T1]{fontenc} % Use 8-bit encoding that has 256 glyphs
  \usepackage[utf8]{inputenc}
  \linespread{1.05} % Line spacing-Palatino needs more space between lines
  \usepackage{microtype} % Slightly tweak font spacing for aesthetics\[IndentingNewLine]
  \usepackage[hmarginratio=1:1,top=32mm,columnsep=20pt]{geometry} % Document margins
  \usepackage{multicol} % Used for the two-column layout of the document
  \usepackage[hang,small,labelfont=bf,up,textfont=it,up]{caption} % Custom captions under//above floats in tables or figures
  \usepackage{booktabs} % Horizontal rules in tables
  \usepackage{float} % Required for tables and figures in the multi-column environment-they need to be placed in specific locations with the[H] (e.g. \begin{table}[H])
  \usepackage{hyperref} % For hyperlinks in the PDF
  \usepackage{lettrine} % The lettrine is the first enlarged letter at the beginning of the text
  \usepackage{paralist} % Used for the compactitem environment which makes bullet points with less space between them
  \usepackage{abstract} % Allows abstract customization
  \renewcommand{\abstractnamefont}{\normalfont\bfseries} 
  %\renewcommand{\abstracttextfont}{\normalfont\small\itshape} % Set the abstract itself to small italic text\[IndentingNewLine]
  \usepackage{titlesec} % Allows customization of titles
  \renewcommand\thesection{\Roman{section}} % Roman numerals for the sections
  \renewcommand\thesubsection{\Roman{subsection}} % Roman numerals for subsections
  \titleformat{\section}[block]{\large\scshape\centering}{\thesection.}{1em}{} % Change the look of the section titles
  \titleformat{\subsection}[block]{\large}{\thesubsection.}{1em}{} % Change the look of the section titles
  \usepackage{fancyhdr} % Headers and footers
  \pagestyle{fancy} % All pages have headers and footers
  \fancyhead{} % Blank out the default header
  \fancyfoot{} % Blank out the default footer
  \fancyhead[C]{X-meeting eXperience $\bullet$ November 2020} % Custom header text
  \fancyfoot[RO,LE]{} % Custom footer text
  %----------------------------------------------------------------------------------------
  % TITLE SECTION
  %---------------------------------------------------------------------------------------- 
 
 \title{\vspace{-15mm}\fontsize{24pt}{10pt}\selectfont\textbf{ Analysis of UVA induced mutagenesis in translesion synthesis-deficient human cells }} % Article title
  
  
  \author{ Nathalia Quintero Ruiz,  Nat\'alia Cestari Moreno,  Tiago Antonio de Souza,  Carlos Frederico Martins Menck,  Camila Corradi }
  
  \affil{ UNIVERSIDADE DE S\~AO PAULO }
  \vspace{-5mm}
  \date{}
  
  %---------------------------------------------------------------------------------------- 
  
  \begin{document}
  
  
  \maketitle % Insert title
  
  
  \thispagestyle{fancy} % All pages have headers and footers
  %----------------------------------------------------------------------------------------  
  % ABSTRACT
  
  %----------------------------------------------------------------------------------------  
  
  \begin{abstract}
  DNA is constantly subject to endogenous and exogenous agents that cause numerous types of lesions,  interfering with its transcription and replication. UVA radiation,  responsible for more than 95\% of the solar ultraviolet radiation that reaches the Earth's surface,  may cause direct and indirect DNA damage,  as pyrimidine cyclobutane dimers (CPDs) and nitrogenous base oxidation. Cells have several mechanisms capable of correcting these problems,  such as Nucleotide Excision Repair,  in addition to pathways that bypasses replication blockage caused by CPDs and oxidized bases,  as Translesion Synthesis (TLS). Deficiencies in POLH/XPV gene,  which codes for an important protein that acts in TLS,  DNA polymerase ? (pol eta),  culminates in a rare,  autosomal recessive syndrome,  Xeroderma Pigmentosum variant (XP-V). Pol eta-deficiency causes an increased frequency of skin cancer in XP-V patients because of their reduced ability to replicate sunlight-induced DNA damage. Decrease of pyrimidine dimers repair induced by UVA light indicates that this process may have a secondary and highly relevant effect on mutagenesis. Furthermore,  XP-V cells demonstrated a higher mutational occurrence even in the absence of sunlight exposure,  indicating that endogenous oxidative stress increase may be related to internal tumors in those patients. Hence,  we analyzed the whole exome sequence of XP-V fibroblasts irradiated with UVA light,  treated or not with the antioxidant N-acetylcysteine (NAC). Preliminary results indicate that NAC promoted partial protection of XP-V cells after UVA irradiation,  considering we observed an important decrease in total number of exclusive point mutations in exons and splicing regions. NAC treatment also reduced C>A transversions,  C>T transitions and CC>TT tandem mutation,  indicating that it promoted a protective effect in oxidatively induced DNA damage and it reduced mutations caused by UVA irradiation at pyrimidine dimers,  suggesting an important role of oxidative stress in cells that have a decreased capacity to remove DNA damage in the absence of pol eta.
  
  Funding:   \\
  \href{http://ab3c.org.br/xpress_pres2020/xmxp2020-297956.html}{Link to Video:}

  \end{abstract}
   
  \end{document} 