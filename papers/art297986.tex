
  \documentclass[twoside]{article}
  \usepackage[affil-it]{authblk}
  \usepackage{lipsum} % Package to generate dummy text throughout this template
  \usepackage{eurosym}
  \usepackage[sc]{mathpazo} % Use the Palatino font
  \usepackage[T1]{fontenc} % Use 8-bit encoding that has 256 glyphs
  \usepackage[utf8]{inputenc}
  \linespread{1.05} % Line spacing-Palatino needs more space between lines
  \usepackage{microtype} % Slightly tweak font spacing for aesthetics\[IndentingNewLine]
  \usepackage[hmarginratio=1:1,top=32mm,columnsep=20pt]{geometry} % Document margins
  \usepackage{multicol} % Used for the two-column layout of the document
  \usepackage[hang,small,labelfont=bf,up,textfont=it,up]{caption} % Custom captions under//above floats in tables or figures
  \usepackage{booktabs} % Horizontal rules in tables
  \usepackage{float} % Required for tables and figures in the multi-column environment-they need to be placed in specific locations with the[H] (e.g. \begin{table}[H])
  \usepackage{hyperref} % For hyperlinks in the PDF
  \usepackage{lettrine} % The lettrine is the first enlarged letter at the beginning of the text
  \usepackage{paralist} % Used for the compactitem environment which makes bullet points with less space between them
  \usepackage{abstract} % Allows abstract customization
  \renewcommand{\abstractnamefont}{\normalfont\bfseries} 
  %\renewcommand{\abstracttextfont}{\normalfont\small\itshape} % Set the abstract itself to small italic text\[IndentingNewLine]
  \usepackage{titlesec} % Allows customization of titles
  \renewcommand\thesection{\Roman{section}} % Roman numerals for the sections
  \renewcommand\thesubsection{\Roman{subsection}} % Roman numerals for subsections
  \titleformat{\section}[block]{\large\scshape\centering}{\thesection.}{1em}{} % Change the look of the section titles
  \titleformat{\subsection}[block]{\large}{\thesubsection.}{1em}{} % Change the look of the section titles
  \usepackage{fancyhdr} % Headers and footers
  \pagestyle{fancy} % All pages have headers and footers
  \fancyhead{} % Blank out the default header
  \fancyfoot{} % Blank out the default footer
  \fancyhead[C]{X-meeting eXperience $\bullet$ November 2020} % Custom header text
  \fancyfoot[RO,LE]{} % Custom footer text
  %----------------------------------------------------------------------------------------
  % TITLE SECTION
  %---------------------------------------------------------------------------------------- 
 
 \title{\vspace{-15mm}\fontsize{24pt}{10pt}\selectfont\textbf{ Environmental plasticity of Staphylococcus aureus extracellular vesicles RNA content }} % Article title
  
  
  \author{ Aur\'elie Nicolas,  Svetlana Chabelskaya,  Vin\'{\i}cius de Rezende Rodovalho,  Yves Le Loir,  Vasco A de C Azevedo,  Brice Felden,  Eric Guedon,  Brenda Silva Rosa da Luz }
  
  \affil{ INAPG- Fran\c{c}a,  UNIVERSIDADE FEDERAL DE MINAS GERAIS }
  \vspace{-5mm}
  \date{}
  
  %---------------------------------------------------------------------------------------- 
  
  \begin{document}
  
  
  \maketitle % Insert title
  
  
  \thispagestyle{fancy} % All pages have headers and footers
  %----------------------------------------------------------------------------------------  
  % ABSTRACT
  
  %----------------------------------------------------------------------------------------  
  
  \begin{abstract}
  The role of bacterial extracellular vesicles (EVs) in cell-to-cell signaling have arisen a lot of interest in the past years. These membranous spheres released by many living cells carry various bioactive macromolecules (e.g. proteins,  nucleic acids),  some of which were shown to be delivered to target cells to perform functional roles. Studies reveled that bacterial EVs charters RNA export,  probably for broad communication with surrounding bacteria,  as well as with their infected/colonized hosts. However,  data in this area is still lacking,  mainly in Gram-positive bacteria. Staphylococcus aureus is a serious human and animal pathogen that releases EVs,  nevertheless,  no studies characterizing the presence of RNA inside its EVs have been reported. Here we address if S. aureus EV cargo comprises RNAs,  and if so,  which RNAs. A high-throughput RNA sequencing approach was used to evaluate samples of producing S. aureus cells (clinical strain HG003) and its derived EVs under different in vitro conditions: early- and late-stationary phases and in presence or absence of a sub lethal concentration of vancomycin. Results showed no significant difference in the particle yields between the conditions tested,  however,  EVs released in late-stationary growth phase were approximately 55 \% larger than those from early-stationary phases. Various RNAs were identified into the S. aureus EVs,  including tRNAs,  rRNAs,  mRNAs,  and sRNAs,  which are also present in the EVs producing cells. EVs enclose mRNAs expressing virulence factors genes,  ribosomal proteins,  transcriptional regulators,  and metabolic enzymes. The sRNA RsaC implicated into the oxidative stress adaptation was also detected. The amount and nature of the RNAs detected into the purified EVs was significantly impacted according to the growth phase and the presence/absence of vancomycin,  while in a much less extent in the EVs producing cells. Finally,  differential  RNA abundance was observed among the environmental conditions tested,  and between EVs and EVs producing cells,  suggesting that not only environment shapes the RNA content,  but the packing of these molecules into EVs is a regulated process. This is the first exploratory work characterizing RNA cargo from S. aureus and its derived EVs under different conditions. Since several RNAs present inside EVs are implicated into staphylococcal virulence and survival (RNAIII,  Spa,  Sbi,  Hld,  RsaC),  our report indicate possible roles of these RNAs on bacterial-host cell communication,  virulence,  and pathogenesis of S. aureus,  paving the way for future functional studies in this area.
  
  Funding:   \\
  \href{http://ab3c.org.br/xpress_pres2020/xmxp2020-297986.html}{Link to Video:}

  \end{abstract}
   
  \end{document} 