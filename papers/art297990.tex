
  \documentclass[twoside]{article}
  \usepackage[affil-it]{authblk}
  \usepackage{lipsum} % Package to generate dummy text throughout this template
  \usepackage{eurosym}
  \usepackage[sc]{mathpazo} % Use the Palatino font
  \usepackage[T1]{fontenc} % Use 8-bit encoding that has 256 glyphs
  \usepackage[utf8]{inputenc}
  \linespread{1.05} % Line spacing-Palatino needs more space between lines
  \usepackage{microtype} % Slightly tweak font spacing for aesthetics\[IndentingNewLine]
  \usepackage[hmarginratio=1:1,top=32mm,columnsep=20pt]{geometry} % Document margins
  \usepackage{multicol} % Used for the two-column layout of the document
  \usepackage[hang,small,labelfont=bf,up,textfont=it,up]{caption} % Custom captions under//above floats in tables or figures
  \usepackage{booktabs} % Horizontal rules in tables
  \usepackage{float} % Required for tables and figures in the multi-column environment-they need to be placed in specific locations with the[H] (e.g. \begin{table}[H])
  \usepackage{hyperref} % For hyperlinks in the PDF
  \usepackage{lettrine} % The lettrine is the first enlarged letter at the beginning of the text
  \usepackage{paralist} % Used for the compactitem environment which makes bullet points with less space between them
  \usepackage{abstract} % Allows abstract customization
  \renewcommand{\abstractnamefont}{\normalfont\bfseries} 
  %\renewcommand{\abstracttextfont}{\normalfont\small\itshape} % Set the abstract itself to small italic text\[IndentingNewLine]
  \usepackage{titlesec} % Allows customization of titles
  \renewcommand\thesection{\Roman{section}} % Roman numerals for the sections
  \renewcommand\thesubsection{\Roman{subsection}} % Roman numerals for subsections
  \titleformat{\section}[block]{\large\scshape\centering}{\thesection.}{1em}{} % Change the look of the section titles
  \titleformat{\subsection}[block]{\large}{\thesubsection.}{1em}{} % Change the look of the section titles
  \usepackage{fancyhdr} % Headers and footers
  \pagestyle{fancy} % All pages have headers and footers
  \fancyhead{} % Blank out the default header
  \fancyfoot{} % Blank out the default footer
  \fancyhead[C]{X-meeting eXperience $\bullet$ November 2020} % Custom header text
  \fancyfoot[RO,LE]{} % Custom footer text
  %----------------------------------------------------------------------------------------
  % TITLE SECTION
  %---------------------------------------------------------------------------------------- 
 
 \title{\vspace{-15mm}\fontsize{24pt}{10pt}\selectfont\textbf{ Dynamics of the feedback loops required for the phenotypic stabilization in the epithelial-mesenchymal transition }} % Article title
  
  
  \author{ Jos\'e Carlos Merino Mombach,  Daner Acunha Silveira }
  
  \affil{ UNIVERSIDADE FEDERAL DE SANTA MARIA }
  \vspace{-5mm}
  \date{}
  
  %---------------------------------------------------------------------------------------- 
  
  \begin{document}
  
  
  \maketitle % Insert title
  
  
  \thispagestyle{fancy} % All pages have headers and footers
  %----------------------------------------------------------------------------------------  
  % ABSTRACT
  
  %----------------------------------------------------------------------------------------  
  
  \begin{abstract}
  The epithelial-mesenchymal transition (EMT) is a complex mechanism in which cells undergo a transition from epithelial to mesenchymal phenotypes (there is also an intermediary hybrid state) in response to microenvironmental alterations and aberrant stimuli triggered by molecules such as TGF-$\beta$. Recent studies in breast cancer progression reported new feedback loops and new participant molecules such as microRNAs 340 and 1199. In this work,  we propose a logical model of EMT contemplating the influence of these new published molecules on the regulatory core of EMT. The model results were compared with theoretical and experimental data for the human breast epithelial cell line MCF10A presenting excellent agreement. We propose that the miRNAs 340 and 1199 should be considered phenotypic stability factors of the hybrid state based on the positive feedback loops they form with ZEB1. Our results highlight new mechanisms related to the EMT dynamics in response to TGF-$\beta$ stimulus in epithelial breast cells and might help the design of therapeutic strategies for breast cancer.
  
  Funding:   \\
  \href{http://ab3c.org.br/xpress_pres2020/xmxp2020-297990.html}{Link to Video:}

  \end{abstract}
   
  \end{document} 