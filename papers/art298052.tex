
  \documentclass[twoside]{article}
  \usepackage[affil-it]{authblk}
  \usepackage{lipsum} % Package to generate dummy text throughout this template
  \usepackage{eurosym}
  \usepackage[sc]{mathpazo} % Use the Palatino font
  \usepackage[T1]{fontenc} % Use 8-bit encoding that has 256 glyphs
  \usepackage[utf8]{inputenc}
  \linespread{1.05} % Line spacing-Palatino needs more space between lines
  \usepackage{microtype} % Slightly tweak font spacing for aesthetics\[IndentingNewLine]
  \usepackage[hmarginratio=1:1,top=32mm,columnsep=20pt]{geometry} % Document margins
  \usepackage{multicol} % Used for the two-column layout of the document
  \usepackage[hang,small,labelfont=bf,up,textfont=it,up]{caption} % Custom captions under//above floats in tables or figures
  \usepackage{booktabs} % Horizontal rules in tables
  \usepackage{float} % Required for tables and figures in the multi-column environment-they need to be placed in specific locations with the[H] (e.g. \begin{table}[H])
  \usepackage{hyperref} % For hyperlinks in the PDF
  \usepackage{lettrine} % The lettrine is the first enlarged letter at the beginning of the text
  \usepackage{paralist} % Used for the compactitem environment which makes bullet points with less space between them
  \usepackage{abstract} % Allows abstract customization
  \renewcommand{\abstractnamefont}{\normalfont\bfseries} 
  %\renewcommand{\abstracttextfont}{\normalfont\small\itshape} % Set the abstract itself to small italic text\[IndentingNewLine]
  \usepackage{titlesec} % Allows customization of titles
  \renewcommand\thesection{\Roman{section}} % Roman numerals for the sections
  \renewcommand\thesubsection{\Roman{subsection}} % Roman numerals for subsections
  \titleformat{\section}[block]{\large\scshape\centering}{\thesection.}{1em}{} % Change the look of the section titles
  \titleformat{\subsection}[block]{\large}{\thesubsection.}{1em}{} % Change the look of the section titles
  \usepackage{fancyhdr} % Headers and footers
  \pagestyle{fancy} % All pages have headers and footers
  \fancyhead{} % Blank out the default header
  \fancyfoot{} % Blank out the default footer
  \fancyhead[C]{X-meeting eXperience $\bullet$ November 2020} % Custom header text
  \fancyfoot[RO,LE]{} % Custom footer text
  %----------------------------------------------------------------------------------------
  % TITLE SECTION
  %---------------------------------------------------------------------------------------- 
 
 \title{\vspace{-15mm}\fontsize{24pt}{10pt}\selectfont\textbf{ A PUTATIVE ENHANCER REGION ACTIVATED BY METFORMIN OVERLAPS WITH SNPS ASSOCIATED WITH VISTAFIN/NAMPT LEVELS }} % Article title
  
  
  \author{ L\'{\i}dia Lana Ferreira Coura,  Marcelo Rizzatti Luizon,  Daniela Alves Pereira }
  
  \affil{ UNIVERSIDADE FEDERAL DE MINAS GERAIS }
  \vspace{-5mm}
  \date{}
  
  %---------------------------------------------------------------------------------------- 
  
  \begin{document}
  
  
  \maketitle % Insert title
  
  
  \thispagestyle{fancy} % All pages have headers and footers
  %----------------------------------------------------------------------------------------  
  % ABSTRACT
  
  %----------------------------------------------------------------------------------------  
  
  \begin{abstract}
  Nicotinamide phosphoribosyltransferase (NAMPT) is an adipocytokine with a potential to be a predictive biomarker or a therapeutical target for several diseases,  such as nonalcoholic fatty liver disease,  type 2 diabetes mellitus,  and obesity. Notably,  NAMPT was shown to be activated by Metformin,  which is the first-line therapy for type 2 diabetes,  and is also used as a treatment for other diseases. The single nucleotide polymorphism (SNP) rs1319501 is located in the promoter region of NAMPT gene,  and it was found to be associated with plasma NAMPT levels. Notably,  the SNPs rs9770242 and rs61330082,  which are located ~1, 500bp upstream from the transcription start site of NAMPT gene,  were in high linkage disequilibrium with the rs1319501 in the European (CEU) population. Moreover,  rs61330082 was associated with visfatin/NAMPT levels and adverse cardiometabolic parameters in a cohort of severely obese children. However,  whether these noncoding SNPs overlap with active regulatory elements,  such as enhancers,  is unknown. Therefore,  we searched for metformin-responsive regulatory elements in the NAMPT locus,  and linked SNPs within them which may be associated with NAMPT levels. First,  we examined publicly available ChIP-seq data for active (H3K27ac) and silenced (H3K27me3) histone marks on human hepatocytes treated with metformin,  GeneHancer to identify active regulatory elements (enhancers and promoters),  and several cis-regulatory elements assignment tools from the Encyclopedia of DNA Elements (ENCODE) to identify enhancers around the NAMPT locus. Next,  we performed the functional annotation of noncoding SNPs located in the NAMPT locus using the Genotype-Tissue Expression (GTEx) project data for SNPs linked to NAMPT expression. The SNPs rs1319501,  rs9770242 and rs61330082 overlap with a metformin-responsive region enriched for the active histone mark H3K27ac upon metformin treatment,  which is located nearby an enhancer element according to GeneHancer (GH07J106288). According to GTEx,  the SNPs rs1319501,  rs9770242 and rs61330082 are eQTLs for NAMPT expression in the heart tissue. These data support that noncoding variation within a metformin-activated enhancer may increase NAMPT gene expression. However,  further studies are needed to reveal whether increased NAMPT expression may represent a beneficial effect. To understand the regulation of NAMPT expression is crucial to reveal its biological functions and the variations under physiological and pathophysiological contexts,  which could help to define NAMPT as a biomarker of disease prognosis,  a predictive or a pharmacogenetic biomarker. Our study highlights noncoding NAMPT SNPs for further functional studies,  which could help to predict NAMPT levels in patients with type 2 diabetes mellitus treated with Metformin.
  
  Funding:   \\
  \href{http://ab3c.org.br/xpress_pres2020/xmxp2020-298052.html}{Link to Video:}

  \end{abstract}
   
  \end{document} 