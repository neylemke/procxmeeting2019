
  \documentclass[twoside]{article}
  \usepackage[affil-it]{authblk}
  \usepackage{lipsum} % Package to generate dummy text throughout this template
  \usepackage{eurosym}
  \usepackage[sc]{mathpazo} % Use the Palatino font
  \usepackage[T1]{fontenc} % Use 8-bit encoding that has 256 glyphs
  \usepackage[utf8]{inputenc}
  \linespread{1.05} % Line spacing-Palatino needs more space between lines
  \usepackage{microtype} % Slightly tweak font spacing for aesthetics\[IndentingNewLine]
  \usepackage[hmarginratio=1:1,top=32mm,columnsep=20pt]{geometry} % Document margins
  \usepackage{multicol} % Used for the two-column layout of the document
  \usepackage[hang,small,labelfont=bf,up,textfont=it,up]{caption} % Custom captions under//above floats in tables or figures
  \usepackage{booktabs} % Horizontal rules in tables
  \usepackage{float} % Required for tables and figures in the multi-column environment-they need to be placed in specific locations with the[H] (e.g. \begin{table}[H])
  \usepackage{hyperref} % For hyperlinks in the PDF
  \usepackage{lettrine} % The lettrine is the first enlarged letter at the beginning of the text
  \usepackage{paralist} % Used for the compactitem environment which makes bullet points with less space between them
  \usepackage{abstract} % Allows abstract customization
  \renewcommand{\abstractnamefont}{\normalfont\bfseries} 
  %\renewcommand{\abstracttextfont}{\normalfont\small\itshape} % Set the abstract itself to small italic text\[IndentingNewLine]
  \usepackage{titlesec} % Allows customization of titles
  \renewcommand\thesection{\Roman{section}} % Roman numerals for the sections
  \renewcommand\thesubsection{\Roman{subsection}} % Roman numerals for subsections
  \titleformat{\section}[block]{\large\scshape\centering}{\thesection.}{1em}{} % Change the look of the section titles
  \titleformat{\subsection}[block]{\large}{\thesubsection.}{1em}{} % Change the look of the section titles
  \usepackage{fancyhdr} % Headers and footers
  \pagestyle{fancy} % All pages have headers and footers
  \fancyhead{} % Blank out the default header
  \fancyfoot{} % Blank out the default footer
  \fancyhead[C]{X-meeting eXperience $\bullet$ November 2020} % Custom header text
  \fancyfoot[RO,LE]{} % Custom footer text
  %----------------------------------------------------------------------------------------
  % TITLE SECTION
  %---------------------------------------------------------------------------------------- 
 
 \title{\vspace{-15mm}\fontsize{24pt}{10pt}\selectfont\textbf{ FUNCTIONAL ANNOTATION OF INTRONIC SNPs OF ARG2 GENE ASSOCIATED WITH FETAL HEMOGLOBIN LEVELS IN PATIENTS WITH SICKLE CELL ANEMIA TREATED WITH HYDROXYUREA }} % Article title
  
  
  \author{ Rahyssa Rodrigues Sales,  Marcelo Rizzatti Luizon,  B\'arbara Lisboa Nogueira }
  
  \affil{ Universidade federal de minas gerais,  UNIVERSIDADE FEDERAL DE MINAS GERAIS }
  \vspace{-5mm}
  \date{}
  
  %---------------------------------------------------------------------------------------- 
  
  \begin{document}
  
  
  \maketitle % Insert title
  
  
  \thispagestyle{fancy} % All pages have headers and footers
  %----------------------------------------------------------------------------------------  
  % ABSTRACT
  
  %----------------------------------------------------------------------------------------  
  
  \begin{abstract}
  Sickle cell anemia (SCA) a $\beta$-hemoglobin disorder. Fetal hemoglobin (HbF) ameliorates the clinical outcomes and severity that are associated to SCA. Hydroxyurea (HU) is the main drug used to treat SCA patients,  which improves their clinical course by raising HbF levels. Notably,  HU was suggested to act as a nitric oxide (NO) donor in SCA. Recently,  HU was shown to modulate NO signalling pathway in red blood cells (RBC),  RBC rheology and oxidative stress through its effects on HbF and possibly on NO bioavailability. However,  the NO-related effects of HU on RBC physiology and NO signalling pathway are not fully known. While BCL11A and HBS1L-MYB are the major loci regulating HbF levels,  other candidate genes were associated with significant changes in HbF levels in SCA patients treated with HU,  including two intronic SNPs (rs10483801 and rs10483802) of ARG2 gene. Arginase 2 was described to play a role in the regulation of extra-urea cycle arginine metabolism,  in down-regulation of NO synthesis,  and also extrahepatic arginase functions to regulate L-arginine bioavailability to nitric oxide synthase (NOS). Therefore,  we examined whether these intronic SNPs of ARG2 (rs10483801 and rs10483802) may be linked with the actual functional regulatory elements that may regulated ARG2 expression. Here,  we performed the identification of cis-regulatory elements at ARG2 locus using several assignment tools,  including The ENCyclopedia Of DNA Elements (ENCODE) ChIP-seq data for the active histone mark H3K27ac,  the ENCODE registry ofcandidate cis-regulatory elements (cCREs) using SCREEN (https://screen.encodeproject.org/). Next,  we performed the functional annotation of these intronic ARG2 SNPs using the Genotype-Tissue Expression (GTEx,  www.gtexportal.org/home/) project and the RegulomeDB (https://regulomedb.org/). Notably,  rs10483801 and rs10483802 SNPs are located ~400 bp distant in the last intron of ARG2 and overlap with H3K27ac peaks for three ENCODE cell lines,  namely K562,  NHEK and NHLF. Moreover,  they are linked to transcription factors and are located next to a region with proximal enhancer-like signature identified by the ENCODE registry of cCREs. These data support the presence of an enhancer element in the last intron of ARG2. Notably,  in SCA hemolysis results in the release and activation of arginase,  an enzyme that reciprocally regulates NO synthase activity and thus,  NO production. Considering that more than half of the patients with SCA present endothelial dysfunction caused by a local decrease of NO bioavailability,  the knowledge of genetic variants which may increase the NO levels constitutes an important alternative to reduce the complications of SCA.
  
  Funding:   \\
  \href{http://ab3c.org.br/xpress_pres2020/xmxp2020-298089.html}{Link to Video:}

  \end{abstract}
   
  \end{document} 