
  \documentclass[twoside]{article}
  \usepackage[affil-it]{authblk}
  \usepackage{lipsum} % Package to generate dummy text throughout this template
  \usepackage{eurosym}
  \usepackage[sc]{mathpazo} % Use the Palatino font
  \usepackage[T1]{fontenc} % Use 8-bit encoding that has 256 glyphs
  \usepackage[utf8]{inputenc}
  \linespread{1.05} % Line spacing-Palatino needs more space between lines
  \usepackage{microtype} % Slightly tweak font spacing for aesthetics\[IndentingNewLine]
  \usepackage[hmarginratio=1:1,top=32mm,columnsep=20pt]{geometry} % Document margins
  \usepackage{multicol} % Used for the two-column layout of the document
  \usepackage[hang,small,labelfont=bf,up,textfont=it,up]{caption} % Custom captions under//above floats in tables or figures
  \usepackage{booktabs} % Horizontal rules in tables
  \usepackage{float} % Required for tables and figures in the multi-column environment-they need to be placed in specific locations with the[H] (e.g. \begin{table}[H])
  \usepackage{hyperref} % For hyperlinks in the PDF
  \usepackage{lettrine} % The lettrine is the first enlarged letter at the beginning of the text
  \usepackage{paralist} % Used for the compactitem environment which makes bullet points with less space between them
  \usepackage{abstract} % Allows abstract customization
  \renewcommand{\abstractnamefont}{\normalfont\bfseries} 
  %\renewcommand{\abstracttextfont}{\normalfont\small\itshape} % Set the abstract itself to small italic text\[IndentingNewLine]
  \usepackage{titlesec} % Allows customization of titles
  \renewcommand\thesection{\Roman{section}} % Roman numerals for the sections
  \renewcommand\thesubsection{\Roman{subsection}} % Roman numerals for subsections
  \titleformat{\section}[block]{\large\scshape\centering}{\thesection.}{1em}{} % Change the look of the section titles
  \titleformat{\subsection}[block]{\large}{\thesubsection.}{1em}{} % Change the look of the section titles
  \usepackage{fancyhdr} % Headers and footers
  \pagestyle{fancy} % All pages have headers and footers
  \fancyhead{} % Blank out the default header
  \fancyfoot{} % Blank out the default footer
  \fancyhead[C]{X-meeting eXperience $\bullet$ November 2020} % Custom header text
  \fancyfoot[RO,LE]{} % Custom footer text
  %----------------------------------------------------------------------------------------
  % TITLE SECTION
  %---------------------------------------------------------------------------------------- 
 
 \title{\vspace{-15mm}\fontsize{24pt}{10pt}\selectfont\textbf{ ToxAnalyzer: A tool for Comparative Toxicogenomics Database (CTD)™ data analysis and visualization }} % Article title
  
  
  \author{ Diego C\'esar Batista Mariano,  Lucianna Helene Silva dos Santos,  Carlos Alberto Tagliati,  Daniel Ribeiro Rodrigues }
  
  \affil{ UNIVERSIDADE FEDERAL DE MINAS GERAIS,  IOC/Fiocruz }
  \vspace{-5mm}
  \date{}
  
  %---------------------------------------------------------------------------------------- 
  
  \begin{document}
  
  
  \maketitle % Insert title
  
  
  \thispagestyle{fancy} % All pages have headers and footers
  %----------------------------------------------------------------------------------------  
  % ABSTRACT
  
  %----------------------------------------------------------------------------------------  
  
  \begin{abstract}
  Toxicology is a field of science that has undergone changes in recent years. Using modern sequencing technologies,  it went from an exclusive animal-based science to a field of data-based decision making. Many databases store important data about toxicology experiments,  which can be useful to understand the mechanisms concerning a chemical product toxicity. One of these databases is the “Comparative Toxicogenomics Database” (CTD). CTD provides information about diseases,  genes,  compounds and their role in toxicity. However,  the volume of data from a simple search is often overwhelming to be manually and individually analyzed. The tools to analyze CTD’s data require previous installation and the knowledge of writing and manipulating programming scripts. Hence,  we developed ToxAnalyzer,  a web application to help users evaluate CTD toxicogenomic data from a chosen compound. Our application withdraws data from CTD’s servers and uses a set of Python scripts to process and display important information about any chemical compound found in CTD. Using the chemical name,  a user can gather information about the publications that report the compound,  the organisms that are involved,  gene interactions,  and other helpful plots. Our tool provides a helpful and quick overview of the complete compound data available in CTD to aid a study hypothesis. If users need more information about an interaction between a specific gene and a chemical compound,  PubMed ID links are available to access the original publications. Therefore,  ToxAnalyzer is a user-friendly tool,  since it can be accessed from anywhere,  with any device with internet access,  requiring absolutely no programming knowledge. ToxAnalyzer is available on https://tox-analyzer.com/.
  
  Funding:   \\
  \href{http://ab3c.org.br/xpress_pres2020/xmxp2020-298211.html}{Link to Video:}

  \end{abstract}
   
  \end{document} 