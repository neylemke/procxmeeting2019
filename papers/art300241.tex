
  \documentclass[twoside]{article}
  \usepackage[affil-it]{authblk}
  \usepackage{lipsum} % Package to generate dummy text throughout this template
  \usepackage{eurosym}
  \usepackage[sc]{mathpazo} % Use the Palatino font
  \usepackage[T1]{fontenc} % Use 8-bit encoding that has 256 glyphs
  \usepackage[utf8]{inputenc}
  \linespread{1.05} % Line spacing-Palatino needs more space between lines
  \usepackage{microtype} % Slightly tweak font spacing for aesthetics\[IndentingNewLine]
  \usepackage[hmarginratio=1:1,top=32mm,columnsep=20pt]{geometry} % Document margins
  \usepackage{multicol} % Used for the two-column layout of the document
  \usepackage[hang,small,labelfont=bf,up,textfont=it,up]{caption} % Custom captions under//above floats in tables or figures
  \usepackage{booktabs} % Horizontal rules in tables
  \usepackage{float} % Required for tables and figures in the multi-column environment-they need to be placed in specific locations with the[H] (e.g. \begin{table}[H])
  \usepackage{hyperref} % For hyperlinks in the PDF
  \usepackage{lettrine} % The lettrine is the first enlarged letter at the beginning of the text
  \usepackage{paralist} % Used for the compactitem environment which makes bullet points with less space between them
  \usepackage{abstract} % Allows abstract customization
  \renewcommand{\abstractnamefont}{\normalfont\bfseries} 
  %\renewcommand{\abstracttextfont}{\normalfont\small\itshape} % Set the abstract itself to small italic text\[IndentingNewLine]
  \usepackage{titlesec} % Allows customization of titles
  \renewcommand\thesection{\Roman{section}} % Roman numerals for the sections
  \renewcommand\thesubsection{\Roman{subsection}} % Roman numerals for subsections
  \titleformat{\section}[block]{\large\scshape\centering}{\thesection.}{1em}{} % Change the look of the section titles
  \titleformat{\subsection}[block]{\large}{\thesubsection.}{1em}{} % Change the look of the section titles
  \usepackage{fancyhdr} % Headers and footers
  \pagestyle{fancy} % All pages have headers and footers
  \fancyhead{} % Blank out the default header
  \fancyfoot{} % Blank out the default footer
  \fancyhead[C]{X-meeting eXperience $\bullet$ November 2020} % Custom header text
  \fancyfoot[RO,LE]{} % Custom footer text
  %----------------------------------------------------------------------------------------
  % TITLE SECTION
  %---------------------------------------------------------------------------------------- 
 
 \title{\vspace{-15mm}\fontsize{24pt}{10pt}\selectfont\textbf{ In silico strategies to identify IP3 receptors and their function in activation of the plasma membrane H(+)-ATPase in Saccharomyces cerevisiae cells }} % Article title
  
  
  \author{ Sayonara Quaresma Alves Campos,  Cleberson Marcelo dos Santos,  Rodrigo Bentes Kato,  Aureliano Claret da Cunha,  ROGELIO LOPES BRANDAO,  Izinara Rosse,  Raquel D. Penido }
  
  \affil{ UNIVERSIDADE FEDERAL DE MINAS GERAIS,  UNIVERSIDADE FEDERAL DE OURO PRETO,  UFOP- UNIVERSIDADE FEDERAL DE OURO PRETO,  UFOP }
  \vspace{-5mm}
  \date{}
  
  %---------------------------------------------------------------------------------------- 
  
  \begin{document}
  
  
  \maketitle % Insert title
  
  
  \thispagestyle{fancy} % All pages have headers and footers
  %----------------------------------------------------------------------------------------  
  % ABSTRACT
  
  %----------------------------------------------------------------------------------------  
  
  \begin{abstract}
  Saccharomyces cerevisiae is a fungal specie commonly known as yeast. It is widely used in several products such as breads,  biofuels,  enzymes production and mainly for alcoholic beverages. Moreover,  it is a model organism,  having a well characterized cellular organization,  short generation time that facilitates its replication,  a small genome with about 6000 genes and orthologous pathways of intracellular signaling to higher eukaryotes. The plasma membrane H(+)-ATPase enzyme is essential for the physiology of this fungi and for the fermentation process. It is known that in the addition of glucose,  the H(+)-ATPase may be involved in the intracellular calcium signaling,  but this pathway has not been elucidated yet. Evidence indicates that a relationship between Inositol 1, 4, 5-trisphosphate (IP3) and the calcium channel Yvc1p is mediated by a receptor,  that are related to the enzyme activations and calcium signaling in yeast. However,  these receptors have not been identified in S. cerevisiae. In this context,  this study,  aimed to identify IP3 receptors in S. cerevisiae using in silico strategies. We’ve searched for similarity between the sequences of IP3 receptors of other species already deposited in GenBank. Therefore,  the gene and protein sequences of IP3 receptors of type 1,  2 and 3 from 729 species deposited were selected using the search terms Inositol 1, 4, 5-Trisphosphate receptor type 1,  type 2,  type 3 and IP3 receptor. The gene sequences were aligned against the genome of S. cerevisiae,  strain S288C,  using the BLASTn. Additionally,  a multiple alignment of these sequences,  at gene and protein level,  was performed using the Muscle algorithm implemented in Molecular Evolutionary Genetics Analysis (MEGA) tool,  to verify the presence of regions that were conserved throughout the evolutionary process in IP3 receptors. No similar sequences to the receptors of any of the other analyzed species was observed in the yeast genome. However,  it was possible to identify from the multiple alignment,  several conserved regions that in some of the individuals are conserved domains. It was also observed that in species of fungi in which IP3 receptors have already been described,  the conserved patterns differ from the other species analyzed. The results indicate that the IP3 receptors in fungi species may be exclusive,  this is the reason that lead us to believe that,  if one confirm the presence of this receptor in S. cerevisiae,  probably will be also with an exclusive sequence. Consequently,  it is still necessary to define what is the standard conserved in the IP3 receptors in fungi species and if it is conserved in S. cerevisiae.
  
  Funding:   \\
  \href{http://ab3c.org.br/xpress_pres2020/xmxp2020-300241.html}{Link to Video:}

  \end{abstract}
   
  \end{document} 