
  \documentclass[twoside]{article}
  \usepackage[affil-it]{authblk}
  \usepackage{lipsum} % Package to generate dummy text throughout this template
  \usepackage{eurosym}
  \usepackage[sc]{mathpazo} % Use the Palatino font
  \usepackage[T1]{fontenc} % Use 8-bit encoding that has 256 glyphs
  \usepackage[utf8]{inputenc}
  \linespread{1.05} % Line spacing-Palatino needs more space between lines
  \usepackage{microtype} % Slightly tweak font spacing for aesthetics\[IndentingNewLine]
  \usepackage[hmarginratio=1:1,top=32mm,columnsep=20pt]{geometry} % Document margins
  \usepackage{multicol} % Used for the two-column layout of the document
  \usepackage[hang,small,labelfont=bf,up,textfont=it,up]{caption} % Custom captions under//above floats in tables or figures
  \usepackage{booktabs} % Horizontal rules in tables
  \usepackage{float} % Required for tables and figures in the multi-column environment-they need to be placed in specific locations with the[H] (e.g. \begin{table}[H])
  \usepackage{hyperref} % For hyperlinks in the PDF
  \usepackage{lettrine} % The lettrine is the first enlarged letter at the beginning of the text
  \usepackage{paralist} % Used for the compactitem environment which makes bullet points with less space between them
  \usepackage{abstract} % Allows abstract customization
  \renewcommand{\abstractnamefont}{\normalfont\bfseries} 
  %\renewcommand{\abstracttextfont}{\normalfont\small\itshape} % Set the abstract itself to small italic text\[IndentingNewLine]
  \usepackage{titlesec} % Allows customization of titles
  \renewcommand\thesection{\Roman{section}} % Roman numerals for the sections
  \renewcommand\thesubsection{\Roman{subsection}} % Roman numerals for subsections
  \titleformat{\section}[block]{\large\scshape\centering}{\thesection.}{1em}{} % Change the look of the section titles
  \titleformat{\subsection}[block]{\large}{\thesubsection.}{1em}{} % Change the look of the section titles
  \usepackage{fancyhdr} % Headers and footers
  \pagestyle{fancy} % All pages have headers and footers
  \fancyhead{} % Blank out the default header
  \fancyfoot{} % Blank out the default footer
  \fancyhead[C]{X-meeting eXperience $\bullet$ November 2020} % Custom header text
  \fancyfoot[RO,LE]{} % Custom footer text
  %----------------------------------------------------------------------------------------
  % TITLE SECTION
  %---------------------------------------------------------------------------------------- 
 
 \title{\vspace{-15mm}\fontsize{24pt}{10pt}\selectfont\textbf{ Insights into plant adaptations to occupy a challenging Amazonian habitat }} % Article title
  
  
  \author{ Cec\'{\i}lio Frois Caldeira,  Guilherme Oliveira,  Mariana Dias }
  
  \affil{ ITV - Instituto Tecnol\'ogico Vale,  UFMG/ITV }
  \vspace{-5mm}
  \date{}
  
  %---------------------------------------------------------------------------------------- 
  
  \begin{document}
  
  
  \maketitle % Insert title
  
  
  \thispagestyle{fancy} % All pages have headers and footers
  %----------------------------------------------------------------------------------------  
  % ABSTRACT
  
  %----------------------------------------------------------------------------------------  
  
  \begin{abstract}
  Canga substrates are a well-known example of a harsh environment for plants. Canga is the name of Brazilian ferruginous field formations,  and it refers to the ecosystems associated with superficial iron crusts typical of Minas Gerais and Par\'a. This ecosystem is marked by high temperature and UV radiation,  acidic and nutrient-depleted soils (especially phosphorus,  magnesium,  and calcium),  and high metal concentrations (such as iron and manganese),  all challenging for the establishment of plants. In the present study,  we generated the first transcriptome data from two native Fabaceae species widely distributed in the canga outcrops in the State of Par\'a,  Parkia platycephala Benth. and Stryphnodendron pulcherrimum (Willd.) Hochr,  to understand the adaptive genetic mechanisms behind the establishment of these plants in the canga environment. Transcriptomics were carried out from leaves of plants growing in canga and forest substrates collected at the Caraj\'as Mineral Province,  Par\'a,  Brazil. A combination of methods was used to recover complete and accurate transcriptomes. We achieved over 95\% of the complete single-copy genes of eudicotyledon orthologs with BUSCO evaluation. Both species had more down-regulated genes in plants growing under canga substrate compared to the forest substrate: 391 up-regulated and 723 down-regulated for P. platycephala and 264 up-regulated and 574 down-regulated for S. pulcherrimum. The gene ontology enrichment analysis showed that the enriched differentially expressed genes (DEGs) of P. platycephala are associated with response to abiotic stimulus with the up-regulation of thiazole and thiamine (vitamin B1) biosynthetic process. For S. pulcherrimum,  the DEGs are mainly associated with the up-regulation of guard cell differentiation and stomatal complex development. However,  the plants’ shared 64 up-regulated and 163 down-regulated genes,  mainly associated with the rhythmic process and polysaccharide catabolic process. The KEGG enrichment pathway analysis revealed also enriched DEGs involved in the plants’ circadian rhythm regulation,  biosynthesis of secondary metabolites,  and starch and sucrose metabolism. The differences in gene expression between P. platycephala and S. pulcherimum in canga substrates were lower than between plants in forest substrates,  suggesting a more conservative strategy when they grow in canga. Cross-species differential expression analysis was conducted using single-copy orthologues shared between species. Samples clustered by species indicated the difference in the adaptive strategies of each species. Looking at the DEGs between canga and forest,  204 gene orthologs were found to be DE between substrates. Our results reveal some insights over the adaptive convergence in the canga environment and suggested different strategies between species.
  
  Funding:   \\
  \href{http://ab3c.org.br/xpress_pres2020/xmxp2020-300276.html}{Link to Video:}

  \end{abstract}
   
  \end{document} 