
  \documentclass[twoside]{article}
  \usepackage[affil-it]{authblk}
  \usepackage{lipsum} % Package to generate dummy text throughout this template
  \usepackage{eurosym}
  \usepackage[sc]{mathpazo} % Use the Palatino font
  \usepackage[T1]{fontenc} % Use 8-bit encoding that has 256 glyphs
  \usepackage[utf8]{inputenc}
  \linespread{1.05} % Line spacing-Palatino needs more space between lines
  \usepackage{microtype} % Slightly tweak font spacing for aesthetics\[IndentingNewLine]
  \usepackage[hmarginratio=1:1,top=32mm,columnsep=20pt]{geometry} % Document margins
  \usepackage{multicol} % Used for the two-column layout of the document
  \usepackage[hang,small,labelfont=bf,up,textfont=it,up]{caption} % Custom captions under//above floats in tables or figures
  \usepackage{booktabs} % Horizontal rules in tables
  \usepackage{float} % Required for tables and figures in the multi-column environment-they need to be placed in specific locations with the[H] (e.g. \begin{table}[H])
  \usepackage{hyperref} % For hyperlinks in the PDF
  \usepackage{lettrine} % The lettrine is the first enlarged letter at the beginning of the text
  \usepackage{paralist} % Used for the compactitem environment which makes bullet points with less space between them
  \usepackage{abstract} % Allows abstract customization
  \renewcommand{\abstractnamefont}{\normalfont\bfseries} 
  %\renewcommand{\abstracttextfont}{\normalfont\small\itshape} % Set the abstract itself to small italic text\[IndentingNewLine]
  \usepackage{titlesec} % Allows customization of titles
  \renewcommand\thesection{\Roman{section}} % Roman numerals for the sections
  \renewcommand\thesubsection{\Roman{subsection}} % Roman numerals for subsections
  \titleformat{\section}[block]{\large\scshape\centering}{\thesection.}{1em}{} % Change the look of the section titles
  \titleformat{\subsection}[block]{\large}{\thesubsection.}{1em}{} % Change the look of the section titles
  \usepackage{fancyhdr} % Headers and footers
  \pagestyle{fancy} % All pages have headers and footers
  \fancyhead{} % Blank out the default header
  \fancyfoot{} % Blank out the default footer
  \fancyhead[C]{X-meeting eXperience $\bullet$ November 2020} % Custom header text
  \fancyfoot[RO,LE]{} % Custom footer text
  %----------------------------------------------------------------------------------------
  % TITLE SECTION
  %---------------------------------------------------------------------------------------- 
 
 \title{\vspace{-15mm}\fontsize{24pt}{10pt}\selectfont\textbf{ Characterization of bacteriocins in Xanthomonas citri }} % Article title
  
  
  \author{ Juan Carlos Ariute,  Flavia Figueira Aburjaile,  Ana Maria Benko-Iseppon,  Milena Gomes Cabral }
  
  \affil{ IOC/Fiocruz,  UNIVERSIDADE FEDERAL DE PERNAMBUCO }
  \vspace{-5mm}
  \date{}
  
  %---------------------------------------------------------------------------------------- 
  
  \begin{document}
  
  
  \maketitle % Insert title
  
  
  \thispagestyle{fancy} % All pages have headers and footers
  %----------------------------------------------------------------------------------------  
  % ABSTRACT
  
  %----------------------------------------------------------------------------------------  
  
  \begin{abstract}
  The resistance against antimicrobial components shows up as a growing concern worldwide and makes treating infectious diseases spread in humans,  animals,  and plants impracticable. In crops,  Xanthomonas citri is a Gram-negative bacterium that infects plants within the Citrus genus,  responsible for developing citrus canker and causing considerable economic losses. However,  this microorganism produces some secondary metabolites,  such as bacteriocins,  that appear as an alternative to elucidate the problem since they can inactivate or kill microorganism targets. Thereby,  this study aims to identify and characterize potential bacteriocins produced by X. citri. We have evaluated in silico 78 complete genomes in GenBank/NCBI database. The RAST server automatically annotated these genomes. After that,  the annotated sequences were submitted to BAGEL4 and antiSMASH to identify genomic regions containing potential bacteriocins. Results obtained from BAGEL4 showed 76 genomes presenting the zoocin A,  which was characterized as a bacteriocin-like inhibitory peptide of class III that attaches with the peptidoglycan region of some bacterias causing its lysis. The only ones that haven't shown zoocin A were Xanthomonas citri pv. mangiferaeindicae XC01 that has revealed a rhodonadin and a microcin and Xanthomonas citri pv. phaseoli fuscans CFBP7767 that hasn't shown any area of interest. Both of them have a lasso structure,  which gives them more stability. Furthermore,  the results obtained from antiSMASH revealed 22 bacteriocins and 55 lasso peptides in those genomes,  except for three: Xanthomonas citri pv. aurantifolii str. 1566,  Xanthomonas citri pv. aurantifolii FDC 1561 and Xanthomonas citri pv. vignicola CFBP7113. Therefore,  we suggest that X. citri can produce metabolites with antimicrobial activity that could be used for industrial applications and to characterize this bacteriocin repertoire in this species.
  
  Funding:  \\
  \href{http://ab3c.org.br/xpress_pres2020/xmxp2020-301064.html}{Link to Video:}

  \end{abstract}
   
  \end{document} 