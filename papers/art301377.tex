
  \documentclass[twoside]{article}
  \usepackage[affil-it]{authblk}
  \usepackage{lipsum} % Package to generate dummy text throughout this template
  \usepackage{eurosym}
  \usepackage[sc]{mathpazo} % Use the Palatino font
  \usepackage[T1]{fontenc} % Use 8-bit encoding that has 256 glyphs
  \usepackage[utf8]{inputenc}
  \linespread{1.05} % Line spacing-Palatino needs more space between lines
  \usepackage{microtype} % Slightly tweak font spacing for aesthetics\[IndentingNewLine]
  \usepackage[hmarginratio=1:1,top=32mm,columnsep=20pt]{geometry} % Document margins
  \usepackage{multicol} % Used for the two-column layout of the document
  \usepackage[hang,small,labelfont=bf,up,textfont=it,up]{caption} % Custom captions under//above floats in tables or figures
  \usepackage{booktabs} % Horizontal rules in tables
  \usepackage{float} % Required for tables and figures in the multi-column environment-they need to be placed in specific locations with the[H] (e.g. \begin{table}[H])
  \usepackage{hyperref} % For hyperlinks in the PDF
  \usepackage{lettrine} % The lettrine is the first enlarged letter at the beginning of the text
  \usepackage{paralist} % Used for the compactitem environment which makes bullet points with less space between them
  \usepackage{abstract} % Allows abstract customization
  \renewcommand{\abstractnamefont}{\normalfont\bfseries} 
  %\renewcommand{\abstracttextfont}{\normalfont\small\itshape} % Set the abstract itself to small italic text\[IndentingNewLine]
  \usepackage{titlesec} % Allows customization of titles
  \renewcommand\thesection{\Roman{section}} % Roman numerals for the sections
  \renewcommand\thesubsection{\Roman{subsection}} % Roman numerals for subsections
  \titleformat{\section}[block]{\large\scshape\centering}{\thesection.}{1em}{} % Change the look of the section titles
  \titleformat{\subsection}[block]{\large}{\thesubsection.}{1em}{} % Change the look of the section titles
  \usepackage{fancyhdr} % Headers and footers
  \pagestyle{fancy} % All pages have headers and footers
  \fancyhead{} % Blank out the default header
  \fancyfoot{} % Blank out the default footer
  \fancyhead[C]{X-meeting eXperience $\bullet$ November 2020} % Custom header text
  \fancyfoot[RO,LE]{} % Custom footer text
  %----------------------------------------------------------------------------------------
  % TITLE SECTION
  %---------------------------------------------------------------------------------------- 
 
 \title{\vspace{-15mm}\fontsize{24pt}{10pt}\selectfont\textbf{ ANTIGENIC VARIABILITY OF DENGUE VIRUS NON-STRUCTURAL PROTEIN 1 AND ITS IMPACT ON THE DEVELOPMENT OF DIAGNOSTIC MODELS }} % Article title
  
  
  \author{ RICARDO LEMES GON\c{C}ALVES,  Breno De Mello Silva,  ANA CLARA GOMES DE SOUZA,  T\'ulio C\'esar Rodrigues Leite,  UBIRATAN DA SILVA BATISTA }
  
  \affil{ UNIVERSIDADE FEDERAL DE OURO PRETO,  UFOP - UNIVERSIDADE FEDERAL  DE OURO PRETO }
  \vspace{-5mm}
  \date{}
  
  %---------------------------------------------------------------------------------------- 
  
  \begin{document}
  
  
  \maketitle % Insert title
  
  
  \thispagestyle{fancy} % All pages have headers and footers
  %----------------------------------------------------------------------------------------  
  % ABSTRACT
  
  %----------------------------------------------------------------------------------------  
  
  \begin{abstract}
  According to the Pan American Health Organization,  over 2 million cases of human infections in the Americas were caused by the Dengue virus (DENV) in 2020 alone with more than 800 deaths. The circulation of the four DENV serotypes (1-4) is still a challenge for the development of efficient diagnoses. The available tests target the structural proteins (E,  M,  and C) and non-structural protein 1 (NS1). However,  global variations in the primary sequences of these proteins are countless and hinder the development of effective vaccines and diagnostic tests. Among these proteins,  NS1 has been increasingly using as a target for new diagnostic methods and vaccines,  albeit very variable. It is secreted into the extracellular medium in a hexameric form and can be detected at high levels in the bloodstream turning it available to the immune system. The present work mapped the variability of circulating NS1 protein main domains in the American continent and evaluated its impact on antigenic regions through in-silico approaches. 132 coding sequences of Dengue virus NS1 protein (39 from DENV1,  33 from DENV2,  34 from DENV3,  and 26 DENV4) from South,  Central,  and North Americas were analyzed. These sequences were aligned using the MEGA X software through the Muscle algorithm. Next,  the variant amino acid (aa) residues were mapped in the exposed wing and $\beta$-ladder domains. This analysis identified 78 residues of aa variants in the NS1 protein for DENV whose 34.61\% of residues show a prevalence higher than 10\%. The Bepipred - 2.0,  NetCTL,  and NetMHCII tools were used to map antigenic regions. Eighty linear B cell,  cytotoxic T,  and T - Helper epitopes were predicted for the 4 serotypes. 65.6\% of the 32 predicted B cell epitopes,  33.3\% of the 18 cytotoxic T cell epitopes,  and 43.3\% of the 30 T-helper epitopes showed amino acid residue variations. Although there is little information in the literature on how these variations may impact the accuracy of tests on the market,  several NS1 variations occur in predicted antigenic regions. Such a situation can be greatly underestimated due to the small amount of sequencing carried out in underdeveloped tropical countries. Together,  these findings suggest that further studies are needed to better understand the distribution and influence of NS1 variability in DENV diagnostic tests and raise a red flag about NS1-based tests to other Flaviviruses.
  
  Funding: UFOP,  CNPQ e CAPES \\
  \href{http://ab3c.org.br/xpress_pres2020/xmxp2020-301377.html}{Link to Video:}

  \end{abstract}
   
  \end{document} 