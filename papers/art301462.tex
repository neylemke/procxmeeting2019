
  \documentclass[twoside]{article}
  \usepackage[affil-it]{authblk}
  \usepackage{lipsum} % Package to generate dummy text throughout this template
  \usepackage{eurosym}
  \usepackage[sc]{mathpazo} % Use the Palatino font
  \usepackage[T1]{fontenc} % Use 8-bit encoding that has 256 glyphs
  \usepackage[utf8]{inputenc}
  \linespread{1.05} % Line spacing-Palatino needs more space between lines
  \usepackage{microtype} % Slightly tweak font spacing for aesthetics\[IndentingNewLine]
  \usepackage[hmarginratio=1:1,top=32mm,columnsep=20pt]{geometry} % Document margins
  \usepackage{multicol} % Used for the two-column layout of the document
  \usepackage[hang,small,labelfont=bf,up,textfont=it,up]{caption} % Custom captions under//above floats in tables or figures
  \usepackage{booktabs} % Horizontal rules in tables
  \usepackage{float} % Required for tables and figures in the multi-column environment-they need to be placed in specific locations with the[H] (e.g. \begin{table}[H])
  \usepackage{hyperref} % For hyperlinks in the PDF
  \usepackage{lettrine} % The lettrine is the first enlarged letter at the beginning of the text
  \usepackage{paralist} % Used for the compactitem environment which makes bullet points with less space between them
  \usepackage{abstract} % Allows abstract customization
  \renewcommand{\abstractnamefont}{\normalfont\bfseries} 
  %\renewcommand{\abstracttextfont}{\normalfont\small\itshape} % Set the abstract itself to small italic text\[IndentingNewLine]
  \usepackage{titlesec} % Allows customization of titles
  \renewcommand\thesection{\Roman{section}} % Roman numerals for the sections
  \renewcommand\thesubsection{\Roman{subsection}} % Roman numerals for subsections
  \titleformat{\section}[block]{\large\scshape\centering}{\thesection.}{1em}{} % Change the look of the section titles
  \titleformat{\subsection}[block]{\large}{\thesubsection.}{1em}{} % Change the look of the section titles
  \usepackage{fancyhdr} % Headers and footers
  \pagestyle{fancy} % All pages have headers and footers
  \fancyhead{} % Blank out the default header
  \fancyfoot{} % Blank out the default footer
  \fancyhead[C]{X-meeting eXperience $\bullet$ November 2020} % Custom header text
  \fancyfoot[RO,LE]{} % Custom footer text
  %----------------------------------------------------------------------------------------
  % TITLE SECTION
  %---------------------------------------------------------------------------------------- 
 
 \title{\vspace{-15mm}\fontsize{24pt}{10pt}\selectfont\textbf{ TEN COMPLETE MITOCHONDRIAL GENOMES OF GYMNOCHARACINI (STETHAPRIONINAE,  CHARACIFORMES) }} % Article title
  
  
  \author{ Iuri Batista da Silva,  Rubens Pasa,  Fabiano Menegidio,  Igor Henrique Rodrigues-Oliveira,  Dina\'{\i}za Abadia Rocha-Reis,  John Seymour (Pat) Heslop-Harrison,  Trude Schwarzacher,  Karine Frehner Kavalco,  Matheus Lewi Cruz Bonaccorsi de Campos }
  
  \affil{ UNIVERSIDADE FEDERAL DE MINAS GERAIS,  UNIVERSIDADE DE MOGI DAS CRUZES }
  \vspace{-5mm}
  \date{}
  
  %---------------------------------------------------------------------------------------- 
  
  \begin{document}
  
  
  \maketitle % Insert title
  
  
  \thispagestyle{fancy} % All pages have headers and footers
  %----------------------------------------------------------------------------------------  
  % ABSTRACT
  
  %----------------------------------------------------------------------------------------  
  
  \begin{abstract}
  Stethaprioninae is a subfamily of characiform fish that comprises small animals popularly known as tetras. Some species of the genus,  such as Astyanax,  share several common features that difficult their recognition,  leading to efforts to identify diagnostic characteristics or molecular signatures for the group. In an attempt to contribute to these efforts,  we are presenting eight new complete mitogenomes of species/cytotypes from Neotropical Ecozone belonging to the Astyanax and Psalidodon genus: A. aeneus,  A. altiparanae,  P. fasciatus (from two basins),  A. lacustris,  P. rivularis (two cytotypes) and P. rioparanaibano. Total genomic DNA was extracted from liver and heart samples of six species. The Whole Genome Sequencing from these species was performed in a Novaseq 6000. We assembled the mitogenomes from raw reads on Novoplasty v3.7 in a parallel cluster computer using the mitogenome of P. paranae from GenBank as seed. We annotated the obtained sequences on MitoAnnotator (MitoFish). In the Galaxy platform,  we accessed the quality of raw reads (using FastQC) and filtered with Fastp tool. For broader comparisons,  we also assembled the mitogenome of two species with raw reads available on European Nucleotide Archive: P. fasciatus from Upper Parana river basin and A. aeneus from Mexico. We perform comparative genomics analysis by BLAST comparison of all Astyanax/Psalidodon mitochondrial genomes against a reference (P. paranae) generated by Blast Ring Image Generator. Our results have shown that all mitogenomes content and gene order were identical,  with 13 protein-coding genes (PCGs),  22 tRNA genes and two rRNA genes,  following an expected order according to already described Characiformes mitogenomes. All PCGs and tRNAs are on the heavy chain,  except the Nd6 gene and eight tRNAs. The length of mitochondrial sequence range from 16, 626bp in P. fasciatus to 16, 812bp in P. rivularis. The average length of D-loop was 1, 061bp. Deepening the knowledge about the D-loop,  can play a fundamental role in understanding the evolutionary history of the Astyanax and Psalidodon genera. In this work,  we observed that the size variation between different Astyanax/Psalidodon mitogenomes occurs mainly due to the extension of the D-loop. In conclusion,  our methodology used in the reconstruction of the mitochondrial genome proved to be satisfactory and able to access the length of this type of genome,  plus the composition and nature of the D-loop,  solving possible gaps in previous methodologies.
  
  Funding:   \\
  \href{http://ab3c.org.br/xpress_pres2020/xmxp2020-301462.html}{Link to Video:}

  \end{abstract}
   
  \end{document} 