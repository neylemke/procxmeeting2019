
  \documentclass[twoside]{article}
  \usepackage[affil-it]{authblk}
  \usepackage{lipsum} % Package to generate dummy text throughout this template
  \usepackage{eurosym}
  \usepackage[sc]{mathpazo} % Use the Palatino font
  \usepackage[T1]{fontenc} % Use 8-bit encoding that has 256 glyphs
  \usepackage[utf8]{inputenc}
  \linespread{1.05} % Line spacing-Palatino needs more space between lines
  \usepackage{microtype} % Slightly tweak font spacing for aesthetics\[IndentingNewLine]
  \usepackage[hmarginratio=1:1,top=32mm,columnsep=20pt]{geometry} % Document margins
  \usepackage{multicol} % Used for the two-column layout of the document
  \usepackage[hang,small,labelfont=bf,up,textfont=it,up]{caption} % Custom captions under//above floats in tables or figures
  \usepackage{booktabs} % Horizontal rules in tables
  \usepackage{float} % Required for tables and figures in the multi-column environment-they need to be placed in specific locations with the[H] (e.g. \begin{table}[H])
  \usepackage{hyperref} % For hyperlinks in the PDF
  \usepackage{lettrine} % The lettrine is the first enlarged letter at the beginning of the text
  \usepackage{paralist} % Used for the compactitem environment which makes bullet points with less space between them
  \usepackage{abstract} % Allows abstract customization
  \renewcommand{\abstractnamefont}{\normalfont\bfseries} 
  %\renewcommand{\abstracttextfont}{\normalfont\small\itshape} % Set the abstract itself to small italic text\[IndentingNewLine]
  \usepackage{titlesec} % Allows customization of titles
  \renewcommand\thesection{\Roman{section}} % Roman numerals for the sections
  \renewcommand\thesubsection{\Roman{subsection}} % Roman numerals for subsections
  \titleformat{\section}[block]{\large\scshape\centering}{\thesection.}{1em}{} % Change the look of the section titles
  \titleformat{\subsection}[block]{\large}{\thesubsection.}{1em}{} % Change the look of the section titles
  \usepackage{fancyhdr} % Headers and footers
  \pagestyle{fancy} % All pages have headers and footers
  \fancyhead{} % Blank out the default header
  \fancyfoot{} % Blank out the default footer
  \fancyhead[C]{X-meeting eXperience $\bullet$ November 2020} % Custom header text
  \fancyfoot[RO,LE]{} % Custom footer text
  %----------------------------------------------------------------------------------------
  % TITLE SECTION
  %---------------------------------------------------------------------------------------- 
 
 \title{\vspace{-15mm}\fontsize{24pt}{10pt}\selectfont\textbf{ IMMUNOINFORMATICS APPROACHS TO DESIGN A CHIMERIC MULTI-EPITOPE VACCINE AGAINST MYCOPLASMA PNEUMONIAE }} % Article title
  
  
  \author{ Vasco A de C Azevedo,  Arun Kumar Jaiswal,  Helioswilton Sales de Campos,  Rodrigo Bentes Kato,  Stephane Fraga de Oliveira Tosta,  Siomar de Castro Soares,  sandeep tiwari,  Marta Giovanetti,  Tha\'{\i}s Cristina Vilela Rodrigues }
  
  \affil{ UFMG - UNIVERSIDADE FEDERAL DE MINAS GERAIS/ UP- Universidade do Porto,  UNIVERSIDADE FEDERAL DE MINAS GERAIS,  INAPG- Fran\c{c}a }
  \vspace{-5mm}
  \date{}
  
  %---------------------------------------------------------------------------------------- 
  
  \begin{document}
  
  
  \maketitle % Insert title
  
  
  \thispagestyle{fancy} % All pages have headers and footers
  %----------------------------------------------------------------------------------------  
  % ABSTRACT
  
  %----------------------------------------------------------------------------------------  
  
  \begin{abstract}
  Pneumonia is a serious health problem with global effects,  being the death cause of over one million people annually. Among the main microorganisms responsible by pneumonia,  Mycoplasma pneumoniae is one of the most common,  with a significant increase in the last years. The vaccines are fundamental in diseases prevention besides to considerably avoid the need of health services and funding resources. In this way,  the proposal of the present study is to construct multi-epitope vaccine against M. pneumoniae through immunoinformatics approach. Multi-epitope vaccines are constructed by epitopes properly selected to induce targeted immune responses and avoid adverse reactions. Therefore,  seven reported vaccine candidates were selected based on the reverse vaccinology approach from one of our published research article and three vaccine candidates through a literature search. Afterwards the search for MHCI,  MHCII and B epitopes were performed. Furthermore,  the overlapping epitopes,  capable to induce both humoral and cellular responses were identified. Those epitopes were filtered according to their immunogenicity,  and population coverage. The epitopes with best features were joined with classical peptide linkers and the heat-labile enterotoxin from Escherichia coli as adjuvant,  then,  the structure of the vaccine was predicted. The vaccine was considered physically stable,  non-toxic,  non-allergen,  not significantly similar to human proteome and with appropriate antigenic and immunogenic properties. The molecular docking of the vaccine with the Toll-Like Receptor 2 (TLR2) was performed and the dynamic simulation will be executed to ensure the affinity and stability between these complexes. In silico cloning was tested in an expression vector with positive results. In addition,  the immune simulation for vaccine efficacy was also tested with promising findings. Through immunoinformatic approaches we constructed an effective multi-epitope vaccine candidate,  that with further tests could contribute to prevention of pneumonia in a massive scale. Besides that,  the study assists to better understanding of the immune mechanisms regarding M. pneumoniae infections and its interaction with the host.
  
  Funding:   \\
  \href{http://ab3c.org.br/xpress_pres2020/xmxp2020-302914.html}{Link to Video:}

  \end{abstract}
   
  \end{document} 