
  \documentclass[twoside]{article}
  \usepackage[affil-it]{authblk}
  \usepackage{lipsum} % Package to generate dummy text throughout this template
  \usepackage{eurosym}
  \usepackage[sc]{mathpazo} % Use the Palatino font
  \usepackage[T1]{fontenc} % Use 8-bit encoding that has 256 glyphs
  \usepackage[utf8]{inputenc}
  \linespread{1.05} % Line spacing-Palatino needs more space between lines
  \usepackage{microtype} % Slightly tweak font spacing for aesthetics\[IndentingNewLine]
  \usepackage[hmarginratio=1:1,top=32mm,columnsep=20pt]{geometry} % Document margins
  \usepackage{multicol} % Used for the two-column layout of the document
  \usepackage[hang,small,labelfont=bf,up,textfont=it,up]{caption} % Custom captions under//above floats in tables or figures
  \usepackage{booktabs} % Horizontal rules in tables
  \usepackage{float} % Required for tables and figures in the multi-column environment-they need to be placed in specific locations with the[H] (e.g. \begin{table}[H])
  \usepackage{hyperref} % For hyperlinks in the PDF
  \usepackage{lettrine} % The lettrine is the first enlarged letter at the beginning of the text
  \usepackage{paralist} % Used for the compactitem environment which makes bullet points with less space between them
  \usepackage{abstract} % Allows abstract customization
  \renewcommand{\abstractnamefont}{\normalfont\bfseries} 
  %\renewcommand{\abstracttextfont}{\normalfont\small\itshape} % Set the abstract itself to small italic text\[IndentingNewLine]
  \usepackage{titlesec} % Allows customization of titles
  \renewcommand\thesection{\Roman{section}} % Roman numerals for the sections
  \renewcommand\thesubsection{\Roman{subsection}} % Roman numerals for subsections
  \titleformat{\section}[block]{\large\scshape\centering}{\thesection.}{1em}{} % Change the look of the section titles
  \titleformat{\subsection}[block]{\large}{\thesubsection.}{1em}{} % Change the look of the section titles
  \usepackage{fancyhdr} % Headers and footers
  \pagestyle{fancy} % All pages have headers and footers
  \fancyhead{} % Blank out the default header
  \fancyfoot{} % Blank out the default footer
  \fancyhead[C]{X-meeting eXperience $\bullet$ November 2020} % Custom header text
  \fancyfoot[RO,LE]{} % Custom footer text
  %----------------------------------------------------------------------------------------
  % TITLE SECTION
  %---------------------------------------------------------------------------------------- 
 
 \title{\vspace{-15mm}\fontsize{24pt}{10pt}\selectfont\textbf{ Multi-omics-based identification of SARS-CoV-2 infection biology and candidate drugs against COVID-19 }} % Article title
  
  
  \author{ Debmalya Barh,  Marianna E. Ivanova,  Vasco A de C Azevedo,  Arist\'oteles G\'oes-Neto,  M. Michael Gromiha,  Preetam Ghosh,  sandeep tiwari }
  
  \affil{ INAPG- Fran\c{c}a,  UNIVERSIDADE FEDERAL DE MINAS GERAIS }
  \vspace{-5mm}
  \date{}
  
  %---------------------------------------------------------------------------------------- 
  
  \begin{document}
  
  
  \maketitle % Insert title
  
  
  \thispagestyle{fancy} % All pages have headers and footers
  %----------------------------------------------------------------------------------------  
  % ABSTRACT
  
  %----------------------------------------------------------------------------------------  
  
  \begin{abstract}
  SARS-CoV-2 has ushered a global pandemic with no effective drug being available at present. Although several FDA-approved drugs are currently under clinical trials for drug repositioning,  there is an on-going global effort for new drug identification. Here,  using multiomics (interactome,  proteome,  transcriptome,  and bibliome) experimental data and subsequent integrated analysis,  we present the biological events associated with SARS-CoV-2 infection and identified several candidate drugs against this viral disease. We found that: (i) Interactome-based infection pathways differ from the other three omics-based profiles. (ii) viral process,  mRNA splicing,  cytokine and interferon signaling,  and ubiquitin mediated proteolysis are important pathways in SARS-CoV-2 infection. (iii) SARS-CoV-2 infection also shares pathways with Influenza A,  Epstein-Barr virus,  HTLV-I,  Measles,  and Hepatitis virus. (iv) Further,  bacterial,  parasitic,  and protozoan infection pathways such as Tuberculosis,  Malaria,  and Leishmaniasis are also shared by this virus. (v) A total of 50 candidate drugs including the prophylaxis agents and pathway specific inhibitors are identified against SARS-CoV-2 infection. (vi) Anticancer antibiotics,  steroids,  Estrogen,  analgesics,  antipsychotic drugs,  anticholesteremics,  antihemophilic factors,  and immunosuppressants are the key drug categories. (vii) Ozone,  Nitric oxide,  and photosensitizer drugs are also identified as possible therapeutic candidates. (viii) Curcumin,  Retinoic acids,  Vitamin D,  Arsenic,  Copper,  and Zinc may be the candidate prophylaxis agents. Nearly 80\% of our identified agents are suggested to have anti-COVID-19 effects or under clinical trials. Our identified drugs,  that are not yet tested,  need validation with caution while an appropriate drug combination from these candidate drugs along with a SARS-CoV-2 specific antiviral agent is needed for effective COVID-19 treatment.
  
  Funding:   \\
  \href{http://ab3c.org.br/xpress_pres2020/xmxp2020-302928.html}{Link to Video:}

  \end{abstract}
   
  \end{document} 