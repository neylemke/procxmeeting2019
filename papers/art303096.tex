
  \documentclass[twoside]{article}
  \usepackage[affil-it]{authblk}
  \usepackage{lipsum} % Package to generate dummy text throughout this template
  \usepackage{eurosym}
  \usepackage[sc]{mathpazo} % Use the Palatino font
  \usepackage[T1]{fontenc} % Use 8-bit encoding that has 256 glyphs
  \usepackage[utf8]{inputenc}
  \linespread{1.05} % Line spacing-Palatino needs more space between lines
  \usepackage{microtype} % Slightly tweak font spacing for aesthetics\[IndentingNewLine]
  \usepackage[hmarginratio=1:1,top=32mm,columnsep=20pt]{geometry} % Document margins
  \usepackage{multicol} % Used for the two-column layout of the document
  \usepackage[hang,small,labelfont=bf,up,textfont=it,up]{caption} % Custom captions under//above floats in tables or figures
  \usepackage{booktabs} % Horizontal rules in tables
  \usepackage{float} % Required for tables and figures in the multi-column environment-they need to be placed in specific locations with the[H] (e.g. \begin{table}[H])
  \usepackage{hyperref} % For hyperlinks in the PDF
  \usepackage{lettrine} % The lettrine is the first enlarged letter at the beginning of the text
  \usepackage{paralist} % Used for the compactitem environment which makes bullet points with less space between them
  \usepackage{abstract} % Allows abstract customization
  \renewcommand{\abstractnamefont}{\normalfont\bfseries} 
  %\renewcommand{\abstracttextfont}{\normalfont\small\itshape} % Set the abstract itself to small italic text\[IndentingNewLine]
  \usepackage{titlesec} % Allows customization of titles
  \renewcommand\thesection{\Roman{section}} % Roman numerals for the sections
  \renewcommand\thesubsection{\Roman{subsection}} % Roman numerals for subsections
  \titleformat{\section}[block]{\large\scshape\centering}{\thesection.}{1em}{} % Change the look of the section titles
  \titleformat{\subsection}[block]{\large}{\thesubsection.}{1em}{} % Change the look of the section titles
  \usepackage{fancyhdr} % Headers and footers
  \pagestyle{fancy} % All pages have headers and footers
  \fancyhead{} % Blank out the default header
  \fancyfoot{} % Blank out the default footer
  \fancyhead[C]{X-meeting eXperience $\bullet$ November 2020} % Custom header text
  \fancyfoot[RO,LE]{} % Custom footer text
  %----------------------------------------------------------------------------------------
  % TITLE SECTION
  %---------------------------------------------------------------------------------------- 
 
 \title{\vspace{-15mm}\fontsize{24pt}{10pt}\selectfont\textbf{ Comparative midgut transcriptome analysis of Helicoverpa armigera feeding on natural conditions }} % Article title
  
  
  \author{ Natalia Faraj Murad,  Karina Lucas da Silva Brand\~ao,  Fernando Luis C\^onsoli,  Celso Omoto,  Marcelo Mendes Brand\~ao,  Andr\'e Ricardo Oliveira Conson }
  
  \affil{ CBMEG/UNICAMP,  UNIVERSIDADE ESTADUAL DE CAMPINAS,  UNIVERSIDADE ESTADUAL DE CAMPINAS }
  \vspace{-5mm}
  \date{}
  
  %---------------------------------------------------------------------------------------- 
  
  \begin{document}
  
  
  \maketitle % Insert title
  
  
  \thispagestyle{fancy} % All pages have headers and footers
  %----------------------------------------------------------------------------------------  
  % ABSTRACT
  
  %----------------------------------------------------------------------------------------  
  
  \begin{abstract}
  The cultivation of different annual crops may provide ideal conditions for feeding and survival of lepidopteran pests presenting generalist feeding habits. With occurrence in Brazil since 2013,  one of the most important species from Noctuidae family is Helicoverpa armigera. It has a high capacity to tolerate many insecticides. Both private companies and public policy makers seeks to develop an integrated pest management approach to reduce the usage of pesticides and the emergence of populations. In addition,  the main enzymes responsible for the digestive process in insects are peptidases involved in the initial digestion of plant proteins. Thus,  to develop efficient ways to control pests,  it is mandatory first to know which genes are involved in the digestive and detoxifying processes. Helicoverpa armigera individuals feeding on natural conditions were collected in order to characterize differentially expressed transcripts associated with soybean,  cotton and bean diets. Total RNA from midgut was extracted and cDNA libraries sequenced (paired-end) using an Illumina Hiseq 2500. A de novo assembly of the short reads using both Mira and Trinity resulted in 145, 284 transcripts (1000 bp N50) and a length of 98.63 Mb. We identified 31 transcripts differentially expressed between dietary conditions. The largest number of differentially expressed transcripts was obtained in the cotton versus soybean contrast,  where 26 transcripts were found up-regulated in cotton diet. From these,  11 transcripts were also found to be up-regulated in cotton diet relative to bean. Functional analysis showed that these transcripts are involved in biological processes like proteolysis,  electron transport chain and lipid catabolic process. This is the first study of H. armigera transcriptome feeding under natural conditions and assembled transcripts are a powerful resource for future research promoting an improved understanding of the gene regulation of digestive peptidases.
  
  Funding: Processo FAPESP 2018/19461-1 \\
  \href{http://ab3c.org.br/xpress_pres2020/xmxp2020-303096.html}{Link to Video:}

  \end{abstract}
   
  \end{document} 