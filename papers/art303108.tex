
  \documentclass[twoside]{article}
  \usepackage[affil-it]{authblk}
  \usepackage{lipsum} % Package to generate dummy text throughout this template
  \usepackage{eurosym}
  \usepackage[sc]{mathpazo} % Use the Palatino font
  \usepackage[T1]{fontenc} % Use 8-bit encoding that has 256 glyphs
  \usepackage[utf8]{inputenc}
  \linespread{1.05} % Line spacing-Palatino needs more space between lines
  \usepackage{microtype} % Slightly tweak font spacing for aesthetics\[IndentingNewLine]
  \usepackage[hmarginratio=1:1,top=32mm,columnsep=20pt]{geometry} % Document margins
  \usepackage{multicol} % Used for the two-column layout of the document
  \usepackage[hang,small,labelfont=bf,up,textfont=it,up]{caption} % Custom captions under//above floats in tables or figures
  \usepackage{booktabs} % Horizontal rules in tables
  \usepackage{float} % Required for tables and figures in the multi-column environment-they need to be placed in specific locations with the[H] (e.g. \begin{table}[H])
  \usepackage{hyperref} % For hyperlinks in the PDF
  \usepackage{lettrine} % The lettrine is the first enlarged letter at the beginning of the text
  \usepackage{paralist} % Used for the compactitem environment which makes bullet points with less space between them
  \usepackage{abstract} % Allows abstract customization
  \renewcommand{\abstractnamefont}{\normalfont\bfseries} 
  %\renewcommand{\abstracttextfont}{\normalfont\small\itshape} % Set the abstract itself to small italic text\[IndentingNewLine]
  \usepackage{titlesec} % Allows customization of titles
  \renewcommand\thesection{\Roman{section}} % Roman numerals for the sections
  \renewcommand\thesubsection{\Roman{subsection}} % Roman numerals for subsections
  \titleformat{\section}[block]{\large\scshape\centering}{\thesection.}{1em}{} % Change the look of the section titles
  \titleformat{\subsection}[block]{\large}{\thesubsection.}{1em}{} % Change the look of the section titles
  \usepackage{fancyhdr} % Headers and footers
  \pagestyle{fancy} % All pages have headers and footers
  \fancyhead{} % Blank out the default header
  \fancyfoot{} % Blank out the default footer
  \fancyhead[C]{X-meeting eXperience $\bullet$ November 2020} % Custom header text
  \fancyfoot[RO,LE]{} % Custom footer text
  %----------------------------------------------------------------------------------------
  % TITLE SECTION
  %---------------------------------------------------------------------------------------- 
 
 \title{\vspace{-15mm}\fontsize{24pt}{10pt}\selectfont\textbf{ How much primer choice affects the perceived biodiversity? A case study of bat diet according to three COI gene regions. }} % Article title
  
  
  \author{ Renato Renison Moreira Oliveira,  Enrico Bernard,  Ronnie Alves,  Eder Soares Pires,  Ang\'elico Fortunato Asenjo Flores,  Gisele Nunes Lopes,  Guilherme Oliveira,  Marcele Laux }
  
  \affil{ ITV - Instituto Tecnol\'ogico Vale,  UNIVERSIDADE FEDERAL DE MINAS GERAIS }
  \vspace{-5mm}
  \date{}
  
  %---------------------------------------------------------------------------------------- 
  
  \begin{document}
  
  
  \maketitle % Insert title
  
  
  \thispagestyle{fancy} % All pages have headers and footers
  %----------------------------------------------------------------------------------------  
  % ABSTRACT
  
  %----------------------------------------------------------------------------------------  
  
  \begin{abstract}
  With the advent of the metabarcoding strategy,  the study of diet components and niche overlap has reached an unprecedented taxonomic resolution,  especially concerning the Arthropoda phylum. The cytochrome c oxidase subunit I (COI),  widely adopted as marker for Metazoa,  presents distinct variation levels at distinct gene regions,  making the primer choice essential to amplification success of targeted taxa. The bat diet recovered from guano eDNA amplified by three distinct COI primer pairs was compared between each primer pair dataset in terms of taxa recovery,  Arthropoda taxonomic coverage,  and community structure response to the spatial component according to diversity indexes and beta-diversity patterns. The guano samples were collected in eight caves located in Amazon biome and one cave sampled along one year in Caatinga biome,  totalizing 13 sample units. The P23 short primer pair (130bp) recovered mostly Chiroptera reads (56\%),  presented the lowest Arthropoda recovery,  but the most even proportion of Arthropoda orders. The P34 primer pair,  which amplifies the longest and more variable amplicons (370bp),  recovered the largest proportion of Arthropoda reads (57\%),  73\% of it recruited by Diptera and Lepidoptera orders. The P56 primer pair,  amplifying a 350bp COI amplicon originally described as the most conserved region,  showed the narrowest taxonomic coverage,  with over 95\% of Arthropoda reads assigned to Lepidoptera. Both P34 and P56 long amplicons recovered the largest portion of reads from Caatinga cave,  showed better Arthropoda recovery and resolution than P23,  and presented a nested beta-diversity structure,  probably as a result from the narrow taxonomic coverage and the time-series sampling design. Even so,  the three primer pair datasets shared similar compositional responses to spatial scale,  with most part of the variance explained by sample units,  followed by regions and biomes. Since there was a spatial-ecological shared response between the datasets,  mostly between P34 and P56,  their Arthropoda community of reads could be complementary,  potentially improving the niche breadth detected. The amplicon variability observed was substantially different from described by the literature,  reinforcing the critical importance of local ecological and biological attributes in primer choice and expected outcome.
  
  Funding:   \\
  \href{http://ab3c.org.br/xpress_pres2020/xmxp2020-303108.html}{Link to Video:}

  \end{abstract}
   
  \end{document} 