
  \documentclass[twoside]{article}
  \usepackage[affil-it]{authblk}
  \usepackage{lipsum} % Package to generate dummy text throughout this template
  \usepackage{eurosym}
  \usepackage[sc]{mathpazo} % Use the Palatino font
  \usepackage[T1]{fontenc} % Use 8-bit encoding that has 256 glyphs
  \usepackage[utf8]{inputenc}
  \linespread{1.05} % Line spacing-Palatino needs more space between lines
  \usepackage{microtype} % Slightly tweak font spacing for aesthetics\[IndentingNewLine]
  \usepackage[hmarginratio=1:1,top=32mm,columnsep=20pt]{geometry} % Document margins
  \usepackage{multicol} % Used for the two-column layout of the document
  \usepackage[hang,small,labelfont=bf,up,textfont=it,up]{caption} % Custom captions under//above floats in tables or figures
  \usepackage{booktabs} % Horizontal rules in tables
  \usepackage{float} % Required for tables and figures in the multi-column environment-they need to be placed in specific locations with the[H] (e.g. \begin{table}[H])
  \usepackage{hyperref} % For hyperlinks in the PDF
  \usepackage{lettrine} % The lettrine is the first enlarged letter at the beginning of the text
  \usepackage{paralist} % Used for the compactitem environment which makes bullet points with less space between them
  \usepackage{abstract} % Allows abstract customization
  \renewcommand{\abstractnamefont}{\normalfont\bfseries} 
  %\renewcommand{\abstracttextfont}{\normalfont\small\itshape} % Set the abstract itself to small italic text\[IndentingNewLine]
  \usepackage{titlesec} % Allows customization of titles
  \renewcommand\thesection{\Roman{section}} % Roman numerals for the sections
  \renewcommand\thesubsection{\Roman{subsection}} % Roman numerals for subsections
  \titleformat{\section}[block]{\large\scshape\centering}{\thesection.}{1em}{} % Change the look of the section titles
  \titleformat{\subsection}[block]{\large}{\thesubsection.}{1em}{} % Change the look of the section titles
  \usepackage{fancyhdr} % Headers and footers
  \pagestyle{fancy} % All pages have headers and footers
  \fancyhead{} % Blank out the default header
  \fancyfoot{} % Blank out the default footer
  \fancyhead[C]{X-meeting eXperience $\bullet$ November 2020} % Custom header text
  \fancyfoot[RO,LE]{} % Custom footer text
  %----------------------------------------------------------------------------------------
  % TITLE SECTION
  %---------------------------------------------------------------------------------------- 
 
 \title{\vspace{-15mm}\fontsize{24pt}{10pt}\selectfont\textbf{ Metformin regulates cells epigenomic landscape leading to decreased proliferation and  inflammation in hepatocytes }} % Article title
  
  
  \author{ Lucio Rezende Queiroz,  Marcelo Rizzatti Luizon,  Paula Silva Matos,  Gloria Regina Franco,  Izabela Mamede Costa Andrade da Concei\c{c}\~ao }
  
  \affil{ UFMG,  UNIVERSIDADE FEDERAL DE MINAS GERAIS,  UFMG - Departamento de Bioqu\'{\i}mica e Imunologia,  UNIVERSIDADE FEDERAL DE MINAS GERAIS }
  \vspace{-5mm}
  \date{}
  
  %---------------------------------------------------------------------------------------- 
  
  \begin{document}
  
  
  \maketitle % Insert title
  
  
  \thispagestyle{fancy} % All pages have headers and footers
  %----------------------------------------------------------------------------------------  
  % ABSTRACT
  
  %----------------------------------------------------------------------------------------  
  
  \begin{abstract}
  Metformin is the first-line oral therapy for type 2 diabetes. It has been approved for use in other diseases,  like polycystic ovary syndrome,  obesity,  and promising clinical trials for cancer. However,  metformin's mechanism remains to be entirely elucidated,  with its potent anti-aging,  anti-carcinogenic,  and epigenetic-regulator effects commonly seen but not thoroughly explained.  
We propose a mechanism for metformin’s beneficial effect on inflammation and proliferation,  hyper and down expressing transcripts which act as potent epigenetic regulators. We analyzed high-throughput RNA-seq data of primary human hepatocytes with the standard laboratory pipeline. Then,  we selected transcripts that acted as epigenomic regulators according to our functional enrichment analysis and queried their translated sequences for the presence of whole domains.
From all differentially expressed transcripts (DETs),  six were present in epigenetic regulation pathways,  four upregulated and two downregulated,  all being protein-coding transcripts with their active domains present. 
The four upregulated DETs belong to the histone lysine demethylase (KDM) subfamily and use a JumonjiC (JmjC) domain that converts a-ketoglutarate (a-KG) to succinate during the demethylation process. High levels of succinate inhibit a-KG conversion,  and metformin is known to reduce intracellular succinate levels,  leading to increased activity of KDMs. At the gene-level,  our candidate KDMs were also up-regulated in hepatocellular carcinoma (HCC),  which does not comply with metformin proposed effects of reducing inflammation and proliferation. Articles that linked KDMs super expression to proliferation did not analyze at the isoform-level,  and a previous study showed that KDM isoforms that retain the JmjC domain show an anti-carcinogenic effect. Contrastingly,  the ones which increase proliferation are short isoforms that lost the JmjC domain. Therefore,  according to our data Metformin would upregulate the anti-carcinogenic KDMs.
The two other DETs were downregulated isoforms of the Methionine Adenosyltransferase 2A (MAT2A). MAT2A is the leading synthesizer of S-adenosylmethionine (SAM),  the main cellular donator of methyl groups. On Liver it is mostly expressed in extra-hepatic tissues,  while its paralogue,  MAT1A,  is present in hepatocytes. A switch in MAT1A:MAT2A ratio in hepatocytes is positively correlated to liver diseases,  such as HCC and fibrosis. Metformin downregulating MAT2A leads to more presence of MAT1A,  which increases SAM,  leading to more methyl groups available.
Our findings point towards a robust epigenetic regulatory axis controlled by isoform-specific differential expression induced by metformin. This mechanism leads to new understanding of metformin’s role in the hepatic microenvironment and new ways in which those pathways can be targeted in hepatic disorders.
  
  Funding:   \\
  \href{http://ab3c.org.br/xpress_pres2020/xmxp2020-303133.html}{Link to Video:}

  \end{abstract}
   
  \end{document} 