
  \documentclass[twoside]{article}
  \usepackage[affil-it]{authblk}
  \usepackage{lipsum} % Package to generate dummy text throughout this template
  \usepackage{eurosym}
  \usepackage[sc]{mathpazo} % Use the Palatino font
  \usepackage[T1]{fontenc} % Use 8-bit encoding that has 256 glyphs
  \usepackage[utf8]{inputenc}
  \linespread{1.05} % Line spacing-Palatino needs more space between lines
  \usepackage{microtype} % Slightly tweak font spacing for aesthetics\[IndentingNewLine]
  \usepackage[hmarginratio=1:1,top=32mm,columnsep=20pt]{geometry} % Document margins
  \usepackage{multicol} % Used for the two-column layout of the document
  \usepackage[hang,small,labelfont=bf,up,textfont=it,up]{caption} % Custom captions under//above floats in tables or figures
  \usepackage{booktabs} % Horizontal rules in tables
  \usepackage{float} % Required for tables and figures in the multi-column environment-they need to be placed in specific locations with the[H] (e.g. \begin{table}[H])
  \usepackage{hyperref} % For hyperlinks in the PDF
  \usepackage{lettrine} % The lettrine is the first enlarged letter at the beginning of the text
  \usepackage{paralist} % Used for the compactitem environment which makes bullet points with less space between them
  \usepackage{abstract} % Allows abstract customization
  \renewcommand{\abstractnamefont}{\normalfont\bfseries} 
  %\renewcommand{\abstracttextfont}{\normalfont\small\itshape} % Set the abstract itself to small italic text\[IndentingNewLine]
  \usepackage{titlesec} % Allows customization of titles
  \renewcommand\thesection{\Roman{section}} % Roman numerals for the sections
  \renewcommand\thesubsection{\Roman{subsection}} % Roman numerals for subsections
  \titleformat{\section}[block]{\large\scshape\centering}{\thesection.}{1em}{} % Change the look of the section titles
  \titleformat{\subsection}[block]{\large}{\thesubsection.}{1em}{} % Change the look of the section titles
  \usepackage{fancyhdr} % Headers and footers
  \pagestyle{fancy} % All pages have headers and footers
  \fancyhead{} % Blank out the default header
  \fancyfoot{} % Blank out the default footer
  \fancyhead[C]{X-meeting eXperience $\bullet$ November 2020} % Custom header text
  \fancyfoot[RO,LE]{} % Custom footer text
  %----------------------------------------------------------------------------------------
  % TITLE SECTION
  %---------------------------------------------------------------------------------------- 
 
 \title{\vspace{-15mm}\fontsize{24pt}{10pt}\selectfont\textbf{ Expression Analysis Integration with Inferencing of Gene Regulatory Networks }} % Article title
  
  
  \author{ Fabr\'{\i}cio Martins Lopes,  David Menotti,  Juliana Costa-Silva }
  
  \affil{ UNIVERSIDADE FEDERAL DO PARAN\'A,  Universidade Tecnol\'ogica Federal do Paran\'a (UTFPR) }
  \vspace{-5mm}
  \date{}
  
  %---------------------------------------------------------------------------------------- 
  
  \begin{document}
  
  
  \maketitle % Insert title
  
  
  \thispagestyle{fancy} % All pages have headers and footers
  %----------------------------------------------------------------------------------------  
  % ABSTRACT
  
  %----------------------------------------------------------------------------------------  
  
  \begin{abstract}
  Application of next generation sequencing technology in cDNA sequencing (RNA-Seq),  in transcriptomic studies has become suitable for transcript discovery,  depicting mechanisms of gene regulation and differential gene expression analysis. Efforts to understand the complex network of reactions and influences that regulate the functioning of organisms involve an interdisciplinary research,  in particular the development of computational models. As advances in generating biological data take place,  computing plays a fundamental role in the processing of biological data. In this context,  several works were produced with the objective of identifying the interaction networks between genes and their functionalities in the most diverse organisms.
This work proposes the development of an interactive software tool analysis of RNA-Seq data. This software tool will create a stream to infer gene regulatory networks,  based on differential expression analysis output,  guaranteeing the order and a pattern of process. Besides,  it will allow the user to choose among different techniques of analysis for each processing step,  resulting in a useful tool for RNA-Seq analysis. It has adopted the Python language programming for development in order to make possible its use in an online page and make easy maintenance. The data analysis provided by the proposed approach is composed by three main steps: i) Expression analysis: BaySeq,  edgeR,  DESeq,  DESeq2,  EBSeq,  limma-voom,  SAMSeq,  sleuth or consexpression can be used; ii) Network Inference: will use the results of the expression analysis to generate the inference of gene networks with RNA-Seq data,  based on the premise that the generated networks are of the small world and scale-free type,  considering the relationships indicated in previous studies. iii) Network validation and integration: To evaluate the characteristics of the generated genetic networks,  KEGG tool and the ontology of this KO tool will be used. The association of the results with a functional annotation results in a complete analysis,  which responds to a need for research in the area.
We expect to produce a software tool for RNA-Seq data analysis and interpretation,  as well as the spread of these analysis techniques. We consider that access to this tool can facilitate transcriptomic studies and help simplify its use in Bioinformatics training or in classes about RNA-Seq analysis.
  
  Funding: CAPES \\
  \href{http://ab3c.org.br/xpress_pres2020/xmxp2020-303167.html}{Link to Video:}

  \end{abstract}
   
  \end{document} 