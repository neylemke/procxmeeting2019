
  \documentclass[twoside]{article}
  \usepackage[affil-it]{authblk}
  \usepackage{lipsum} % Package to generate dummy text throughout this template
  \usepackage{eurosym}
  \usepackage[sc]{mathpazo} % Use the Palatino font
  \usepackage[T1]{fontenc} % Use 8-bit encoding that has 256 glyphs
  \usepackage[utf8]{inputenc}
  \linespread{1.05} % Line spacing-Palatino needs more space between lines
  \usepackage{microtype} % Slightly tweak font spacing for aesthetics\[IndentingNewLine]
  \usepackage[hmarginratio=1:1,top=32mm,columnsep=20pt]{geometry} % Document margins
  \usepackage{multicol} % Used for the two-column layout of the document
  \usepackage[hang,small,labelfont=bf,up,textfont=it,up]{caption} % Custom captions under//above floats in tables or figures
  \usepackage{booktabs} % Horizontal rules in tables
  \usepackage{float} % Required for tables and figures in the multi-column environment-they need to be placed in specific locations with the[H] (e.g. \begin{table}[H])
  \usepackage{hyperref} % For hyperlinks in the PDF
  \usepackage{lettrine} % The lettrine is the first enlarged letter at the beginning of the text
  \usepackage{paralist} % Used for the compactitem environment which makes bullet points with less space between them
  \usepackage{abstract} % Allows abstract customization
  \renewcommand{\abstractnamefont}{\normalfont\bfseries} 
  %\renewcommand{\abstracttextfont}{\normalfont\small\itshape} % Set the abstract itself to small italic text\[IndentingNewLine]
  \usepackage{titlesec} % Allows customization of titles
  \renewcommand\thesection{\Roman{section}} % Roman numerals for the sections
  \renewcommand\thesubsection{\Roman{subsection}} % Roman numerals for subsections
  \titleformat{\section}[block]{\large\scshape\centering}{\thesection.}{1em}{} % Change the look of the section titles
  \titleformat{\subsection}[block]{\large}{\thesubsection.}{1em}{} % Change the look of the section titles
  \usepackage{fancyhdr} % Headers and footers
  \pagestyle{fancy} % All pages have headers and footers
  \fancyhead{} % Blank out the default header
  \fancyfoot{} % Blank out the default footer
  \fancyhead[C]{X-meeting eXperience $\bullet$ November 2020} % Custom header text
  \fancyfoot[RO,LE]{} % Custom footer text
  %----------------------------------------------------------------------------------------
  % TITLE SECTION
  %---------------------------------------------------------------------------------------- 
 
 \title{\vspace{-15mm}\fontsize{24pt}{10pt}\selectfont\textbf{ Effects of Alignment Characteristics on Distance-based Tree Reconstruction }} % Article title
  
  
  \author{ Roberto Tadeu Raittz,  Jeroniza Nunes  Marchaukoski,  Dieval Guizelini,  ALEXANDRE GORI DE CASTILHO,  Guilherme Taborda Ribas }
  
  \affil{ UNIVERSIDADE FEDERAL DO PARAN\'A }
  \vspace{-5mm}
  \date{}
  
  %---------------------------------------------------------------------------------------- 
  
  \begin{document}
  
  
  \maketitle % Insert title
  
  
  \thispagestyle{fancy} % All pages have headers and footers
  %----------------------------------------------------------------------------------------  
  % ABSTRACT
  
  %----------------------------------------------------------------------------------------  
  
  \begin{abstract}
  The analysis and comparison of nucleotide and amino acid sequences are the cornerstones of computational biology and bioinformatics studies. Trees are important tools for comparative studies between gene sequences,  proteins and genomes. As well as in studies of phylogeny,  protein homology and taxonomy based on molecular characteristics. The constructions of the trees are directly affected by the multiple alignment algorithms,  by the measures of distances between the sequences and in the research ambiguity of the interpretation of the distance values,  observed in the comparison of the sequences composition and the evolutionary events. In this work,  we revisited classical hierarchical clustering methods,  applied to nucleotide and amino acid sequences,  to measure and identify the effects on the resulting dendrogram when one varies the size and number of aligned sequences. And,  also,  to verify whether any of those methods can reproduce the reference tree - or at least close for some alignment variation. The Average-linkage is widely studied and has known problems of consistency in phylogenetic reconstruction,  especially when the distance matrix is not time corrected with evolutive models. However,  with the nowadays softwares for simulating theoretical alignments and trees,  it is  possible drawing massive experiments to verify how the input data can affect the final tree obtained,  and whether there are variations that can identify ranges where distance and linkage methods can be consistent. In the completed stages,  840 data sets were simulated and,  from different clustering methods,  5, 880 trees were generated. The data sets were obtained by the Cartesian combination between size and sequence quantity. Where sizes ranged from 10 to 10, 000 bases,  in nucleotide and amino acid sequences and; sequence amounts ranged from 25 to 100 sequences. We used the geodesic distance method to do trees comparisons. Among the methods evaluated for nucleotide sequences,  the best methods are: single,  complete,  average and weighted. While for amino acid sequences,  the highlighted methods are: median and centroid. We conclude that the larger the sequence size,  the greater the consensus of the trees produced and,  fairly often,  the average-linkage method is not the most suitable for the reconstruction trees when compared to the other methods covered in this study. And,  at this point in the research,  none of these clustering methods reproduces the reference tree.
  
  Funding: Capes \\
  \href{http://ab3c.org.br/xpress_pres2020/xmxp2020-303178.html}{Link to Video:}

  \end{abstract}
   
  \end{document} 