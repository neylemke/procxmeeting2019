
  \documentclass[twoside]{article}
  \usepackage[affil-it]{authblk}
  \usepackage{lipsum} % Package to generate dummy text throughout this template
  \usepackage{eurosym}
  \usepackage[sc]{mathpazo} % Use the Palatino font
  \usepackage[T1]{fontenc} % Use 8-bit encoding that has 256 glyphs
  \usepackage[utf8]{inputenc}
  \linespread{1.05} % Line spacing-Palatino needs more space between lines
  \usepackage{microtype} % Slightly tweak font spacing for aesthetics\[IndentingNewLine]
  \usepackage[hmarginratio=1:1,top=32mm,columnsep=20pt]{geometry} % Document margins
  \usepackage{multicol} % Used for the two-column layout of the document
  \usepackage[hang,small,labelfont=bf,up,textfont=it,up]{caption} % Custom captions under//above floats in tables or figures
  \usepackage{booktabs} % Horizontal rules in tables
  \usepackage{float} % Required for tables and figures in the multi-column environment-they need to be placed in specific locations with the[H] (e.g. \begin{table}[H])
  \usepackage{hyperref} % For hyperlinks in the PDF
  \usepackage{lettrine} % The lettrine is the first enlarged letter at the beginning of the text
  \usepackage{paralist} % Used for the compactitem environment which makes bullet points with less space between them
  \usepackage{abstract} % Allows abstract customization
  \renewcommand{\abstractnamefont}{\normalfont\bfseries} 
  %\renewcommand{\abstracttextfont}{\normalfont\small\itshape} % Set the abstract itself to small italic text\[IndentingNewLine]
  \usepackage{titlesec} % Allows customization of titles
  \renewcommand\thesection{\Roman{section}} % Roman numerals for the sections
  \renewcommand\thesubsection{\Roman{subsection}} % Roman numerals for subsections
  \titleformat{\section}[block]{\large\scshape\centering}{\thesection.}{1em}{} % Change the look of the section titles
  \titleformat{\subsection}[block]{\large}{\thesubsection.}{1em}{} % Change the look of the section titles
  \usepackage{fancyhdr} % Headers and footers
  \pagestyle{fancy} % All pages have headers and footers
  \fancyhead{} % Blank out the default header
  \fancyfoot{} % Blank out the default footer
  \fancyhead[C]{X-meeting eXperience $\bullet$ November 2020} % Custom header text
  \fancyfoot[RO,LE]{} % Custom footer text
  %----------------------------------------------------------------------------------------
  % TITLE SECTION
  %---------------------------------------------------------------------------------------- 
 
 \title{\vspace{-15mm}\fontsize{24pt}{10pt}\selectfont\textbf{ LEAGUE OF BRAZILIAN BIOINFORMATICS: CHALLENGES TO PROMOTE SCIENTIFIC TRAINING }} % Article title
  
  
  \author{ Lucas Miguel de Carvalho,  Raquel Riyuzo de Almeida Franco,  Elvira C Alves Horacio,  Maira Rodrigues de Camargo Neves,  Nilson Coimbra,  Mayla Abrahim Costa,  Flavia Figueira Aburjaile,  Sheila Tiemi Nagamatsu,  Neli Jose da Fonseca Junior }
  
  \affil{ UNIVERSIDADE ESTADUAL DE CAMPINAS,  FIOCRUZ - IOC,  RSG BRAZIL,  EMBL-EBI,  UNIVERSIDADE FEDERAL DE MINAS GERAIS,  CTBE/CNPEM,  IOC/Fiocruz,  UNIVERSIDADE DE S\~AO PAULO }
  \vspace{-5mm}
  \date{}
  
  %---------------------------------------------------------------------------------------- 
  
  \begin{document}
  
  
  \maketitle % Insert title
  
  
  \thispagestyle{fancy} % All pages have headers and footers
  %----------------------------------------------------------------------------------------  
  % ABSTRACT
  
  %----------------------------------------------------------------------------------------  
  
  \begin{abstract}
  The scientific training to become a bioinformatician includes multidisciplinary abilities such as biology,  mathematics,  statistics,  biochemistry,  and computer science. Besides,  it requires the development of soft-skills such as teamwork,  scientific communication,  resilience,  critical thinking,  and research. In order to improve and promote the ongoing training of the Brazilian bioinformatics community,  we organize a national competition,  with the main goal to develop human resources and abilities in computational biology at the international level.  The competition framework was designed in three phases,  and the competitors had to organize themselves in groups (until 2-3 participants). The first phase was a one-day challenge with 60 multiple choice questions of Biology,  Computer Science,  and Bioinformatics. In the second phase,  they were challenged to solve 5five programming questions in 5 days. While the third phase included the development of an original project evaluated during the 15th X-meeting. The first edition of the Brazilian League of Bioinformatics (LBB) counted with 168 competitors and 59 groups. We reached 76\% out of 26 Brazilian States. The participants were majority men (67.2\%) from the southeast of Brazil (around 55\%). Also,  they were distributed into undergraduate students (14.4\%),  graduate students (12.6\% master and 16.8\%,  Ph.D.),  and professionals in the field. The first phase selected 46 teams to proceed in the competition,  while the second phase selected the three top-performing teams. The Brazilian League of Bioinformatics included mostly multi-diverse groups,  however,  the finals were composed only of men. During the competition we were able to stimulate teamwork in the main areas of bioinformatics,  with the engagement of all research-level competitors. Furthermore,  we identified opportunities to deliver and offer better training to the community and we intend to apply the acquired experience in the second edition of the LBB,  which will occur in 2021.
  
  Funding:   \\
  \href{http://ab3c.org.br/xpress_pres2020/xmxp2020-303431.html}{Link to Video:}

  \end{abstract}
   
  \end{document} 