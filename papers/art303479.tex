
  \documentclass[twoside]{article}
  \usepackage[affil-it]{authblk}
  \usepackage{lipsum} % Package to generate dummy text throughout this template
  \usepackage{eurosym}
  \usepackage[sc]{mathpazo} % Use the Palatino font
  \usepackage[T1]{fontenc} % Use 8-bit encoding that has 256 glyphs
  \usepackage[utf8]{inputenc}
  \linespread{1.05} % Line spacing-Palatino needs more space between lines
  \usepackage{microtype} % Slightly tweak font spacing for aesthetics\[IndentingNewLine]
  \usepackage[hmarginratio=1:1,top=32mm,columnsep=20pt]{geometry} % Document margins
  \usepackage{multicol} % Used for the two-column layout of the document
  \usepackage[hang,small,labelfont=bf,up,textfont=it,up]{caption} % Custom captions under//above floats in tables or figures
  \usepackage{booktabs} % Horizontal rules in tables
  \usepackage{float} % Required for tables and figures in the multi-column environment-they need to be placed in specific locations with the[H] (e.g. \begin{table}[H])
  \usepackage{hyperref} % For hyperlinks in the PDF
  \usepackage{lettrine} % The lettrine is the first enlarged letter at the beginning of the text
  \usepackage{paralist} % Used for the compactitem environment which makes bullet points with less space between them
  \usepackage{abstract} % Allows abstract customization
  \renewcommand{\abstractnamefont}{\normalfont\bfseries} 
  %\renewcommand{\abstracttextfont}{\normalfont\small\itshape} % Set the abstract itself to small italic text\[IndentingNewLine]
  \usepackage{titlesec} % Allows customization of titles
  \renewcommand\thesection{\Roman{section}} % Roman numerals for the sections
  \renewcommand\thesubsection{\Roman{subsection}} % Roman numerals for subsections
  \titleformat{\section}[block]{\large\scshape\centering}{\thesection.}{1em}{} % Change the look of the section titles
  \titleformat{\subsection}[block]{\large}{\thesubsection.}{1em}{} % Change the look of the section titles
  \usepackage{fancyhdr} % Headers and footers
  \pagestyle{fancy} % All pages have headers and footers
  \fancyhead{} % Blank out the default header
  \fancyfoot{} % Blank out the default footer
  \fancyhead[C]{X-meeting eXperience $\bullet$ November 2020} % Custom header text
  \fancyfoot[RO,LE]{} % Custom footer text
  %----------------------------------------------------------------------------------------
  % TITLE SECTION
  %---------------------------------------------------------------------------------------- 
 
 \title{\vspace{-15mm}\fontsize{24pt}{10pt}\selectfont\textbf{ Comparative analysis of genetic networks of Anticarsia gemmatalis H\"ubner,  1818 (Lepidoptera: Erebidae) obtained from transcriptomes of strains resistant and susceptible to the protein Cry of Bacillus thuringiensis }} % Article title
  
  
  \author{ Laurival Ant\^onio Vilas Boas,  Fabr\'{\i}cio Martins Lopes,  Rog\'erio Fernandes de Souza,  Freddy Eddinson Ninaja Zegarra }
  
  \affil{ Universidade Tecnol\'ogica Federal do Paran\'a (UTFPR) }
  \vspace{-5mm}
  \date{}
  
  %---------------------------------------------------------------------------------------- 
  
  \begin{document}
  
  
  \maketitle % Insert title
  
  
  \thispagestyle{fancy} % All pages have headers and footers
  %----------------------------------------------------------------------------------------  
  % ABSTRACT
  
  %----------------------------------------------------------------------------------------  
  
  \begin{abstract}
  The larvae of the pest Anticarsia gemmatalis H\"ubner,  1818 (Lepidoptera: Erebidae) are capable of generating serious economic losses in the soybean industry and,  although efficient methods are known to combat them,  such as chemicals,  less aggressive techniques are chosen to human health. Biological technology that uses organisms or by-products of biocontrollers,  such as Cry toxins from Bacillus thuringiensis,  is capable of acting efficiently and specifically on the insect pest without generating serious environmental impacts. However,  cases of resistance of A. gemmatalis to Cry toxin have been reported. Therefore,  the objective of this project is to perform a comparative analysis of genetic networks of transcripts of A. gemmatalis obtained from a resistant colony and another sensitive to the Cry protein of B. thuringiensis subsp. Kurstaki Lineage HD-73. The project is based on the data set of the work of Forim et al.,  2017,  which reports information on transcriptomes of A. gemmatalis resistant and susceptible to this toxin. Each treatment and control will be carried out in triplicate for resistant and susceptible,  resulting in 12 experimental units. Our consensus software are adopted to analyze differentially expressed genes. The produced data set (active and inhibited genes) are performed adopting the DimReduction software to infer the genetic networks. Subsequently,  the metabolic pathways corresponding to the target genes are investigated using the computational tools of the KEGG database. As a result,  some relevant genes in the process of resistance and susceptibility to Cry protein are identified,  as well as the metabolic pathways in which they participate.
  
  Funding:   \\
  \href{http://ab3c.org.br/xpress_pres2020/xmxp2020-303479.html}{Link to Video:}

  \end{abstract}
   
  \end{document} 