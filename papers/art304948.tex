
  \documentclass[twoside]{article}
  \usepackage[affil-it]{authblk}
  \usepackage{lipsum} % Package to generate dummy text throughout this template
  \usepackage{eurosym}
  \usepackage[sc]{mathpazo} % Use the Palatino font
  \usepackage[T1]{fontenc} % Use 8-bit encoding that has 256 glyphs
  \usepackage[utf8]{inputenc}
  \linespread{1.05} % Line spacing-Palatino needs more space between lines
  \usepackage{microtype} % Slightly tweak font spacing for aesthetics\[IndentingNewLine]
  \usepackage[hmarginratio=1:1,top=32mm,columnsep=20pt]{geometry} % Document margins
  \usepackage{multicol} % Used for the two-column layout of the document
  \usepackage[hang,small,labelfont=bf,up,textfont=it,up]{caption} % Custom captions under//above floats in tables or figures
  \usepackage{booktabs} % Horizontal rules in tables
  \usepackage{float} % Required for tables and figures in the multi-column environment-they need to be placed in specific locations with the[H] (e.g. \begin{table}[H])
  \usepackage{hyperref} % For hyperlinks in the PDF
  \usepackage{lettrine} % The lettrine is the first enlarged letter at the beginning of the text
  \usepackage{paralist} % Used for the compactitem environment which makes bullet points with less space between them
  \usepackage{abstract} % Allows abstract customization
  \renewcommand{\abstractnamefont}{\normalfont\bfseries} 
  %\renewcommand{\abstracttextfont}{\normalfont\small\itshape} % Set the abstract itself to small italic text\[IndentingNewLine]
  \usepackage{titlesec} % Allows customization of titles
  \renewcommand\thesection{\Roman{section}} % Roman numerals for the sections
  \renewcommand\thesubsection{\Roman{subsection}} % Roman numerals for subsections
  \titleformat{\section}[block]{\large\scshape\centering}{\thesection.}{1em}{} % Change the look of the section titles
  \titleformat{\subsection}[block]{\large}{\thesubsection.}{1em}{} % Change the look of the section titles
  \usepackage{fancyhdr} % Headers and footers
  \pagestyle{fancy} % All pages have headers and footers
  \fancyhead{} % Blank out the default header
  \fancyfoot{} % Blank out the default footer
  \fancyhead[C]{X-meeting eXperience $\bullet$ November 2020} % Custom header text
  \fancyfoot[RO,LE]{} % Custom footer text
  %----------------------------------------------------------------------------------------
  % TITLE SECTION
  %---------------------------------------------------------------------------------------- 
 
 \title{\vspace{-15mm}\fontsize{24pt}{10pt}\selectfont\textbf{ CHANGES INTHE EXPRESSION PROFILE OF TRANSCRIPT ISOFORMS IN MICE HEARTS CAUSED BY TRYPANOSOMA CRUZI INFECTION }} % Article title
  
  
  \author{ Raphael Tavares da Silva,  Tiago Bruno Rezende de Castro,  Stellamaris Soares,  Carlos Renato,  Andr\'ea Mara Macedo,  Gloria Regina Franco,  Nayara Evelin de Toledo }
  
  \affil{ UNIVERSIDADE FEDERAL DE MINAS GERAIS,  UFMG - Departamento de Bioqu\'{\i}mica e Imunologia,  UNIVERSIDADE FEDERAL DE MINAS GERAIS }
  \vspace{-5mm}
  \date{}
  
  %---------------------------------------------------------------------------------------- 
  
  \begin{document}
  
  
  \maketitle % Insert title
  
  
  \thispagestyle{fancy} % All pages have headers and footers
  %----------------------------------------------------------------------------------------  
  % ABSTRACT
  
  %----------------------------------------------------------------------------------------  
  
  \begin{abstract}
  Since the description of Chagas disease,  caused by the protozoan parasite Trypanosoma cruzi,  the reasons for its different clinical manifestations have yet to be completely revealed. Our group has previously shown that different strains of T. cruzi (JG- T.cruzi II and Col1.7G2-T. cruzi I) had a differential tissue distribution in BALB/c mice upon infection. Studies of differential gene expression,  seeking to elucidate which host genes could be involved in this phenomenon,  evaluated RNA-Seq data of BALB/c hearts infected with either JG,  Col1.7G2 or a mixture of both strains and showed that Col1.7G2 is a stronger activator of the immune response,  while JG effectively downregulates the oxidative stress response,  basal metabolism and protein translation in the host. The mixture-infected group showed both profiles simultaneously. In this study we aimed to quantify differential alternative splicing,  as well as identify differentially expressed transcript isoforms from the same transcriptomic data. By comparing Col1.7G2 infected mice with control group,  we identified a total of 594 differentially expressed transcripts,  including 543 upregulated and 51 downregulated. By comparing JG with control group,  a total of 901 transcripts were considered differentially expressed,  including 256 upregulated and 645 downregulated. Despite the greater number of protein coding biotypes,  many non-coding transcripts from protein coding genes were found.  Functional enrichment analysis showed that Col1.7G2 induced a higher inflammatory response while JG exhibit a weaker activation of immune response genes. Furthermore,  JG-infected mice showed a reduction in expression of genes responsible for energetic metabolism,  mitochondrial oxidative phosphorylation,  and protein synthesis,  corroborating our previous studies of gene expression analysis. Increase in splicing events were observed,  including a rise in the number of skipping exons,  retained introns and usage of alternate 5’ and 3’ splice sites. Our future steps include to correlate alternative transcriptome changes with expressed proteins identified by mass spectrometry.
  
  Funding:   \\
  \href{http://ab3c.org.br/xpress_pres2020/xmxp2020-304948.html}{Link to Video:}

  \end{abstract}
   
  \end{document} 