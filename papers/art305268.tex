
  \documentclass[twoside]{article}
  \usepackage[affil-it]{authblk}
  \usepackage{lipsum} % Package to generate dummy text throughout this template
  \usepackage{eurosym}
  \usepackage[sc]{mathpazo} % Use the Palatino font
  \usepackage[T1]{fontenc} % Use 8-bit encoding that has 256 glyphs
  \usepackage[utf8]{inputenc}
  \linespread{1.05} % Line spacing-Palatino needs more space between lines
  \usepackage{microtype} % Slightly tweak font spacing for aesthetics\[IndentingNewLine]
  \usepackage[hmarginratio=1:1,top=32mm,columnsep=20pt]{geometry} % Document margins
  \usepackage{multicol} % Used for the two-column layout of the document
  \usepackage[hang,small,labelfont=bf,up,textfont=it,up]{caption} % Custom captions under//above floats in tables or figures
  \usepackage{booktabs} % Horizontal rules in tables
  \usepackage{float} % Required for tables and figures in the multi-column environment-they need to be placed in specific locations with the[H] (e.g. \begin{table}[H])
  \usepackage{hyperref} % For hyperlinks in the PDF
  \usepackage{lettrine} % The lettrine is the first enlarged letter at the beginning of the text
  \usepackage{paralist} % Used for the compactitem environment which makes bullet points with less space between them
  \usepackage{abstract} % Allows abstract customization
  \renewcommand{\abstractnamefont}{\normalfont\bfseries} 
  %\renewcommand{\abstracttextfont}{\normalfont\small\itshape} % Set the abstract itself to small italic text\[IndentingNewLine]
  \usepackage{titlesec} % Allows customization of titles
  \renewcommand\thesection{\Roman{section}} % Roman numerals for the sections
  \renewcommand\thesubsection{\Roman{subsection}} % Roman numerals for subsections
  \titleformat{\section}[block]{\large\scshape\centering}{\thesection.}{1em}{} % Change the look of the section titles
  \titleformat{\subsection}[block]{\large}{\thesubsection.}{1em}{} % Change the look of the section titles
  \usepackage{fancyhdr} % Headers and footers
  \pagestyle{fancy} % All pages have headers and footers
  \fancyhead{} % Blank out the default header
  \fancyfoot{} % Blank out the default footer
  \fancyhead[C]{X-meeting eXperience $\bullet$ November 2020} % Custom header text
  \fancyfoot[RO,LE]{} % Custom footer text
  %----------------------------------------------------------------------------------------
  % TITLE SECTION
  %---------------------------------------------------------------------------------------- 
 
 \title{\vspace{-15mm}\fontsize{24pt}{10pt}\selectfont\textbf{ Role of functional polymorphisms in the p53 pathway genes and the genetic susceptibility to Congenital ZIKV Syndrome }} % Article title
  
  
  \author{ Eduarda Sgarioni,  Igor Araujo Vieira,  Ana Cl\'audia P Ter\c{c}as-Tretell,  Marcial Francis Galera,  Maria Denise Fernandes Carvalho de Andrade,  Lavinia Schuler-Faccini,  Fernanda Sales Luiz Vianna,  Patr\'{\i}cia Ashton-Prolla,  Lucas R Fraga,  Julia do Amaral Gomes }
  
  \affil{ Unichristus - Centro Universit\'ario Christus,  UNIVERSIDADE FEDERAL DO RIO GRANDE DO SUL }
  \vspace{-5mm}
  \date{}
  
  %---------------------------------------------------------------------------------------- 
  
  \begin{document}
  
  
  \maketitle % Insert title
  
  
  \thispagestyle{fancy} % All pages have headers and footers
  %----------------------------------------------------------------------------------------  
  % ABSTRACT
  
  %----------------------------------------------------------------------------------------  
  
  \begin{abstract}
  Congenital ZIKV Syndrome (CZS) occurs in up to 42\% of individuals exposed to ZIKV prenatally. Deregulation in gene expression and protein levels of components of the p53 signaling pathway,  such as p53 and MDM2,  due to ZIKV infection has been reported. Here,  we evaluate functional polymorphisms in genes of the p53 signaling pathway as risk factors to CZS. This study was approved by the Ethics Committees of all participating hospitals. Forty children born with CZS (case group) and forty-eight children exposed to ZIKV,  but born without congenital anomalies (control group) were included in this study. Case group was recruited in five Brazilian research and/or assistance centers in North (n=4),  Northeast (n=21),  Midwest (n=14) and South (n=1) regions of Brazil. Control group was recruited specially from Midwest (n=46),  but also from North (n=1) and South (n=1) regions. Sociodemographic and pregnancy characteristics were obtained from questionnaires. Clinical data were obtained from chart review and consultation done by physicians. Blood samples were collected from participants for the DNA extraction. Genotyping was performed using the TaqMan\textsuperscript{\textcopyright} Genotyping Assay method in the Step One PlusTM Real-Time PCR System. Gestational and sociodemographic information as well as the genotypic and allelic frequencies of functional polymorphisms in TP53,  MDM2,  MIR605 and LIF genes were compared between the two groups. Quantitative variables were compared between groups by Student's t test or Mann–Whitney U test and categorical variables were compared by chi-squared test or Fisher’s Exact Test. Hardy-Weinberg equilibrium was tested for all polymorphisms. A p-value <0.05 was considered statistically significant. We found children with CZS exposed predominantly in the first trimester and controls in the third trimester (p<0.001). Moreover,  children with CZS were predominantly from families with a lower socioeconomic level (p=0.008). We did not find a statistically significant association between the investigated polymorphisms and development of CZS; however,  we found an association of the TP53 rs1042522,  which is associated with a more potent p53-induced apoptosis,  and the phenotype lissencephaly in children with CZS (p=0.007). Our findings suggest that the ZIKV infection in the first trimester of pregnancy as well as the socioeconomic level may be environmental risk factors to CZS as well as the TP53 rs1042522 could be a possible genetic risk factor for the development of lissencephaly in children with CZS.
  
  Funding: CNPq; INAGEMP; FIPE-HCPA \\
  \href{http://ab3c.org.br/xpress_pres2020/xmxp2020-305268.html}{Link to Video:}

  \end{abstract}
   
  \end{document} 