
  \documentclass[twoside]{article}
  \usepackage[affil-it]{authblk}
  \usepackage{lipsum} % Package to generate dummy text throughout this template
  \usepackage{eurosym}
  \usepackage[sc]{mathpazo} % Use the Palatino font
  \usepackage[T1]{fontenc} % Use 8-bit encoding that has 256 glyphs
  \usepackage[utf8]{inputenc}
  \linespread{1.05} % Line spacing-Palatino needs more space between lines
  \usepackage{microtype} % Slightly tweak font spacing for aesthetics\[IndentingNewLine]
  \usepackage[hmarginratio=1:1,top=32mm,columnsep=20pt]{geometry} % Document margins
  \usepackage{multicol} % Used for the two-column layout of the document
  \usepackage[hang,small,labelfont=bf,up,textfont=it,up]{caption} % Custom captions under//above floats in tables or figures
  \usepackage{booktabs} % Horizontal rules in tables
  \usepackage{float} % Required for tables and figures in the multi-column environment-they need to be placed in specific locations with the[H] (e.g. \begin{table}[H])
  \usepackage{hyperref} % For hyperlinks in the PDF
  \usepackage{lettrine} % The lettrine is the first enlarged letter at the beginning of the text
  \usepackage{paralist} % Used for the compactitem environment which makes bullet points with less space between them
  \usepackage{abstract} % Allows abstract customization
  \renewcommand{\abstractnamefont}{\normalfont\bfseries} 
  %\renewcommand{\abstracttextfont}{\normalfont\small\itshape} % Set the abstract itself to small italic text\[IndentingNewLine]
  \usepackage{titlesec} % Allows customization of titles
  \renewcommand\thesection{\Roman{section}} % Roman numerals for the sections
  \renewcommand\thesubsection{\Roman{subsection}} % Roman numerals for subsections
  \titleformat{\section}[block]{\large\scshape\centering}{\thesection.}{1em}{} % Change the look of the section titles
  \titleformat{\subsection}[block]{\large}{\thesubsection.}{1em}{} % Change the look of the section titles
  \usepackage{fancyhdr} % Headers and footers
  \pagestyle{fancy} % All pages have headers and footers
  \fancyhead{} % Blank out the default header
  \fancyfoot{} % Blank out the default footer
  \fancyhead[C]{X-meeting eXperience $\bullet$ November 2020} % Custom header text
  \fancyfoot[RO,LE]{} % Custom footer text
  %----------------------------------------------------------------------------------------
  % TITLE SECTION
  %---------------------------------------------------------------------------------------- 
 
 \title{\vspace{-15mm}\fontsize{24pt}{10pt}\selectfont\textbf{ Large runs of homozigozity regions in Arapaima gigas (Pirarucu) suggest the existence of genomic regions in single-copy and a sex-related dosage-compensation mechanism. }} % Article title
  
  
  \author{ Tetsu Sakamoto,  Jorge Estefano Santana De Souza,  SIDNEY EMANUEL B DOS SANTOS,  Jos\'e Miguel Ortega,  Renata Lilian Dantas Cavalcante }
  
  \affil{ UNIVERSIDADE FEDERAL DE MINAS GERAIS,  UNIVERSIDADE FEDERAL DO RIO GRANDE DO NORTE,  UNIVERSIDADE FEDERAL DO PAR\'A }
  \vspace{-5mm}
  \date{}
  
  %---------------------------------------------------------------------------------------- 
  
  \begin{document}
  
  
  \maketitle % Insert title
  
  
  \thispagestyle{fancy} % All pages have headers and footers
  %----------------------------------------------------------------------------------------  
  % ABSTRACT
  
  %----------------------------------------------------------------------------------------  
  
  \begin{abstract}
  Arapaima gigas,  known as Pirarucu,  is the largest freshwater bony fish in the world that dwells in the Amazon Basin. It is an Osteoglossiformes with great potential for aquaculture because of its large size,  fast growing rate,  and quality of flesh. Albeit these advantages,  Pirarucu’s reproductive management in captivity is limited,  in particular,  due to the lack of external sexual dimorphism,  which makes sex identification difficult in this species. Here we used genomic data from Pirarucu to seek molecular markers which could assist in their sex identification and in unraveling the molecular mechanisms involved in sex differentiation. Previous studies on genome comparison of both sexes identified male specific regions,  suggesting the XY system for Pirarucu. However,  because of its small size,  the topic is still in debate. In this study,  we aimed to search for single-copy genomic regions in one sex of Pirarucu,  which could be associated with the sex determination mechanism,  by identifying sex-specific runs of homozygosity (ROH) regions. We used genomic data from six adults,  three males and three females available at NCBI under the Bioproject IDs PRJEB22808 and PRJNA540910. BWA (v. 0.7.12) and Samtools (v. 1.7) were used for the genome indexing,  mapping,  and alignment steps of reads against the reference genome of Pirarucu ID GCA\_900497675.1 deposited also in NCBI. Bedtools genomecov (v. 2.24) was used to analyze the total sample coverage. For calling variants,  Varscan (v. 2.4.0) was used. The results were processed using R (v. 3.6.0) and the graphics were generated with ggplot2 library (v.3.2.1). To identify ROH regions,  we verified the occurrence of heterozygous sites along the genome of each sample and selected those scaffolds that presented (1) the average of sites in heterozygosis = to 5 for one sex and (2) the size of the scaffold being = to 500.000 bp. We identified 22 scaffolds with large ROH in one sex and with the antagonistic scenario in the other one,  suggesting that these regions are in single-copy in one sex and being affected by the dosage compensation mechanism in the opposite sex. These scaffolds are promissory candidates of a molecular marker for sex differentiation in Pirarucu,  although analyses with more samples and molecular tests are needed. Furthermore,  identifying ROH for comparative genomic analysis demonstrated to be an interesting strategy to assist us in the unraveling of the sex determination system in Pirarucu.
  
  Funding:  \\
  \href{http://ab3c.org.br/xpress_pres2020/xmxp2020-305296.html}{Link to Video:}

  \end{abstract}
   
  \end{document} 