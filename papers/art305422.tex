
  \documentclass[twoside]{article}
  \usepackage[affil-it]{authblk}
  \usepackage{lipsum} % Package to generate dummy text throughout this template
  \usepackage{eurosym}
  \usepackage[sc]{mathpazo} % Use the Palatino font
  \usepackage[T1]{fontenc} % Use 8-bit encoding that has 256 glyphs
  \usepackage[utf8]{inputenc}
  \linespread{1.05} % Line spacing-Palatino needs more space between lines
  \usepackage{microtype} % Slightly tweak font spacing for aesthetics\[IndentingNewLine]
  \usepackage[hmarginratio=1:1,top=32mm,columnsep=20pt]{geometry} % Document margins
  \usepackage{multicol} % Used for the two-column layout of the document
  \usepackage[hang,small,labelfont=bf,up,textfont=it,up]{caption} % Custom captions under//above floats in tables or figures
  \usepackage{booktabs} % Horizontal rules in tables
  \usepackage{float} % Required for tables and figures in the multi-column environment-they need to be placed in specific locations with the[H] (e.g. \begin{table}[H])
  \usepackage{hyperref} % For hyperlinks in the PDF
  \usepackage{lettrine} % The lettrine is the first enlarged letter at the beginning of the text
  \usepackage{paralist} % Used for the compactitem environment which makes bullet points with less space between them
  \usepackage{abstract} % Allows abstract customization
  \renewcommand{\abstractnamefont}{\normalfont\bfseries} 
  %\renewcommand{\abstracttextfont}{\normalfont\small\itshape} % Set the abstract itself to small italic text\[IndentingNewLine]
  \usepackage{titlesec} % Allows customization of titles
  \renewcommand\thesection{\Roman{section}} % Roman numerals for the sections
  \renewcommand\thesubsection{\Roman{subsection}} % Roman numerals for subsections
  \titleformat{\section}[block]{\large\scshape\centering}{\thesection.}{1em}{} % Change the look of the section titles
  \titleformat{\subsection}[block]{\large}{\thesubsection.}{1em}{} % Change the look of the section titles
  \usepackage{fancyhdr} % Headers and footers
  \pagestyle{fancy} % All pages have headers and footers
  \fancyhead{} % Blank out the default header
  \fancyfoot{} % Blank out the default footer
  \fancyhead[C]{X-meeting eXperience $\bullet$ November 2020} % Custom header text
  \fancyfoot[RO,LE]{} % Custom footer text
  %----------------------------------------------------------------------------------------
  % TITLE SECTION
  %---------------------------------------------------------------------------------------- 
 
 \title{\vspace{-15mm}\fontsize{24pt}{10pt}\selectfont\textbf{ Unrevealing novel potential therapeutic targets in fungi extracellular vesicle proteins: a comparative study between pathogenic and non-pathogenic fungi species }} % Article title
  
  
  \author{ Let\'{\i}cia Graziela Costa Santos,  Marcio Rodrigues,  Fabio Passetti,  Vinicius da Silva Coutinho Parreira }
  
  \affil{ Fiocruz-PR,  Fiocruz,  Instituto Carlos Chagas,  Fiocruz }
  \vspace{-5mm}
  \date{}
  
  %---------------------------------------------------------------------------------------- 
  
  \begin{document}
  
  
  \maketitle % Insert title
  
  
  \thispagestyle{fancy} % All pages have headers and footers
  %----------------------------------------------------------------------------------------  
  % ABSTRACT
  
  %----------------------------------------------------------------------------------------  
  
  \begin{abstract}
  Extracellular vesicles (EV) are double-membrane vesicles associated to intercellular communication. In higher eukaryotes,  EVs are regularly used to cell-cell communication from close cells to far tissues. In pathogenic organisms,  EVs play a role in pathogen-host interplay. The EVs contain different molecules,  including nucleic acids,  proteins,  polysaccharides,  and lipids. Since the discovery of EV production in the fungus Cryptococcus neoformans,  the importance of EVs release in its pathogenicity have been elucidated. Shotgun proteomics is the standard strategy to study the proteome composition of a given sample. Therefore,  shotgun proteomics has been used to identify fungi EV proteins in many clinical samples. To date,  few studies have addressed the proteomic content of EVs from multiple pathogenic fungi species. In this context,  orthology analysis allows to identify relations about genes and its proteins in different species through the peptide sequences comparison. Our main objective was to use an orthology approach to compare EV shotgun proteomics data derived from 8 pathogenic and 1 non-pathogenic species,  as follows: Aspergillus fumigatus,  Candida albicans,  Cryptococcus deuterogattii,  C. neoformans,  Histoplasma capsulatum,  Paracoccidioides brasiliensis,  Sporothrix brasiliensis,  Sporothrix schenckii,  and Saccaromyces cerevisiae. Proteins detected through shotgun proteomics were compared using an orthology approach based on Uniprot and FungiDB databases. We integrated data for 11, 433 proteins detected in fungi EVs,  resulting in 3, 834 different orthogroups. Proteins with the Hsp70 Pfam domain were clustered in the unique orthogroup (OG6\_100083) identified for all fungi species. Such proteins are associated with stress response,  survival,  and morphological changes of different fungi species. Although no orthogroup limited to pathogenic fungi EV was found,  we identified 5 orthogroups exclusive for S. cerevisiae. Using the criteria of at least 6 pathogenic fungi species to define a cluster,  we detected the following unique pathogenic orthogroups,  which are listed according to the most frequent Pfam annotation: ATPase family associated with various cellular activities (OG6\_101915),  Nucleoside diphosphate kinase (OG6\_100304),  Ribosomal S17 (OG6\_100832),  Core histone H2A/H2B/H3/H4 (OG6\_100082),  and RNA recognition motif (OG6\_100425).  Taken together,  our data suggest that Hsp70-related proteins play a key role in fungi EVs,  regardless the pathogenic status. Using an orthology approach,  we identified at least 5 protein domains that may be investigated as novel potential therapeutic targets against pathogenic fungi.
  
  Funding: CAPES,  CNPq,  and Fiocruz \\
  \href{http://ab3c.org.br/xpress_pres2020/xmxp2020-305422.html}{Link to Video:}

  \end{abstract}
   
  \end{document} 