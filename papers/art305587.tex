
  \documentclass[twoside]{article}
  \usepackage[affil-it]{authblk}
  \usepackage{lipsum} % Package to generate dummy text throughout this template
  \usepackage{eurosym}
  \usepackage[sc]{mathpazo} % Use the Palatino font
  \usepackage[T1]{fontenc} % Use 8-bit encoding that has 256 glyphs
  \usepackage[utf8]{inputenc}
  \linespread{1.05} % Line spacing-Palatino needs more space between lines
  \usepackage{microtype} % Slightly tweak font spacing for aesthetics\[IndentingNewLine]
  \usepackage[hmarginratio=1:1,top=32mm,columnsep=20pt]{geometry} % Document margins
  \usepackage{multicol} % Used for the two-column layout of the document
  \usepackage[hang,small,labelfont=bf,up,textfont=it,up]{caption} % Custom captions under//above floats in tables or figures
  \usepackage{booktabs} % Horizontal rules in tables
  \usepackage{float} % Required for tables and figures in the multi-column environment-they need to be placed in specific locations with the[H] (e.g. \begin{table}[H])
  \usepackage{hyperref} % For hyperlinks in the PDF
  \usepackage{lettrine} % The lettrine is the first enlarged letter at the beginning of the text
  \usepackage{paralist} % Used for the compactitem environment which makes bullet points with less space between them
  \usepackage{abstract} % Allows abstract customization
  \renewcommand{\abstractnamefont}{\normalfont\bfseries} 
  %\renewcommand{\abstracttextfont}{\normalfont\small\itshape} % Set the abstract itself to small italic text\[IndentingNewLine]
  \usepackage{titlesec} % Allows customization of titles
  \renewcommand\thesection{\Roman{section}} % Roman numerals for the sections
  \renewcommand\thesubsection{\Roman{subsection}} % Roman numerals for subsections
  \titleformat{\section}[block]{\large\scshape\centering}{\thesection.}{1em}{} % Change the look of the section titles
  \titleformat{\subsection}[block]{\large}{\thesubsection.}{1em}{} % Change the look of the section titles
  \usepackage{fancyhdr} % Headers and footers
  \pagestyle{fancy} % All pages have headers and footers
  \fancyhead{} % Blank out the default header
  \fancyfoot{} % Blank out the default footer
  \fancyhead[C]{X-meeting eXperience $\bullet$ November 2020} % Custom header text
  \fancyfoot[RO,LE]{} % Custom footer text
  %----------------------------------------------------------------------------------------
  % TITLE SECTION
  %---------------------------------------------------------------------------------------- 
 
 \title{\vspace{-15mm}\fontsize{24pt}{10pt}\selectfont\textbf{ A novel mitovirus from the sand fly Lutzomyia longipalpis shows sRNA profiles consistent with siRNA pathway activation }} % Article title
  
  
  \author{ Flavia Viana Ferreira,  Felipe Ferreira da Silva,  Liliane Santana Oliveira Kashiwabara,  Jo\~ao Trindade Marques,  Arist\'oteles G\'oes-Neto,  Eric Roberto Guimar\~aes Rocha Aguiar,  Arthur Gruber,  Paula Luize Camargos Fonseca }
  
  \affil{ UNIVERSIDADE TECNOL\'OGICA FEDERAL DO PARAN\'A,  UNIVERSIDADE FEDERAL DE MINAS GERAIS,  UNIVERSIDADE DE S\~AO PAULO,  UNIVERSIDADE ESTADUAL DE SANTA CRUZ }
  \vspace{-5mm}
  \date{}
  
  %---------------------------------------------------------------------------------------- 
  
  \begin{document}
  
  
  \maketitle % Insert title
  
  
  \thispagestyle{fancy} % All pages have headers and footers
  %----------------------------------------------------------------------------------------  
  % ABSTRACT
  
  %----------------------------------------------------------------------------------------  
  
  \begin{abstract}
  Hematophagous insects act as major reservoirs of infectious agents due to their intimate contact with a large variety of vertebrate hosts. Lutzomyia longipalpis is the main vector of Leishmania chagasi in the New World,  but its role as a host of viruses is poorly understood. In this work,  publicly available L. longipalpis RNA libraries were subjected to progressive assembly using viral profile HMMs as seeds. A viral sequence presented a size distribution of small RNAs consistent with the activation of the siRNA pathway. This sequence 2, 697-base corresponds to a monopartite ssRNA(+) genome of a virus called Lul-MV-1. A single ORF encoding an RNA-directed RNA polymerase covers almost the entire genome and uses a typical organellar genetic code with tryptophan being mostly coded by UGA. A phylogenetic analysis positioned Lul-MV1 in a monophyletic clade composed of mitoviruses mostly found in fungal,  but also in crustaceans. To determine whether the virus was infecting a fungus from the sand fly microbiota or the phlebotomine itself,  we analyzed some molecular characteristics of the genome. Dinucleotide composition and codon usage showed profiles similar to mitochondrial DNA of invertebrate hosts. Also,  base preference and size of sRNAs were analogous to those observed in viruses that infect sand flies,  suggesting that L. longipalpis is the putative host. Finally,  RT-PCR of different insect pools confirmed the presence of Lul-MV-1 in seven out of eight tested samples. Concluding,  the strategy used in this work permitted to identify and characterize for the first time of a mitovirus infecting an insect host.
  
  Funding:   \\
  \href{http://ab3c.org.br/xpress_pres2020/xmxp2020-305587.html}{Link to Video:}

  \end{abstract}
   
  \end{document} 