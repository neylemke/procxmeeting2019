
  \documentclass[twoside]{article}
  \usepackage[affil-it]{authblk}
  \usepackage{lipsum} % Package to generate dummy text throughout this template
  \usepackage{eurosym}
  \usepackage[sc]{mathpazo} % Use the Palatino font
  \usepackage[T1]{fontenc} % Use 8-bit encoding that has 256 glyphs
  \usepackage[utf8]{inputenc}
  \linespread{1.05} % Line spacing-Palatino needs more space between lines
  \usepackage{microtype} % Slightly tweak font spacing for aesthetics\[IndentingNewLine]
  \usepackage[hmarginratio=1:1,top=32mm,columnsep=20pt]{geometry} % Document margins
  \usepackage{multicol} % Used for the two-column layout of the document
  \usepackage[hang,small,labelfont=bf,up,textfont=it,up]{caption} % Custom captions under//above floats in tables or figures
  \usepackage{booktabs} % Horizontal rules in tables
  \usepackage{float} % Required for tables and figures in the multi-column environment-they need to be placed in specific locations with the[H] (e.g. \begin{table}[H])
  \usepackage{hyperref} % For hyperlinks in the PDF
  \usepackage{lettrine} % The lettrine is the first enlarged letter at the beginning of the text
  \usepackage{paralist} % Used for the compactitem environment which makes bullet points with less space between them
  \usepackage{abstract} % Allows abstract customization
  \renewcommand{\abstractnamefont}{\normalfont\bfseries} 
  %\renewcommand{\abstracttextfont}{\normalfont\small\itshape} % Set the abstract itself to small italic text\[IndentingNewLine]
  \usepackage{titlesec} % Allows customization of titles
  \renewcommand\thesection{\Roman{section}} % Roman numerals for the sections
  \renewcommand\thesubsection{\Roman{subsection}} % Roman numerals for subsections
  \titleformat{\section}[block]{\large\scshape\centering}{\thesection.}{1em}{} % Change the look of the section titles
  \titleformat{\subsection}[block]{\large}{\thesubsection.}{1em}{} % Change the look of the section titles
  \usepackage{fancyhdr} % Headers and footers
  \pagestyle{fancy} % All pages have headers and footers
  \fancyhead{} % Blank out the default header
  \fancyfoot{} % Blank out the default footer
  \fancyhead[C]{X-meeting eXperience $\bullet$ November 2020} % Custom header text
  \fancyfoot[RO,LE]{} % Custom footer text
  %----------------------------------------------------------------------------------------
  % TITLE SECTION
  %---------------------------------------------------------------------------------------- 
 
 \title{\vspace{-15mm}\fontsize{24pt}{10pt}\selectfont\textbf{ Unraveling potential probiotic features in the genome of Lactobacillus paragasseri }} % Article title
  
  
  \author{ Flavia Figueira Aburjaile,  Rodrigo Profeta Silveira Santos,  Marcus Vinicius Can\'ario Viana,  Vasco Ariston de Carvalho Azevedo,  Rodrigo Dias de Oliveira Carvalho }
  
  \affil{ IOC/Fiocruz,  UNIVERSIDADE FEDERAL DE MINAS GERAIS }
  \vspace{-5mm}
  \date{}
  
  %---------------------------------------------------------------------------------------- 
  
  \begin{document}
  
  
  \maketitle % Insert title
  
  
  \thispagestyle{fancy} % All pages have headers and footers
  %----------------------------------------------------------------------------------------  
  % ABSTRACT
  
  %----------------------------------------------------------------------------------------  
  
  \begin{abstract}
  The vaginal microbiota is dominated by lactobacilli,  which exerts important health-promoting effects and prevention of bacterial vaginosis (BV). Lactobacillus paragasseri is a sister taxon related to L. gasseri and has been recently described in 2018 as a new species. Therefore,  its role in the vaginal environment must be investigated. Moreover,  previous studies suggest the vaginal ecosystem as a potential source of probiotic strains to be used in the treatment of other dysbiosis related diseases such as gastrointestinal disorders. Currently,  comparative genomics studies have been shown as a promising tool in the prediction of the possible mechanisms related to their beneficial activity. In this work two L. paragasseri strains,  CRI16 and CRI18 were recovered from healthy women of reproductive age. The strains were isolated from healthy subjects while CRI22 was isolated from a BV patient. The genomes were assembled with SPAdes and annotated via RAST. Eight genome sequences of L. paragasseri were obtained from NCBI and included in the comparative analysis. Genes involved in protective mechanisms,  such as bacteriocins were predicted using BAGEL. The metabolic pathway of other health-promoting features such as vitamins and short-chain fatty acids and the adaptation to gastrointestinal stress conditions were characterized via PATRIC. The presence of genetic mobile elements such as plasmids and phages was evaluated in the strains to evaluate their safety for human use. In this context,  antibiotic resistance genes and virulence factors were investigated using CARD and VFDB databases respectively. Our results revealed several beneficial features such as the presence of three and six bacteriocin genes in CRI16 and CRI18 respectively. Metabolic pathway analysis indicates both strains are able to produce vitamins K2,   B1 and B9. Furthermore,  they also seem to be able to produce Lactate and to degrade biogenic amines,  including putrescine and spermidine although they may produce cadaverine as well. Interestingly,  the strains present enzymes genes involved in bile salts deconjugation,  suggesting adaptation to gastrointestinal conditions. Regarding safety issues,  one intact prophage sequence was predicted in both strains and only CRI16 harbors a plasmid of 40kb,  although no antibiotic resistance or virulence gene was found. Our study represents the first step in the characterization of the L. paragasseri strains CRI16 and CRI18 suggesting them as potential candidates for probiotic use.
  
  Funding: CAPES \\
  \href{http://ab3c.org.br/xpress_pres2020/xmxp2020-305812.html}{Link to Video:}

  \end{abstract}
   
  \end{document} 