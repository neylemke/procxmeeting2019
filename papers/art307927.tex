
  \documentclass[twoside]{article}
  \usepackage[affil-it]{authblk}
  \usepackage{lipsum} % Package to generate dummy text throughout this template
  \usepackage{eurosym}
  \usepackage[sc]{mathpazo} % Use the Palatino font
  \usepackage[T1]{fontenc} % Use 8-bit encoding that has 256 glyphs
  \usepackage[utf8]{inputenc}
  \linespread{1.05} % Line spacing-Palatino needs more space between lines
  \usepackage{microtype} % Slightly tweak font spacing for aesthetics\[IndentingNewLine]
  \usepackage[hmarginratio=1:1,top=32mm,columnsep=20pt]{geometry} % Document margins
  \usepackage{multicol} % Used for the two-column layout of the document
  \usepackage[hang,small,labelfont=bf,up,textfont=it,up]{caption} % Custom captions under//above floats in tables or figures
  \usepackage{booktabs} % Horizontal rules in tables
  \usepackage{float} % Required for tables and figures in the multi-column environment-they need to be placed in specific locations with the[H] (e.g. \begin{table}[H])
  \usepackage{hyperref} % For hyperlinks in the PDF
  \usepackage{lettrine} % The lettrine is the first enlarged letter at the beginning of the text
  \usepackage{paralist} % Used for the compactitem environment which makes bullet points with less space between them
  \usepackage{abstract} % Allows abstract customization
  \renewcommand{\abstractnamefont}{\normalfont\bfseries} 
  %\renewcommand{\abstracttextfont}{\normalfont\small\itshape} % Set the abstract itself to small italic text\[IndentingNewLine]
  \usepackage{titlesec} % Allows customization of titles
  \renewcommand\thesection{\Roman{section}} % Roman numerals for the sections
  \renewcommand\thesubsection{\Roman{subsection}} % Roman numerals for subsections
  \titleformat{\section}[block]{\large\scshape\centering}{\thesection.}{1em}{} % Change the look of the section titles
  \titleformat{\subsection}[block]{\large}{\thesubsection.}{1em}{} % Change the look of the section titles
  \usepackage{fancyhdr} % Headers and footers
  \pagestyle{fancy} % All pages have headers and footers
  \fancyhead{} % Blank out the default header
  \fancyfoot{} % Blank out the default footer
  \fancyhead[C]{X-meeting eXperience $\bullet$ November 2020} % Custom header text
  \fancyfoot[RO,LE]{} % Custom footer text
  %----------------------------------------------------------------------------------------
  % TITLE SECTION
  %---------------------------------------------------------------------------------------- 
 
 \title{\vspace{-15mm}\fontsize{24pt}{10pt}\selectfont\textbf{ First-ever described Virome of the Amazonian Lake Bolonha: contributions to the understanding of water-related public health concerns }} % Article title
  
  
  \author{ Bruna Ver\^onica Azevedo Gois,  Kenny da Costa Pinheiro,  Andressa de Oliveira Arag\~ao,  Ana Lidia Queiroz Cavalcante,  adriana ribeiro carneiro folador,  Rommel Thiago Juc\'a Ramos,  Wylerson Guimar\~aes Nogueira }
  
  \affil{ UFPA - UNIVERSIDADE FEDERAL DO PAR\'A/GEAM,  Universidade Federal do Par\'a -UFPA,  UNIVERSIDADE FEDERAL DE MINAS GERAIS,  UNIVERSIDADE ESTADUAL DA PARA\'IBA,  UNIVERSIDADE FEDERAL DO PAR\'A,  UNIVERSIDADE DO ESTADO DO PAR\'A }
  \vspace{-5mm}
  \date{}
  
  %---------------------------------------------------------------------------------------- 
  
  \begin{document}
  
  
  \maketitle % Insert title
  
  
  \thispagestyle{fancy} % All pages have headers and footers
  %----------------------------------------------------------------------------------------  
  % ABSTRACT
  
  %----------------------------------------------------------------------------------------  
  
  \begin{abstract}
  The availability of safe water supplies and adequate sanitation is vital to protect populational health and is one of the basic human rights,  according to the World Health Organization. Despite the importance of water supplies,  the understanding of the ecology of freshwater viruses still lacks an in-depth description of its diversity,  as well as the microbiological interactions that occur in these environments. The Amazonian Lake Bologna,  from Bel\'em,  capital of the Brazilian State of Par\'a,  is a key source of water that supplies the city and all of its metropolitan region,  yet it remains unexplored regarding the contents of its virome and viral diversity composition. Therefore,  this work's main aim is to clarify in terms of taxonomic diversity the species of DNA viruses that are present in this lake,  especially bacteriophages and cyanophages,  since they can act both as transducers of resistance genes and reporters of water quality for human consumption. For this work,  we used the metagenomic sequencing data generated by Alves et al. (2020),  and we analyzed it at the taxonomic level using the tools Kraken2,  Bracken,  and Pavian. Later,  the data was assembled using Genome Detective,  which performs the assembly of viruses. The results observed in this work suggest the existence of a widely diverse viral community and an established microbial phage regulated dynamics in the Lake Bolonha. This work is the first-ever to describe the virome of Lake Bolonha using a metagenomic approach based on high-throughput sequencing,  as it contributes to the understanding of water-related public health concerns regarding the spreading of antibiotic resistance genes and population control of native bacteria and cyanobacteria.
  
  Funding: CAPES; FAPESPA; FAPEMIG \\
  \href{http://ab3c.org.br/xpress_pres2020/xmxp2020-307927.html}{Link to Video:}

  \end{abstract}
   
  \end{document} 