
\chapter{Organizing Committee}

\begin{description}

 \item[AB3C President]: Ney Lemke (UNESP)

\item[AB3C Vice President]: Marcelo Brand\~ao (Unicamp) 

  
\item[AB3C Secretaries]:

\begin{itemize}
 \item Fabr\'{\i}cio Martins Lopes  (UTFPR)
 \item Vasco Ariston de Carvalho Azevedo (UFMG)
\end{itemize}

\item[AB3C Financial Department]:

\begin{itemize}
\item Nicole Scherer (INCA) 
\item Arthur Gruber (USP)
\end{itemize}


\item[Poster Session Organizers]:

\begin{itemize}
\item Alexandre Paschoal (UTFPR)
\item André Yoshiaki Kashiwabara (USP)
\end{itemize}

\item[Highlight Track Organizer]:
\begin{itemize}
\item Arthur Gruber (USP)
\end{itemize} 
\end{description}


\newpage
\chapter{Introduction}
The Brazilian Association of Bioinformatics and Computational Biology (AB3C) is
a scientific society funded in July 12th 2004.
Since its creation, AB3C has been responsible for the annual conference entitled
''X-Meeting'' which is the main Bioinformatics and Computation Biology event in
Brazil. This year its 13th edition occurred in Campos do Jordão from October
\nth{30} to November \nth{1} .

	
Bioinformatics is now a strategic area for Brazil and all Latin America and,
therefore, it is also strategic to the development of Science, Technology and
Economy. The X-Meeting is a Brazilian event with international reach which has
an average of 200 participants. The Conference is an opportunity for students,
researchers and companies to interact and difuse knowledge. The AB3C has been a
pioneer society in the field of Bioinformatics in Brazil and we have a history
of ten past very productive meetings.

%\begin{figure}[h]
    %\begin{center}
  %\includegraphics[scale=0.7]{wordcloud}
%\end{center}
%\caption{Word Cloud for the words used on the Conference Papers Titles}
%\end{figure}
%
%
%\begin{landscape}
%\begin{figure}[h]
    %\begin{center}
  %\includegraphics[scale=1.5,angle=0]{grafo}
%\end{center}
%\caption{Graph representing the network of collaborations of the X-meeting 2016. An interactive 
%version of this graph can be seen at \url{https://neylemke.github.io/assets/grafo.html}}
%\end{figure}
%\end{landscape}
